\documentclass[./devocionario.tex]{subfiles}

\begin{document}

    \subsection*{La Señal de la Cruz}
    \begin{tabular} { p{0.5\textwidth} p{0.5\textwidth} }
        En el Nombre del Padre, y del Hijo, y del Espíritu Santo. Amén.
        
        &
        
        In Nómine Pátris, et Filii, et Spíritus Sancti. Amen.
    \end{tabular}

    \subsection*{Padre Nuestro}
    \begin{tabular} { p{0.5\textwidth} p{0.5\textwidth} }
        Padre Nuestro que estás en los cielos, santificado sea tu Nombre. Venga a nosotros tu Reino. 
        Hágase tu voluntad, así en la tierra como en el cielo. El pan nuestro de cada día dánosle hoy. 
        Y perdónamos nuestras deuda, así como nosotros perdonamos a nuestros deudores. 
        Y no nos dejes caer en la tentación: mas líbranos del mal. Amén.
        
        &
        
        Pater noster, qui es in cœlis, sanctifificétur nomen tuum. Advéniat regnum tuum. 
        Fiat voluntas tua, sicut in cœlo et in terra. Panem nóstrum quotidiánum da nobis hódie. 
        Et dimite nosbis debita nostra, sicut et nos dimittimus debitóribus nostris. 
        Et ne nos indúcas in tentatiónem: sed libera nos a malo. Amen.
    \end{tabular}

    \subsection*{Ave María}
    \begin{tabular} { p{0.5\textwidth} p{0.5\textwidth} }
        Dios de salve, María, llena eres de gracia, el Señor es contigo; dendita eres entre todas las mujeres, 
        y bendito es el fruto de tu vientre, Jesús. Santa María, Madre de Dios, ruega por nosotros pecadores, 
        ahora u en el hora de nuestra muerte. Amén.
        
        &
        
        Ave María, grátia plena, Dóminus tecum; benedicta tu in muliéribus, et benedictum fructus ventris tui, 
        Jesus. Sancta Maria, Mater Dei, ora pro nobis peccatóribus, nunc et in hora mortis nostræ. Amen.
    \end{tabular}

    \subsection*{Gloria Patri}
    \begin{tabular} { p{0.5\textwidth} p{0.5\textwidth} }
        \textbf{Gloria al Padre, al Hijo, y al Espíritu Santo.} 
        
        &

        \textbf{Gloria Patri, et Filio, et Spíritui Sancto.}\\
        
        \textit{Como era en el principio, ahora, y siempre, y por los siglos de los siglos. Amén.}
        
        &

        \textit{Sicut erat in pcincípio et nunc, et semper et in sæcula sæculórum, Amen.}
    \end{tabular}

    \subsection*{Confíteor}
    \begin{tabular} { p{0.5\textwidth} p{0.5\textwidth} }
        Yo pecador me confieso a Dios todopoderoso, a la bienaventurada siempre Virgen María, al bienaventurado San Miguel Arcángel, 
        al bienaventurado San Juan Bautista, a los Santos Apóstoles Pedro y Pablo, a todos los Santos y a vos, Padre, que pequé mucho 
        de pensamiento, palabra y obra: por mi culpa, por mi culpa, por mi grandísima culpa. Por tanto, ruego a la bienaventurada 
        siempre Virgen María, al bienaventurado San Miguel Arcángel, al bienaventurado San Juan Bautista, a los Santos Apóstoles 
        Pedro y Pablo, a todos los Santos, y a vos, Padre, que roguéis por mi a Dios Nuestro Señor.
        
        &

        Confíteor Deo omnipoténti, beátæ Maríæ semper Virigini, beáto Michaéli Archángelo, beáto Joánni Baptístæ, sanctis Apóstolis Petro et Paulo, 
        ómníbus Sanctis et tibi, pater, quia peccávi nimis, cogitatióne, verbo et ópere, mea culpa, mea culpa, mea máxima culpa. Ideo precor beátam 
        Maríam semper Virgínem, beátum Michaélem Archángelum, beátum Joánnem Bastístam, sanctis Apóstolos Petrum et Paolum, omnes Sanctos, et te, pater, 
        oráre pro me ad Dóminum Deum nostrum.\\

        \textbf{El Señor omnipotente tenga piedad de nosotros y, perdonados nuestros pecados, nos lleve a la vida eterna. Amén.}

        &

        \textbf{Misereátur nostri omnipotens Deus, et dimissis pecátis nostris, perdúcat nos ad vitam ætérnam. Amen.}\\

        \textit{El Señor omnipotente y misericordioso nos conceda la indulgencia, la absolución y el perdón de nuestros pecados. Amén.}

        &

        \textit{Indulgéntiam, absolutiónem, et remissiónem pecatórum nostrórum, tribuat nobis omnipotens et miséricors Dóminus. Amen.}
    \end{tabular}

    \subsection*{Acto de contrición}
    Señor mío Jesucristo, Dios y Hombre verdadero, Creador y Redentor mío: por ser vos quién sois, y porque os amo sobre todas las cosas, 
    me pesa de todo corazón de haberos ofendido, propongo firmemente nunca más pecar, y apartarme de todas las ocasiones de ofenderos, 
    confesarme, y cumplir la penitencia que me fuere impuesta; ofrézcoos mi vida, obras y trabajos en satisfacción de todos mis pecados; 
    y confío en vuestra bondad y misericordia infinita me los perdonaréis por los merecimientos de vuestra preciosísima sangre, pasión y muerte, 
    y me daréis gracia para enmendarme y para perseverar en vuestro santo servicio hasta el fin de mi vida. Amén.

    \subsection*{Credo Apostólico}
    \begin{tabular} { p{0.5\textwidth} p{0.5\textwidth} }
        Creo en Dios, Padre todopoderoso. Creador del cielo y de la tierra. Y en Jesucristo, su único Hijo, Nuestro Señor, 
        que fue concebido por obra y gracia del Espíritu Santo; nació de Santa María Vírgen; padeció bajo el poder de Poncio Pilato, 
        fue crucificado, muerto y sepultado; descendió a los infiernos; al tercer día resucitó de entre los muertos; subió a los cielos, 
        está sentado a la derecha de Dios Padre todopoderoso; desde allí ha de venir a juzgar a vivos y muertos. 
        Creo en el Espíritu Santo, la Santa Iglesia Católica, la comunión de los Santos, el perdón de los pecados, 
        la resurección de la carne y la vida eterna. Amén.

        &

        Credo in Deum, Patrem omnipoténtem. Creatórem cœli et terræ. Et in Jesum CHristum, Filium ejus únicum, Dóminum nostrum; 
        qui concéptus est de Spíritu Sancto; natus ex María Virgine; passus sub Póntio Pilato, crucifíxus, mortuus et sepúltus: 
        descéndit ad inferos; tértia die resurréxit a mórtuis: ascéndit ad cœlos, sedet ad dexteram Dei Patris omnipoténtis; 
        inde ventúrus est judicáre vivos et mórtuos. Credo in Spíritum Sanctum, sanctam Ecclésiam cathólicam, Sanctórum communiónem, 
        remisiónem peccatórum, carnis resurrectiónem, vitam ætérnam. Amen
    \end{tabular}

    \subsection*{Salve}
    \begin{tabular} { p{0.5\textwidth} p{0.5\textwidth} }
        Dios te salve, Reina y Madre de mi­se­ri­cordia, vida, dulzura y esperanza nuestra; Dios te salve. 
        A ti llamamos los desterrados hijos de Eva; a ti suspiramos, gimiendo y llorando en este valle de lágrimas. 
        Ea, pues, Señora, abogada nuestra, vuelve a nosotros esos tus ojos mi­se­ri­cordiosos. Y después de este destierro, muéstranos a Jesús, 
        fruto bendito de tu vientre. ¡Oh cle­men­tísima, oh piadosa, oh dulce siempre Virgen María!

        &

        Salve, Regina, Mater mi­se­ri­córdiæ, vita, dulcédo et spes nostra, salve. Ad te clamámus, éxsules fílii Hevæ. 
        Ad te suspirámus geméntes et flentes in hac lacrimárum valle. Éia ergo, advocáta nostra, illos tuos mi­se­ri­córdes óculos ad nos convérte. 
        Et Iesum benedíctum fructum ventris tui, nobis, post hoc exsílium, osténde. O clemens, o pia, o dulcis Virgo Maria!\\
        
        \textbf{Ruega por nosotros, Santa Madre de Dios.} & \textbf{Ora pro nobis, Sancta Dei Génetrix.}\\
        
        \textit{Para que seamos dignos de alcanzar las promesas de nuestro Señor Jesucristo.}

        &

        \textit{Ut digni efficiámur pro­mi­ssiónibus Christi.}
    \end{tabular}
    
\end{document}