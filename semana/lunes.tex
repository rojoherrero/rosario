\noindent\textcolor{red}{Lectio i \hfill i Apoc 14, 13}

\primeraletragranderoja{E}{n}aquellos días: Oí la voz del cielo, que me decía: Escribe: Bienaventurados los muertos que mueren en el Señor. Ya desde ahora dice el Espíritu que descansen
de sus trabajos, puesto que sus obras los van acompañando.

\vspace{0.5em}

\noindent\textcolor{red}{Lectio ii \hfill Ioan 6, 37-40}

\primeraletragranderoja{E}{n}aquel tiempo: Dijo Jesús a las turbas de los judíos: Todos los que el Padre me da, vendrán a mi; y al que viniere a mi, no le desecharé; porque he descendido
del cielo, no para hacer mi voluntad, sino la de aquel que me ha enviado. Y la voluntad de mi Padre, que me ha enviado, es que yo no pierda ninguno de los que
me ha dado, sino que los resucite a todos en el último día. Por tanto, la voluntad de mi Padre, que me ha enviado, es que todo el que ve al Hijo, y cree en él,
tenga vida eterna, y yo le resucitaré en el último día.

\vspace{0.5em}

\textbf{Orémos}
\primeraletragrande{O}{h}Dios, dador del perdón y que deseáis la salvación del hombre: rogamos a vuestra clemencia que a las almas de todos lso fieles, que de este mundo salieron,
les concedáis por intercesión de la bienaventurada siempre Virgen María y de todos sus Santos, llegar a la participación de la eterna felicidad. Por nuestro
Señor Jesucristo, tu Hijo, que con contigo vive y reina en unión del Espíritu Santo Dios por todos los siglos de los siglos. \respuesta{Amen}

\vspace{0.5mm}

\begin{otherlanguage}{latin}
      \textcolor{red}{P}ater noster, qui es in c{\ae}lis, sanctificétur nomen tuum. Advéniat regnum tuum.
Fiat volúntas tua, sicut in c{\oe}lo et in terra. Panem nóstrum quotidiánum da nobis hódie.
Et dimitte nobis debita nostra, sicut et nos dimittimus debitóribus nostris.
Et ne nos indúcas in tentatiónem: sed libera nos a malo. Amen.

      \vspace{0.25em}

      \textcolor{red}{A}ve María, grátia plena, Dóminus tecum; benedicta tu in muliéribus, et benedíctus fructus ventris tui,
Iesus. Sancta Maria, Mater Dei, ora pro nobis peccatóribus, nunc et in hora mortis nostr{\ae}. Amen.
      
      \vspace{0.25em}

      \versiculorespuestaseguido{Réquiem {\ae}térnam dona eis, Dómine}
    {Et lux perpétua lúceat eis}

      \vspace{0.25em}

      \versiculorespuestaseguido{Requiéscant in pace}{Amen}
\end{otherlanguage}