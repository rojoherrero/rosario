\noindent\textcolor{red}{Lectio i\hfill Lc 3, 23}

\primeraletragranderoja{J}{esús,}al empezar, tenía unos treinta años, y era, según se creía, hijo de José.

\vspace{0.5em}

\noindent\textcolor{red}{Lectio ii\hfill Mt 1, 18-21}

\primeraletragranderoja{E}{stando}desposada la Madre de Jesús, María, con José, sin que antes hubiesen estado juntos, se halló que había concebido en su seno por obra del Espíritu Santo. Mas José, su esposo,
siendo como era justo, y no queriendo infamarla, deliberó dejarla secretamente. Estando él en este pensamiento, he aquí que un ángel del Señor se le apareció en suerños, diciendo:
José, hijo de David, no tengas recelos en recibir a María tu esposa, porque lo que se ha engendrado en su seno, es obra del Espíritu Santo. Así que dará a luz un hijo, a quien pondrás
el nombre de Jesús, pues él es el que ha de salvar a su pueblo de sus pecados.

\begin{center}
      \textcolor{red}{Himno}
\end{center}

\vspace{-0.5em}

\primeraletragrande{O}{h}José! Que los coros celestiales celebren vuestras grandezas; que los cantos de todos los cristianos hagan resonar vuestras alabanzas. Glorioso ya por vuestros méritos, os
unisteis por una casta alianza a la augusta Virgen.

Cuando dominado por la duda y la ansiedad, os asombráis del estado en que se halla vuestra esposa, un Ángel viene a deciros que el hijo que ella ha concebido es obra del Espíritu Santo.

El Señor ha nacido, y le estrecháis en vuestros brazos; partís con él hacia las lejanas playas de Egipto; después de haberle perdido en Jerusalén, le encontráis de nuevos; así vuestros gozos
van mezclados con lágrimas.

Otros son glorificados después de una santa muerte, y los que han merecido la palma son recibidos en el seno de la gloria; por vos, por un admirable destino, semejante a los Santos, y aun
más dichoso, disfrutáis ya en esta vida de la presencia de Dios.

Oh Trinidad soberana, oíd nuestras preces, concedednos el perdón; que los méritos de José nos ayuden a subir al cielo, para siempre el cántico de acción de gracias y de felicidad. Amén.

\vspace{0.5em}

\textbf{Orémos}
\lettrine[lines=2]{O}h Dios, que con inefable pro­videncia elegisteis al bien­ aventurado José para esposo de vuestra santísima Madre; os ro­gamos nos concedáis tener por intercesor en el cielo al que ve­
neramos como protector en la tierra. Vos que vives y reinas con Dios Padre en unidad del Espíritu Santo Dios por los siglos de los siglos. \respuesta{Amén}

\vspace{0.5em}

\begin{otherlanguage}{latin}
      \textcolor{red}{P}ater noster, qui es in c{\ae}lis, sanctificétur nomen tuum. Advéniat regnum tuum.
Fiat volúntas tua, sicut in c{\oe}lo et in terra. Panem nóstrum quotidiánum da nobis hódie.
Et dimitte nobis debita nostra, sicut et nos dimittimus debitóribus nostris.
Et ne nos indúcas in tentatiónem: sed libera nos a malo. Amen.

      \vspace{0.25em}

      \textcolor{red}{A}ve María, grátia plena, Dóminus tecum; benedicta tu in muliéribus, et benedíctus fructus ventris tui,
Iesus. Sancta Maria, Mater Dei, ora pro nobis peccatóribus, nunc et in hora mortis nostr{\ae}. Amen.

      \vspace{0.25em}

      \versiculorespuestaseguido{Glória Patri, et Filio, et Spirítui Sancto}{Sicut erat in princípio et nunc, et semper et in s{\ae}cula s{\ae}culórum. Amen}

      \vspace{0.25em}

      \versiculorespuestaseguido{Ora pro nobis, sancte Joseph}{Ut digni efficiamur promissionisbus Christi. Amen}
\end{otherlanguage}