\noindent\textcolor{red}{Lectio i \hfill 1 Cor 11, 23-29}

\primeraletragranderoja{H}{ermanos:}Yo recibí del Señor lo que también os tengo enseñado, y es que el Señor Jesús la noche misma en que había de ser entregado, tomó el pan, y dando gracias, lo partió, y dijo:
Tomad y comed; este es mi cuerpo, que por vosotros será entregado; haced esto en memoria mía. Y de la misma manera tomó también el cáliz después de haber cenado, diciendo: Este es el
Nuevo Testamento en mi sangre. Haced esto, cuantas veces le bebiereis, en memoria mía. Pues todas las veces que comiereis este pan, y bebiereis este cáliz, anunciaréis la muerte del
Señor hasta que venga. De manera que cualquiera que comiere este pan, o bebiere el cáliz del Señor indignamente, reo será del Cuerpo y la Sangre del Señor. Por lo tanto, examínese a
sí mismo el hombre; y de esta suerte, coma del aquel pan y beba de aquel cáliz; porque quien le come y bebe indignamente, se traga y bebe su propia condenación, no haciendo el debido
discernimiento del cuerpo del Señor.

\vspace{0.5em}

\noindent\textcolor{red}{Lectio ii \hfill Ioan 6, 56-59}

\primeraletragranderoja{E}{n}aquel tiempo: Dijo Jesús a las turbas de los judíos: Mi carne verdaderamente es comida, y mi Sangre es verdaderamente bebida. Quien como mi carne y bebe mi Sangre, en mi mora y yo en él.
Así como el Padre, que me ha enviado, vive, y yo vivo por mi Padre; así quien me come, también él vivirá por mi. Este es el Pan que ha bajado del Cielo. No sucederá como a vuestros padres,
que comieron el maná y, no obstante esto, murieron. Quien come este pan vivirá eternamente.

\vspace{0.5mm}

\textbf{Orémus}
\primeraletragrande{O}{h}Dios, que dejasteis la memoria de vuestra Pasión en ese Sacramento admirable: concedednos que de tal suerte veneremos los sagrados misterios de vuestro Cuerpo 
y vuestra Sangre, que experimentemos continuamente en nuestras almas el fruto de vuestra redención: Vos que vivís y reináis con Dios Padre, en la unidad del Espíritu Santo Dios, 
por los siglos de los siglos, Amén.

\vspace{0.5em}

\begin{otherlanguage}{latin}
      \versiculorespuestaseguido{Adóremus in {\ae}térnum Sanctíssimum Sacraméntum}{Adóremus in {\ae}térnum Sanctíssimum Sacraméntum}

      \vspace{0.25em}
      
      \textcolor{red}{P}ater noster, qui es in c{\ae}lis, sanctificétur nomen tuum. Advéniat regnum tuum.
Fiat volúntas tua, sicut in c{\oe}lo et in terra. Panem nóstrum quotidiánum da nobis hódie.
Et dimitte nobis debita nostra, sicut et nos dimittimus debitóribus nostris.
Et ne nos indúcas in tentatiónem: sed libera nos a malo. Amen.

      \vspace{0.25em}

      \textcolor{red}{A}ve María, grátia plena, Dóminus tecum; benedicta tu in muliéribus, et benedíctus fructus ventris tui,
Iesus. Sancta Maria, Mater Dei, ora pro nobis peccatóribus, nunc et in hora mortis nostr{\ae}. Amen.

      \vspace{0.25em}

      \versiculorespuestaseguido{Deum, in adjutórium meum inténde}{Dómine, ad adjuvándum me festina}

      \textcolor{red}{Lo anterior se repite tres veces}

      \vspace{0.5em}

      \versiculorespuestaseguido{Adóremus in {\ae}térnum Sanctíssimum Sacraméntum}{Adóremus in {\ae}térnum Sanctíssimum Sacraméntum}
\end{otherlanguage}

