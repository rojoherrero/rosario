\noindent\textcolor{red}{Lectio i \hfill Eccli 24, 11-13.15-20}

\primeraletragranderoja{E}{n}todas las cosas busqué el descanso y en la heredad del Señor fijé mi morada. Entonces el Creador de todas las cosas dió sus órdenes, y me habló; y el que a mi me dió el ser
descansó en mi tabernáculo, y me dijo: Habita en Jacob, y sea Israel tu herencia, y arráigate en medio de mis escogidos. Y así fijé mi estancia en Sión, y fué el lugar de mi reposo
la ciudad santa, y en Jerusalén está el trono mío. Y me arraigué en un pueblo glorioso, y en la porción de mi Dios, la cual es su herencia: y mi habitación fué en la plena reunión
de los santos. Elevada estoy cual cedro sobre el Líbano y cual ciprés sobre el monte Sión. Extendí mis ramas como palma de Cades y como rosal plantado en Jericó; me alcé como hermoso
olivo en los campos, y como plátano en las plazas junto al agua. Como cinamomo y bálsamo aromático desprendí fragancia. Como mirra exhalé suave olor.

\vspace{0.5em}

\noindent\textcolor{red}{Lectio ii \hfill Lc 1, 26-28.42}

\lettrine[lines=2]{\textcolor{red}{E}}n aquel tiempo: envió Dios al ángel Gabriel a Nazaret, a una virgen desposada con un varón de nombre José, de la casa de David; el nombre de la virgen era María. 
Entrando a ella, le dijo: Dios te salve, llena de gracia, el Señor es contigo. {!`}Bendita tú entre las mujeres y bendito el fruto de tu vientre!.

\vspace{0.5em}

\noindent\textcolor{red}{Lectio iii \hfill Ioan 19, 25-27}

\lettrine[lines=2]{\textcolor{red}{E}}staba junto a la cruz de Jesús su Madre y la hermana de su Madre, María de Cleofás, y María Magdalena. Jesús, pues, viendo a la Madre, y junto a ella
al discípulo a quien amaba, dice a su Madre: Mujer, he ahí a tu hijo. Luego dice al discípulo: He ahí a tu Madre. Y desde aquella hora la tomó el discípulo en su compañía.

\vspace{0.25em}

\begin{center}
      \textcolor{red}{Himno}
\end{center}

\vspace{-0.5em}

\lettrine[lines=2]{U}n mensajero de la corte ceíestial, descubriendo los se­cretos divinos, saluda llena de gracia a la Virgen, Madre de Dios.

Maria visita a su parienta, Ia madre de Juan Bautista, el cual anuncia, desde el seno materno, la presencia de Jesús.

El Verbo, engendrado desde la- eternidad por la mente del Padre, nace del seno de la V ir ­gen Madre, pequeño infante so­ metido a la muerte.

Aquel niño divino es presen­ tado en el templo; el legislador «e sujeta a la ley; ya allí se ofrece en sacrificio el Redentor, 
rescatado con el rescate de los pobres.

La Madre encuentra al Hijo cuya pérdida lloraba: hállalo en el templo enseñando a los doctores misterios por ellos toda­ vía ignorados.

Gloria a Vos, oh Jesús, naci­do de la Virgen, juntamente con el Padre y el Espíritu Santo, por los siglos de los siglos. Amén.

\vspace{0.5em}

\begin{otherlanguage}{latin}
      \textcolor{red}{P}ater noster, qui es in c{\ae}lis, sanctificétur nomen tuum. Advéniat regnum tuum.
Fiat volúntas tua, sicut in c{\oe}lo et in terra. Panem nóstrum quotidiánum da nobis hódie.
Et dimitte nobis debita nostra, sicut et nos dimittimus debitóribus nostris.
Et ne nos indúcas in tentatiónem: sed libera nos a malo. Amen.

      \vspace{0.25em}

      \textcolor{red}{A}ve María, grátia plena, Dóminus tecum; benedicta tu in muliéribus, et benedíctus fructus ventris tui,
Iesus. Sancta Maria, Mater Dei, ora pro nobis peccatóribus, nunc et in hora mortis nostr{\ae}. Amen.\footnote{\versiculorespuestaseguido{Regina caeli l{\ae}táre, allelúja}{Quia quem meruisti portáre, allelúja}

\vspace{1mm}

\versiculorespuestaseguido{Resurréxit sicut dixit, allelúja}{Ora pro nobis Deum, allelúja}

\vspace{1mm}

\versiculorespuestaseguido{Gaude et l{\ae}táre, Virgo María, allelúja}{Quia surréxit Dóminus vere, allelúja}

\vspace{1mm}

\textbf{Orémus}.-- \textcolor{red}{D}eus, qui per resurrectiónem Filii tui Dómini nostri Jesu Christi,
mundum l{\ae}tificáre dignátus es: pr{\ae}sta, qu{\'\ae}sumus, ut per ejus Genetricem Vírginem Maríam,
perpétu{\ae} capiámus gáudia vit{\ae}. Per eúmdem Christum Dóminum nostrum. \respuesta{Amen} 
 \\[0.5em] \indent\versiculorespuestaseguido{Regina caeli l{\ae}táre, allelúja}{Quia quem meruisti portáre, allelúja}

\vspace{1mm}

\versiculorespuestaseguido{Resurréxit sicut dixit, allelúja}{Ora pro nobis Deum, allelúja}

\vspace{1mm}

\versiculorespuestaseguido{Gaude et l{\ae}táre, Virgo María, allelúja}{Quia surréxit Dóminus vere, allelúja}

\vspace{1mm}

\textbf{Orémus}.-- \textcolor{red}{D}eus, qui per resurrectiónem Filii tui Dómini nostri Jesu Christi,
mundum l{\ae}tificáre dignátus es: pr{\ae}sta, qu{\'\ae}sumus, ut per ejus Genetricem Vírginem Maríam,
perpétu{\ae} capiámus gáudia vit{\ae}. Per eúmdem Christum Dóminum nostrum. \respuesta{Amen} 
}

      \vspace{0.25em}

      \versiculorespuestaseguido{Deum, in adjutórium meum inténde}{Dómine, ad adjuvándum me festina}
\end{otherlanguage}