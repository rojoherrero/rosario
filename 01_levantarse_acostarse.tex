\documentclass[9pt]{article}
\usepackage[utf8]{inputenc}
\usepackage[latin,spanish]{babel}
%\usepackage[spanish]{babel}
\usepackage[T1]{fontenc}
\usepackage[a5paper, margin=10mm]{geometry}
\usepackage{lettrine}
\usepackage{xcolor}
\usepackage{yfonts}
\usepackage{gregoriotex}
\usepackage{multicol}
\usepackage{enumitem}
%\setlength\parindent{0pt}
\pagenumbering{gobble}

\setlength{\columnseprule}{0.5pt}
\def\columnseprulecolor{\color{red}}

\newcommand{\primeraletragranderoja}[2]{
    \lettrine[lines=2]{\textcolor{red}{#1}}{#2}
}

\newcommand{\primeraletragranderojasola}[1]{
    \lettrine[lines=2]{\textcolor{red}{#1}}{}
}

\newcommand{\primeraletragrande}[2]{
    \lettrine[lines=2]{#1}#2
}

\newcommand{\letraroja}[2]{
    \textcolor{red}{#1}#2
}

\newcommand{\versiculo}[1]{
    \textcolor{red}{\Vbar.} #1.
}

\newcommand{\respuesta}[1]{
    \textcolor{red}{\Rbar.} #1.
}

\newcommand{\textopequenorojo}[1]{
    {\textcolor{red}{\small{#1}}}
}

\newcommand{\versiculorespuesta}[2]{
    \versiculo{#1}\\
    \respuesta{#2}
}

\newcommand{\versiculorespuestaseguido}[2]{
    \versiculo{#1}\respuesta{#2}
}

\newcommand{\redcross}{
    \textcolor{red}{\grecross}
}

\newcommand{\lineahorizontal}[2]{
    \begin{center}
        {\rule{#1cm}{#2pt}}
    \end{center}
}

\newcommand{\lineahorizontalroja}[2]{
    \begin{center}
        \textcolor{red}{\rule{#1cm}{#2pt}}
    \end{center}
}

\newcommand{\latinderecha}[1]{
    \begin{otherlanguage}{latin}
        \begin{rightcolumn}
            \input{#1}    
        \end{rightcolumn}
    \end{otherlanguage}
}

\newcommand{\castellanoizquierda}[1]{
    \begin{leftcolumn}
        \input{#1}
    \end{leftcolumn}
}

\newcommand{\castellanoizquierdasincro}[1]{
    \begin{leftcolumn*}
        \input{#1}
    \end{leftcolumn*}
}

\newcommand{\castellanoizquierdasincronota}[2]{
    \begin{leftcolumn*}[#1]
        \input{#2}
    \end{leftcolumn*}
}

\newcommand{\filacastellanolatin}[2]{
    {    
        \begin{leftcolumn}
            \input{#1}
        \end{leftcolumn}
        \begin{otherlanguage}{latin}
            \begin{rightcolumn}
                \input{#2}    
            \end{rightcolumn}
        \end{otherlanguage}
    }
}

\newcommand{\filacastellanolatinsincro}[2]{
    {    
        \begin{leftcolumn*}
            \input{#1}
        \end{leftcolumn*}
        \begin{otherlanguage}{latin}
            \begin{rightcolumn}
                \input{#2}    
            \end{rightcolumn}
        \end{otherlanguage}
    }
}

\newcommand{\filacastellanolatinsincronota}[3]{
    \castellanoizquierdasincronota{#1}{#2}
    \latinderecha{#3}
}

\newcommand{\titulomisterios}[2]{
    {    
        \begin{minipage}[t]{0.6\textwidth}
            \subsection*{#1}
        \end{minipage}\begin{minipage}[t]{0.395\textwidth}
            \begin{flushright}
                \textcolor{red}{#2}
            \end{flushright}
        \end{minipage}
    }
}

\newcommand{\titulomisterio}[2]{
    {
        \begin{minipage}[t]{0.6\textwidth}
            \section*{#1}
        \end{minipage}\begin{minipage}[t]{0.395\textwidth}
            \begin{flushright}
                \textcolor{red}{#2}
            \end{flushright}
        \end{minipage}
    }
}

\newcommand{\iralfinal}{
    \begin{center}
        \textcolor{red}{Una vez terminamos nos vamos a la \cpageref{final-prayer} para las oraciones finales.}
    \end{center}
}




\begin{document}

\begin{center}
      \textsc{\Large{Oraciones al Levantarse}}
\end{center}

\begin{multicols}{2}
      Por la señal de la Santa Cruz{\redcross}, de nuestros enemigos{\redcross}líbranos, Señor Dios nuestro{\redcross}. En el Nombre del Padre, y del{\redcross}Hijo, y del Espíritu Santo. Amén.

      \vspace{1mm}

      \begin{otherlanguage}{latin}
            \versiculorespuestaseguido{Deum, in adjutórium meum inténde}{Dómine, ad adjuvándum me festina}
            
            \textcolor{red}{P}ater noster, qui es in c{\ae}lis, sanctificétur nomen tuum. Advéniat regnum tuum.
Fiat volúntas tua, sicut in c{\oe}lo et in terra. Panem nóstrum quotidiánum da nobis hódie.
Et dimitte nobis debita nostra, sicut et nos dimittimus debitóribus nostris.
Et ne nos indúcas in tentatiónem: sed libera nos a malo. Amen.

            \textcolor{red}{A}ve María, grátia plena, Dóminus tecum; benedicta tu in muliéribus, et benedíctus fructus ventris tui,
Iesus. Sancta Maria, Mater Dei, ora pro nobis peccatóribus, nunc et in hora mortis nostr{\ae}. Amen.

            \textcolor{red}{C}redo in Deum, Patrem omnipoténtem. Creatórem c{\oe}li et terr{\ae}. Et in Jesum Christum, Filium ejus únicum, Dóminum nostrum; 
qui concéptus est de Spíritu Sancto; natus ex María Virgine; passus sub Póntio Pilato, crucifíxus, mortuus et sepúltus:
descéndit ad inferos; tértia die resurréxit a mórtuis: ascéndit ad c{\oe}los, sedet ad dexteram Dei Patris omnipoténtis;
inde ventúrus est judicáre vivos et mórtuos. Credo in Spíritum Sanctum, sanctam Ecclésiam cathólicam, Sanctórum communiónem,
remissiónem peccatórum, carnis resurrectiónem, vitam {\ae}térnam. Amen.

            \versiculorespuestaseguido{Deum, in adjutórium meum inténde}{Dómine, ad adjuvándum me festina}
      \end{otherlanguage}

      \vspace{1mm}

      \textcolor{red}{A la Santísima Trinidad}

      \textcolor{red}{Y}o os adoro, único Dios en tres personas, y ante vuestra Majestad me abismo en mi nada. Vos sois la única fuente de todo ser, sólo Vos sois la vida, la verdad, la hermosura
      y el bien. Con ser pobre y tan indigno, con todo, os glorifíco y alabo; os diy gracias y os amo en unión de vuestro Hijo amadísimo Jesucristo, Salvador y amante Padre nuestro en el amor de sul
      sacratísimo Corazón y por sus méritos infinitos. Yo quiero agradaros y serviros, obedeceros y amaros siempre, en unión de la Inmaculada Virgen María, Madre de Dios y Madre nuestra.
      Por vuestro amor quiero también amar y servir a mis prójimos: dadme Señor, vuestro santo Espíritu, para que me ilumine, me corrija y me guíe por el camino de vuestros mandamientos, a una
      perfección siempre creciente hasta llegar a la celestial felicidad, en la que os adoremos por toda la eternidad.
      
      \vspace{1mm}

      \textcolor{red}{Para pedir la pureza de intención}

      \textcolor{red}{S}eñor mío Jesucristo, en unión con aquella divina intención con la que Vos mismo rendisteis a Dios en la tierra alabanza y honor por medio de vuestro santísimo Corazón, y se lo rendís
      actualmente sin interrupción hasta el fin de los siglos por todo el universo en el Santísimo Sacramento del altar, os ofrezco durante todo el día de hoy sin esceptuar la menor parte de él,
      a imitación del santísimo COrazón de la beatísima y siempre pura Virgen María, desde el fondo del alma, todas mis intenciones y pensamientos, todas mis aspiraciones y deseos y todas mis
      palabras y obras.

      \vspace{1mm}

      \textcolor{red}{Consagración a María Santísima}

      \textcolor{red}{{!`}O}h Señora mía! {!`}Oh Madre mía! Yo me entrego del todo a vos. Y, en prueba de mi filial afecto, os entrego en este día mis ojos, mis oídos, mi lengua, mi Corazón,
      en una palabra, todo mi ser. Ya que soy todo vuestro, {!`}oh Madre de piedad!, guardadme y defenderme como cosa y posesión vuestra \textcolor{red}{(}Tres Avemarías\textcolor{red}{)}.

      \vspace{1mm}

      \textcolor{red}{Al Ángel de la Guarda}

      \textcolor{red}{A}ngele Dei, qui custos es mei, me tibi commissum pietate superna hac die et semper illumina, custodi, rege et guberna. Amén

      \vspace{1mm}

      \textcolor{red}{A San José}

      \textcolor{red}{O}h Dios, que con inefable providencia, te has dignado escoger a San José  para esposo de tu santa Madre: te suplicamos que hagas que, así como nosotros le veneramos
      como protector en la tierra, así él sea nuestro intercesor en los cielos. Señor que vives y reinas por los siglos de los siglos. Amén.

      \vspace{1mm}

      \textcolor{red}{Ángelus}

      \versiculorespuestaseguido{Angelus Dómini nuntiávit Marí{\ae}}{Et concépit de Spíritu Sancto}

\vspace{1mm}

\textit{Ave Maria}\ldots

\vspace{1mm}

\versiculorespuestaseguido{Ecce Amcilla Dómini}{Fiat mihi secúndum verbum tuum}

\vspace{1mm}

\textit{Ave Maria}\ldots

\vspace{1mm}

\versiculorespuestaseguido{Et Verbum caro factum est}{Et habitávit in nobis}

\vspace{1mm}

\textit{Ave Maria}\ldots

\vspace{1mm}

\versiculorespuestaseguido{Ora pro nobis, Sancta Dei Génetrix}{Ut digni efficiámur promissiónibus Christi. Amen}

\vspace{1mm}

\textbf{Orémus}.-- \textcolor{red}{G}rátiam tuam qu{\'\ae}sumus. Dómine, méntibus nostris infúnde ut qui Angelo nuntiáte.
Christi Filii tui incarnatiónem cognóvimus, per passiónem ejus et crucem ad resurrectiónis glóriam perducámur.
Per eúmdem Christum Dóminum nostrum.\\[1mm]
\respuesta{Amen} 

      \vspace{2mm}

      \lineahorizontalroja{3}{0.5}
      \textcolor{red}{En Tiempo Pascual decimos el Regina C{\ae}li}

      \begin{otherlanguage*}{latin}
            \versiculorespuestaseguido{Regina caeli l{\ae}táre, allelúja}{Quia quem meruisti portáre, allelúja}

\vspace{1mm}

\versiculorespuestaseguido{Resurréxit sicut dixit, allelúja}{Ora pro nobis Deum, allelúja}

\vspace{1mm}

\versiculorespuestaseguido{Gaude et l{\ae}táre, Virgo María, allelúja}{Quia surréxit Dóminus vere, allelúja}

\vspace{1mm}

\textbf{Orémus}.-- \textcolor{red}{D}eus, qui per resurrectiónem Filii tui Dómini nostri Jesu Christi,
mundum l{\ae}tificáre dignátus es: pr{\ae}sta, qu{\'\ae}sumus, ut per ejus Genetricem Vírginem Maríam,
perpétu{\ae} capiámus gáudia vit{\ae}. Per eúmdem Christum Dóminum nostrum. \respuesta{Amen} 

            \lineahorizontalroja{3}{0.5}
            
            \vspace{1mm}

            \versiculorespuestaseguido{Regina caeli l{\ae}táre, allelúja}{Quia quem meruisti portáre, allelúja}

\vspace{1mm}

\versiculorespuestaseguido{Resurréxit sicut dixit, allelúja}{Ora pro nobis Deum, allelúja}

\vspace{1mm}

\versiculorespuestaseguido{Gaude et l{\ae}táre, Virgo María, allelúja}{Quia surréxit Dóminus vere, allelúja}

\vspace{1mm}

\textbf{Orémus}.-- \textcolor{red}{D}eus, qui per resurrectiónem Filii tui Dómini nostri Jesu Christi,
mundum l{\ae}tificáre dignátus es: pr{\ae}sta, qu{\'\ae}sumus, ut per ejus Genetricem Vírginem Maríam,
perpétu{\ae} capiámus gáudia vit{\ae}. Per eúmdem Christum Dóminum nostrum. \respuesta{Amen} 

      \end{otherlanguage*}

      \vspace{1mm}

      \textcolor{red}{S}eñor, Dios omnipotente, que nos has habéis hecho llegar al principio de este día; salvadnos hoy con vuestra virtud; para que en este día no caigamos en ningún pecado;
      sino que nuestra palabra, pensamientos y obras se dirijan siempre al cumplimiento de vuestra santa ley. Por jesucristo nuestro Sñor, vuestro Hijo, que con Vos vive y reina en unidad
      del Espíritu Santo por todos los siglos de los siglos. Amén.\\[2mm]
      En el Nombre del Padre, y del{\redcross}Hijo, y del Espíritu Santo. Amén.
\end{multicols}

\newpage

\begin{center}
      \textsc{\Large{Oraciones al Acostarse}}
\end{center}

\begin{multicols}{2}
      \noindent\textcolor{red}{Lección breve}\hfill\textcolor{red}{1 Petr. 5, 8-9}
      
      \primeraletragranderoja{H}{ermanos:}Sed sobrios y velad; porque vuestro enemigo el diablo anda alrededor como león rugiente, buscando a quien devorar; resistidle firmes en la fe.
      Y Vos, Señor, tened misericordia de nosotros.

      \vspace{1mm}

      \versiculorespuestaseguido{Demos gracias a Dios}{Nuestro auxilio está en el nombre del Señor}

      \begin{otherlanguage}{latin}
            \textcolor{red}{P}ater noster, qui es in c{\ae}lis, sanctificétur nomen tuum. Advéniat regnum tuum.
Fiat volúntas tua, sicut in c{\oe}lo et in terra. Panem nóstrum quotidiánum da nobis hódie.
Et dimitte nobis debita nostra, sicut et nos dimittimus debitóribus nostris.
Et ne nos indúcas in tentatiónem: sed libera nos a malo. Amen.
      \end{otherlanguage}\vspace{-3mm}

      \textcolor{red}{Hacemos un breve examen de conciencia y damos gracias a Dios por el día}

      \vspace{1mm}

      \begin{otherlanguage}{latin}
            \input{oraciones/contricion/confiteor/latin_simple.tex}
      \end{otherlanguage}

      \vspace{1mm}

      \hfill\textcolor{red}{Salmo 4}
      \primeraletragranderoja{C}{uando}le invoqué escuchóme mi Dios, que es justo: en la tribulación me consolasteis.

      Tened piedad de mi y oid mi plegaria.

      Hijos de los hombres {?`}hasta cuándo seréis de torpe corazón?{?`}por qué amáis la vanidad, y váis en busca de la mentira?

      Sabed que el Señor ha distinguido a su santo; el Señor me oye cuando lo invoco.

      Airaos, mas no pequéis: de lo que decís en vuestros corazones, arrepentíos en vuestros hechos.
      
      Inmolad sacrificios de justicia, esperad en el Señor. Muchos dicen: <<{?`}Quién nos hará ver los bienes?>>

      Señor, refléjese en nosotros la luz de vuestra faz; Vos dais la alegría a mi corazón.

      Por la cosecha del trigo, vino y aceite se engrandecieron.

      En paz me duermo y descanso.

      Porque Vos solo, Señor, me habéis robustecido en mi esperanza.

      \vspace{1mm}

      \begin{otherlanguage}{latin}
            \versiculorespuestaseguido{Deum, in adjutórium meum inténde}{Dómine, ad adjuvándum me festina}
      \end{otherlanguage}

      \vspace{1mm}

      \hfill\textcolor{red}{Salmo 90}
      \primeraletragranderoja{E}{l}que descansa bajo la guarda del Altísimo, está bajo la tutela del Dios del cielo.

      Dirá al Señor: <<Vos sois mi defensa y mi refugio; mi Dios, en quien espero>>.

      Él es quien me libra del lazo del cazador y de los palabras malignas.

      Te cubre con sus alas, y bajo sus plumas estarás lleno de esperanzas.

      Te rodea su fidelidad como un escudo; no te amedrentarán los terrores de la noche.

      Ni la flecha que vuele durante el día, ni el mal que avance entre nieblas, ni los asaltos de demonio en plano día.

      Mil caerán a tu lado, y a tu diestra diez mil, pero a ti no se llegará la muerte.

      Antes contemplarás con tus ojos y verás el castigo de los impíos.

      Porque Vos sois, Señor, mi esperanza. Has elegido al Altísimo para tu refugio.

      A ti no se te llegará ningún mal, ni se acercará el azote a tu morada.

      Pues respecto a ti mandó a sus Ángeles que te guarden en todos tus caminos.

      Te llevarán en sus manos, no sea que tu pie tropiece en piedra.

      Andarás sobre el áspid y el basilisco y pisotearás al león y al dragón.

      Ya que espera en mi, yo le libraré; le protegeré , porque conoce mi nombre.

      Me invoca, yo lo atiendo, en la tribulación estoy con él; le salvaré y le llenaré de gloria.

      Le colmaré de dilatados días; y le haré ver mi salvación.

      \vspace{1mm}

      \begin{otherlanguage}{latin}
            \versiculorespuestaseguido{Deum, in adjutórium meum inténde}{Dómine, ad adjuvándum me festina}
      \end{otherlanguage}

      \vspace{1mm}

      \noindent\textcolor{red}{Himno}
      \primeraletragranderoja{A}{}ti, {!`}oh Creador del mundo!,\\
      antes que el día se acabe,\\
      pedimos por clemencia,\\
      nos protejas y nos guardes.\\
      Malos sueños, alejaos;\\
      atrás, noctrunos fantasmas.\\
      Reprime, oh Dios, al maligno;\\
      puros nuestros cuerpos guarda.\\
      Humildes te lo pedimos,\\
      oh clementísimo Padre\\
      que reinas eternamente\\
      con tu Hijo y el Paráclito. Amén.

      \vspace{1mm}

      \versiculorespuestaseguido{Guardadnos, Señor, como la pupila del ojo}{Protegednos bajo la sombra de vuestras alas}

      \textbf{Oremos:}.-- Os rogamos, Señor, que visites esta habitación y ahuyentéis lejos de ella todas las asechanzas del enemigo; moren en ella vuestros santos Ángeles
      que nos guarden en paz; y vuestra bendición sea siempre sobre nosotros. Por nuestro Señor Jesucristo, tu Hijo, que contigo vive y reina en unidad
      con el Espíritu Santo, Dios, por los siglos de los siglos. Amén.

      \vspace{1mm}

      \textcolor{red}{Consagración a María Santísima}

      \textcolor{red}{{!`}O}h Señora mía! {!`}Oh Madre mía! Yo me entrego del todo a vos. Y, en prueba de mi filial afecto, os entrego en esta noche mis ojos, mis oídos, mi lengua, mi Corazón,
      en una palabra, todo mi ser. Ya que soy todo vuestro, {!`}oh Madre de piedad!, guardadme y defenderme como cosa y posesión vuestra \textcolor{red}{(}Tres Avemarías\textcolor{red}{)}.
      
      \vspace{1mm}

      \textcolor{red}{A nuestro Ángel de la Guarda}

      \textcolor{red}{A}ngele Dei, qui custos es mei, me tibi commissum pietate superna hac nocte et semper illumina, custodi, rege et guberna. Amen

      \vspace{1mm}

      \textcolor{red}{Al Corazón de Jesús por loa moribundos}

      {!`}Oh clementísimo Jesús, amadorde las almas! os ruego, por la agonía de vuestro Corazón santísimo y por los dolores de vuestra inmaculada Madre,
      que lavéis con vuestra sangre a todos los pecadores que están ahora en la agonía y que hoy van a morir. Amén.\versiculorespuestaseguido{Corazón agonizante de Jesús}
      {tened compasión de los moribundos}

      \vspace{1mm}

      Jesús, José y María, os doy el corazón y alma mía; Jesús, José y María, asistidme en mi última agonía; Jesús, José y María, con vos descanse en paz el alma mía.

      \vspace{1mm}

      \versiculorespuestaseguido{Sancte Joseph}{Ora pro nobis}

      \vspace{1mm}

      \versiculorespuestaseguido{Ave María puríssima}{Sine labe originali concépta}

      \vspace{1mm}

      En el Nombre del Padre, y del{\redcross}Hijo, y del Espíritu Santo. Amén.
\end{multicols}
\end{document}


