\documentclass[a4paper,11pt,sans]{article}

\usepackage[utf8]{inputenc}
\usepackage[spanish]{babel}
\usepackage{multicol}

\begin{document}
  \section*{\hfil Misterios Gozosos \hfil}
    \subsection*{\hfil La Anunciación de la Santísima Virgen María \hfil}
      
      %\begin{multicols}{2}
      %  \textsuperscript{26}En el sexto mes fué enviado el Ángel Gabriel de parte de Dios a una ciudad de Galilea, llamada Nazaret, \textsuperscript{27}a una doncella desposada con un varón llamada José, de la familia de 
      %  David, y el nombre de la doncella era María.
      %\columnbreak
          
      %\end{multicols}

      % PRIMER
      \textsuperscript{26}En el sexto mes fué enviado el Ángel Gabriel de parte de Dios a una ciudad de Galilea, llamada Nazaret, \textsuperscript{27}a una 
      doncella desposada con un varón llamada José, de la familia de David, y el nombre de la doncella era María.
      
      \begin{center}
        Lc. 1,26- 27           
      \end{center}
      
      %SEGUNDO
      \textsuperscript{28}Y habiendo entrado a ella, dijo: "Dios te salve, llena de gracia, el Señor es contigo, bendita tú entre las mujeres 
      \textsuperscript{42}y levanto la voz con gran clamor y dijo: "Bendita tú entre las mujeres y bendito el fruto de tu vientre"
      
      \begin{center}
        Lc. 1, 28, 42      
      \end{center}
      
      %TERCERO
      \textsuperscript{29}Ella, al oír estas palabras, se turbó, y discurría qué podía ser esta salutación

      \begin{center}
        Lc. 1, 29         
      \end{center}
      
      %CUARTO
      \textsuperscript{30}Y le dijo el ángel: "No temas, María, pues hallaste gracia a los ojos de Dios".

      \begin{center}
        Lc. 1, 30         
      \end{center}
      
      %QUINTO
      \textsuperscript{31}``He aquí que concebirás en tu seno y darás a luz un Hijo, a quién darás por nombre Jesús".

      \begin{center}
        Lc. 1, 31      
      \end{center}

      %SEXTO
      \textsuperscript{32}``Este será grande, y será llamado Hijo del Altísimo, y le dará el Señor Dios el trono de David su padre, \textsuperscript{33}y reinará
      sobre la casa de Jacob etérnamente, y su reinado no tendrá fin"

      \begin{center}
        Lc. 1, 32-33        
      \end{center}
      
      %SEPTIMO
      \textsuperscript{34}DIjo María al ángel: "¿cómo será eso, pues, no conozco varón?".

      \begin{center}
        Lc. 1, 34         
      \end{center}
      
      %OCTAVO
      \textsuperscript{35}Y respondiendo el ángel, le dijo: ``el Espíritu Santo descenderá sobre ti, y el poder del Altísimo te cobijará con su sombra";

      \begin{center}
        Lc. 1, 35       
      \end{center}
      
      %NOVENO
      ``por lo cual también lo que nacerá será llamado santo, Hijo de Dios".

      \begin{center}
        Lc. 1, 35      
      \end{center}      
      
      %DECIMO
      \textsuperscript{38}María dijo: ``he aquí la esclava del Señor; hágase en mí según tu palabra".

      \begin{center}
        Lc. 1, 38      
      \end{center}
            
    \subsection*{\hfil La Visitación de Nuestra Señora \hfil}
      
      % PRIMER
      \textsuperscript{39}Por aquellos días, levantándose María, se dirigió presurosa a la montaña, a un ciudad de Judá, \textsuperscript{40}y entró en la casa
      de Zacarías y saludó a Isabel
      \begin{center}
        Lc. 1, 39-40        
      \end{center}
      
      %SEGUNDO
      \textsuperscript{41}Y aconteció que, al oír Isabel la salutación de María. dió saltos de gozo el niño en su seno, y fué llena Isabel del Espíritu Santo.
      \begin{center}
        Lc. 1, 41        
      \end{center}
      
      %TERCERO
      \textsuperscript{42}y levantó la voz con gran clamor y dijo: "Bendita tu entre las mujeres y bendito esl fruto de tu vientre".

      \begin{center}
        Lc. 1, 42         
      \end{center}
      
      %CUARTO
      \textsuperscript{45}Y dichosa la que creyó que tendrá cumplimiento las cosas que le han sido dichas de parte del Señor
      \begin{center}
        Lc. 1, 45         
      \end{center}
      
      %QUINTO
      \textsuperscript{46}Y dijo María:
      \begin{center}
        Engrandece mi alma al Señor, \\
        \textsuperscript{47}y se regocijó mi espíritu en Dios, mi Salvador; \\
        \textsuperscript{48}porque puso sus ojos en la bajeza de su esclava.
      \end{center}

      \begin{center}
        Lc. 1, 46-48        
      \end{center}

      %SEXTO
      \begin{center}
        Pues he aquí que desde ahora \\
        me llamarán dichosa todas las generaciones;\\
        \textsuperscript{49}porque hizo en mi favor grandes cosas el Poderoso,
        y cuyo nombre es "Santo";
      \end{center}

      \begin{center}
        Lc. 1, 48-49          
      \end{center}
      
      %SEPTIMO
      \begin{center}
        y cuyo nombre es "Santo"; \\
        \textsuperscript{50}y su misericordia por generaciones y generaciones \\
        para con aquellos que le temen
      \end{center}

      \begin{center}
        Lc. 1, 49-50          
      \end{center}
      
      %OCTAVO
      \begin{center}
        \textsuperscript{51}Hizo ostentación de poder con su brazo: \\
        desbarató a los soberbios en los proyectos de su corazón
      \end{center}

      \begin{center}
        Lc. 1, 51        
      \end{center}
      
      %NOVENO
      \begin{center}
        \textsuperscript{52}derrocó de su trono a los potentados \\
        y enalteció a los humildes.
      \end{center}

      \begin{center}
        Lc. 1, 52       
      \end{center}      
      
      %DECIMO
      \begin{center}
        \textsuperscript{53}llenó de bienes a los hambrientos \\
        y despidió vacíos a los ricos.
      \end{center}

      \begin{center}
        Lc. 1, 53        
      \end{center}
            
    \subsection*{\hfil La Natividad de Nuestro Señor Jesucristo \hfil}
      
      % PRIMER
      \begin{center}
        \textsuperscript{6}Y sucedió que estando ellos allí se le complieron a ella los díaas del parto
      \end{center}

      \begin{center}
        Lc. 2,6        
      \end{center}
      
      %SEGUNDO
      \begin{center}
        \textsuperscript{7}Y dió a luz a su hijo primogénito, y le envolvió en pañales...
      \end{center}

      \begin{center}
        Lc. 2, 7          
      \end{center}
      
      %TERCERO
      \begin{center}
        \ldots y le recostó en un pesebre, pues no había para ellos lugar en el mesón
      \end{center}
      
      \begin{center}
        Lc. 2, 7         
      \end{center}
      
      %CUARTO
      \begin{center}
        \textsuperscript{8}Y había unos pastores en aquella misma comarca, que pernoctaban al raso y velaban por turno para guardar su ganado, \textsuperscript{9}y un ángel
        del Señor se presentó ante ellos, y la gloria del Señor los envolvió en sus fulgores, y se atemorizaron con gran temor.
      \end{center}
      \begin{center}
        Lc. 2, 8-9         
      \end{center}
      
      %QUINTO
      \begin{center}
        \textsuperscript{10}Y les dijo el ángel: ``no temáis, pues he aquí que os traigo una buena nueva, que será de grande alegría para todo el pueblo:"
      \end{center}
      \begin{center}
        Lc. 2, 10         
      \end{center}

      %SEXTO
      \begin{center}
        \textsuperscript{11}``que os ha nacido hoy en la ciudad de David un Salvador, que es el Mesías, el Señor"
      \end{center}
      \begin{center}
        Lc. 2, 11       
      \end{center}
      
      %SEPTIMO
      \begin{center}
        \textsuperscript{14}Gloria a Dios en las alturas \\
        y en la tierra a los hombres del [divino] agrado.
      \end{center}

      \begin{center}
        Lc. 2, 14        
      \end{center}
      
      %OCTAVO
      \begin{center}
        \textsuperscript{1}Nacido Jesús en Belén de la Judea en los días de Herodes el rey, he aquí que unos magos venidos de las regiones orientales llegaron a Jerusalén.
        \textsuperscript{11}Y entrando en la casa, vieron al niño con María, su madre;
      \end{center}

      \begin{center}
        Mt. 2, 1, 11        
      \end{center}
      
      %NOVENO
      \begin{center}
        y postrándose en tierra le adoraron, y abrieron sus tesoros le ofrecieron presentes, oro, incienso y mirra.
      \end{center}

      \begin{center}
        Mt. 2, 11         
      \end{center}      
      
      %DECIMO
      \begin{center}
        \textsuperscript{19}Pero María guardaba todas estas palabras confiriéndolas en su corazón
      \end{center}
      \begin{center}
        Lc. 2, 19        
      \end{center}
            
    \subsection*{\hfil La Presentación del Niño Jesús en el Templo \hfil}
      
      % PRIMER
      \begin{center}
        \textsuperscript{22}Y cuando se les cumplieron los días de la purificación según la ley de Moisés (Lev. 12, 6), le subieron a Jerusalén para presentarle al Señor
      \end{center}
      \begin{center}
        Lc. 2, 22         
      \end{center}
      
      %SEGUNDO
      \begin{center}
        \textsuperscript{25}y he aquí había un hombre en Jerusalén por nombre Simeón. Y era este hombre justo y temeroso de Dios que aguardaba la consolación de Israel, 
        y el Espíritu Santo estaba sobre él;
      \end{center}
      \begin{center}
        Lc. 2, 25         
      \end{center}
      
      %TERCERO
      \begin{center}
        \textsuperscript{26}y le había sido revelado por el Espíritu Santo qye no vería la muerte antes de ver al Ungido del Señor.  
      \end{center}
      
      \begin{center}
        Lc. 2, 26       
      \end{center}
      
      %CUARTO
      \begin{center}
        \textsuperscript{27}Y vino al templo impulsado por el Espíritu Santo. Y cuando sus padres intrudujeron al niño Jesús para cumplir las prescipciones usuales
        de la ley tocantes a El, \textsuperscript{28}Simeón le recibió en sus brazos y bendijo a Dios diciendo:
      \end{center}
      \begin{center}
        Lc. 2, 27-28         
      \end{center}
      
      %QUINTO
      \begin{center}
        \textsuperscript{29}Ahora dejas ir a tu siervo, Señor, según tu palabra, en paz;
      \end{center}
      \begin{center}
        Lc. 2, 29        
      \end{center}

      %SEXTO
      \begin{center}
        \textsuperscript{30}pues ya vieron mis ojos tu salud, \textsuperscript{31}que preparaste a la faz de todos los pueblos;
      \end{center}
      \begin{center}
        Lc. 2, 30-31         
      \end{center}
      
      %SEPTIMO
      \begin{center}
        \textsuperscript{31}luz para iluminación de los gentiles, y gloria de tu pueblo Israel.
      \end{center}
      \begin{center}
        Lc. 2, 32        
      \end{center}
      
      %OCTAVO
      \begin{center}
        \textsuperscript{34}Y les bendijo Simeón, y dijo a María, su madre: ``He aquí que éste está puesto para caída y resurgimiento de muchos en Israel, y como
         señal a quien se contradice''
      \end{center}
      \begin{center}
        Lc. 2, 34       
      \end{center}
      
      %NOVENO
      \begin{center}
        \textsuperscript{34}``y a ti misma una espada te traspasará el alma, para que salgan a la luz de muchos corazones los pensamientos''
      \end{center}
      \begin{center}
        Lc. 2, 35         
      \end{center}      
      
      %DECIMO
      \begin{center}
        \textsuperscript{39}Y así que cumplieron todas las cosas ordenadas en la ley del Señor, se volvieron a Galilea, a su ciudad de Nazaret. \textsuperscript{40}EL niño crecía
        y se robustecía llenándose de sabiduría, y la gracia de Dios estaba en Él.
      \end{center}
      \begin{center}
        Lc. 2, 39-40         
      \end{center}
            
    \subsection*{\hfil La Pérdida y Hallazgo del Niño Jesús en el Templo \hfil}
      
      % PRIMER
      \begin{center}
        \textsuperscript{42}Y cuando fué de doce años, habiendo ellos subido, según la costumbre de la fiesta,
      \end{center}
      \begin{center}
        Lc. 2, 42     
      \end{center}
      
      %SEGUNDO
      \begin{center}
        \textsuperscript{43}y acabados los días, al volverse ellos, quedóse el niño Jesús en Jerusalén, sin que lo advirtiesen sus padres
      \end{center}
      \begin{center}
        Lc. 2, 43        
      \end{center}
      
      %TERCERO
      \begin{center}
        \textsuperscript{45}y no hallándole, se tornaron a Jerusalén para burcarlo. \textsuperscript{46}Y sucedió que después de tres días le hallaron en el templo,
        sentado en medio de los maestros, escuchándoles y haciéndoles preguntas;
      \end{center}
      \begin{center}
        Lc. 2, 45-46       
      \end{center}
      
      %CUARTO
      \begin{center}
        sentado en medio de los maestros, escuchándoles y haciéndoles preguntas;
      \end{center}
      \begin{center}
        Lc. 2, 46       
      \end{center}
      
      %QUINTO
      \begin{center}
        \textsuperscript{47}y se pasmaban todos los que le oían de su inteligencia y de sus respuestas.
      \end{center}
      \begin{center}
        Lc. 2, 47     
      \end{center}

      %SEXTO
      \begin{center}
        \textsuperscript{48}Y sus padres, al verle, quedaron sorprendidos; y le dijo su madre: ``hijo, ¿por qué lo hiciste así con nosotros? Mira que tu padre
        y yo, llenos de aflicción, te andábamos buscando''
      \end{center}
      \begin{center}
        Lc. 2, 48     
      \end{center}
      
      %SEPTIMO
      \begin{center}
        \textsuperscript{49}Díjoles Él: ``¿pues por qué me buscabais? ¿No sabíais que había yo de estar en casa de mi padre?''
      \end{center}
      \begin{center}
        Lc. 2, 49        
      \end{center}
      
      %OCTAVO
      \textsuperscript{50}Y ellos no comprendieron lo que les dijo.
      \begin{center}
        Lc. 2, 50        
      \end{center}
      
      %NOVENO
      \textsuperscript{51}Y bajó en su compañía y se fue a Nazaret, y vivía sometido a ellos. Y su madre guardaba todas estas
      cosas en su corazón.
      \begin{center}
        Lc. 2,51       
      \end{center}      
      
      %DECIMO
      \textsuperscript{52}Y Jesús progresaba en sabiduría, en estatura y en gracia delante de Dios y de los hombres.
      \begin{center}
        Lc. 2, 52        
      \end{center}
            
    \newpage
        
  \section*{\hfil Misterios Dolorosos \hfil}
    \subsection*{\hfil La Agonía de Nuestro Señor en el Huerto de los Olivos \hfil}
      
      % UNO
      \textsuperscript{36}Entonces llego Jesús con ellos a una granja llamada Getsemaní, y dice a los discípulos: ``Sentaos aquí mientras voy allá para hacer oración''. 
      \textsuperscript{37}Y llevando consigo a Pedro y a los dos hijos de Zebedeo, comenzó a ponerse triste y a sentir abatimiento.
      \begin{center}
        Mt. 26; 36-37     
      \end{center}

      % DOS
      \textsuperscript{38}Entonces les dice: ``triste em gran manera está mi alma hasta la muerte; quedad aquí y velad conmigo''
      \begin{center}
        Mt. 26, 38
      \end{center}

      % TRES
      \textsuperscript{35}Y adelantándose un poco, caía sobre la tierra, y rogaba que, a ser posible, pasase de Él aquella hora
      \begin{center}
        Mc. 14, 35  
      \end{center}

      % CUATRO
      \textsuperscript{42}diciendo: ``Padre, si quieres, traspasa de mi este cáliz; mas no se haga mi voluntad, sino la tuya''.
      \begin{center}
        Lc. 22, 42
      \end{center}

      % CINCO
      \textsuperscript{43}Y se le apareció un ángel venido del cielo, que le confortaba
      \begin{center}
        Lc. 22, 43
      \end{center}

      % SEIS
      \textsuperscript{44}Y venido en agonía, oraba más intensamente. 
      \begin{center}
        Lc. 22, 44
      \end{center}

      % SIETE
      Y se hizo su sudor como grumos de sangre, que caían hasta el suerlo.
      \begin{center}
        Lc. 22, 44
      \end{center}

      % OCHO
      \textsuperscript{40}Y viene a los discípulos y los halla durmiendo, y dice a Pedro: ``¿así no pudísteis velar una hora conmigo?''
      \begin{center}
        Mt. 26, 40
      \end{center}

      % NUEVE
      \textsuperscript{41}``Velad y orad, para que no entréis en tentación;''
      \begin{center}
        Mt. 26, 41
      \end{center}

      % DIEZ
      ``el espíritu sí, está animoso, más la carne es flaca''
      \begin{center}
        Mt. 26, 41
      \end{center}

    \subsection*{\hfil La flagelación de Nuestro Señor Jesucristo \hfil}
      
      % UNO
      \textsuperscript{1}Y luego al amanecer, después de celebrar consejo, los sumos sacerdotes con los ancianos y los escribas, es decir, todo el sanhedrín, atando a Jesús,
      le llevaron de allí y le entregaron a Pilato. \textsuperscript{2}Y le interrogó Pilato: ``¿Tú eres el Rey de los judíos?''. El le respondió: ``Tú lo dices''.
      \begin{center}
        Mc. 15, 1-2
      \end{center}

      % DOS
      \textsuperscript{36}Respondió: ``Mi reino no es de este mundo. Si de este mundo fuera mi reino, mis ministros lucharían para que yo no fuera entregado a los judíos
      Más ahora mi reino no es de aquí''.
      \begin{center}
        Jn. 18, 36
      \end{center}

      % TRES
      \textsuperscript{37}Díjole, pues, Pilato: ``¿luego, rey eres tu?''. Respondión Jesús : ``Tú dices que yo soy rey. Yo para eso he nacido y para esto he venido
      al mundo: para dar testimonio a favor de la verdad. Todo el que es de la verdad, oye mi voz''.
      \begin{center}
        Jn. 18, 37
      \end{center}

      % CUATRO
      \textsuperscript{4}Pilato dijo a los sumos sacerdotes y a los turbas: ``ningún delito hallo en este hombre. \textsuperscript{16}Le castigaré, pues, y le soltaré''.
      \begin{center}
        Lc. 23, 4, 16
      \end{center}

      % CINCO
      \textsuperscript{1}Entonces, pues, tomó Pilato a Jesús y le azotó.
      \begin{center}
        Jn. 19, 1
      \end{center}

      % SEIS
      \textsuperscript{8}De opresión u juicio fué tomado, y a sus contemporáneos, ¿quién tendrá en cuenta?. \textsuperscript{3}Fué despreciado y abandonado de los hombres,
      varón de dolores y familiarizado de los hombres
      \begin{center}
        Is. 53, 8, 3
      \end{center}

      % SIETE
      \textsuperscript{4}Mas nuestros sufrimientos él los ha llevado, nuestros dolores él los cargó sobre sí,
      \begin{center}
        Is. 53, 4
      \end{center}

      % OCHO
      \textsuperscript{5}Fué traspasado por causa de nuestros pecados, molido por causa de nuestras iniquidades; 
      \begin{center}
        Is. 53, 5
      \end{center}

      % NUEVE
      mientras nosotros le tuvimos por azotado, por herido de Dios y abatido
      \begin{center}
        Is. 53, 4 
      \end{center}

      % DIEZ
      el castigo de nuestra paz cayó sobre Él y por sus verdugones se nos curó.
      \begin{center}
        Is. 53, 5
      \end{center}
      
    \subsection*{\hfil La coronación de espinas de Nuestro Señor Jesucristo \hfil}
      
      % UNO
      \textsuperscript{16}Los soldados se lo llevaron dentro del palacio, que es el pretorio, y convocan a toda la cohorte (Marcos).
      \begin{center}
        Mc. 15, 16
      \end{center}
      \textsuperscript{28}Y habiéndole quitado sus vestidos, le envolvieron en uns clámide de grana (Mateo).
      \begin{center}
        Mt. 27, 28
      \end{center}

      % DOS
      \textsuperscript{29}y trenzando una corna de espinas, la pusieron sobre la cabeza, y una caña en la mano derecha; 
      \begin{center}
        Mt. 27, 29
      \end{center}

      % TRES
      y doblando la rodilla delante de Él, le mofaban, diciendo: ``Salud, Rey de los judíos''
      \begin{center}
        Mt. 27, 29
      \end{center}

      % CUATRO
      \textsuperscript{30}Y escupiendo en Él, tomaron la caña y le daba golpes en la cabeza.
      \begin{center}
        Mt 27, 30
      \end{center}

      % CINCO
      \textsuperscript{4}Salió Pilato otra vez fuera, y les dice: ``Ved, os le traigo para que conozcáis que no hallo en Él delito alguno''.
      \begin{center}
        Jn. 19, 4
      \end{center}

      % SEIS
      \textsuperscript{5}Salió, pues, Jesús afuera, llevando la corona de espinas y el manto de púrpura. Y les dice: ``ved aquí el hombre''
      \begin{center}
        Jn. 19, 5
      \end{center}

      % SIETE
      \textsuperscript{5}Y les dice: ``ved aquí el hombre''. \textsuperscript{15}Gritaron, pues, ellos: ``quita, quita; crucifícale''
      \begin{center}
        Jn. 19, 5, 15
      \end{center}

      % OCHO
      \textsuperscript{14}Pilato, queriendo dar satisfacción a la turba les soltó a Barrabás. Y entregó a Jesús, después de azotarle, para que fuese crucificado.
      \begin{center}
        Mc. 15, 14
      \end{center}

      % NUEVE
      Díceles Pilato: ``¿A vuestro rey he de crucificar?''. Respondieron los pontífices: ``no tenemos rey, sino César''
      \begin{center}
        Jn. 19, 15
      \end{center}

      % DIEZ
      \textsuperscript{16}Entonces, pués, se le entregó para que fuera crucificado. Se apoderaron, pues, de Jesús.
      \begin{center}
        Jn. 19, 16
      \end{center}

    \subsection*{\hfil Jesús con la Cruz a cuestas \hfil}
      
      % UNO
      \textsuperscript{23}Si alguno quiere venir en pos de Mí, niégese a sí mismo
      \begin{center}
        Lc. 9, 23
      \end{center}

      % DOS
      y tome a cuestas su cruz cada día y sígame
      \begin{center}
        Lc. 9, 23
      \end{center}

      % TRES
      \textsuperscript{17}y, llevando a cuestas su cruz, salió hacia el lugar llamado el Cráneo, que en hebreo se dice Gólgota.
      \begin{center}
        Jn. 19, 17
      \end{center}

      % CUATRO
      \textsuperscript{26}Y como le hubieron sacado, echaron mano de un tal Simón de Cirene que venía del campo, le pusieron en hombros la cruz para que la llevase
      detrás de Jesús.
      \begin{center}
        Lc. 23, 26
      \end{center}

      % CINCO
      \textsuperscript{29}Tomad mi yugo sobre vuestros, y aprended de mi, 
      \begin{center}
        Mt. 11, 29  
      \end{center}

      % SEIS
      pues soy manso y humilde de Corazón, y hallaréis reposo para vuestras almas.
      \begin{center}
        Mt. 11, 29
      \end{center}

      % SIETE
      Porque mi yugo es suave, y mi carga, ligera.
      \begin{center}
        Mt. 11, 30
      \end{center}

      % OCHO
      \textsuperscript{27}Seguíanle gran mucheduncbre de pueblo y de mujeres, las cuales le plañían y lamentaban.
      \begin{center}
        Lc. 23, 27
      \end{center}

      % NUEVE
      \textsuperscript{28}Volviéndose Jesús a ellas, les dijo: ``Hijas de Jerusalén: no lloréis sobre mi, sino llorad má bien sobre vosotras mismas y sobre
      vuestros hijos''
      \begin{center}
        Lc. 23, 28
      \end{center}

      % DIEZ
      \textsuperscript{31}``Porque si en el leño verde esto hacen, ¿en el seco qué se hará?''.
      \begin{center}
        Lc. 23, 31
      \end{center}

    \subsection*{\hfil La Crucifixión y Muerte del Redentor \hfil}
      
      % UNO
      \textsuperscript{33}Y cuando hubieron llegado al lugar llamado ``Cráneo'', allí crucificaron a Él y a los malhechores, uno a la derecha y el otro a la izquierda.
      \begin{center}
        Lc. 23, 33
      \end{center}

      % DOS
      \textsuperscript{34}Y Jesús decía: ``Padre, perdónalos, porque no saben lo que hacen''.
      \begin{center}
        Lc. 23, 34
      \end{center}

      % TRES
      \textsuperscript{39}Uno de los malhechores que estaba colgado \textsuperscript{42}decía a Jesús: ``acuérdate de mi cuando vinieres en la gloria de tu realeza''.
      \begin{center}
        Lc. 23, 39, 42
      \end{center}

      % CUATRO
      \textsuperscript{43}Díjole: ``en verdad te digo que hoy estarás conmigo en el paraíso''
      \begin{center}
        Lc. 23, 43
      \end{center}

      % CINCO
      \textsuperscript{26}Jesús, pues, viendo a la Madre, y junto a ella al discípulo a quien amaba,
      \begin{center}
        Jn. 19, 26
      \end{center}

      % SEIS
      \textsuperscript{27}dice a tu Madre: ``mujer, he ahí a tu hijo''. \textsuperscript{27}Luego dice al discípulo: ``He aquí a tu Madre''
      \begin{center}
        Jn. 19, 26-27
      \end{center}

      % SIETE
      Y desde aquella hora la tomó el discípulo en su compañía.
      \begin{center}
        Jn. 19, 27
      \end{center}

      % OCHO
      \textsuperscript{45}habiendo faltado el sol; y se rasgó por medio el velo del santuario.
      \begin{center}
        Lc. 23, 45
      \end{center}

      % NUEVE
      \textsuperscript{46}Y clamando con voz poderosa, Jesús dijo: ``Padre, en tus manos encomiendo mis espíritu''.
      \begin{center}
        Lc. 23, 46 
      \end{center}

      % DIEZ
      \textsuperscript{30}Jesús dijo: ``Consumado está''. E inclinando la cabeza entregó el espíritu.
      \begin{center}
        Jn. 19, 30
      \end{center}
 
    \newpage
         
  \section*{\hfil Miserios Gloriosos \hfil}
    \subsection*{\hfil La Resurección del Señor \hfil}
      \begin{multicols}{2}

      \columnbreak
                           
      \end{multicols}
      \begin{center}
        Lc. 1,26- 27           
      \end{center}
    \subsection*{\hfil La Ascensión de Jesucristo a los cielos \hfil}
      \begin{multicols}{2}

      \columnbreak
                           
      \end{multicols}         
      \begin{center}
        Lc. 1,26- 27           
      \end{center}
    \subsection*{\hfil La Venida del Espíritu Santo sobre los Apóstoles \hfil}
      \begin{multicols}{2}

      \columnbreak
                           
      \end{multicols}         
      \begin{center}
        Lc. 1,26- 27           
      \end{center}
    \subsection*{\hfil La Asunción de Nuestra Señora a los cielos \hfil}
      \begin{multicols}{2}

      \columnbreak
                           
      \end{multicols}         
      \begin{center}
        Lc. 1,26- 27           
      \end{center}
    \subsection*{\hfil La Coronación de la Santísima Virgen María \hfil}
      \begin{multicols}{2}

      \columnbreak
                           
      \end{multicols}         
      \begin{center}
        Lc. 1,26- 27           
      \end{center}
\end{document}
