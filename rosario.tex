\documentclass[a4paper,11pt, oneside]{report}

\usepackage[utf8]{inputenc}
\usepackage[spanish]{babel}
\usepackage[T1]{fontenc}

\title{Devocionario}
\author{Sergio Rojo Herrero}
\date{Junio 2019}

\begin{document}
  
  \begin{titlepage}
    \maketitle    
  \end{titlepage}

  \tableofcontents{}

  \newpage

  \chapter{Oraciones}

    \section{La Señal de la Cruz}
      En el Nombre del Padre, y del Hijo, y del Espíritu Santo. Amén.

      \medskip

      \textit{In Nómine Pátris, et Filii, et Spíritus Sancti. Amen.}

    \section{Oración Dominical (Padrenuestro)}
      
      Padre Nuestro que estás en los cielos, santificado sea tu Nombre. Venga a nosotros tu Reino. Hágase tu voluntad, así en la tierra como
      en el cielo. El pan nuestro de cada día dánosle hoy. Y perdónamos nuestras deuda, así como nosotros perdonamos a nuestros deudores.
      Y no nos dejes caer en la tentación: mas líbranos del mal. Amén.
      
      \medskip

      \textit{Pater noster, qui es in c{\oe}lis, sanctifificétur nomen tuum. Advéniat regnum tuum. Fiat voluntas tua, sicut in c{\oe}lo et in terra.
      Panem nóstrum quotidiánum da nobis hódie. Et dimite nosbis debita nostra, sicut et nos dimittimus debitóribus nostris. Et ne nos indúcas
      in tentatiónem: sed libera nos a malo. Amen.}

    \section{Salutación Evangélica (Avemaría)}
      Dios de salve, María, llena eres de gracia, el Señor es contigo; dendita eres entre todas las mujeres, y bendito es el fruto de tu
      vientre, Jeeús. Santa María, Madre de Dios, ruega por nosotros pecadores, ahora u en el hora de nuestra muerte. Amén.
      
      \medskip

      \textit{Ave María, grátia plena, Dóminus tecum; benedicta tu in muliéribus , et benedictum fructus ventris tui, Jesus.
      Sancta Maria, Mater Dei, ora pro nobis peccatóribus, nunc et in hora mortis nostr{\ae}. Amen.}

    \section{Símbolo de los Apóstoles (Credo Apostólico)}

      Creo en Dios, Padre todopoderoso. Creador del cielo y de la tierra. Y en Jesucristo, su único Hijo, Nuestro Señor, que fue concebido por
      obra y gracia del Espíritu Santo; nació de Santa María Vírgen; padeció bajo el poder de Poncio Pilato, fue crucificado, muerto y sepultado;
      descendió a los infiernos; al tercer día resucitó de entre los muertos; subió a los cielos, está sentado a la derecha de Dios Padre todopoderoso;
      desde allí ha de venir a juzgar a vivos y muertos. Creo en el Espíritu Santo, la Santa Iglesia Católica, la comunión de los Santos, el perdón
      de los pecados, la resurección de la carne y la vida eterna. Amén.

      \medskip

      \textit{Credo in Deum, Patrem omnipoténtem. Creatórem c{\oe}li et terr{\ae}. Et in Jesum CHristum, Filium ejus únicum, Dóminum nostrum; qui concéptus
      est de Spíritu Sancto; natus ex María Virgine; passus sub Póntio Pilato, crucifíxus, mortuus et sepúltus: descéndit ad inferos; tértia die resurréxit
      a mórtuis: ascéndit ad c{\oe}los, sedet ad dexteram Dei Patris omnipoténtis; inde ventúrus est judicáre vivos et mórtuos. Credo in Spíritum Sanctum,
      sanctam Ecclésiam cathólicam, Sanctórum communiónem, remisiónem peccatórum, carnis resurrectiónem, vitam {\ae}térnam. Amen.}
      
  \newpage

  \chapter{Santo Rosario}

  \section{Misterios Gozosos}
    
    \subsection{La Anunciación de la Santísima Virgen María}


            
    \subsection{La Visitación de Nuestra Señora}
      

            
    \subsection{La Natividad de Nuestro Señor Jesucristo}
      


    \subsection{La Presentación del Niño Jesús en el Templo}
      

            
    \subsection{La Pérdida y Hallazgo del Niño Jesús en el Templo}
      
        
  \section{Misterios Dolorosos}
    
    \subsection{La Agonía de Nuestro Señor en el Huerto de los Olivos}
      
      

    \subsection{La flagelación de Nuestro Señor Jesucristo}
      

      
    \subsection{La coronación de espinas de Nuestro Señor Jesucristo}
      


    \subsection{Jesús con la Cruz a cuestas}
      


    \subsection{La Crucifixión y Muerte del Redentor}
      

         
  \section{Miserios Gloriosos}
    \subsection{La Resurección del Señor}
      


    \subsection{La Ascensión de Jesucristo a los cielos}
        


    \subsection{La Venida del Espíritu Santo sobre los Apóstoles}



    \subsection{La Asunción de Nuestra Señora a los cielos}



    \subsection{La Coronación de la Santísima Virgen María}

  \chapter{Novenas}

\end{document}
