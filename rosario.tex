\documentclass[a4paper,11pt, oneside]{report}

\usepackage[utf8]{inputenc}
\usepackage[spanish]{babel}
\usepackage[T1]{fontenc}
\usepackage{tocloft}

\title{Santo Rosario}
\author{Sergio Rojo Herrero}
\date{Julio 2019}

\begin{document}
  
  \begin{titlepage}
    \maketitle    
  \end{titlepage}

  \section*{La Señal de la Cruz}
    En el Nombre del Padre, y del Hijo, y del Espíritu Santo. Amén.

    \medskip

    \textit{In Nómine Pátris, et Filii, et Spíritus Sancti. Amen.}

  \section*{Oración Dominical (Padrenuestro)}
    
    Padre Nuestro que estás en los cielos, santificado sea tu Nombre. Venga a nosotros tu Reino. Hágase tu voluntad, así en la tierra como
    en el cielo. El pan nuestro de cada día dánosle hoy. Y perdónamos nuestras deuda, así como nosotros perdonamos a nuestros deudores.
    Y no nos dejes caer en la tentación: mas líbranos del mal. Amén.
    
    \medskip

    \textit{Pater noster, qui es in c{\oe}lis, sanctifificétur nomen tuum. Advéniat regnum tuum. Fiat voluntas tua, sicut in c{\oe}lo et in terra.
    Panem nóstrum quotidiánum da nobis hódie. Et dimite nosbis debita nostra, sicut et nos dimittimus debitóribus nostris. Et ne nos indúcas
    in tentatiónem: sed libera nos a malo. Amen}.

  \section*{Salutación Evangélica (Avemaría)}
    Dios de salve, María, llena eres de gracia, el Señor es contigo; dendita eres entre todas las mujeres, y bendito es el fruto de tu
    vientre, Jeeús. Santa María, Madre de Dios, ruega por nosotros pecadores, ahora u en el hora de nuestra muerte. Amén.
    
    \medskip

    \textit{Ave María, grátia plena, Dóminus tecum; benedicta tu in muliéribus, et benedictum fructus ventris tui, Jesus.
    Sancta Maria, Mater Dei, ora pro nobis peccatóribus, nunc et in hora mortis nostr{\ae}. Amen.}

  \section*{Símbolo de los Apóstoles (Credo Apostólico)}

    Creo en Dios, Padre todopoderoso. Creador del cielo y de la tierra. Y en Jesucristo, su único Hijo, Nuestro Señor, que fue concebido por
    obra y gracia del Espíritu Santo; nació de Santa María Vírgen; padeció bajo el poder de Poncio Pilato, fue crucificado, muerto y sepultado;
    descendió a los infiernos; al tercer día resucitó de entre los muertos; subió a los cielos, está sentado a la derecha de Dios Padre todopoderoso;
    desde allí ha de venir a juzgar a vivos y muertos. Creo en el Espíritu Santo, la Santa Iglesia Católica, la comunión de los Santos, el perdón
    de los pecados, la resurección de la carne y la vida eterna. Amén.

    \medskip

    \textit{Credo in Deum, Patrem omnipoténtem. Creatórem c{\oe}li et terr{\ae}. Et in Jesum CHristum, Filium ejus únicum, Dóminum nostrum; qui concéptus
    est de Spíritu Sancto; natus ex María Virgine; passus sub Póntio Pilato, crucifíxus, mortuus et sepúltus: descéndit ad inferos; tértia die resurréxit
    a mórtuis: ascéndit ad c{\oe}los, sedet ad dexteram Dei Patris omnipoténtis; inde ventúrus est judicáre vivos et mórtuos. Credo in Spíritum Sanctum,
    sanctam Ecclésiam cathólicam, Sanctórum communiónem, remisiónem peccatórum, carnis resurrectiónem, vitam {\ae}térnam. Amen.}

  \section*{Gloria Patri}

    V. Gloria al Padre, al Hijo, y al Espíritu Santo.\\
    \indent R. Como era en el principio, ahora, y siempre, y por los siglos de los siglos. Amén.

    \medskip

    \textit{V. Gloria Patri, et Filio, et Spíritui Sancto.}\\
    \indent \textit{R. Sicut erat in pcincípio et nunc, et semper et in s{\ae}cula s{\ae}culórum, Amen.}

  \section*{Confíteor}

    Yo pecador me confieso a Dios todopoderoso, a la bienaventurada siempre Virgen María, al bienaventurado San Miguel Arcángel,
    al bienaventurado San Juan Bautista, a los Santos Apóstoles Pedro y Pablo, y a todos los Santos, porque pequé gravemente de 
    pensamiento, palabra y obra: por mi culpa, por mi culpa, por mi grandísima culpa. Por tanto, ruego a la bienaventurada siempre
    Virgen María, al bienaventurado San Miguel Arcángel, al bienaventurado San Juan Bautista, a los Santos Apóstoles Pedro y Pablo,
    y a todos los Santos, que roguéis por mi a Dios Nuestro Señor.\par\smallbreak{}
    R. El Señor omnipotente tenga piedad de nosotros y, perdonados nuestros pecados, nos lleve a la vida eterna. Amén.\par\smallbreak{}
    V. El Señor omnipotente y misericordioso nos conceda la indulgencia, la absolución y el perdón de nuestros pecados. Amén.

    \medskip

    \textit{Confíteor Deo omnipoténti, beát\ae Marí\ae Virigini, beáto Michaéli Archángelo, beáto Joánni Baptíst\ae, sanctis Apóstolis
    Petro et Paulo, et omníbus Sanctis, quia peccávi nimis, cogitatióne et ópere, mea culpa, mea culpa, mea máxima culpa. Ideo precor
    beátam Maríam semper Virgínem, beátum Michaélem Archángelum, beátum Joánnem Bastístam, sanctis Apóstolos Petrum et Paolum, et omnes
    Sanctos, oráre pro me ad Dóminum Deum nostrum.}\par\smallbreak{}
    \textit{R. Misereátur nostri omnipotens Deus, et dimissis pecátis nostris, perdúcat nos ad vitam {\ae}térnam. Amen}\par\smallbreak{}
    \textit{V. Indulgéntiam, absolutiónem, et remissiónem pecatórum nostrórum, tribuat nobis omnipotens et miséricors Dóminus. Amen.}
    
  \chapter*{Versión «corta»}

    \section*{Misterios Gozosos (Lunes y Jueves)}
      
      \subsection*{La Anunciación de la Santísima Virgen María}
        Y habiendo entrado a ella, dijo: «Dios te salve, llena de gracia, el Señor es contigo, bendita tu entre las mujeres». Ella, al oír estas palabras, se turbó,
        y discurría que podría ser esta salucatión. Y le dijo el ángel: «No temas, María, pues hallaste gracia a los ojos de Dios. He aquí que concebirás en tu seno
        y darás a luz un Hijo, a quien daras por nombre Jesús. Este será grande, y será llamado Hijo del Altísimo, y le dará el Señor Dios el trono de DAvid su padre,
        y reinará sobre la casa de Jacob etérnamete, y su reinado no tendrá fin». Lucas 1, 28-33.
        
        \medskip
        1 Paternoster, 10 Avemarías y 1 Gloria
        
        \medskip
        ¡Oh Jesús mío! Perdonadnos. Libradnos del fuego enterno del infierno. Llevad al Cielo a todas las almas, y socorred especialmente a las más 
        necesitadas

      \subsection*{La Visitación de Nuestra Señora}
        Y aconteció que, al oir Isabel la salutación de María, dió saltos de gozo el niño en su seno, y fue llena Isabel del Espíritu Santo, y levantó la voz con gran
        clamor y dijo: «Bendita tu entre las mujeres y bendito el fruto de tu vientre. ¿Y de dónde a mí esto que venga la madre de mi Señor a mí?». 
        Lucas 1, 41-43.
        
        \medskip
        1 Paternoster, 10 Avemarías y 1 Gloria
        
        \medskip
        ¡Oh Jesús mío! Perdonadnos. Libradnos del fuego enterno del infierno. Llevad al Cielo a todas las almas, y socorred especialmente a las más 
        necesitadas
                      
      \subsection*{La Natividad de Nuestro Señor Jesucristo}
        Y dió a luz su hijo primogénito, y le envolvió en pañales y le recostó en un pesebre, pues no había para ellos lugar en el mesón.
        Y les dijo el Ángel: «No tenáis, pues he aquí que os traigo una buena nueva, que será de grande alegría para todo el pueblo: que os ha nacido hoy en la ciudad de DAvid
        un Salvador, que es el Mesías, el Señor». Lucas 2, 7.10-11.
        
        \medskip
        1 Paternoster, 10 Avemarías y 1 Gloria
        
        \medskip
        ¡Oh Jesús mío! Perdonadnos. Libradnos del fuego enterno del infierno. Llevad al Cielo a todas las almas, y socorred especialmente a las más 
        necesitadas
    
      \subsection*{La Presentación del Niño Jesús en el Templo}
        Y cuando se les cunplieron los días de la purificación según la ley de Moisés\footnote{Levítico 12, 6}, le subieron a Jerusalen para presentarle al Señor, según está escrito
        en la Ley del Señor que «todo primogénito del sexo masculino será consagrado al Señor\footnote{Éxodo 13, 2; 12, 15}», y para ofrecer como sacrificio, según lo que 
        se ordena en la Ley del Señor, «un par de tórtolas o dos palominos\footnote{Levítico 12, 8; 5, 11}». Lucas 2, 22-24.
        
        \medskip
        1 Paternoster, 10 Avemarías y 1 Gloria
        
        \medskip
        ¡Oh Jesús mío! Perdonadnos. Libradnos del fuego enterno del infierno. Llevad al Cielo a todas las almas, y socorred especialmente a las más 
        necesitadas
              
      \subsection*{La Pérdida y Hallazgo del Niño Jesús en el Templo}
        Y no hallándole, se tornaron a Jerusalén para burcarle. Y sucedió que después de tres días le hallaron en el templo, sentado en medio de los maestros,
        escuchándolos y haciéndoles preguntas; y se pasmaban todos los que le oían de su inteligencia y de sus respuestas. Y sus padres padres, al verle, quedaron
        sorprendidos; y le dijo su madre: «Hijo, ¿por qué lo niciste así con nosotros? Mira que tu padre y yo, llenos de aflicción, te andábamos buscando»
        
        \medskip
        1 Paternoster, 10 Avemarías y 1 Gloria
        
        \medskip
        ¡Oh Jesús mío! Perdonadnos. Libradnos del fuego enterno del infierno. Llevad al Cielo a todas las almas, y socorred especialmente a las más 
        necesitadas

    \section*{Misterios Dolorosos (Martes y Viernes)}
      
      \subsection*{La Agonía de Nuestro Señor en el Huerto de los Olivos}
        Y saliendo de allí, se dirigió, según costrumbre, al monte de los Olivos; y le siguieron también los discípulos.
        Y Él, arrancándose de ellos, se apartó a la distancia como de un tiro de piedra, y puesto de rodillas oraba. 
        Y venido en agonía, oraba más intensamente. Y se hizo su sudor como grumos de sangre, que caían hasta el suelo
        Estando Él hablando todavía, he aquí una turba, y el que se llamaba Judas, uno de los Doce, iba delante de ellos. Y se llegó a Jesús para besarle.
        Mas Jesús le dijo: «¡Judas! ¿Con un beso entregas al Hijo del hombre?». Lucas 22, 39. 41. 44. 47-48.

        \medskip
        1 Paternoster, 10 Avemarías y 1 Gloria
        
        \medskip
        ¡Oh Jesús mío! Perdonadnos. Libradnos del fuego enterno del infierno. Llevad al Cielo a todas las almas, y socorred especialmente a las más 
        necesitadas
      
      \subsection*{La flagelación de Nuestro Señor Jesucristo}
        Entonces, pues, tomó Pilato a Jesús y le azotó. Juan 19, 1.
        
        \medskip
        Fue despreciado y abandonado de los hombres, varón de dolores y familiarizado con el sufrimiento, y como uno ante el cual
        se oculta el rostro, le despreciamos y no le estimamos. Isaías 53, 5.
        
        \medskip
        1 Paternoster, 10 Avemarías y 1 Gloria
        
        \medskip
        ¡Oh Jesús mío! Perdonadnos. Libradnos del fuego enterno del infierno. Llevad al Cielo a todas las almas, y socorred especialmente a las más 
        necesitadas
      
      \subsection*{La coronación de espinas de Nuestro Señor Jesucristo}

        Y habiéndole quitado sus vestidos, le envolvieron en una clámide de grana, y trenzando una corona de espinas, la pusieron sobre su cabeza, y una
        caña en su mano derecha; y doblando la rodilla delante de Él, le mofaban, diciendo: «Salud, Rey de los judíos». Mateo 27, 28-31.
        
        \medskip
        1 Paternoster, 10 Avemarías y 1 Gloria
        
        \medskip
        ¡Oh Jesús mío! Perdonadnos. Libradnos del fuego enterno del infierno. Llevad al Cielo a todas las almas, y socorred especialmente a las más 
        necesitadas
    
      \subsection*{Jesús con la Cruz a cuestas}
        Gritaron, pues, ellos: «Quita, quita, crucifícale». Díceles Pilato: «¿A vuestro rey voy he de crucificar?». Respondieron los pontífices: «No tenemos rey,
        sino César». Entonces, pues, se le entregó para que fuera crucificando. Se apoderaron, pues, de Jesús, y llevándo a cuestas su cruz, salió hacia el lugar
        llamado el Cráneo, que en hebreo se dice Gólgota\ldots Juan 19, 15-17.

        \medskip
        \noindent 1 Paternoster, 10 Avemarías y 1 Gloria
        
        \medskip
        ¡Oh Jesús mío! Perdonadnos. Libradnos del fuego enterno del infierno. Llevad al Cielo a todas las almas, y socorred especialmente a las más 
        necesitadas
      
      \subsection*{La Crucifixión y Muerte del Redentor}
        {\ldots}en donde le crucificaron, y con Él otros dos, a una mano y a otra, y en medio Jesús. Juan 19, 18-19.

        \medskip
        Y era ya como la hora sexta, y se produjeron tinieblas sobre toda la tierra hasta la hora nona, habiendo faltado el sol; y se rasgó por medio 
        el velo del santuario. Y clamando con voz poderosa, Jesús dijo: «Padre, en tus manos encomiendo mi espíritu\footnote{Sal. 30, 6}». 
        Y, dicho esto, expiró. Lucas 23, 44-46.
        
        \medskip
        1 Paternoster, 10 Avemarías y 1 Gloria
        
        \medskip
        ¡Oh Jesús mío! Perdonadnos. Libradnos del fuego enterno del infierno. Llevad al Cielo a todas las almas, y socorred especialmente a las más 
        necesitadas
        
    \section*{Miserios Gloriosos (Miércoles, Sábados y Domingos)}
      \subsection*{La Resurección del Señor}
        Y muy de madrugada, el primer día de la semana vienen al monumento, salido ya el sol. Y mirando atentamente, observan que la losa había
        sido corrida a un lado; porque era enormemente grande. Y entrando en el monumento, vieron un joven sentado a la derecha, vestido de un largo
        ropaje blanco, y quedaron espantadas. El les dice: «No os espantéis. A Jesús buscáis el Nazareno, el crucificado; resucitó, no está aquí. Mirad
        el lugar donde le pusieron. Pero id, decid a sus discípulos, y a Pedro, que va delante de vosotros a Galilea; allí le veŕéis, conforme os dijo».
        Marcos 16, 2.4-7.

        \medskip
        Y eran María Magdalena, y Juana, y María la de Santiago; y las demás que iban con ellas dijeron esto mismo a los apóstoles. 
        Lucas 24, 10.

        \medskip
        1 Paternoster, 10 Avemarías y 1 Gloria
        
        \medskip
        ¡Oh Jesús mío! Perdonadnos. Libradnos del fuego enterno del infierno. Llevad al Cielo a todas las almas, y socorred especialmente a las más 
        necesitadas
      
      \subsection*{La Ascensión de Jesucristo a los cielos}
        Y los sacó afuera hasta llegar junto a Betania, y alzando sus manos los bendijo. Y aconteció que, mientras los bendecía, se desprendió de ellos,
        y era llevado en alto al cielo. Y ellos, habiéndole adorado, se tornaron a Jerusalén con grande gozo, y estaban continuamente en el templo
        adorando a Dios. Lucas 24, 50-52.

        \medskip
        1 Paternoster, 10 Avemarías y 1 Gloria
        
        \medskip
        ¡Oh Jesús mío! Perdonadnos. Libradnos del fuego enterno del infierno. Llevad al Cielo a todas las almas, y socorred especialmente a las más 
        necesitadas
        
      \subsection*{La Venida del Espíritu Santo sobre los Apóstoles}
        Y al cumplirse el día de Pentecostés, estaban todos juntos en el mismo lugar. Y se produjo de súbito desde el cielo un estruendo como de viento
        que soplaba vehementemente, y llenó toda la casa donde se hallaban sentados. Y vieron aparecer lenguas como de fuego, que, repartiéndose, se 
        posaban sobre cada uno de ellos. Hechos 2, 1-4.

        \medskip
        1 Paternoster, 10 Avemarías y 1 Gloria
        
        \medskip
        ¡Oh Jesús mío! Perdonadnos. Libradnos del fuego enterno del infierno. Llevad al Cielo a todas las almas, y socorred especialmente a las más 
        necesitadas

      \subsection*{La Asunción de Nuestra Señora a los cielos}
        Antes de los siglos, desde el principio me creó y hasta la eternidad no cesaré. En la tienda santa, ante Él he ejercido ministerio
        He arraigado en pueblo ilustre, en la porción del Señor, heredad suya. Yo soy la madre de la hermosa dilección, y del temor, y del
        conocimiento y santa esperanza. Eclesiástico 24, 14.16.24.

        \medskip
        1 Paternoster, 10 Avemarías y 1 Gloria
        
        \medskip
        ¡Oh Jesús mío! Perdonadnos. Libradnos del fuego enterno del infierno. Llevad al Cielo a todas las almas, y socorred especialmente a las más 
        necesitadas

      \subsection*{La Coronación de la Santísima Virgen María}
        Y una gran señal fué vista en el cielo: una Mujer vestida del sol, y la luna debajo de sus pies, y sobre su cabeza una corona de doce estrellas.
        Apocalipsis 12, 1.

        \medskip
        Doy con todas estas cosas los bienes eternos a mis hijos y a los por Él designados. Quién me obedece no se avergonzará y los que obran por
        no pecarán. Los que me esclarecen tendrán vida eterna. Eclesiástico 24, 25.30-31.

        \medskip
        1 Paternoster, 10 Avemarías y 1 Gloria
        
        \medskip
        ¡Oh Jesús mío! Perdonadnos. Libradnos del fuego enterno del infierno. Llevad al Cielo a todas las almas, y socorred especialmente a las más 
        necesitadas

  \chapter*{Versión «larga»}

    \section*{ Misterios Gozosos (Lunes y Jueves)}
      
    \subsection*{ La Anunciación de la Santísima Virgen María }

      \textit{Pater Noster}

      \begin{enumerate}
        %PRIMERO
        \item En el sexto mes fué enviado el Ángel Gabriel de parte de Dios a una ciudad de Galilea, llamada Nazaret, a una 
        doncella desposada con un varón llamada José, de la familia de David, y el nombre de la doncella era María. (Lc. 1,26- 27)

        \textit{Ave María}

        %SEGUNDO
        \item Y habiendo entrado a ella, dijo: »Dios te salve, llena de gracia, el Señor es contigo, bendita tú entre las mujeres 
        y levanto la voz con gran clamor y dijo: »Bendita tú entre las mujeres y bendito el fruto de tu vientre». (Lc. 1, 28, 42)

        \textit{Ave María}

        %TERCERO
        \item Ella, al oír estas palabras, se turbó, y discurría qué podía ser esta salutación. (Lc. 1, 29)

        \textit{Ave María}

        %CUARTO
        \item Y le dijo el ángel: «No temas, María, pues hallaste gracia a los ojos de Dios». (Lc. 1, 30)

        \textit{Ave María}

        %QUINTO
        \item «He aquí que concebirás en tu seno y darás a luz un Hijo, a quién darás por nombre Jesús». (Lc. 1, 31)

        \textit{Ave María}

        %SEXTO
        \item «Este será grande, y será llamado Hijo del Altísimo, y le dará el Señor Dios el trono de David su padre, 
        y reinará sobre la casa de Jacob etérnamente, y su reinado no tendrá fin». (Lc. 1, 32-33)

        \textit{Ave María}

        %SEPTIMO
        \item Y respondiendo el ángel, le dijo: «el Espíritu Santo descenderá sobre ti, 
        y el poder del Altísimo te cobijará con su sombra»; (Lc. 1, 35)
        
        \textit{Ave María}

        %OCTAVO
        \item En el sexto mes fué enviado el Ángel Gabriel de parte de Dios a una ciudad de Galilea, llamada Nazaret, a una 
        doncella desposada con un varón llamada José, de la familia de David, y el nombre de la doncella era María. (Lc. 1,26- 27)

        \textit{Ave María}

        %NOVENO
        \item «por lo cual también lo que nacerá será llamado santo, Hijo de Dios». (Lc. 1, 35)

        \textit{Ave María}

        %DECIMO
        \item María dijo: «he aquí la esclava del Señor; hágase en mí según tu palabra». (Lc. 1, 38)
        
      \end{enumerate}

      \textit{Ave María} \\
      \indent\textit{Gloria} \\
      \indent¡Oh Jesús mío! Perdonadnos. Libradnos del fuego enterno del infierno. Llevad al Cielo a todas las almas, y socorred especialmente a las más 
      necesitadas
            
    \subsection*{ La Visitación de Nuestra Señora }
      
      \textit{Pater Noster} 

      \begin{enumerate}
        
        %PRIMERO
        \item Por aquellos días, levantándose María, se dirigió presurosa a la montaña, a un ciudad de Judá, y entró en la casa
        de Zacarías y saludó a Isabel. (Lc. 1, 39-40)
        
        \textit{Ave María}

        %SEGUNDO
        \item Y aconteció que, al oír Isabel la salutación de María. dió saltos de gozo el niño en su seno, y fué llena Isabel del Espíritu Santo. (Lc. 1, 41)
        
        \textit{Ave María}

        %TERCERO
        \item y levantó la voz con gran clamor y dijo: »Bendita tu entre las mujeres y bendito esl fruto de tu vientre». (Lc. 1, 42)
        
        \textit{Ave María}

        %CUARTO
        \item Y dichosa la que creyó que tendrá cumplimiento las cosas que le han sido dichas de parte del Señor. (Lc. 1, 45)
        
        \textit{Ave María}

        %QUINTO
        \item Y dijo María: «Engrandece mi alma al Señor, y se regocijó mi espíritu en Dios, mi Salvador;
        porque puso sus ojos en la bajeza de su esclava». (Lc. 1, 46-48)

        \textit{Ave María}

        %SEXTO
        \item «Pues he aquí que desde ahora me llamarán dichosa todas las generaciones; 
        porque hizo en mi favor grandes cosas el Poderoso, (Lc. 1, 48-49)
        
        \textit{Ave María}

        %SEPTIMO
        \item «y cuyo nombre es Santo; y su misericordia por generaciones y generaciones para con aquellos que le temen». (Lc. 1, 49-50)
        
        \textit{Ave María}

        %OCTAVO
        \item Hizo ostentación de poder con su brazo: desbarató a los soberbios en los proyectos de su corazón. (Lc. 1, 51)
        
        \textit{Ave María}

        %NOVENO
        \item «derrocó de su trono a los potentados y enalteció a los humildes». (Lc. 1, 52)
        
        \textit{Ave María}

        %DECIMO
        \item «llenó de bienes a los hambrientos y despidió vacíos a los ricos». (Lc. 1, 53)

      \end{enumerate}

      \textit{Ave María} \\
      \indent\textit{Gloria} \\
      \indent¡Oh Jesús mío! Perdonadnos. Libradnos del fuego enterno del infierno. Llevad al Cielo a todas las almas, y socorred especialmente a las más 
      necesitadas
            
    \subsection*{ La Natividad de Nuestro Señor Jesucristo }
      
      \textit{Pater Noster} 

      \begin{enumerate}
        
        %PRIMERO
        \item Y sucedió que estando ellos allí se le complieron a ella los díaas del parto. (Lc. 2,6)
        
        \textit{Ave María}

        %SEGUNDO
        \item Y dió a luz a su hijo primogénito, y le envolvió en pañales\ldots (Lc. 2, 7)
        
        \textit{Ave María}

        %TERCERO
        \item \ldots y le recostó en un pesebre, pues no había para ellos lugar en el mesón. (Lc. 2, 7)
        
        \textit{Ave María}

        %CUARTO
        \item Y había unos pastores en aquella misma comarca, que pernoctaban al raso y velaban por turno para guardar su ganado, y un ángel
        del Señor se presentó ante ellos, y la gloria del Señor los envolvió en sus fulgores, y se atemorizaron con gran temor. (Lc. 2, 8-9)
        
        \textit{Ave María}

        %QUINTO
        \item Y les dijo el ángel: «no temáis, pues he aquí que os traigo una buena nueva, que será de grande alegría para todo el pueblo:» (Lc. 2, 10)

        \textit{Ave María}

        %SEXTO
        \item «que os ha nacido hoy en la ciudad de David un Salvador, que es el Mesías, el Señor» (Lc. 2, 11)
        
        \textit{Ave María}

        %SEPTIMO
        \item Gloria a Dios en las alturas y en la tierra a los hombres del [divino] agrado. (Lc. 2, 14)

        \textit{Ave María}

        %OCTAVO
        \item Nacido Jesús en Belén de la Judea en los días de Herodes el rey, he aquí que unos magos venidos de las regiones orientales llegaron a Jerusalén.
        Y entrando en la casa, vieron al niño con María, su madre; (Mt. 2, 1, 11)
        
        \textit{Ave María}

        %NOVENO
        \item y postrándose en tierra le adoraron, y abrieron sus tesoros le ofrecieron presentes, oro, incienso y mirra. (Mt. 2, 11)
        
        \textit{Ave María}

        %DECIMO
        \item Pero María guardaba todas estas palabras confiriéndolas en su corazón. (Lc. 2, 19)

      \end{enumerate}

      \textit{Ave María} \\
      \indent\textit{Gloria} \\
      \indent¡Oh Jesús mío! Perdonadnos. Libradnos del fuego enterno del infierno. Llevad al Cielo a todas las almas, y socorred especialmente a las más 
      necesitadas

    \subsection*{ La Presentación del Niño Jesús en el Templo }
      
      \textit{Pater Noster}

      \begin{enumerate}
        
        %PRIMERO
        \item Y cuando se les cumplieron los días de la purificación según la ley de Moisés (Lev. 12, 6), 
        le subieron a Jerusalén para presentarle al Señor. (Lc. 2, 22)
        
        \textit{Ave María}

        %SEGUNDO
        \item y he aquí había un hombre en Jerusalén por nombre Simeón. Y era este hombre justo y temeroso de Dios que aguardaba la consolación de Israel, 
        y el Espíritu Santo estaba sobre él; (Lc. 2, 25)
        
        \textit{Ave María}

        %TERCERO
        \item y le había sido revelado por el Espíritu Santo qye no vería la muerte antes de ver al Ungido del Señor. (Lc. 2, 26)
        
        \textit{Ave María}

        %CUARTO
        \item Y vino al templo impulsado por el Espíritu Santo. Y cuando sus padres intrudujeron al niño Jesús para cumplir las prescipciones usuales
        de la ley tocantes a El, Simeón le recibió en sus brazos y bendijo a Dios diciendo: (Lc. 2, 27-28)
        
        \textit{Ave María}

        %QUINTO
        \item Ahora dejas ir a tu siervo, Señor, según tu palabra, en paz; (Lc. 2, 29)

        \textit{Ave María}

        %SEXTO
        \item pues ya vieron mis ojos tu salud, que preparaste a la faz de todos los pueblos; (Lc. 2, 30-31)
        
        \textit{Ave María}

        %SEPTIMO
        \item luz para iluminación de los gentiles, y gloria de tu pueblo Israel. (Lc. 2, 32)
        
        \textit{Ave María}

        %OCTAVO
        \item Y les bendijo Simeón, y dijo a María, su madre: «He aquí que éste está puesto para caída y resurgimiento de muchos en Israel, y como
        señal a quien se contradice». (Lc. 2, 34)
        
        \textit{Ave María}

        %NOVENO
        \item «y a ti misma una espada te traspasará el alma, para que salgan a la luz de muchos corazones los pensamientos». (Lc. 2, 35)
        
        \textit{Ave María}

        %DECIMO
        \item Y así que cumplieron todas las cosas ordenadas en la ley del Señor, se volvieron a Galilea, a su ciudad de Nazaret. El niño crecía
        y se robustecía llenándose de sabiduría, y la gracia de Dios estaba en Él. (Lc. 2, 39-40)

      \end{enumerate}

      \textit{Ave María} \\
      \indent\textit{Gloria} \\
      \indent¡Oh Jesús mío! Perdonadnos. Libradnos del fuego enterno del infierno. Llevad al Cielo a todas las almas, y socorred especialmente a las más 
      necesitadas
            
    \subsection*{ La Pérdida y Hallazgo del Niño Jesús en el Templo }
      
      \textit{Pater Noster}

      \begin{enumerate}

        %PRIMERO
        \item Y cuando fué de doce años, habiendo ellos subido, según la costumbre de la fiesta, (Lc. 2, 42)
        
        \textit{Ave María}

        %SEGUNDO
        \item y acabados los días, al volverse ellos, quedóse el niño Jesús en Jerusalén, sin que lo advirtiesen sus padres. (Lc. 2, 43)
        
        \textit{Ave María}

        %TERCERO
        \item y no hallándole, se tornaron a Jerusalén para buscarlo. Y sucedió que después de tres días le hallaron en el templo,
        sentado en medio de los maestros, escuchándoles y haciéndoles preguntas; (Lc. 2, 45-46)
        
        \textit{Ave María}

        %CUARTO
        \item sentado en medio de los maestros, escuchándoles y haciéndoles preguntas; (Lc. 2, 46)
        
        \textit{Ave María}

        %QUINTO
        \item y se pasmaban todos los que le oían de su inteligencia y de sus respuestas. (Lc. 2, 47)

        \textit{Ave María}

        %SEXTO
        \item Y sus padres, al verle, quedaron sorprendidos; y le dijo su madre: «hijo, ¿por qué lo hiciste así con nosotros? Mira que tu padre
        y yo, llenos de aflicción, te andábamos buscando» (Lc. 2, 48)
        
        \textit{Ave María}

        %SEPTIMO
        \item Díjoles Él: «¿pues por qué me buscabais? ¿No sabíais que había yo de estar en casa de mi padre?» (Lc. 2, 49)
        
        \textit{Ave María}

        %OCTAVO
        \item Y ellos no comprendieron lo que les dijo. (Lc. 2, 50)
        
        \textit{Ave María}

        %NOVENO
        \item Y bajó en su compañía y se fue a Nazaret, y vivía sometido a ellos. Y su madre guardaba todas estas
        cosas en su corazón. (Lc. 2, 51)
        
        \textit{Ave María}

        %DECIMO
        \item Y Jesús progresaba en sabiduría, en estatura y en gracia delante de Dios y de los hombres. (Lc. 2, 52)

      \end{enumerate}

      \textit{Ave María} \\
      \indent\textit{Gloria} \\
      \indent¡Oh Jesús mío! Perdonadnos. Libradnos del fuego enterno del infierno. Llevad al Cielo a todas las almas, y socorred especialmente a las más 
      necesitadas
        
  \section*{ Misterios Dolorosos (Martes y Viernes)}
    
  \textit{Pater Noster}

    \subsection*{ La Agonía de Nuestro Señor en el Huerto de los Olivos }
      
      \begin{enumerate}

        %PRIMERO
        \item Entonces llego Jesús con ellos a una granja llamada Getsemaní, y dice a los discípulos: «Sentaos aquí mientras voy allá para hacer oración». 
        Y llevando consigo a Pedro y a los dos hijos de Zebedeo, comenzó a ponerse triste y a sentir abatimiento. (Mt. 26, 36-37)

        \textit{Ave María}

        %SEGUNDO
        \item Entonces les dice: «triste em gran manera está mi alma hasta la muerte; quedad aquí y velad conmigo». (Mt. 26, 38)

        \textit{Ave María}

        %TERCERO
        \item Y adelantándose un poco, caía sobre la tierra, y rogaba que, a ser posible, pasase de Él aquella hora. (Mc. 14, 35)

        \textit{Ave María}

        %CUARTO
        \item diciendo: «Padre, si quieres, traspasa de mi este cáliz; mas no se haga mi voluntad, sino la tuya». (Lc. 22, 42)

        \textit{Ave María}

        %QUINTO
        \item Y se le apareció un ángel venido del cielo, que le confortaba. (Lc. 22, 43)

        \textit{Ave María}

        %SEXTO
        \item Y venido en agonía, oraba más intensamente. (Lc. 22, 44)

        \textit{Ave María}

        %SEPTIMO
        \item Y se hizo su sudor como grumos de sangre, que caían hasta el suerlo. (Lc. 22, 44)

        \textit{Ave María}

        %OCTAVO
        \item Y viene a los discípulos y los halla durmiendo, y dice a Pedro: «¿así no pudísteis velar una hora conmigo?». (Mt. 26, 40)

        \textit{Ave María}

        %NOVENO
        \item «Velad y orad, para que no entréis en tentación;». (Mt. 26, 41)

        \textit{Ave María}

        %DECIMO
        \item «el espíritu sí, está animoso, más la carne es flaca». (Mt. 26, 41)

      \end{enumerate}

      \textit{Ave María} \\
      \indent\textit{Gloria} \\
      \indent¡Oh Jesús mío! Perdonadnos. Libradnos del fuego enterno del infierno. Llevad al Cielo a todas las almas, y socorred especialmente a las más 
      necesitadas

    \subsection*{ La flagelación de Nuestro Señor Jesucristo }
      
      \textit{Pater Noster}

      \begin{enumerate}
        
        %PRIMERO
        \item Y luego al amanecer, después de celebrar consejo, los sumos sacerdotes con los ancianos y los escribas, es decir, todo el sanhedrín, atando a Jesús,
        le llevaron de allí y le entregaron a Pilato. Y le interrogó Pilato: «¿Tú eres el Rey de los judíos?». El le respondió: «Tú lo dices». (Mc. 15, 1-2)

        \textit{Ave María}

        %SEGUNDO
        \item Respondió: «Mi reino no es de este mundo. Si de este mundo fuera mi reino, mis ministros lucharían para que yo no fuera entregado a los judíos
        Más ahora mi reino no es de aquí». (Jn. 18, 36)

        \textit{Ave María}

        %TERCERO
        \item Díjole, pues, Pilato: «¿luego, rey eres tu?». Respondión Jesús : «Tú dices que yo soy rey. Yo para eso he nacido y para esto he venido
        al mundo: para dar testimonio a favor de la verdad. Todo el que es de la verdad, oye mi voz». (Jn. 18, 37)

        \textit{Ave María}

        %CUARTO
        \item Pilato dijo a los sumos sacerdotes y a los turbas: «ningún delito hallo en este hombre. 
        Le castigaré, pues, y le soltaré». (Lc. 23, 4, 16)

        \textit{Ave María}

        %QUINTO
        \item Entonces, pues, tomó Pilato a Jesús y le azotó. (Jn. 19, 1)

        \textit{Ave María}

        %SEXTO
        \item De opresión u juicio fué tomado, y a sus contemporáneos, ¿quién tendrá en cuenta?. Fué despreciado y abandonado de los hombres,
        varón de dolores y familiarizado de los hombres. (Is. 53, 8, 3)

        \textit{Ave María}

        %SEPTIMO
        \item Mas nuestros sufrimientos él los ha llevado, nuestros dolores él los cargó sobre sí, (Is. 53, 4)

        \textit{Ave María}

        %OCTAVO
        \item Fué traspasado por causa de nuestros pecados, molido por causa de nuestras iniquidades; (Is. 53, 5)

        \textit{Ave María}

        %NOVENO
        \item mientras nosotros le tuvimos por azotado, por herido de Dios y abatido. (Is. 53, 4)

        \textit{Ave María}

        %DECIMO
        \item el castigo de nuestra paz cayó sobre Él y por sus verdugones se nos curó. (Is. 53, 5)

      \end{enumerate}

      \textit{Ave María} \\
      \indent\textit{Gloria} \\
      \indent¡Oh Jesús mío! Perdonadnos. Libradnos del fuego enterno del infierno. Llevad al Cielo a todas las almas, y socorred especialmente a las más 
      necesitadas
      
    \subsection*{ La coronación de espinas de Nuestro Señor Jesucristo }
      
      \textit{Pater Noster}

      \begin{enumerate}

        %PRIMERO
        \item Los soldados se lo llevaron dentro del palacio, que es el pretorio, y convocan a toda la cohorte (Mc. 15, 16).
        Y habiéndole quitado sus vestidos, le envolvieron en una clámide de grana (Mt. 27, 28).

        \textit{Ave María}

        %SEGUNDO
        \item y trenzando una corna de espinas, la pusieron sobre la cabeza, y una caña en la mano derecha; (Mt. 27, 29)

        \textit{Ave María}

        %TERCERO
        \item y doblando la rodilla delante de Él, le mofaban, diciendo: «Salud, Rey de los judíos». (Mt. 27, 29)

        \textit{Ave María}

        %CUARTO
        \item Y escupiendo en Él, tomaron la caña y le daba golpes en la cabeza. (Mt 27, 30)

        \textit{Ave María}

        %QUINTO
        \item Salió Pilato otra vez fuera, y les dice: «Ved, os le traigo para que conozcáis que no hallo en Él delito alguno». (Jn. 19, 4)

        \textit{Ave María}

        %SEXTO
        \item Salió, pues, Jesús afuera, llevando la corona de espinas y el manto de púrpura. Y les dice: «ved aquí el hombre». (Jn. 19, 5)

        \textit{Ave María}

        %SEPTIMO
        \item Y les dice: «ved aquí el hombre». Gritaron, pues, ellos: «quita, quita; crucifícale». (Jn. 19, 5, 15)

        \textit{Ave María}

        %OCTAVO
        \item Pilato, queriendo dar satisfacción a la turba les soltó a Barrabás. Y entregó a Jesús, después de azotarle, para que fuese crucificado. (Mc. 15, 14)

        \textit{Ave María}

        %NOVENO
        \item Díceles Pilato: «¿A vuestro rey he de crucificar?». Respondieron los pontífices: «no tenemos rey, sino César». (Jn. 19, 15)

        \textit{Ave María}

        %DECIMO
        \item Entonces, pués, se le entregó para que fuera crucificado. Se apoderaron, pues, de Jesús. (Jn. 19, 16)

      \end{enumerate}

      \textit{Ave María} \\
      \indent\textit{Gloria} \\
      \indent¡Oh Jesús mío! Perdonadnos. Libradnos del fuego enterno del infierno. Llevad al Cielo a todas las almas, y socorred especialmente a las más 
      necesitadas

    \subsection*{ Jesús con la Cruz a cuestas }
      
      \textit{Pater Noster}

      \begin{enumerate}
        
        %PRIMERO
        \item Si alguno quiere venir en pos de Mí, niégese a sí mismo. (Lc. 9, 23)

        \textit{Ave María}

        %SEGUNDO
        \item y tome a cuestas su cruz cada día y sígame. (Lc. 9, 23)

        \textit{Ave María}

        %TERCERO
        \item y, llevando a cuestas su cruz, salió hacia el lugar llamado el Cráneo, que en hebreo se dice Gólgota. (Jn. 19, 17)

        \textit{Ave María}

        %CUARTO
        \item Y como le hubieron sacado, echaron mano de un tal Simón de Cirene que venía del campo, le pusieron en hombros la cruz para que la llevase
        detrás de Jesús. (Lc. 23, 26)

        \textit{Ave María}

        %QUINTO
        \item Tomad mi yugo sobre vuestros, y aprended de mi, (Mt. 11, 29)

        \textit{Ave María}

        %SEXTO
        \item pues soy manso y humilde de Corazón, y hallaréis reposo para vuestras almas. (Mt. 11, 29)

        \textit{Ave María}

        %SEPTIMO
        \item Porque mi yugo es suave, y mi carga, ligera. (Mt. 11, 30)

        \textit{Ave María}

        %OCTAVO
        \item Seguíanle gran mucheduncbre de pueblo y de mujeres, las cuales le plañían y lamentaban. (Lc. 23, 27)

        \textit{Ave María}

        %NOVENO
        \item Volviéndose Jesús a ellas, les dijo: «Hijas de Jerusalén: no lloréis sobre mi, sino llorad má bien sobre vosotras mismas y sobre
        vuestros hijos». (Lc. 23, 28)

        \textit{Ave María}

        %DECIMO
        \item «Porque si en el leño verde esto hacen, ¿en el seco qué se hará?». (Lc. 23, 31)

      \end{enumerate}

      \textit{Ave María} \\
      \indent\textit{Gloria} \\
      \indent¡Oh Jesús mío! Perdonadnos. Libradnos del fuego enterno del infierno. Llevad al Cielo a todas las almas, y socorred especialmente a las más 
      necesitadas

    \subsection*{ La Crucifixión y Muerte del Redentor }
      
      \textit{Pater Noster}

      \begin{enumerate}
        
        %PRIMERO
        \item Y cuando hubieron llegado al lugar llamado «Cráneo», allí crucificaron a Él y a los malhechores, uno a la derecha y el otro a la izquierda. (Lc. 23, 33)

        \textit{Ave María}

        %SEGUNDO
        \item Y Jesús decía: «Padre, perdónalos, porque no saben lo que hacen». (Lc. 23, 34)

        \textit{Ave María}

        %TERCERO
        \item Uno de los malhechores que estaba colgado decía a Jesús: «acuérdate de mi cuando vinieres en la gloria de tu realeza». (Lc. 23, 39, 42)

        \textit{Ave María}

        %CUARTO
        \item Díjole: «en verdad te digo que hoy estarás conmigo en el paraíso». (Lc. 23, 43)

        \textit{Ave María}

        %QUINTO
        \item Jesús, pues, viendo a la Madre, y junto a ella al discípulo a quien amaba, (Jn. 19, 26)

        \textit{Ave María}

        %SEXTO
        \item dice a tu Madre: «mujer, he ahí a tu hijo». Luego dice al discípulo: «He aquí a tu Madre». (Jn. 19, 26-27)

        \textit{Ave María}

        %SEPTIMO
        \item Y desde aquella hora la tomó el discípulo en su compañía. (Jn. 19, 27)

        \textit{Ave María}

        %OCTAVO
        \item habiendo faltado el sol; y se rasgó por medio el velo del santuario. (Lc. 23, 45)

        \textit{Ave María}

        %NOVENO
        \item Y clamando con voz poderosa, Jesús dijo: «Padre, en tus manos encomiendo mis espíritu». (Lc. 23, 46 )

        \textit{Ave María}

        %DECIMO
        \item Jesús dijo: «Consumado está». E inclinando la cabeza entregó el espíritu. (Jn. 19, 30)

      \end{enumerate}

      \textit{Ave María} \\
      \indent\textit{Gloria} \\
      \indent¡Oh Jesús mío! Perdonadnos. Libradnos del fuego enterno del infierno. Llevad al Cielo a todas las almas, y socorred especialmente a las más 
      necesitadas
        
  \section*{ Miserios Gloriosos (Miércoles, Sábados y Domingos)}
    \subsection*{ La Resurección del Señor }
      
      \textit{Pater Noster}

      \begin{enumerate}
        
        %PRIMERO
        \item «En verdad, en verdad os digo que vosotros lloraréis y os lamentaréis, y el mundo se rogocijará;
        vosotros os afligiréis, pero vuestra aflicción está de parto». (Jn. 16, 20)

        \textit{Ave María}

        %SEGUNDO  
        \item «Pues así también vosotros, ahora cierto tenéis congoja; mas otra vez os veré, y se gozará vuestro corazon,
        y vuestro gozo nadie os lo quita». (Jn. 16, 22)

        \textit{Ave María}

        %TERCERO
        \item Más el primer día de la semana, apenas rayó el alba, se vinieron al monumento llevando consigo los aromas
        que habían preparado. (Lc. 24, 1)

        \textit{Ave María}

        %CUARTO
        \item De pronto se produjo un gran temblor de tierra, pues un ángel del Señor, bajado de cielo y acercándose, hizo rodar
        de su sitio la losa, y se sentó sobre ella. (Mt. 28, 2)
        
        \textit{Ave María}

        %QUINTO
        \item Tomando la palabra el ángel, dijo a las mujeres: «no tengáis miedo vosotras, que ya sé que buscáis a Jesús el crucificado». (Mt. 28, 5)

        \textit{Ave María}

        %SEXTO
        \item «no está aquí; resucitó, como dijo. Venid, ved el lugar dode estuvo puesto». (Mt. 28, 6)

        %SEPTIMO
        \item «Y marchando a toda prisa, decid a sus discípulos que resucitó de entre los muertos, y he aquí que se os adelanta en ir a Galilea:
        allí le veréis. Conque os lo tengo dicho». (Mt. 28, 7)

        \textit{Ave María}

        %OCTAVO
        \item Y partiendo a toda prisa del monumento, con temor y grande gozo corrieron a dar la nueva a sus discípulos. (Mt. 28, 8)

        \textit{Ave María}

        %NOVENO
        \item «Yo soy la resurección y la vida; quien cree en mi, aun cuando muera, vivirá». (Jn. 11, 25)

        \textit{Ave María}

        %DECIMO
        \item «y todo el que vive y cree en mi, no morirá para siempre. ¿Crees esto?». (Jn. 11,26)
        
      \end{enumerate}

      \textit{Ave María} \\
      \indent\textit{Gloria} \\
      \indent¡Oh Jesús mío! Perdonadnos. Libradnos del fuego enterno del infierno. Llevad al Cielo a todas las almas, y socorred especialmente a las más 
      necesitadas

    \subsection*{ La Ascensión de Jesucristo a los cielos }
        
      \textit{Pater Noster}

      \begin{enumerate} 

        %PRIMERO
        \item Y los sacó afuera hasta llegar junto a Betania, y alzando sus manos los bendijo. (Lc. 24, 50)

        \textit{Ave María}

        %SEGUNDO
        \item Y acercándose Jesús, les habló diciendo: «Me fué dada toda potestad en el cielo y sobre la tierra». (Mt. 28, 18)

        \textit{Ave María}

        %TERCERO
        \item «Id, pues, amaestrad a todas las gentes». (Mt. 28, 19)

        \textit{Ave María}

        %CUARTO
        \item «bautizándoles en el nombre del Padre y del Hijo y del Espíritu Santo». (Mt. 28, 19)

        \textit{Ave María}

        %QUINTO
        \item «enseñándoles a guardar todas cuantas cosas os ordené». (Mt. 28, 20)

        \textit{Ave María}

        %SEXTO
        \item El que creyere y fuere bautizado, se salvará; (Mc. 16, 16)

        \textit{Ave María}

        %SEPTIMO
        \item mas el que no creyere, será condenado. (Mc. 16, 16)

        \textit{Ave María}

        %OCTAVO
        \item «Y sabed que estoy con vosotros todos los días hasta la consumación de los siglos» (Mt. 28, 20)

        \textit{Ave María}

        %NOVENO
        \item Y aconteció que, mientras los bendecía, se desprendió de ellos, y era llevado en alto al cielo. (Lc. 24, 51)

        \textit{Ave María}

        %DECIMO
        \item Con esto el Señor Jesús, despues de hablarles, fue elevado al cielo y se sentó a la diestra de Dios. (Mc. 16, 19)

      \end{enumerate}

      \textit{Ave María} \\
      \indent\textit{Gloria} \\
      \indent¡Oh Jesús mío! Perdonadnos. Libradnos del fuego enterno del infierno. Llevad al Cielo a todas las almas, y socorred especialmente a las más 
      necesitadas

    \subsection*{ La Venida del Espíritu Santo sobre los Apóstoles }

      \textit{Pater Noster}

      \begin{enumerate}

        %PRIMERO
        \item Y al cumplirse el día de Pentecostés, estaban todos juntos en el mismo lugar. (Hch. 2, 1)

        \textit{Ave María}

        %SEGUNDO
        \item Y se produjo de súbito desde el cielo un estruendo como de viento que soplaba vehemente, u llenó toda la casa
        donde se hallaban sentados. (Hch. 2, 2)

        \textit{Ave María}

        %TERCERO
        \item Y vieron aparecer lenguas como de fuego, que, repartiéndose, se posaban sobre cada uno de ellos. (Hch. 2, 3)

        \textit{Ave María}

        %CUARTO
        \item Y se llenaron todos del Espíritu Santo, y comenzaron a hablar en lenguas diferentes, según que el Espíritu Santo les movía
        a expresarse. (Hch. 2, 4)

        \textit{Ave María}

        %QUINTO
        \item Hallábanse en Jerusalén judíos allí domiciliados, hombres religiosos de toda nación de las que están debajo del cielo. (Hch. 2, 5)

        \textit{Ave María}

        %SEXTO
        \item Puesto de pie Pedro, acompañado de los Once, alzó en voz y les habló en estos términos. (Hch. 2, 14)

        \textit{Ave María}

        %SEPTIMO
        \item «Arrepentíos, dice, y bautícese cada uno de vosotros en el nombre del Jesu-Cristo para remisión de vuestros pecados, y recibiréis el don
        del Espíritu Santo». (Hch. 2, 38)

        \textit{Ave María}

        %OCTAVO
        \item Ellos, pues, acogieron su palabra, fueron bautizados; y fueron agragados en aqueñ día como unas tres mil almas. (Hch. 2,41)

        \textit{Ave María}

        %NOVENO
        \item SI tu Espíritu envías, son creados, y la faz de la tierra así renuevas. (Sal. 104, 30)

        \textit{Ave María}

        %DECIMO
        \item ¡Ven, Espíritu Santo, y desde el Cielo envía rayos de tu virtud!¡!Ven, Padre de los pobres!¡Ven, Dador de tus Dones!¡Ven, de almas luz!
        
      \end{enumerate}

      \textit{Ave María} \\
      \indent\textit{Gloria} \\
      \indent¡Oh Jesús mío! Perdonadnos. Libradnos del fuego enterno del infierno. Llevad al Cielo a todas las almas, y socorred especialmente a las más 
      necesitadas

    \subsection*{ La Asunción de Nuestra Señora a los cielos }

      \textit{Pater Noster}

      \begin{enumerate}

        %PRIMERO
        \item Bendita tú, hija, ante el Dios Altísimo sobre todas las mujeres de la tierra\ldots (Jdt. 13, 18)

        \textit{Ave María}

        %SEGUNDO
        \item Pues no se apartará etérnamente tu esperanza del corazón de los hombre\ldots (Jdt. 13, 19)

        \textit{Ave María}

        %TERCERO
        \item Y esto haga contigo Dios para eterno encubrimiento, que te visite con sus bienes; por cuanto no perdonate
        a tu vida, lastimada por la humillación de nuestro linaje\ldots (Jdt. 13, 20)

        \textit{Ave María}

        %CUARTO
        \item {\ldots}Tú eres enaltecimiento de Jerusalén, tú gloria grande de Israel, tú grande honor de nuestro linaje. (Jdt. 15, 9)

        \textit{Ave María}

        %QUINTO
        \item Oye, hija, mira; tu oído aplica; tu pueblo olvida y la mansión paterna; deja que tu hermosura
        el rey codicie, que es tu señor. A él le doblega. (Sal. 45; 11-12)

        \textit{Ave María}

        %SEXTO
        \item Y se abrió el templo de Dios, que está en el cielo, y fué vista el arca de la alianza en el templo,
        y se produjeron relámpagos, y voces, y truenos, y temblor de tierra, y fuerte granizada. (Ap. 11, 19)

        \textit{Ave María}

        %SEPTIMO
        \item Y una gran señal fué vista en el cielo: una Mujer vestida del sol\ldots (Ap. 12, 1)

        \textit{Ave María}

        %OCTAVO
        \item {\ldots}y la luna debajo de sus pies, y sobre su cabeza una corona de doce estrellas. (Ap. 12, 1)

        \textit{Ave María}

        %NOVENO
        \item Del rey la hija toda hermosa entra; vestidos áureos se adorno son. (Sal. 45, 14)

        \textit{Ave María}

        %DECIMO
        \item Entonad a Yahveh cántico nuevo, que portentos ha obrado. Su diestra le ha traído la victoria y aquel su brazo santo. (Sal. 98, 1)

      \end{enumerate}

      \textit{Ave María} \\
      \indent\textit{Gloria} \\
      \indent¡Oh Jesús mío! Perdonadnos. Libradnos del fuego enterno del infierno. Llevad al Cielo a todas las almas, y socorred especialmente a las más 
      necesitadas

    \subsection*{La Coronación de la Santísima Virgen María }

      \textit{Pater Noster}

      \begin{enumerate}

        %PRIMERO
        \item ¿Quién es esa que aparece resplandeciente como la aurora,
        hermosa cual luna, deslumbradora como el sol, imponente como batallones? (Cant. 6, 10)

        \textit{Ave María}

        %SEGUNDO
        \item Como el arco iris, que se aparece en las nubes; como flor entre el ramaje en días primaverales, como azucena junto
        a la corriente de las aguas, como las flores del Líbano en días de verano; como el incienso que arde sobre la ofrenda,
        como el vaso de oro fínamente trabajado. (Eclo. 50, 8-9)

        \textit{Ave María}

        %TERCERO
        \item Yo soy la madre del amor, del temor, de la ciencia y de la santa esperanza. (Eclo. 24, 24)

        \textit{Ave María}

        %CUARTO
        \item EN mi está toda la gracia del camino y de la verdad, en mi toda esperanza de la vida y de la virtud. (Eclo. 24, 25)

        \textit{Ave María}

        %QUINTO
        \item Venid a mí cuantos me deseáis y saciaos de mis frutos. (Eclo. 24, 26)

        \textit{Ave María}

        %SEXTO
        \item Porque recordarme es más dulce que la miel, y poseerme más rico que el panal de miel. (Eclo. 24, 27)

        \textit{Ave María}

        %SEPTIMO
        \item Ahora, pues, hijos míos, oídme; y felices quienes guardan mis caminos. Escochad la corrección y sed sabios, (Prov. 8, 32-33)

        \textit{Ave María}

        %OCTAVO
        \item y no la rechacéis. Feliz el hombre que me escucha, velando a mis puertas cada día,
        guardando las jambas de mis entradas. (Prov. 8, 33-34)

        \textit{Ave María}

        %NOVENO
        \item Pues quien me halla, ha hallado la vida y alcanza el favor de Yahveh. Más quien peca contra mi, se perjudica a si mismo,
        y cuantos me odian aman la muerte. (Prov. 8, 35)

        \textit{Ave María}

        %DECIMO
        \item Tiene Él escrito en su vestido y en su manto Rey de reyes y Señor de los que dominan (Ap. 18, 16).
        Está la Reina a su derecha, adornada con oro finísimo (Sal. 44, 10).
      \end{enumerate}

      \textit{Ave María} \\
      \indent\textit{Gloria} \\
      \indent¡Oh Jesús mío! Perdonadnos. Libradnos del fuego enterno del infierno. Llevad al Cielo a todas las almas, y socorred especialmente a las más 
      necesitadas
      
\end{document}
