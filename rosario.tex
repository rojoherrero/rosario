\documentclass[a4paper,11pt]{article}

\usepackage[utf8]{inputenc}
\usepackage[spanish]{babel}
\usepackage{subfiles}
\usepackage[T1]{fontenc}

\begin{document}
  \section*{\hfil Misterios Gozosos \hfil}
    \subsection*{\hfil La Anunciación de la Santísima Virgen María \hfil}

      % PRIMER
      \subfile{mysteries/joyful/joyful_01_01}
      \medskip

      %SEGUNDO
      \subfile{mysteries/joyful/joyful_01_02}
      \medskip

      %TERCERO
      \subfile{mysteries/joyful/joyful_01_03}
      \medskip

      %CUARTO
      \subfile{mysteries/joyful/joyful_01_04}
      \medskip

      %QUINTO
      \subfile{mysteries/joyful/joyful_01_05}
      \medskip

      %SEXTO
      \subfile{mysteries/joyful/joyful_01_06}
      \medskip

      %SEPTIMO
      \subfile{mysteries/joyful/joyful_01_07}
      \medskip
      
      %OCTAVO
      \subfile{mysteries/joyful/joyful_01_08}
      \medskip

      %NOVENO
      \subfile{mysteries/joyful/joyful_01_09}
      \medskip

      %DECIMO
      \subfile{mysteries/joyful/joyful_01_10}
      \medskip
            
    \subsection*{\hfil La Visitación de Nuestra Señora \hfil}
      
      % PRIMER
      \subfile{mysteries/joyful/joyful_02_01}
      \medskip
      
      %SEGUNDO
      \subfile{mysteries/joyful/joyful_02_02}
      \medskip
      
      %TERCERO
      \subfile{mysteries/joyful/joyful_02_03}
      \medskip
      
      %CUARTO
      \subfile{mysteries/joyful/joyful_02_04}
      \medskip
      
      %QUINTO
      \subfile{mysteries/joyful/joyful_03_05}
      \medskip

      %SEXTO
      \subfile{mysteries/joyful/joyful_02_06}
      \medskip
      
      %SEPTIMO
      \subfile{mysteries/joyful/joyful_02_07}
      \medskip
      
      %OCTAVO
      \subfile{mysteries/joyful/joyful_02_08}
      \medskip
      
      %NOVENO
      \subfile{mysteries/joyful/joyful_02_09}
      \medskip
      
      %DECIMO
      \subfile{mysteries/joyful/joyful_02_10}
      \medskip
            
    \subsection*{\hfil La Natividad de Nuestro Señor Jesucristo \hfil}
      
      % PRIMER
      \subfile{mysteries/joyful/joyful_03_01}
      \medskip
      
      %SEGUNDO
      \subfile{mysteries/joyful/joyful_03_02}
      \medskip
      
      %TERCERO
      \subfile{mysteries/joyful/joyful_03_03}
      \medskip
      
      %CUARTO
      \subfile{mysteries/joyful/joyful_03_04}
      \medskip
      
      %QUINTO
      \subfile{mysteries/joyful/joyful_03_05}
      \medskip

      %SEXTO
      \subfile{mysteries/joyful/joyful_03_06}
      \medskip
      
      %SEPTIMO
      \subfile{mysteries/joyful/joyful_03_07}
      \medskip

      %OCTAVO
      \subfile{mysteries/joyful/joyful_03_08}
      \medskip
      
      %NOVENO
      \subfile{mysteries/joyful/joyful_03_09}
      \medskip
      
      %DECIMO
      \subfile{mysteries/joyful/joyful_03_10}

            
    \subsection*{\hfil La Presentación del Niño Jesús en el Templo \hfil}
      
      % PRIMER
      \subfile{mysteries/joyful/joyful_04_01}
      \medskip
      
      %SEGUNDO
      \subfile{mysteries/joyful/joyful_04_02}
      \medskip
      
      %TERCERO
      \subfile{mysteries/joyful/joyful_04_03}
      \medskip
      
      %CUARTO
      \subfile{mysteries/joyful/joyful_04_04}
      \medskip
      
      %QUINTO
      \subfile{mysteries/joyful/joyful_04_05}
      \medskip

      %SEXTO
      \subfile{mysteries/joyful/joyful_04_06}
      \medskip
      
      %SEPTIMO
      \subfile{mysteries/joyful/joyful_04_07}
      \medskip
      
      %OCTAVO
      \subfile{mysteries/joyful/joyful_04_08}
      \medskip
      
      %NOVENO
      \subfile{mysteries/joyful/joyful_04_09}
      \medskip
      
      %DECIMO
      \subfile{mysteries/joyful/joyful_04_10}
      \medskip
            
    \subsection*{\hfil La Pérdida y Hallazgo del Niño Jesús en el Templo \hfil}
      
      % PRIMER
      \subfile{mysteries/joyful/joyful_05_01}
      \medskip
      
      %SEGUNDO
      \subfile{mysteries/joyful/joyful_05_02}
      \medskip
      
      %TERCERO
      \subfile{mysteries/joyful/joyful_05_03}
      \medskip
      
      %CUARTO
      \subfile{mysteries/joyful/joyful_05_04}
      \medskip
      
      %QUINTO
      \subfile{mysteries/joyful/joyful_05_05}
      \medskip

      %SEXTO
      \subfile{mysteries/joyful/joyful_05_06}
      \medskip
      
      %SEPTIMO
      \subfile{mysteries/joyful/joyful_05_07}
      \medskip
      
      %OCTAVO
      \subfile{mysteries/joyful/joyful_05_08}
      \medskip
      
      %NOVENO
      \subfile{mysteries/joyful/joyful_05_09}
      \medskip     
      
      %DECIMO
      \subfile{mysteries/joyful/joyful_05_10}
      \medskip
            
    \newpage
        
  \section*{\hfil Misterios Dolorosos \hfil}
    \subsection*{\hfil La Agonía de Nuestro Señor en el Huerto de los Olivos \hfil}
      
      % UNO
      \textsuperscript{36}Entonces llego Jesús con ellos a una granja llamada Getsemaní, y dice a los discípulos: «Sentaos aquí mientras voy allá para hacer oración». 
      \textsuperscript{37}Y llevando consigo a Pedro y a los dos hijos de Zebedeo, comenzó a ponerse triste y a sentir abatimiento.
      \begin{flushright}
        Mt. 26; 36-37     
      \end{flushright}

      % DOS
      \textsuperscript{38}Entonces les dice: «triste em gran manera está mi alma hasta la muerte; quedad aquí y velad conmigo»
      \begin{flushright}
        Mt. 26, 38
      \end{flushright}

      % TRES
      \textsuperscript{35}Y adelantándose un poco, caía sobre la tierra, y rogaba que, a ser posible, pasase de Él aquella hora
      \begin{flushright}
        Mc. 14, 35  
      \end{flushright}

      % CUATRO
      \textsuperscript{42}diciendo: «Padre, si quieres, traspasa de mi este cáliz; mas no se haga mi voluntad, sino la tuya».
      \begin{flushright}
        Lc. 22, 42
      \end{flushright}

      % CINCO
      \textsuperscript{43}Y se le apareció un ángel venido del cielo, que le confortaba
      \begin{flushright}
        Lc. 22, 43
      \end{flushright}

      % SEIS
      \textsuperscript{44}Y venido en agonía, oraba más intensamente. 
      \begin{flushright}
        Lc. 22, 44
      \end{flushright}

      % SIETE
      Y se hizo su sudor como grumos de sangre, que caían hasta el suerlo.
      \begin{flushright}
        Lc. 22, 44
      \end{flushright}

      % OCHO
      \textsuperscript{40}Y viene a los discípulos y los halla durmiendo, y dice a Pedro: «¿así no pudísteis velar una hora conmigo?»
      \begin{flushright}
        Mt. 26, 40
      \end{flushright}

      % NUEVE
      \textsuperscript{41}«Velad y orad, para que no entréis en tentación;»
      \begin{flushright}
        Mt. 26, 41
      \end{flushright}

      % DIEZ
      «el espíritu sí, está animoso, más la carne es flaca»
      \begin{flushright}
        Mt. 26, 41
      \end{flushright}

    \subsection*{\hfil La flagelación de Nuestro Señor Jesucristo \hfil}
      
      % UNO
      \textsuperscript{1}Y luego al amanecer, después de celebrar consejo, los sumos sacerdotes con los ancianos y los escribas, es decir, todo el sanhedrín, atando a Jesús,
      le llevaron de allí y le entregaron a Pilato. \textsuperscript{2}Y le interrogó Pilato: «¿Tú eres el Rey de los judíos?». El le respondió: «Tú lo dices».
      \begin{flushright}
        Mc. 15, 1-2
      \end{flushright}

      % DOS
      \textsuperscript{36}Respondió: «Mi reino no es de este mundo. Si de este mundo fuera mi reino, mis ministros lucharían para que yo no fuera entregado a los judíos
      Más ahora mi reino no es de aquí».
      \begin{flushright}
        Jn. 18, 36
      \end{flushright}

      % TRES
      \textsuperscript{37}Díjole, pues, Pilato: «¿luego, rey eres tu?». Respondión Jesús : «Tú dices que yo soy rey. Yo para eso he nacido y para esto he venido
      al mundo: para dar testimonio a favor de la verdad. Todo el que es de la verdad, oye mi voz».
      \begin{flushright}
        Jn. 18, 37
      \end{flushright}

      % CUATRO
      \textsuperscript{4}Pilato dijo a los sumos sacerdotes y a los turbas: «ningún delito hallo en este hombre. \textsuperscript{16}Le castigaré, pues, y le soltaré».
      \begin{flushright}
        Lc. 23, 4, 16
      \end{flushright}

      % CINCO
      \textsuperscript{1}Entonces, pues, tomó Pilato a Jesús y le azotó.
      \begin{flushright}
        Jn. 19, 1
      \end{flushright}

      % SEIS
      \textsuperscript{8}De opresión u juicio fué tomado, y a sus contemporáneos, ¿quién tendrá en cuenta?. \textsuperscript{3}Fué despreciado y abandonado de los hombres,
      varón de dolores y familiarizado de los hombres
      \begin{flushright}
        Is. 53, 8, 3
      \end{flushright}

      % SIETE
      \textsuperscript{4}Mas nuestros sufrimientos él los ha llevado, nuestros dolores él los cargó sobre sí,
      \begin{flushright}
        Is. 53, 4
      \end{flushright}

      % OCHO
      \textsuperscript{5}Fué traspasado por causa de nuestros pecados, molido por causa de nuestras iniquidades; 
      \begin{flushright}
        Is. 53, 5
      \end{flushright}

      % NUEVE
      mientras nosotros le tuvimos por azotado, por herido de Dios y abatido
      \begin{flushright}
        Is. 53, 4 
      \end{flushright}

      % DIEZ
      el castigo de nuestra paz cayó sobre Él y por sus verdugones se nos curó.
      \begin{flushright}
        Is. 53, 5
      \end{flushright}
      
    \subsection*{\hfil La coronación de espinas de Nuestro Señor Jesucristo \hfil}
      
      % UNO
      \textsuperscript{16}Los soldados se lo llevaron dentro del palacio, que es el pretorio, y convocan a toda la cohorte (Marcos).
      \begin{flushright}
        Mc. 15, 16
      \end{flushright}
      \textsuperscript{28}Y habiéndole quitado sus vestidos, le envolvieron en uns clámide de grana (Mateo).
      \begin{flushright}
        Mt. 27, 28
      \end{flushright}

      % DOS
      \textsuperscript{29}y trenzando una corna de espinas, la pusieron sobre la cabeza, y una caña en la mano derecha; 
      \begin{flushright}
        Mt. 27, 29
      \end{flushright}

      % TRES
      y doblando la rodilla delante de Él, le mofaban, diciendo: «Salud, Rey de los judíos»
      \begin{flushright}
        Mt. 27, 29
      \end{flushright}

      % CUATRO
      \textsuperscript{30}Y escupiendo en Él, tomaron la caña y le daba golpes en la cabeza.
      \begin{flushright}
        Mt 27, 30
      \end{flushright}

      % CINCO
      \textsuperscript{4}Salió Pilato otra vez fuera, y les dice: «Ved, os le traigo para que conozcáis que no hallo en Él delito alguno».
      \begin{flushright}
        Jn. 19, 4
      \end{flushright}

      % SEIS
      \textsuperscript{5}Salió, pues, Jesús afuera, llevando la corona de espinas y el manto de púrpura. Y les dice: «ved aquí el hombre»
      \begin{flushright}
        Jn. 19, 5
      \end{flushright}

      % SIETE
      \textsuperscript{5}Y les dice: «ved aquí el hombre». \textsuperscript{15}Gritaron, pues, ellos: «quita, quita; crucifícale»
      \begin{flushright}
        Jn. 19, 5, 15
      \end{flushright}

      % OCHO
      \textsuperscript{14}Pilato, queriendo dar satisfacción a la turba les soltó a Barrabás. Y entregó a Jesús, después de azotarle, para que fuese crucificado.
      \begin{flushright}
        Mc. 15, 14
      \end{flushright}

      % NUEVE
      Díceles Pilato: «¿A vuestro rey he de crucificar?». Respondieron los pontífices: «no tenemos rey, sino César»
      \begin{flushright}
        Jn. 19, 15
      \end{flushright}

      % DIEZ
      \textsuperscript{16}Entonces, pués, se le entregó para que fuera crucificado. Se apoderaron, pues, de Jesús.
      \begin{flushright}
        Jn. 19, 16
      \end{flushright}

    \subsection*{\hfil Jesús con la Cruz a cuestas \hfil}
      
      % UNO
      \textsuperscript{23}Si alguno quiere venir en pos de Mí, niégese a sí mismo
      \begin{flushright}
        Lc. 9, 23
      \end{flushright}

      % DOS
      y tome a cuestas su cruz cada día y sígame
      \begin{flushright}
        Lc. 9, 23
      \end{flushright}

      % TRES
      \textsuperscript{17}y, llevando a cuestas su cruz, salió hacia el lugar llamado el Cráneo, que en hebreo se dice Gólgota.
      \begin{flushright}
        Jn. 19, 17
      \end{flushright}

      % CUATRO
      \textsuperscript{26}Y como le hubieron sacado, echaron mano de un tal Simón de Cirene que venía del campo, le pusieron en hombros la cruz para que la llevase
      detrás de Jesús.
      \begin{flushright}
        Lc. 23, 26
      \end{flushright}

      % CINCO
      \textsuperscript{29}Tomad mi yugo sobre vuestros, y aprended de mi, 
      \begin{flushright}
        Mt. 11, 29  
      \end{flushright}

      % SEIS
      pues soy manso y humilde de Corazón, y hallaréis reposo para vuestras almas.
      \begin{flushright}
        Mt. 11, 29
      \end{flushright}

      % SIETE
      Porque mi yugo es suave, y mi carga, ligera.
      \begin{flushright}
        Mt. 11, 30
      \end{flushright}

      % OCHO
      \textsuperscript{27}Seguíanle gran mucheduncbre de pueblo y de mujeres, las cuales le plañían y lamentaban.
      \begin{flushright}
        Lc. 23, 27
      \end{flushright}

      % NUEVE
      \textsuperscript{28}Volviéndose Jesús a ellas, les dijo: «Hijas de Jerusalén: no lloréis sobre mi, sino llorad má bien sobre vosotras mismas y sobre
      vuestros hijos»
      \begin{flushright}
        Lc. 23, 28
      \end{flushright}

      % DIEZ
      \textsuperscript{31}«Porque si en el leño verde esto hacen, ¿en el seco qué se hará?».
      \begin{flushright}
        Lc. 23, 31
      \end{flushright}

    \subsection*{\hfil La Crucifixión y Muerte del Redentor \hfil}
      
      % UNO
      \textsuperscript{33}Y cuando hubieron llegado al lugar llamado «Cráneo», allí crucificaron a Él y a los malhechores, uno a la derecha y el otro a la izquierda.
      \begin{flushright}
        Lc. 23, 33
      \end{flushright}

      % DOS
      \textsuperscript{34}Y Jesús decía: «Padre, perdónalos, porque no saben lo que hacen».
      \begin{flushright}
        Lc. 23, 34
      \end{flushright}

      % TRES
      \textsuperscript{39}Uno de los malhechores que estaba colgado \textsuperscript{42}decía a Jesús: «acuérdate de mi cuando vinieres en la gloria de tu realeza».
      \begin{flushright}
        Lc. 23, 39, 42
      \end{flushright}

      % CUATRO
      \textsuperscript{43}Díjole: «en verdad te digo que hoy estarás conmigo en el paraíso»
      \begin{flushright}
        Lc. 23, 43
      \end{flushright}

      % CINCO
      \textsuperscript{26}Jesús, pues, viendo a la Madre, y junto a ella al discípulo a quien amaba,
      \begin{flushright}
        Jn. 19, 26
      \end{flushright}

      % SEIS
      \textsuperscript{27}dice a tu Madre: «mujer, he ahí a tu hijo». \textsuperscript{27}Luego dice al discípulo: «He aquí a tu Madre»
      \begin{flushright}
        Jn. 19, 26-27
      \end{flushright}

      % SIETE
      Y desde aquella hora la tomó el discípulo en su compañía.
      \begin{flushright}
        Jn. 19, 27
      \end{flushright}

      % OCHO
      \textsuperscript{45}habiendo faltado el sol; y se rasgó por medio el velo del santuario.
      \begin{flushright}
        Lc. 23, 45
      \end{flushright}

      % NUEVE
      \textsuperscript{46}Y clamando con voz poderosa, Jesús dijo: «Padre, en tus manos encomiendo mis espíritu».
      \begin{flushright}
        Lc. 23, 46 
      \end{flushright}

      % DIEZ
      \textsuperscript{30}Jesús dijo: «Consumado está». E inclinando la cabeza entregó el espíritu.
      \begin{flushright}
        Jn. 19, 30
      \end{flushright}
 
    \newpage
         
  \section*{\hfil Miserios Gloriosos \hfil}
    \subsection*{\hfil La Resurección del Señor \hfil}
      %PRIMERO
      \textsuperscript{20}«En verdad, en verdad os digo que vosotros lloraréis y os lamentaréis, y el mundo se rogocijará;
      vosotros os afligiréis, pero vuestra aflicción está de parto»
      \begin{flushright}
        Jn. 16, 20        
      \end{flushright}      
      %SEGUNDO
      \textsuperscript{22}«Pues así también vosotros, ahora cierto tenéis congoja; mas otra vez os veré, y se gozará vuestro corazon,
      y vuestro gozo nadie os lo quita»
      \begin{flushright}
        Jn. 16, 22       
      \end{flushright}      
      %TERCERO
      \textsuperscript{1}Más el primer día de la semana, apenas rayó el alba, se vinieron al monumento llevando consigo los aromas
      que habían preparado.
      \begin{flushright}
        Lc. 24, 1        
      \end{flushright}      
      %CUARTO
      \textsuperscript{2}De pronto se produjo un gran temblor de tierra, pues un ángel del Señor, bajado de cielo y acercándose, hizo rodar
      de su sitio la losa, y se sentó sobre ella.
      \begin{flushright}
        Mt. 28, 2        
      \end{flushright}      
      %QUINTO
      \textsuperscript{5}Tomando la palabra el ángel, dijo a las mujeres: «no tengáis miedo vosotras, que ya sé que buscáis a Jesús el crucificado;»
      \begin{flushright}
        Mt. 28, 5        
      \end{flushright}      
      %SEXTO
      \textsuperscript{6}«no está aquí; resucitó, como dijo. Venid, ved el lugar dode estuvo puesto».
      \begin{flushright}
        Mt. 28, 6       
      \end{flushright}      
      %SEPTIMO
      \textsuperscript{7}«Y marchando a toda prisa, decid a sus discípulos que resucitó de entre los muertos, y he aquí que se os adelanta en ir a Galilea:
      allí le veréis. Conque os lo tengo dicho».
      \begin{flushright}
        Mt. 28, 7     
      \end{flushright}      
      %OCTAVO
      \textsuperscript{8}Y partiendo a toda prisa del monumento, con temor y grande gozo corrieron a dar la nueva a sus discípulos
      \begin{flushright}
        Mt. 28, 8       
      \end{flushright}      
      %NOVENO
      \textsuperscript{25}«Yo soy la resurección y la vida; quien cree en mi, aun cuando muera, vivirá»
      \begin{flushright}
        Jn. 11, 25    
      \end{flushright}      
      %DECIMO
      \textsuperscript{26}«y todo el que vive y cree en mi, no morirá para siempre. ¿Crees esto?»
      \begin{flushright}
        Jn. 11,26     
      \end{flushright}
    \subsection*{\hfil La Ascensión de Jesucristo a los cielos \hfil}
      %PRIMERO
      \textsuperscript{50}Y los sacó afuera hasta llegar junto a Betania, y alzando sus manos los bendijo.
      \begin{flushright}
        Lc. 24, 50     
      \end{flushright}      
      %SEGUNDO
      \textsuperscript{18}Y acercándose Jesús, les habló diciendo: «Me fué dada toda potestad en el cielo y sobre la tierra».
      \begin{flushright}
        Mt. 28, 18      
      \end{flushright}      
      %TERCERO
      \textsuperscript{19}«Id, pues, amaestrad a todas las gentes».
      \begin{flushright}
        Mt. 28, 19
      \end{flushright}      
      %CUARTO
      \textsuperscript{19}«bautizándoles en el nombre del Padre y del Hijo y del Espíritu Santo».
      \begin{flushright}
        Mt. 28, 19      
      \end{flushright}      
      %QUINTO
      \textsuperscript{20}«enseñándoles a guardar todas cuantas cosas os ordené».
      \begin{flushright}
        Mt. 28, 20  
      \end{flushright}      
      %SEXTO
      \textsuperscript{16}El que creyere y fuere bautizado, se salvará;
      \begin{flushright}
        Mc. 16, 16
      \end{flushright}      
      %SEPTIMO
      \textsuperscript{16}mas el que no creyere, será condenado.
      \begin{flushright}
        Mc. 16, 16
      \end{flushright}      
      %OCTAVO
      \textsuperscript{28}«Y sabed que estoy con vosotros todos los días hasta la consumación de los siglos»
      \begin{flushright}
        Mt. 28, 20
      \end{flushright}      
      %NOVENO
      \textsuperscript{51}Y aconteció que, mientras los bendecía, se desprendió de ellos, y era llevado en alto al cielo
      \begin{flushright}
        Lc. 24, 51
      \end{flushright}      
      %DECIMO
      \textsuperscript{19}Con esto el Señor Jesús, despues de hablarles, fue elevado al cielo y se sentó a la diestra de Dios.
      \begin{flushright}
        Mc. 16, 19
      \end{flushright}
    \subsection*{\hfil La Venida del Espíritu Santo sobre los Apóstoles \hfil}
      %PRIMERO
      \textsuperscript{1}Y al cumplirse el día de Pentecostés, estaban todos juntos en el mismo lugar.
      \begin{flushright}
        Hch. 2, 1
      \end{flushright}      
      %SEGUNDO
      \textsuperscript{2}Y se produjo de súbito desde el cielo un estruendo como de viento que soplaba vehemente, u llenó toda la casa
      donde se hallaban sentados.
      \begin{flushright}
        Hch. 2, 2
      \end{flushright}      
      %TERCERO
      \textsuperscript{3}Y vieron aparecer lenguas como de fuego, que, repartiéndose, se posaban sobre cada uno de ellos.
      \begin{flushright}
        Hch. 2, 3
      \end{flushright}      
      %CUARTO
      \textsuperscript{4}Y se llenaron todos del Espíritu Santo, y comenzaron a hablar en lenguas diferentes, según que el Espíritu Santo les movía
      a expresarse.
      \begin{flushright}
        Hch. 2, 4
      \end{flushright}      
      %QUINTO
      \textsuperscript{5}Hallábanse en Jerusalén judíos allí domiciliados, hombres religiosos de toda nación de las que están debajo del cielo.
      \begin{flushright}
        Hch. 2, 5
      \end{flushright}      
      %SEXTO
      \textsuperscript{14}Puesto de pie Pedro, acompañado de los Once, alzó en voz y les habló en estos términos
      \begin{flushright}
        Hch. 2, 14
      \end{flushright}      
      %SEPTIMO
      \textsuperscript{38}«Arrepentíos, dice, y bautícese cada uno de vosotros en el nombre del Jesu-Cristo para remisión de vuestros pecados, y recibiréis el don
      del Espíritu Santo»
      \begin{flushright}
        Hch. 2, 38
      \end{flushright}      
      %OCTAVO
      \textsuperscript{41}Ellos, pues, acogieron su palabra, fueron bautizados; y fueron agragados en aqueñ día como unas tres mil almas.
      \begin{flushright}
        Hch. 2,41
      \end{flushright}      
      %NOVENO
      \textsuperscript{30}SI tu Espíritu envías, son creados, y la faz de la tierra así renuevas.
      \begin{flushright}
        Sal. 104, 30
      \end{flushright}      
      %DECIMO
      ¡Ven, Espíritu Santo, y desde el Cielo envía rayos de tu virtud!¡!Ven, Padre de los pobres!¡Ven, Dador de tus Dones!¡Ven, de almas luz!
      \begin{flushright}
        Secuencia de Pentecostés
      \end{flushright}
    \subsection*{\hfil La Asunción de Nuestra Señora a los cielos \hfil}
      %PRIMERO
      \textsuperscript{18}Bendita tú, hija, ante el Dios Altísimo sobre todas las mujeres de la tierra...
      \begin{flushright}
        Jdt. 13, 18
      \end{flushright}      
      %SEGUNDO
      \textsuperscript{19}Pues no se apartará etérnamente tu esperanza del corazón de los hombre...
      \begin{flushright}
        Jdt. 13, 19
      \end{flushright}
      %TERCERO
      \textsuperscript{20}Y esto haga contigo Dios para eterno encubrimiento, que te visite con sus bienes; por cuanto no perdonate
      a tu vida, lastimada por la humillación de nuestro linaje...
      \begin{flushright}
        Jdt. 13, 20
      \end{flushright}      
      %CUARTO
      \textsuperscript{9}...Tú eres enaltecimiento de Jerusalén, tú gloria grande de Israel, tú grande honor de nuestro linaje
      \begin{flushright}
        Jdt. 15, 9
      \end{flushright}      
      %QUINTO
      \textsuperscript{11}Oye, hija, mira; tu oído aplica; tu pueblo olvida y la mansión paterna; \textsuperscript{12}deja que tu hermosura
      el rey codicie, que es tu señor. A él le doblega
      \begin{flushright}
        Sal. 45; 11-12
      \end{flushright}      
      %SEXTO
      \textsuperscript{19}Y se abrió el templo de Dios, que está en el cielo, y fué vista el arca de la alianza en el templo,
      y se produjeron relámpagos, y voces, y truenos, y temblor de tierra, y fuerte granizada.
      \begin{flushright}
        Ap. 11, 19
      \end{flushright}      
      %SEPTIMO
      \textsuperscript{1}Y una gran señal fué vista en el cielo: una Mujer vestida del sol...
      \begin{flushright}
        Ap. 12, 1
      \end{flushright}      
      %OCTAVO
      \textsuperscript{1}...y la luna debajo de sus pies, y sobre su cabeza una corona de doce estrellas.
      \begin{flushright}
        Ap. 12, 1
      \end{flushright}      
      %NOVENO
      \textsuperscript{14}Del rey la hija toda hermosa entra; vestidos áureos se adorno son.
      \begin{flushright}
        Sal. 45, 14
      \end{flushright}      
      %DECIMO
      \textsuperscript{1}Entonad a Yahveh cántico nuevo, que portentos ha obrado. Su diestra le ha traído la victoria y aquel su brazo santo.
      \begin{flushright}
        Sal. 98, 1
      \end{flushright}
    \subsection*{\hfil La Coronación de la Santísima Virgen María \hfil}
      %PRIMERO
      \textsuperscript{10}¿Quién es esa que aparece resplandeciente como la aurora,
      hermosa cual luna, deslumbradora como el sol, imponente como batallones?
      \begin{flushright}
        Cant. 6, 10
      \end{flushright}      
      %SEGUNDO
      \textsuperscript{8}Como el arco iris, que se aparece en las nubes; como flor entre el ramaje en días primaverales, como azucena junto
      a la corriente de las aguas, como las flores del Líbano en días de verano; \textsuperscript{9}como el incienso que arde sobre la ofrenda,
      como el vaso de oro fínamente trabajado.
      \begin{flushright}
        Eclo. 50, 8-9
      \end{flushright}      
      %TERCERO
      \textsuperscript{24}Yo soy la madre del amor, del temor, de la ciencia y de la santa esperanza.
      \begin{flushright}
        Eclo. 24, 24
      \end{flushright}      
      %CUARTO
      \textsuperscript{25}EN mi está toda la gracia del camino y de la verdad, en mi toda esperanza de la vida y de la virtud.
      \begin{flushright}
        Eclo. 24, 25
      \end{flushright}      
      %QUINTO
      \textsuperscript{26}Venid a mí cuantos me deseáis y saciaos de mis frutos
      \begin{flushright}
        Eclo. 24, 26
      \end{flushright}      
      %SEXTO
      \textsuperscript{27}Porque recordarme es más dulce que la miel, y poseerme más rico que el panal de miel.
      \begin{flushright}
        Eclo. 24, 27
      \end{flushright}      
      %SEPTIMO
      \textsuperscript{32}Ahora, pues, hijos míos, oídme; y felices quienes guardan mis caminos. \textsuperscript{33}Escochad la corrección y sed sabios,
      \begin{flushright}
        Prov. 8, 32-33
      \end{flushright}      
      %OCTAVO
      \textsuperscript{33}y no la rechacéis. Feliz el hombre que me escucha, velando a mis puertas cada día,
      guardando las jambas de mis entradas
      \begin{flushright}
        Prov. 8, 33-34
      \end{flushright}      
      %NOVENO
      \textsuperscript{35}Pues quien me halla, ha hallado la vida y alcanza el favor de Yahveh. Más quien peca contra mi, se perjudica a si mismo,
      y cuantos me odian aman la muerte.
      \begin{flushright}
        Prov. 8, 35
      \end{flushright}
      %DECIMO
      \textsuperscript{16}Tiene Él escrito en su vestido y en su manto Rey de reyes y Señor de los que dominan\textsubscript{Ap. 18, 16}.  \\
      \textsuperscript{10}Está la Reina a su derecha, adornada con oro finísimo\textsubscript{Sal. 44, 10}.
      \begin{flushright}
        Ap. 18, 16. Sal. 44, 10
      \end{flushright}
\end{document}
