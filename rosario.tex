\documentclass[11pt,a4paper]{book}
\usepackage[utf8]{inputenc}
\usepackage[spanish]{babel}
\usepackage[T1]{fontenc}
\usepackage{geometry}
\usepackage[spanish]{cleveref}
\geometry{
    a4paper,
    total={170mm,257mm},
    left=20mm,
    top=20mm,
 }
\usepackage{subfiles}
\usepackage{comment}
\usepackage{longtable}

%\setlength\parindent{0pt}

\title{Rosario Bíblico Meditado}
\author{Sergio}
\date{\today}

\begin{document}

    \begin{titlepage}
    \maketitle        
    \end{titlepage}

    \chapter*{Oraciones}

    \section*{La Señal de la Cruz}\label{sec:senal-cruz}
    \begin{longtable} { p{0.5\textwidth} p{0.5\textwidth} }
        Por la señal de la Santa Cruz, de nuestros enemigos libranos Señor, Dios nuestro. En el Nombre del Padre, 
        y del Hijo, y del Espíritu Santo. Amén.
        
        &
        
        Per signum crucis de inimícis nostris libera nos, Deus noster. In Nómine Pátris, et Filii, et Spíritus Sancti. Amen.
    \end{longtable}

    \section*{Padre Nuestro}\label{sec:paternoster}
    \begin{longtable} { p{0.5\textwidth} p{0.5\textwidth} }
        Padre Nuestro que estás en los cielos, santificado sea tu Nombre. Venga a nosotros tu Reino. 
        Hágase tu voluntad, así en la tierra como en el cielo. El pan nuestro de cada día dánosle hoy. 
        Y perdónamos nuestras deuda, así como nosotros perdonamos a nuestros deudores. 
        Y no nos dejes caer en la tentación: mas líbranos del mal. Amén.
        
        &
        
        Pater noster, qui es in cælis, sanctificétur nomen tuum. Advéniat regnum tuum. 
        Fiat volúntas tua, sicut in cœlo et in terra. Panem nóstrum quotidiánum da nobis hódie. 
        Et dimitte nobis debita nostra, sicut et nos dimittimus debitóribus nostris. 
        Et ne nos indúcas in tentatiónem: sed libera nos a malo. Amen.
    \end{longtable}

    \section*{Avemaría}\label{sec:avemaria}
    \begin{longtable} { p{0.5\textwidth} p{0.5\textwidth} }
        Dios de salve, María, llena eres de gracia, el Señor es contigo; bendita eres entre todas las mujeres, 
        y bendito es el fruto de tu vientre, Jesús. Santa María, Madre de Dios, ruega por nosotros pecadores, 
        ahora u en el hora de nuestra muerte. Amén.
        
        &
        
        Ave María, grátia plena, Dóminus tecum; benedicta tu in muliéribus, et benedíctus fructus ventris tui, 
        Iesus. Sancta Maria, Mater Dei, ora pro nobis peccatóribus, nunc et in hora mortis nostræ. Amen.
    \end{longtable}

    \section*{Gloria Patri}\label{sec_gloria}
    \begin{longtable} { p{0.5\textwidth} p{0.5\textwidth} }
        V. Gloria al Padre, al Hijo, y al Espíritu Santo. &
        V. Gloria Patri, et Filio, et Spíritui Sancto.\\
        
        R. Como era en el principio, ahora, y siempre, y por los siglos de los siglos. Amén. &
        R. Sicut erat in princípio et nunc, et semper et in sæcula sæculórum, Amen.
    \end{longtable}

    \section*{Confíteor}\label{sec:confiteor}
    \begin{longtable} { p{0.5\textwidth} p{0.5\textwidth} }
        Yo pecador me confieso a Dios todopoderoso, a la bienaventurada siempre Virgen María, al bienaventurado San Miguel Arcángel, 
        al bienaventurado San Juan Bautista, a los Santos Apóstoles Pedro y Pablo, a todos los Santos y a vos, Padre, que pequé mucho 
        de pensamiento, palabra y obra: por mi culpa, por mi culpa, por mi grandísima culpa. Por tanto, ruego a la bienaventurada 
        siempre Virgen María, al bienaventurado San Miguel Arcángel, al bienaventurado San Juan Bautista, a los Santos Apóstoles 
        Pedro y Pablo, a todos los Santos, y a vos, Padre, que roguéis por mi a Dios Nuestro Señor. Amén. 
        &
        Confíteor Deo omnipoténti, beátæ Maríæ semper Virigini, beáto Michaéli Archángelo, beáto Joánni Baptístæ, sanctis Apóstolis Petro et Paulo, 
        ómníbus Sanctis (et tibi, pater), quia peccávi nimis, cogitatióne, verbo et ópere, mea culpa, mea culpa, mea máxima culpa. Ideo precor beátam 
        Maríam semper Virgínem, beátum Michaélem Archángelum, beátum Joánnem Baptístam, sanctis Apóstolos Petrum et Paulum, omnes Sanctos (et te, pater), 
        oráre pro me ad Dóminum Deum nostrum. Amen.
    \end{longtable}

    \section*{Acto de contrición}\label{sec:contricion}
    Señor mío Jesucristo, Dios y Hombre verdadero, Creador y Redentor mío: por ser vos quién sois, y porque os amo sobre todas las cosas, 
    me pesa de todo corazón de haberos ofendido, propongo firmemente nunca más pecar, y apartarme de todas las ocasiones de ofenderos, 
    confesarme, y cumplir la penitencia que me fuere impuesta; ofrézcoos mi vida, obras y trabajos en satisfacción de todos mis pecados; 
    y confío en vuestra bondad y misericordia infinita me los perdonaréis por los merecimientos de vuestra preciosísima sangre, pasión y muerte, 
    y me daréis gracia para enmendarme y para perseverar en vuestro santo servicio hasta el fin de mi vida. Amén.

    \section*{Credo Apostólico}\label{sec:credo}
    \begin{longtable} { p{0.5\textwidth} p{0.5\textwidth} }
        Creo en Dios, Padre todopoderoso. Creador del cielo y de la tierra. Y en Jesucristo, su único Hijo, Nuestro Señor, 
        que fue concebido por obra y gracia del Espíritu Santo; nació de Santa María Vírgen; padeció bajo el poder de Poncio Pilato, 
        fue crucificado, muerto y sepultado; descendió a los infiernos; al tercer día resucitó de entre los muertos; subió a los cielos, 
        está sentado a la derecha de Dios Padre todopoderoso; desde allí ha de venir a juzgar a vivos y muertos. 
        Creo en el Espíritu Santo, la Santa Iglesia Católica, la comunión de los Santos, el perdón de los pecados, 
        la resurección de la carne y la vida eterna. Amén. 
        &
        Credo in Deum, Patrem omnipoténtem. Creatórem cœli et terræ. Et in Jesum Christum, Filium ejus únicum, Dóminum nostrum; 
        qui concéptus est de Spíritu Sancto; natus ex María Virgine; passus sub Póntio Pilato, crucifíxus, mortuus et sepúltus: 
        descéndit ad inferos; tértia die resurréxit a mórtuis: ascéndit ad cœlos, sedet ad dexteram Dei Patris omnipoténtis; 
        inde ventúrus est judicáre vivos et mórtuos. Credo in Spíritum Sanctum, sanctam Ecclésiam cathólicam, Sanctórum communiónem, 
        remissiónem peccatórum, carnis resurrectiónem, vitam ætérnam. Amen.
    \end{longtable}

    \section*{Salve}\label{sec:salve}
    \begin{longtable} { p{0.5\textwidth} p{0.5\textwidth} }
        Dios te salve, Reina y Madre de mi­se­ri­cordia, vida, dulzura y esperanza nuestra; Dios te salve. 
        A ti llamamos los desterrados hijos de Eva; a ti suspiramos, gimiendo y llorando en este valle de lágrimas. 
        Ea, pues, Señora, abogada nuestra, vuelve a nosotros esos tus ojos mi­se­ri­cordiosos. Y después de este destierro, muéstranos a Jesús, 
        fruto bendito de tu vientre. ¡Oh cle­men­tísima, oh piadosa, oh dulce siempre Virgen María! &
        Salve, Regina, Mater mi­se­ri­córdiæ, vita, dulcédo et spes nostra, salve. Ad te clamámus, éxsules fílii Hevæ. 
        Ad te suspirámus geméntes et flentes in hac lacrimárum valle. Éia ergo, advocáta nostra, illos tuos mi­se­ri­córdes óculos ad nos convérte. 
        Et Iesum benedíctum fructum ventris tui, nobis, post hoc exsílium, osténde. O clemens, o pia, o dulcis Virgo Maria!\\
        
        V. Ruega por nosotros, Santa Madre de Dios. & V. Ora pro nobis, Sancta Dei Génetrix.\\
        
        R. Para que seamos dignos de alcanzar las promesas de nuestro Señor Jesucristo. Amén. & 
        R. Ut digni efficiámur pro­mi­ssiónibus Christi. Amen.
    \end{longtable}

    \section*{Oraciones de Fátima}

    \subsection*{Oración del Perdón}\label{subsec:forgivenessPrayer}
    ¡Dios mío, yo creo, adoro, espero y te amo! Te pido perdón por los que no creen, no adoran, no esperan, no te aman.

    \subsection*{Oración del Ángel}\label{subsec:angelPrayer}
    Santísima Trinidad: Padre, Hijo y Espíritu Santo, Os adoro profundamente y Os ofrezco el Preciosísimo Cuerpo, Sangre, Alma y Divinidad de Jesucristo, 
    presente en todos los tabernáculos del mundo, en reparación por las ofensas, sacrilegios e indiferencias con los que El es ofendido. Por los méritos 
    infinitos del Sagrado Corazón de Jesús y del Inmaculado Corazón de María, Os pido la conversión de los pecadores.

    \chapter*{Rosario \footnote{Los que lleven consigo el Rosario debidamente bendecido y lo besaren devotamente diciento las palabras:
        «Ave María, grátia plena, Dóminus tecum; benedicta tu in muliéribus, et benedictum fructus ventris tui, Iesus», Su Santidad el 
        Papa Pío XII concede perpetuamente 500 días de indulgencia una vez el día}}

    \begin{center}
        Por la señal{\ldots}\\
        Confíteor\\
        Acto de contrición
    \end{center}

    \begin{longtable} {p{0.45\textwidth} p{0.45\textwidth} }
        V. Abre, Señor, mis labios & V. Dómine, lábia mea apéries\\
        R. Y mi boca cantará tus alabanzas & R. Et os meum annuntiábit laudem tuam\\
        V. Apresútare, Señor, a socorrerme & V. Deum, in adjutórium meum inténde\\
        R. Ven, oh Dios, en mi ayuda & R. Dómine, ad adjuvándum me festina\\ 
        V. Gloria al Padre{\ldots} & V. Gloria Patri{\ldots}\\
        R. Como era{\ldots} & R. Sicut erat{\ldots}\\
    \end{longtable}

    \begin{center}
        \label{sec:endTenPrayers}
        María, Madre de gracia, Madre de Misericordia. Defendednos del enemigo y amparadnos ahora y en la hora de nuestra muerte. Amén.
        
        ¡Oh Jesús mío! Perdonadnos. Libradnos del fuego enterno del infierno. Llevad al Cielo a todas las almas, y socorred especialmente 
        a las más necesitadas
    \end{center}


    \begin{center}
        1 Paternoster, 3 Avemarías y Gloria.
    \end{center}

    % EMPIEZAN LOS MISTERIOS
    
    \section*{Misterios Gozosos}
    Lunes, jueves y domingos de Adviento y Navidad.

    \subsection*{I Misterio: La Anunciación de la Santísima Virgen María}
    Y habiendo entrado a ella, dijo: «Dios te salve, llena de gracia, el Señor es contigo, bendita tu entre las mujeres». 
    Ella, al oír estas palabras, se turbó, y discurría que podría ser esta salucatión. Y le dijo el ángel: «No temas, María, 
    pues hallaste gracia a los ojos de Dios. He aquí que concebirás en tu seno y darás a luz un Hijo, a quien daras por 
    nombre Jesús. Este será grande, y será llamado Hijo del Altísimo, y le dará el Señor Dios el trono de DAvid su padre, 
    y reinará sobre la casa de Jacob etérnamete, y su reinado no tendrá fin».

    \begin{flushright}
        \emph{Lucas 1, 27-33}
    \end{flushright}

    1 Paternoster, 10 Avemarías, 1 Gloria, ¡On Jesús mío!... y María, Madre de gracia... (\cpageref{sec:endTenPrayers})

    \rule{\textwidth}{0.5pt}
    
    \begin{enumerate}
        \item \textbf{\emph{Paternóster}}. Al principio era el Verbo, y el Verbo estaba en Dios y el Verbo era Dios. El estaba al principio en Dios. Y el Verbo se hizo carne, 
                y habitó entre nosotros; y contemplamos su gloria, gloria cual del Unigénito procedente del Padre, lleno de gracia y de verdad. 
                \emph{Jn. 1, 1-2.14}. \textbf{\emph{Avemaría}}.

        \item Pues bien, el Señor mismo os dará una señal: He aquí que la virgen concebirá y parirá un hijo, 
                a quien ella denominará con el nombre de Emmanuel. Y brotará un retoño del trono de Jesé y retoñará de sus raices un vástago. 
                Sobre el que reposará el espíritu de Yavé, espíritu de sabiduría y de inteligencia, espíritu de consejo y de fortaleza, espíritu 
                de entendimiento y de temor de Yavé. \emph{Is. 7,14; 11, 1-4}. \textbf{\emph{Avemaría}}.                

        \item He aquí que se le apareció en sueños un ángel del Señor y le dijo: José, hijo de David, no temas recibir en tu casa a María, tu esposa,
                pues lo concebido en ella es obra del Espiritu Santo \emph{Mt. 1, 20}. \textbf{\emph{Avemaría}}.

        \item Dará a luz un hijo, a quien pondrás por nombre Jesús, porque salvará a su pueblo de sus pecados. Todo esto sucedió para que se cumplieran
                lo que el Señor había anunciado por el profeta. \emph{Mt. 1, 21-22}. \textbf{\emph{Avemaría}}.
    
        \item Fué enviado el Ángel Gabriel de parte de Dios a una ciudad de Galilea, llamada Nazaret, 
                a una doncella desposada con un varón llamada José, de la familia de David, y el nombre de la doncella era María. \emph{Lc. 1, 26-27}. \textbf{\emph{Avemaría}}.

        \item Y habiendo entrado a ella, dijo: «Dios te salve, llena de gracia, el Señor es contigo, bendita tú entre las mujeres». 
                Ella, al oír estas palabras, se turbó, y discurría qué podía ser esta salutación. \emph{Lc. 1, 28-29}. \textbf{\emph{Avemaría}}.

        \item Y le dijo el ángel: «No temas, María, pues hallaste gracia a los ojos de Dios. He aquí que concebirás en tu seno y darás a luz un Hijo, 
                a quién darás por nombre Jesús». \emph{Lc. 1, 30-31}. \textbf{\emph{Avemaría}}.

        \item «Este será grande, y será llamado Hijo del Altísimo, y le dará el Señor Dios el trono de David su padre, 
                y reinará sobre la casa de Jacob etérnamente, y su reinado no tendrá fin».  \emph{Lc. 1, 32-33}. \textbf{\emph{Avemaría}}.

        \item Sube a un alto monte, anuncia a Sión la buena nueva. Alza con fuerza la voz, tú que llevas la buena nueva A Jerusalén. Alzadla, no temáis nada, decid a las ciudades de Judá:
                He aquí a vuestro Dios. \emph{Is. 40, 9}. \textbf{\emph{Avemaría}}.

        \item He aquí al Señor, Yahveh, que viene con fortaleza. Su brazo dominará. Ved que viene con él su salario y va delante de Él su fruto. El apacentará a su rebaño como pastores,
                El le reunirá con su brazo; El llevará en su seno a los corderos y cuidará a las perdidas \emph{Is. 40, 10-11}. 
                \textbf{\emph{Avemaría}} y \textbf{\emph{Gloria}}
    \end{enumerate}

    \rule{\textwidth}{0.5pt}
    ¡On Jesús mío!... y María, Madre de gracia... (\cpageref{sec:endTenPrayers})

    \subsection*{II Misterio:La Visitación de Nuestra Señora}

    Por aquellos días, levantándose María, se dirigió presurosa a la montaña, a un ciudad de Judá, y entró en la casa de Zacarías y saludó a Isabel. 
    Y aconteció que, al oir Isabel la salutación de María, dió saltos de gozo el niño en su seno, y fue llena Isabel del Espíritu Santo, 
    y levantó la voz con gran clamor y dijo: «Bendita tu entre las mujeres y bendito el fruto de tu vientre. ¿Y de dónde a mí esto que venga la madre de mi Señor a mí? 
    Porque he aquí que, como sonó la voz de tu salutación en mi oídos, dió saltos de alborozo el niño en mi seno.
    
    \begin{flushright}
        \emph{Lucas 1, 39-45}
    \end{flushright}    

    1 Paternoster, 10 Avemarías, 1 Gloria, ¡On Jesús mío!... y María, Madre de gracia... (\cpageref{sec:endTenPrayers})
    
    \rule{\textwidth}{0.5pt}

    \begin{enumerate}
        \item \textbf{\emph{Paternóster}}. Por aquellos días, levantándose María, se dirigió presurosa a la montaña, a un ciudad de Judá, y entró en la casa de Zacarías y saludó a Isabel. \emph{Lc. 1, 39-40}. \textbf{\emph{Avemaría}}.

        \item Y aconteció que, al oír Isabel la salutación de María. dió saltos de gozo el niño en su seno, y fué llena Isabel del Espíritu Santo. \emph{Lc. 1, 41}. \textbf{\emph{Avemaría}}.

        \item y levantó la voz con gran clamor y dijo: «Bendita tu entre las mujeres y bendito esl fruto de tu vientre». \emph{Lc. 1, 42}. \textbf{\emph{Avemaría}}.

        \item ¿Y de dónde a mí esto que venga la madre de mi Señor a mí?». \emph{Lc. 1, 43}. \textbf{\emph{Avemaría}}.

        \item Porque he aquí que. como sonó la voz de tu salutación en mi oídos, dió saltos de alborozo el niño en mi seno. \emph{Lc. 1, 44}. \textbf{\emph{Avemaría}}.

        \item Y dichosa la que creyó que tendrá cumplimiento las cosas que le han sido dichas de parte del Señor. \emph{Lc. 1, 45}. \textbf{\emph{Avemaría}}.

        \item Regocíjate y alégrate, hija de Sión, porque he aquí que yo estoy para llegar y habitaré en medio de ti, dice Yahveh. \emph{Zac. 2,14}. \textbf{\emph{Avemaría}}.

        \item Oigo que se grita: En el desierto depejad el camino a Yahveh, enderezad en la estepa una calzada para nuestro Dios \emph{Is. 40,3}. \textbf{\emph{Avemaría}}.

        \item ¡Gritad de júbilo, exultad juntamente, ruinas de Jerusalén, pues Yahveh se ha compadecido de su pueblo, ha redimido a Jerusalén! \emph{Is. 52,9}. \textbf{\emph{Avemaría}}.

        \item Súbitamente hago aproximarse mi justicia; mi salvación brotará como luz y mis brazos juzgarán a los pueblos. En mi esperarán las islas y en mi brazo confiarán \emph{Is. 51,5}.. \textbf{\emph{Avemaría}} y \textbf{\emph{Gloria}}
    \end{enumerate}    

    \rule{\textwidth}{0.5pt}
    ¡On Jesús mío!... y María, Madre de gracia... (\cpageref{sec:endTenPrayers})

    \subsection*{III Misterio: La Natividad del Nuestro Señor Jesucristo}

    Y dió a luz su hijo primogénito, y le envolvió en pañales y le recostó en un pesebre, pues no había para ellos lugar en el mesón. 
    Y había unos pastores en aquella misma comarca, que pernoctaban al raso y velaban por turno para guardar su ganado, 
    y un ángel del  Señor se presentó ante ellos. Y les dijo el Ángel: «No tenáis, pues he aquí que os traigo una buena nueva, 
    que será de grande alegría para todo el pueblo: que os ha nacido hoy en la ciudad de David un Salvador, que es el Mesías, el Señor».

    \begin{flushright}
        \emph{Lucas 2, 7-8.10-11}
    \end{flushright}    

    1 Paternoster, 10 Avemarías, 1 Gloria, ¡On Jesús mío!... y María, Madre de gracia... (\cpageref{sec:endTenPrayers})
    
    \rule{\textwidth}{0.5pt}

    \begin{enumerate}
        \item \textbf{\emph{Paternóster}}. Pero tú, Belén de Efratá, pequeño entre los clanes de Judá, de ti saldrá quien señoreará en Israel, cuyos
            orígenes serán de antiguo, de días de muy remota antigüedad. \emph{Miq, 4, 2}. \textbf{\emph{Avemaría}}.

        \item El pueblo que camina en las tinieblas vió una gran luz; una luz ha resplandecido sobre los que habitaban en la tierra de sombras de muerte.
            Pues un niño nos ha nacido, un hijo se nos ha dado, sobre cuyo hombro está el principado y cuyo nombre se llamará Consejero maravilloso, 
            Dios fuerte, Padre eterno, Principe de la Paz. \emph{Is. 9, 2.5}. \textbf{\emph{Avemaría}}.

        \item José subió de Galilea, de la ciudad de Nazaret, a Judea, a la ciudad de David, que se llama Belén, por ser el de la casa y de la familia de David 
            \emph{Lc. 2, 4-5}. \textbf{\emph{Avemaría}}.

        \item Y sucedió que estando ellos allí se le complieron a ella los días del parto. Y dió a luz a su hijo primogénito, 
            y le envolvió en pañales y le recostó en un pesebre, pues no había para ellos lugar en el mesón. 
            \emph{Lc. 2, 6-7}. \textbf{\emph{Avemaría}}.

        \item Y había unos pastores en aquella misma comarca, que pernoctaban al raso y velaban por turno para guardar su ganado, 
            y un ángel del Señor se presentó ante ellos, y la gloria del Señor los envolvió en sus fulgores, y se atemorizaron con gran temor. 
            \emph{Lc. 2, 8-9}. \textbf{\emph{Avemaría}}.

        \item Y les dijo el ángel: «no temáis, pues he aquí que os traigo una buena nueva, que será de grande alegría para todo el pueblo: 
            que os ha nacido hoy en la ciudad de David un Salvador, que es el Mesías, el Señor». Esto tendréis por señal: encontraréis un 
            niño envuelto en pañales y reclinado en un pesebre. \emph{Lc. 2, 10-12}. \textbf{\emph{Avemaría}} y \textbf{\emph{Gloria}}
        
        \item Al instante se juntó con el ángel una multitud del ejército celestial, que alababa a Dios diciendo: «Gloria a Dios en las alturas y paz en
            la tierra a los hombres de buena voluntad» \emph{Lc. 2, 13-14}. \textbf{\emph{Avemaría}}.

        \item Así que los ángeles se fueron al cielo, se dijeron los pastores unos a otros: Vamos a Belén a ver esto que el Señor nos ha anunciado. fueron
            con presteza y encontraron a María, a José y al Niño acostado en un pesebre, y viéndole, contaron lo que se les había dicho acerca del
            Niño. \emph{Lc. 2, 15-17}. \textbf{\emph{Avemaría}}.   

        \item Nacido, pues, Jesús en Belén de Judá en los días del rey Herodes, llegaron de Oriente a Jerusalén unos magos, diciendo: ¿Dónde está el rey de los judíos
            que acaba de nacer? Porque hemos visto su estrella al oriente y hemos venido a adorarle. \emph{Mt. 2, 1-2}. \textbf{\emph{Avemaría}}.            
        
        \item Mas a cuantos le recibieron dioles poder de venir a ser hijos de Dios, a aquellos que creen en su nombre; que no de la sangre, ni de la voluntad carnal,
            ni de la voluntad de varón, sino de Dios son nacidos. \emph{Jn. 1, 12-13}. \textbf{\emph{Avemaría}} y \textbf{\emph{Gloria}}
    \end{enumerate}    

    \rule{\textwidth}{0.5pt}
    ¡On Jesús mío!... y María, Madre de gracia... (\cpageref{sec:endTenPrayers})

    \subsection*{IV Misterio:: La Purificación de nuestra Serñora y La Presentación del Niño Jesús en el Templo}

    Y cuando se les cumplieron los días de la purificación según la ley de Moisés (Levítico 12, 6), 
    le subieron a Jerusalen para presentarle al Señor, según está escrito en la Ley del Señor que «todo primogénito 
    del sexo masculino será consagrado al Señor\footnote{Éxodo 13, 2; 12, 15\label{primogenito}}», y para ofrecer como sacrificio, 
    según lo que se ordena en la Ley del Señor, «un par de tórtolas o dos palominos\footnote{Levítico 12, 8; 5, 11\label{sacrificio}}». 
    \begin{flushright}
        \emph{Lucas 2, 22-24}
    \end{flushright}

    1 Paternoster, 10 Avemarías, 1 Gloria, ¡On Jesús mío!... y María, Madre de gracia... (\cpageref{sec:endTenPrayers})    

    \rule{\textwidth}{0.5pt}

    \begin{enumerate}
        \item \textbf{\emph{Paternóster}}. ¡Alzaos, oh vosotras las puertas eternales, para que el Rey de la gloria entre!
            ¿Y quién es este Rey de la gloria? El Señor fuerte, Señor [fuerte] y potente, poderoso en la liza. \emph{Sal 24, 7-8}. \textbf{\emph{Avemaría}}.

        \item Habló después Yahveh a Moises diciendo: «Conságradme todo primogénito; todo primer nacido entre los hijos de Israel, tanto en hombres como en animales es mío». 
            \emph{Ex. 13, 1-2}. \textbf{\emph{Avemaría}}.

        \item Cuando se hubieron complido los ocho días para circuncidar al Niño, le dieron el nombre de Jesús, impuesto por el ángel antes de será concebido en el seno. 
            \emph{Lc. 2, 21}. \textbf{\emph{Avemaría}}.

        \item Y cuando se les cunplieron los días de la purificación según la ley de Moisés, le subieron a Jerusalen para presentarle al Señor, 
            según está escrito en la Ley del Señor que «todo primogénito del sexo masculino será consagrado al Señor», 
            y para ofrecer como sacrificio, según lo que se ordena en la Ley del Señor, «un par de tórtolas o dos palominos». \emph{Lc. 2, 22-24}. \textbf{\emph{Avemaría}}.

        \item Proseguí viendo en la visión nocturna, y he aquí que en las nubes del cielo venía como un hombre, y llegó hasta el anciano y fué llevado hasta Él. 
            \emph{Dan 7, 13}. \textbf{\emph{Avemaría}}.

        \item Y he aquí había un hombre en Jerusalén por nombre Simeón. Y era este hombre justo y temeroso de Dios que aguardaba la consolación de Israel, 
            y el Espíritu Santo estaba sobre él; \emph{Lc. 2, 25}. \textbf{\emph{Avemaría}}.

        \item Y le había sido revelado por el Espíritu Santo que no vería la muerte antes de ver al Ungido del Señor. \emph{Lc. 2, 26}. \textbf{\emph{Avemaría}}.

        \item Y vino al templo impulsado por el Espíritu Santo. Y cuando sus padres intrudujeron al niño Jesús para cumplir 
            las prescipciones usuales de la ley tocantes a El, Simeón le recibió en sus brazos y bendijo a Dios diciendo. \emph{Lc. 2, 27-28}. \textbf{\emph{Avemaría}}.

        \item Ahora dejas ir a tu siervo, Señor, según tu palabra, en paz; pues ya vieron mis ojos tu salud, que preparaste a la faz de todos los pueblos; 
            luz para iluminación de los gentiles, y gloria de tu pueblo Israel. \emph{Lc. 2, 29-32}. \textbf{\emph{Avemaría}}.

        \item Y así que cumplieron todas las cosas ordenadas en la ley del Señor, se volvieron a Galilea, a su ciudad de Nazaret. 
            El niño crecía y se robustecía llenándose de sabiduría, y la gracia de Dios estaba en Él. \emph{Lc. 2, 39-40}. \textbf{\emph{Avemaría}} y \textbf{\emph{Gloria}}
    \end{enumerate}    

    \rule{\textwidth}{0.5pt}
    ¡On Jesús mío!... y María, Madre de gracia... (\cpageref{sec:endTenPrayers})

    \subsection*{V Misterio: La pérdida y hallazgo del Niño Jesús en el Templo}

    Y no hallándole, se tornaron a Jerusalén para buscarle. Y sucedió que después de tres días le hallaron en el templo, 
    sentado en medio de los maestros, escuchándolos y haciéndoles preguntas; y se pasmaban todos los que le oían de su 
    inteligencia y de sus respuestas. Y sus padres, al verle, quedaron sorprendidos; y le dijo su madre: 
    «Hijo, ¿por qué lo niciste así con nosotros? Mira que tu padre y yo, llenos de aflicción, te andábamos buscando». 
    \begin{flushright}
        \emph{Lucas 2, 46-48}
    \end{flushright}    

    1 Paternoster, 10 Avemarías, 1 Gloria, ¡On Jesús mío!... y María, Madre de gracia... (\cpageref{sec:endTenPrayers})    

    \rule{\textwidth}{0.5pt}

    \begin{enumerate}
        \item Pero aquella misma noche tuvo Natán palabra de Yahveh: «Anda y ve a decir a David, mi siervo: Así habla Yahveh: ¿Vas a edificarme tú una casa
            para que yo habite en ella?» \emph{2 Sam. 7, 4}. \textbf{\emph{Avemaría}}.

        \item \textbf{\emph{Paternóster}}. Iban sus padres cada año a Jerusalén por la fiesta de la Pascua. 
            Y cuando fué de doce años, habiendo ellos subido, según la costumbre de la fiesta, \emph{Lc. 2, 41-42}. \textbf{\emph{Avemaría}}.

        \item Y acabados los días, al volverse ellos, quedóse el niño Jesús en Jerusalén, sin que lo advirtiesen sus padres. 
            Y creyendo ellos que Él andaría en la comitiva, caminaron una jornada; y le buscaban entre los parientes y conocidos. 
            \emph{Lc. 2, 43-44}. \textbf{\emph{Avemaría}}.

        \item Y no hallándole, se tornaron a Jerusalén para buscarlo. Y sucedió que después de tres días le hallaron en el templo, 
            sentado en medio de los maestros, escuchándoles y haciéndoles preguntas; y se pasmaban todos los que le oían de su inteligencia 
            y de sus respuestas. \emph{Lc. 2, 45-47}. \textbf{\emph{Avemaría}}.

        \item Y sus padres, al verle, quedaron sorprendidos; y le dijo su madre: «hijo, ¿por qué lo hiciste así con nosotros? 
            Mira que tu padre y yo, llenos de aflicción, te andábamos buscando». \emph{Lc. 2, 48}. \textbf{\emph{Avemaría}}.

        \item Díjoles Él: «¿pues por qué me buscabais? ¿No sabíais que había yo de estar en casa de mi padre?». Y ellos no comprendieron lo que les dijo. 
            \emph{Lc. 2, 49-50}. \textbf{\emph{Avemaría}}.

        \item Y bajó en su compañía y se fue a Nazaret, y vivía sometido a ellos. Y su madre guardaba todas estas cosas en su corazón. 
            Y Jesús progresaba en sabiduría, en estatura y en gracia delante de Dios y de los hombres. \emph{Lc. 2, 51-52}. \textbf{\emph{Avemaría}}.

        \item Jesús les respondió y dijo: Mi doctrina no es mía, sino del que me ha enviado. Quien quisiere hacer la voluntad de Él conocerá si mi doctrina
            es de Dios o si es mía. \emph{Jn. 7, 16-17}. \textbf{\emph{Avemaría}}.

        \item El que de sí mismo habla, busca su propia gloria; pero el que busca la gloria del quen le ha enviado, ése es veraz y no hay en el injusticia.
             \emph{Jn. 7, 18}. \textbf{\emph{Avemaría}}.

        \item Y la ciudad no tiene necesidad de sol ni de luna, para que alumbren en ella; porque la gloria de Dios la ilumina y su antorcha es el Cordero. 
            \emph{Ap 21,23}. \textbf{\emph{Avemaría}} y \textbf{\emph{Gloria}}
    \end{enumerate}    

    \rule{\textwidth}{0.5pt}

    ¡On Jesús mío!..., María, Madre de gracia... (\cpageref{sec:endTenPrayers}) y Oraciones finales (\cpageref{sec:final-prayer}).

    \newpage    

    \section*{Misterios Dolorosos}
    Martes, viernes y domingos de Cuaresma

    \subsection*{I Misterio: La Agonía de Nuestro Señor en el Huerto de los Olivos}
    
    Y lleva consigo a Pedro y a Santiago y a Juan, y comenzó a sentir espanto y abatimiento; y le dice: 
    «triste en gran manera está mi corazón hasta la muerte; quedad aquí y velad». Y apartándose un poco, 
    caía sobre tierra, y rogaba que, a ser posible, pasase el Él aquella hora, y decía: «Abba, Padre, todas las cosas te son posibles: 
    traspasa de mi este cáliz; más no se haga lo que yo quiero, sino lo que tú quieres».
    
    \begin{flushright}
        \emph{Marcos 14, 33-36}
    \end{flushright}
    
    1 Paternoster, 10 Avemarías, 1 Gloria, ¡On Jesús mío!... y María, Madre de gracia... (\cpageref{sec:endTenPrayers})

    \rule{\textwidth}{0.5pt}

    \begin{enumerate}
        \item \textbf{\emph{Paternóster}}. Hijo, si te acercares a servir al Señor Dios, prepara tu alma a la 
            tentación Gobierna tu corazón y muéstrate firme y no te apresures en tiempo de evasión. \emph{Eci. 2,1-2}. \textbf{\emph{Avemaría}}.

        \item Y llegan a una granja, cuyo nombre es Getsemaní, y dice a sus discípulos: «sentaos aquí mientras hago oración». 
            Y lleva consigo a Pedro y a Santiago y a Juan, y comenzó a sentir espanto y abatimiento; 
            y le dice: «triste en gran manera está mi corazón hasta la muerte; quedad aquí y velad». \emph{Mc. 14,32-34}. \textbf{\emph{Avemaría}}.

        \item Y apartándose un poco, caía sobre tierra, y rogaba que, a ser posible, pasase Él aquella hora, y decía: 
            «Abba, Padre, todas las cosas te son posibles: traspasa de mi este cáliz; más no se haga lo que yo quiero, sino lo que tú». 
            \emph{Mc. 14,35-36}. \textbf{\emph{Avemaría}}.

        \item Y viene, y los halla durmiendo, y dice a Pedro: «¡Simón!¿Duermes?¿No pudiste velar una hora? Velad y orad para que no entréis en tentación; 
            el espíritu, sí, está pronto, más la carne es flaca». \emph{Mc. 14,37-38\emph}. \textbf{\emph{Avemaría}}.

        \item Y de nuevo habiéndse retirado se puso a orar, repitiendo las mismas palabras. Y volviendo los halló otra vez durmiendo, 
            porque estaban sus ojos cargados, y no sabían qué responderle. \emph{Mc. 14,39-40}. \textbf{\emph{Avemaría}}.

        \item Y viene tercera vez y les dice: «Ya por mi, dormid y descansad... Ya está: llegó la hora; 
            he aquí que es entregado el Hijo del hombre en las manos de los pecadores. Levantaos, vamos; mirad, 
            el que me entrega está aquí cerca». \emph{Mc. 14, 41-42}. \textbf{\emph{Avemaría}}.

        \item Y luego, estando Él hablando todavía, se presenta Judas, uno de los Doce, y con él una turba con espadas y bastones, 
            de parte de los escribas y los ancianos. Y así que llegó, luego acercándose dijo: «Rabí». 
            Y le dió un fuerte beso. Ellos le echaron las manos y le sujetaron. \emph{Mc. 14,43.45-46}. \textbf{\emph{Avemaría}}. 

        \item En aquella hora dijo Jesús a las turbas: «¡cómo contra un salteador habéis salido con espadas y bastones a prenderme! 
            Cada día en el templo me sentaba para enseñar, y no me predisteis. Mas todo esto ha pasado para que se cumplan las Escrituras de los profetas». 
            Entonces los discípulo todos, abandonándole, huyeron». \emph{Mt. 26, 55-56}. \textbf{\emph{Avemaría}}.

        \item No volverá atrás la colerá de Yahveh hasta que ejecute y lleve a efecto los designios de su corazón; 
            al fin de los tiempos adquiriréis de ello inteligencia. \emph{Sal. 55, 4-5}. \textbf{\emph{Avemaría}}.

        \item {[Y Samuel exclamó:]} «¿Acaso se complace Yahveh tanto en holocaustos y sacrificios cuanto en que se obedezca su voz? 
            He aquí que la obediencia vale más que el sacrificio y la docilidad más que la grosura de carneros». 
            \emph{1 Sam 15, 22}. \textbf{\emph{Avemaría}} y \textbf{\emph{Gloria}}
    \end{enumerate}    

    \rule{\textwidth}{0.5pt}
    ¡On Jesús mío!... y María, Madre de gracia... (\cpageref{sec:endTenPrayers})

    \subsection*{II Misterio: La Flagelación de Nuestro Señor Jesucristo}

    «Yo no hallo en Él delito alguno. Es costumbre vuestra que yo os suelte un preso por la Pascua: 
    ¿queréis, pues, que os suelte al rey de los Judíos?». Gritaron, pues, de nuevo, diciendo: «No, a ése, sino a Barrabás». 
    Era este Barrabás un salteador. Entonces, pues, tomó Pilato a Jesús y le azotó.

    \begin{flushright}
        \emph{Juan 18,38-40;19,1}
    \end{flushright}    

    1 Paternoster, 10 Avemarías, 1 Gloria, ¡On Jesús mío!... y María, Madre de gracia... (\cpageref{sec:endTenPrayers})

    \rule{\textwidth}{0.5pt}

    \begin{enumerate}
        \item \textbf{\emph{Paternóster}}. Y luego al amanecer, después de celebrar consejo, los sumos sacerdotes con los ancianos y los escribas, 
            es decir, todo el sanhedrín, atando a Jesús, le llevaron de allí y le entregaron a Pilato. 
            Y le interrogó Pilato: «¿Tú eres el Rey de los judíos?». El le respondió: «Tú lo dices». \emph{Mc. 15, 1-2}. \textbf{\emph{Avemaría}}.

        \item Respondió Jesús: «Mi reino no es de este mundo. Si de este mundo fuera mi reino, mis ministros lucharían para 
            que yo no fuera entregado a los judíos. Más ahora mi reino no es de aquí». \emph{Jn. 18, 36}. \textbf{\emph{Avemaría}}.

        \item Díjole, pues, Pilato: «¿luego, rey eres tu?». Respondió Jesús: «Tú dices que yo soy rey. Yo para eso he nacido y 
            para esto he venido al mundo: para dar testimonio a favor de la verdad. 
            Todo el que es de la verdad, oye mi voz». \emph{Jn. 18, 37}. \textbf{\emph{Avemaría}}.

        \item Dícele Pilato: «¿Qué es verdad?». Dicho esto, de nuevo, salió a los judíos y les dice: 
            «Yo no hallo en Él delito alguno. Es costumbre vuestra que yo os suelte un preso por la Pascua: ¿queréis, pues, 
            que os suelte al rey de los Judíos?». Gritaron, pues, de nuevo, diciendo: «No, a ése, sino a Barrabás». 
            Era este Barrabás un salteador. \emph{Jn. 18, 38-40}. \textbf{\emph{Avemaría}}.

        \item Entonces, pues, tomó Pilato a Jesús y le azotó. \emph{Jn. 19, 1}. \textbf{\emph{Avemaría}}.
        
        \item He dado mis espaldas a los que me herían, y mis mejillas a lso que me arrancaban la barba. Y no escondí mi rostro ante las injurias y los esputos
            El Señor, Yahveh, me ha socorrido, y por eso no cedí ante la ignominia e hice mi rostro como el pedernal, sabiendo que no sería confundido. \emph{Is. 50, 6-7}. \textbf{\emph{Avemaría}}.

        \item Sube ante El como un retoño, como retoño de raíz en tierra árida. No hay en el parecer, no hay hermosura que atraiga las miradas, no hay en él belleza que agrade.
            \emph{Is. 53, 2}. \textbf{\emph{Avemaría}}.

        \item Hízose para nosotros censura de nuestros criterios: pesado es para nosotros aun el verlo; 
            pues discordante de los otros es su vida, y muy otros sus caminos. \emph{Sab 2,14-15}. \textbf{\emph{Avemaría}}.

        \item Despreciado, desecho de los hombres, varón de dolores, conocedor de todos los quebrantos, ante quie se vuelve el rostro, menospreciado, estimado en nada. 
            \emph{Is. 53, 3}. \textbf{\emph{Avemaría}}.

        \item ¡Oh vosotros los que pasáis por el camino, mirad y ved si hay dolor semejante al dolor que me hiere, 
            pues me ha afligido Yahveh en el día del ardor de su cólera! \emph{Lm 3,1-3}. \textbf{\emph{Avemaría}} y \textbf{\emph{Gloria}}
    \end{enumerate}

    \rule{\textwidth}{0.5pt}
    ¡On Jesús mío!... y María, Madre de gracia... (\cpageref{sec:endTenPrayers})

    \subsection*{III Misterio: La Coronación de espinas de Nuestro Señor Jesucristo}

    Entonces los soldados del gobernador, tomando a Jesús y conduciéndole al pretorio, 
    reunieron en torno a Él toda la cohorte. Y habiéndole quitado sus vestidos, le envolvieron en una clámide de grana, 
    y trenzando una corona de espinas, la pusieron sobre su cabeza, y una caña en su mano derecha; 
    y doblando la rodilla delante de Él, le mofaban, diciendo: «Salud, Rey de los judíos». 
    Y escupiendo en Él, tomaron la caña y le daban golpes en la cabeza.

    \begin{flushright}
        \emph{Mateo 27, 27-30}
    \end{flushright}    

    1 Paternoster, 10 Avemarías, 1 Gloria, ¡On Jesús mío!... y María, Madre de gracia... (\cpageref{sec:endTenPrayers})

    \rule{\textwidth}{0.5pt}

    \begin{enumerate}
        \item \textbf{\emph{Paternóster}}. Del vestido miserable no te burles, ni te mofes de quien se halla en día aciago; 
            porque maravillosas son las obras de Yahveh, y sus acciones son desconocidas de los hombres. \emph{Eci 11,4}. \textbf{\emph{Avemaría}}.

        \item Entonces los soldados del gobernador, tomando a Jesús y conduciéndole al pretorio, reunieron en torno a Él toda la cohorte. 
            Y habiéndole quitado sus vestidos, le envolvieron en una clámide de grana. \emph{Mt. 27, 28}. \textbf{\emph{Avemaría}}.

        \item y trenzando una corna de espinas, la pusieron sobre la cabeza, y una caña en la mano derecha. \emph{Mt. 27, 29}. \textbf{\emph{Avemaría}}.

        \item y doblando la rodilla delante de Él, le mofaban, diciendo: «Salud, Rey de los judíos». \emph{Mt. 27, 29}. \textbf{\emph{Avemaría}}.

        \item Y escupiendo en Él, tomaron la caña y le daba golpes en la cabeza. \emph{Mt 27, 30}. \textbf{\emph{Avemaría}}.

        \item Hazte más pequeño cuanto más grande eres, y ante Dios hallarás gracia. \emph{Eci 3,18}. \textbf{\emph{Avemaría}}.
        
        \item Pero fue él, ciertamente, quien tomó sobre sí nuestras enfermedades y cargó con nuestros dolores, y nosotros le tuvimos por castigado y herido por Dios y humillado. \emph{Is. 53, 4}. \textbf{\emph{Avemaría}}.

        \item Fué traspasado por causa de nuestros pecados, molido por causa de nuestra iniquidades; el castigo [precio] de nuestra paz cayó sobre Él y por sus verdugones se nos curó; \emph{Is. 53, 5}. \textbf{\emph{Avemaría}}.

        \item Tened los mismos sentimientos que tuvo Cristo Jesús, quien, existiendo en la forma de Dios, no reputó codiciable tesoro mantenerse igual a Dios, antes se anonadó, tomando
            la forma de siervo y haciéndose semejante a los hombres; y en la condición de hombre se humilló, hecho obediente hasta la muerte, y muerte en cruz \emph{Flp. 2, 5-8}. 
            \textbf{\emph{Avemaría}}.

        \item Maltratado y afligido, no abrió la boca, como cordero llevado al matadero, como oveja muda ante los trasquiladores. 
            Fue arrebatado por un juicio inicuo, sin que nadie defendiera su causa cuando era arrancado de la tierra de los vivientes y muerto por las iniquidades de su pueblo.
            \emph{Is 53, 7-8}. \textbf{\emph{Avemaría}} y \textbf{\emph{Gloria}}
    \end{enumerate}    

    \rule{\textwidth}{0.5pt}
    ¡On Jesús mío!... y María, Madre de gracia... (\cpageref{sec:endTenPrayers})

    \subsection*{IV Misterio: El Señor con la cruz a cuestas}

    Entonces, pues, se le entregó para que fuera crucificando. Se apoderaron, pues, de Jesús, 
    y llevando a cuestas su cruz, salió hacia el lugar llamado el Cráneo, que en hebreo se dice Gólgota.     

    \begin{flushright}
        \emph{Juan 19, 16-17}
    \end{flushright}    

    1 Paternoster, 10 Avemarías, 1 Gloria, ¡On Jesús mío!... y María, Madre de gracia... (\cpageref{sec:endTenPrayers})

    \rule{\textwidth}{0.5pt}

    \begin{enumerate}
        \item \textbf{\emph{Paternóster}}. Tengo por cierto que los padecimientos del tiempo presente no son nada en comparación con la gloria que ha
            de manifestarse en nosotros. \emph{Rom. 8, 18}. \textbf{\emph{Avemaría}}.

        \item Cuanto a mi, no quiera Dios que me gloríe sino en la cruz de nuestro Señor Jesucristo, por quien el mundo está crucificando
            para mi y yo para el mundo. \emph{Gál. 6, 14}. \textbf{\emph{Avemaría}}.

        \item Sed, hermanos, imitadores míos y atended a los que andan según el modelo que en nosotros tenéis, porque son muchos los que andan, de quienes frecuentemente
            os dije, y ahora con lágrimas os lo digo, que son enemigos de la cruz de Cristo. \emph{Flp. 3, 17-18}. \textbf{\emph{Avemaría}}.

        \item Si alguno quiere venir en pos de Mí, niégese a sí mismo y tome a cuestas su cruz cada día y sígame. \emph{Lc. 9, 23}. \textbf{\emph{Avemaría}}.

        \item Gritaron, pues, ellos: «Quita, quita, crucifícale». Díceles Pilato: «¿A vuestro rey voy he de crucificar?». 
            Respondieron los pontífices: «No tenemos rey, sino César». Entonces, pues, se le entregó para que fuera crucificando. 
            Se apoderaron, pues, de Jesús. \emph{Jn. 19, 15-16}. \textbf{\emph{Avemaría}}.

        \item Y, llevando a cuestas su cruz, salió hacia el lugar llamado el Cráneo, que en hebreo se dice Gólgota. \emph{Jn. 19, 17}. \textbf{\emph{Avemaría}}.

        \item Y como le hubieron sacado, echaron mano de un tal Simón de Cirene que venía del campo, le pusieron en hombros la cruz para que la llevase detrás de Jesús. 
            \emph{Lc. 23, 26}. \textbf{\emph{Avemaría}}.

        \item Pues, ¿qué mérito tendríais si, delinquiendo y castigados por ello, lo soportáis? Pero si por haber hecho el bien padecéis y lo lleváis con paciencia, 
            esto es del agrado de Dios. Pues para esto fuisteis llamados, ya que también Cristo padeció por vosotros y os dejó ejemplo para que sigáis sus pasos. Él,
            em quien no hubo pacado y cuya boca no se halló engaño, ultrajado, no replica con injurias, y atormentado, no amenazaba, sino que lo remitía al que juzga 
            con justicia \emph{1Pe. 2, 20-23}. \textbf{\emph{Avemaría}}.            

        \item Llevó nuestros pecados en su cuerpo sobre el madero, para que, muertos al pecado, viviéramos para la justici, y por sus heridas hemos sido crucificados.
            Porque «erais como ovejas descarriadas»; mas ahora os habéis vuelto al pastor y guardían de vuestras almas. \emph{1Pe. 24-25}. \textbf{\emph{Avemaría}}.

        \item Tomad mi yugo sobre vuestros, y aprended de mi, pues soy manso y humilde de Corazón, y hallaréis reposo para vuestras almas. 
            Porque mi yugo es suave, y mi carga, ligera. \emph{Mt. 11, 29-30}. \textbf{\emph{Avemaría}} y \textbf{\emph{Gloria}}
    \end{enumerate}    

    \rule{\textwidth}{0.5pt}
    ¡On Jesús mío!... y María, Madre de gracia... (\cpageref{sec:endTenPrayers})

    \subsection*{V Misterio: Crucifixión y Muerte del Redentor}

    Y era ya como la hora sexta, y se produjeron tinieblas sobre toda la tierra hasta la hora nona, 
    habiendo faltado el sol; y se rasgó por medio el velo del santuario Y clamando con voz poderosa, 
    Jesús dijo: «Padre, en tus manos encomiendo mi espíritu». Y dicho esto, expiró.

    \begin{flushright}
        \emph{Lucas 23, 44-46}
    \end{flushright}    

    1 Paternoster, 10 Avemarías, 1 Gloria, ¡On Jesús mío!... y María, Madre de gracia... (\cpageref{sec:endTenPrayers})

    \rule{\textwidth}{0.5pt}

    \begin{enumerate}
        \item \textbf{\emph{Paternóster}}. Y respondió Abraham: «Dios proveerá de cordero para el holocausto, hijo mío». \emph{Gn. 22, 8}. \textbf{\emph{Avemaría}}.
        
        \item Es que quiso quebrantarle Yahveh con padecimientos. Ofreciendo su vida en sacrificio por el pecado, tendrá posteridad y vivirá largos días, y en sus manos
            prosperará la obra de Yahveh. \emph{Is. 53, 10}. \textbf{\emph{Avemaría}}.

        \item Y cuando hubieron llegado al lugar llamado «Cráneo», allí crucificaron a Él y a los malhechores, uno a la derecha y el otro a la izquierda. Y Jesús decía: 7
            «Padre, perdónalos, porque no saben lo que hacen». Y al repartir sus vestidos, echaron suertes. \emph{Lc. 23, 33-34}. \textbf{\emph{Avemaría}}.

        \item Había también por encima de Él una inscripción escrita en letras griegas, latinas y hebreas: Este es el Rey de los judios. \emph{Lc. 23, 38}. \textbf{\emph{Avemaría}}.

        \item Uno de los malhechores crucificados le insultaba, diciendo: ¿No eres tú el Mesías? Sálvate, pues, a ti mismo y a nosotros. Pero el otro,
            tomando la palabra, le respondió, diciendo: ¿Ni tú, que estás sufriendo el mismo suplicio, temes a Dios? En nosotros se cumple la justicia, pues
            recibimos el justo castigo de nuestras obras; pero éste nada malo ha hecho. Y decía: Jesús, acuérdate de mi cuando llegues a tu reino. El le dijo: En verdad te digo, 
            hoy serás conmigo en el paraíso. \emph{Lc. 23, 39-43}. \textbf{\emph{Avemaría}}.

        \item Jesús, pues, viendo a la Madre, y junto a ella al discípulo a quien amaba, dice a tu Madre: «Mujer, he ahí a tu hijo». 
            Luego dice al discípulo: «He aquí a tu Madre». Y desde aquella hora la tomó el discípulo en su compañía. \emph{Jn. 19, 26-27}. \textbf{\emph{Avemaría}}.

        \item Y era ya como la hora sexta, y se produjeron tinieblas sobre toda la tierra hasta la hora nona, habiendo faltado el sol; 
            y se rasgó por medio el velo del santuario.Y clamando con voz poderosa, Jesús dijo: «Padre, en tus manos encomiendo mis espíritu». Y dicho esto, expiró. 
            \emph{Lc. 23, 44-46}. \textbf{\emph{Avemaría}}.

        \item Mas viendo a Jesús, cuando vinieron, como le vieron ya muerto, no le quebrantaron las piernas, sino que uno de los soldados 
            con una lanza le traspasó el costado, y salió al punto sangre y agua. Pues acontecieron estas cosas para que se cumpliese las
            Escrituras: «No le será quebrantado hueso alguno»\footnote{Ex. 12, 46; Núm. 9, 12}. Y asimismo otra Escritura dice: «Verán al que
            traspasaron»\footnote{Zac. 12, 10}. \emph{Jn 19, 33-34.36}. \textbf{\emph{Avemaría}}.

        \item Después de esto, José de Arimatea, que era discípulo de Jesús, si bien oculto por miedo a los judíos, rogó a Pilato le permitiese
            quitar el cuerpo de Jesús. Y se lo permitió Pilato. Vino, pues, y quitó su cuerpo. Vino también Nicodemo, el que la primera vez había
            venido a Él de noche, trayendo una mixtura de mirra y áloe, como cien libras.
            \emph{Jn. 19, 38-39}. \textbf{\emph{Avemaría}}.            

        \item Tomaron pues, el cuerpo de Jesús y lo fajaron con bandas y aromas, según es costumbre sepultar entre los judíos. Había cerca
            del sitio donde fue crucificado un huerto, y en el huerto un sepulcro nuevo, en el cual nadie aún había sido depositado.
            Allí, a causa de la Parasceve de los judíos, por estar cerca del monumento, pusieron a Jesús. 
            \emph{Jn. 19, 40-42}. \textbf{\emph{Avemaría}} y \textbf{\emph{Gloria}}
    \end{enumerate}    

    \rule{\textwidth}{0.5pt}
    ¡On Jesús mío!..., María, Madre de gracia... (\cpageref{sec:endTenPrayers}) y Oraciones finales (\cpageref{sec:final-prayer}).

    \newpage

    \section*{Misterios Gloriosos.}
    Miércoles, sábados y domingos de Pascua y Pentecostés.

    \subsection*{I Misterio: La Resurrección del Señor}

    Tomando la palabra el ángel, dijo a las mujeres: «no tengáis miedo vosotras, que ya sé que buscáis a Jesús el crucificado; 
    no está aquí; resucitó, como dijo. Venid, ved el lugar donde estuvo puesto. Y marchando a toda prisa, 
    decid a sus discípulos que resucitó de entre los muertos, y he aquí que se os adelanta en ir a Galilea: allí le veréis. 
    Conque os lo tengo dicho». Y partiendo a toda prisa del monumento, con temor y grande gozo corrieron a dar la nueva a sus discípulos. 

    \begin{flushright}
        \emph{Mateo 28, 5-8}
    \end{flushright}    

    1 Paternoster, 10 Avemarías, 1 Gloria, ¡On Jesús mío!... y María, Madre de gracia... (\cpageref{sec:endTenPrayers})

    \rule{\textwidth}{0.5pt}

    \begin{enumerate}
        \item \textbf{\emph{Paternóster}}. Alzase Dios, y al presentarse ante ellos desbándese sus enemigos, y huyen aquellos que le odiaron. \emph{Sal 68, 2}. \textbf{\emph{Avemaría}}.
        
        \item Y si Cristo no resucitó, vana es nuestra predicación. Vana es nuestra fe. \emph{1Cor. 15, 14}. \textbf{\emph{Avemaría}}.

        \item Bienaventurados los que están afligidos, porque ellos serán consolados. \emph{Mt 5,5}. \textbf{\emph{Avemaría}}.

        \item «Pues así también vosotros, ahora cierto tenéis congoja; mas otra vez os veré, y se gozará vuestro corazon, y vuestro gozo nadie os lo quita». \emph{Jn. 16, 22}. \textbf{\emph{Avemaría}}.

        \item De pronto se produjo un gran temblor de tierra, pues un ángel del Señor, bajado de cielo y acercándose, hizo rodar de su sitio la losa, y se sentó sobre ella. 
            Era su aspecto como relámpago, y su vestidura blanca como la nieve. Tomando la palabra el ángel, dijo a las mujeres: «no tengáis miedo vosotras, que ya sé que buscáis 
            a Jesús el crucificado; no está aquí; resucitó, como dijo. Venid, ved el lugar donde estuvo puesto». \emph{Mt. 28, 2.5-6}. \textbf{\emph{Avemaría}}.

        \item «Y marchando a toda prisa, decid a sus discípulos que resucitó de entre los muertos, y he aquí que se os adelanta en ir a Galilea: allí le veréis. Conque os lo tengo dicho». 
            Y partiendo a toda prisa del monumento, con temor y grande gozo corrieron a dar la nueva a sus discípulos. \emph{Mt. 28, 7-8}. \textbf{\emph{Avemaría}}.

        \item Pues así tambíén vosotros, ahora cierto tenéis congoja; mas otra vez os veré, y se gozará vuestro corazón, y vuestro gozo nadie os lo quitará. \emph{Jn. 16, 22}. \textbf{\emph{Avemaría}}.

        \item Si hemos muerto con Cristo, también viviremos con Él; pues sabemos que Cristo, resucitado de entre los muertos, ya no muere, la muerte no tiene ya dominio sobre Él. Porque
            muriendo, murió al pecado una vez para siempre; pero viviendo, vive para Dios. \emph{Rom. 6, }. \textbf{\emph{Avemaría}}.
        
        \item Pues de gracia habéis sido salvados por la fe, y esto no os viene de vosotros, es don de Dios; no viene de las obras, para que nadie se gloríe; que hechura suya somos,
            creados en Cristo Jesús, para hacer buenas obras, que Dios de antemano preparó, para que en ella anduviésemos \emph{Ef. 2, 8-10}. \textbf{\emph{Avemaría}}.        

        \item Fuése María Magdalena a dar la nueva a los discípulos: «he visto al Señor, y me ha dicho esto y esto». Siendo, pues, tarde aquel día, primero de la semana, y estando cerradas, 
            por miedo a los judíos, las puertas de la casa donde estaban los discípulos, vino Jesús y se presentó en medio de ellos y les dice: «la paz sea con vosotros». Y en diciendo esto, 
            les mostró las manos y el costado. Se gozaron, pues, los discípulos al ver al Señor. \emph{Jn. 20, 18-20}. \textbf{\emph{Avemaría}} y \textbf{\emph{Gloria}}.
    \end{enumerate}    

    \rule{\textwidth}{0.5pt}
    ¡On Jesús mío!... y María, Madre de gracia... (\cpageref{sec:endTenPrayers})

    \subsection*{II Misterio: Las Ascensión Jesucristo a los cielos}

    [Y les dijo:] «Id al mundo entero y predicad el Evangelio a toda la creación. El que creyere y fuere bautizado, 
    se salvará, mas el que no creyere, será condenado. Con esto el Señor Jesús, despues de hablarles, 
    fue elevado al cielo y se sentó a la diestra de Dios. Y ellos, partiéndose de allí, 
    predicaron por todas partes, cooperando el Señor y confirmando la palabra con las señales que la acompañaban. 

    \begin{flushright}
        \emph{Marcos 16, 15-16.19-20}
    \end{flushright}    

    1 Paternoster, 10 Avemarías, 1 Gloria, ¡On Jesús mío!... y María, Madre de gracia... (\cpageref{sec:endTenPrayers})

    \rule{\textwidth}{0.5pt}

    \begin{enumerate}
        \item \textbf{\emph{Paternóster}}. Que no entró Cristo en un santuario hecho por mano de hombre, figura del verdadero, sino en el mismo cielo, para compadecer ahora en
            la presencia de Dios a favor nuestro. \emph{Heb. 9, 24}. \textbf{\emph{Avemaría}}.

        \item Y acercándose Jesús, les habló diciendo: «Me fué dada toda potestad en el cielo y sobre la tierra». \emph{Mt. 28,18}. \textbf{\emph{Avemaría}}.

        \item {[Y les dijo:]} «Id al mundo entero y predicad el Evangelio a toda la creación. El que creyere y fuere bautizado, 
            se salvará, mas el que no creyere, será condenado». \emph{Mc. 16, 15-16}. \textbf{\emph{Avemaría}}.

        \item «Y a los que hubieren creído les acompañarán estas señales: en mi nombre lanzarán demonios, hablarán lenguas nuevas, en sus manos tomarán serpientes, 
            y si bebieren ponzoña mortífera, no les dañará; pondrán sus manos sobre los enfermos, y se hallarán bien». \emph{Mc. 16, 17}. \textbf{\emph{Avemaría}}.

        \item «Enseñándoles a guardar todas cuantas cosas os ordené. Y sabed que estoy con vosotros todos los días hasta la consumación de los siglos». 
            \emph{Mt. 28, 20}. \textbf{\emph{Avemaría}}.

        \item Con esto el Señor Jesús, despues de hablarles, fue elevado al cielo y se sentó a la diestra de Dios. \emph{Mc. 16, 19}. \textbf{\emph{Avemaría}}.

        \item «Varones galileos, ¿qué hacéis ahí plantados mirando fíjamente al cielo?; Este mismo Jesucristo, que ha sido quitado de entre vosotros 
            para ser elevado al cielo, así vendrá, de la misma manera que le habéis contemplado irse al cielo». \emph{Hch. 1, 11}. \textbf{\emph{Avemaría}}.

        \item Y ellos, partiéndose de allí, predicaron por todas partes, cooperando el Señor y confirmando la palabra con las señales que la acompañaban. 
            \emph{Mc. 16,20}. \textbf{\emph{Avemaría}}.

        \item Teniendo, pues, un Potífice grande, que ha penetrado los cielos, Jesús, el Hijo de Dios, mantengamos firme la fe que profesamos. Lleguémonos, pues, 
            con segura confianza al trono de la gracia, para que alcancemos misericordia y hallemos gracia en orden a ser socorridos en el tiempo oportuno. 
            \emph{Heb 4, 14.16}. \textbf{\emph{Avemaría}}.

        \item {[Jesús dijo:]} Yo soy el camino, la verdad y la vida; nadie viene al Padre sino por mi. \emph{Heb 4,16}. \textbf{\emph{Avemaría}} y \textbf{\emph{Gloria}}.
    \end{enumerate}    

    \rule{\textwidth}{0.5pt}
    ¡On Jesús mío!... y María, Madre de gracia... (\cpageref{sec:endTenPrayers})

    \subsection*{III Misterio glorioso: La Venida del Espíritu Santo sobre los Apóstoles}

    Y al cumplirse el día de Pentecostés, estaban todos juntos en el mismo lugar. Y se produjo de súbito desde el cielo un estruendo como de viento que soplaba vehementemente, 
    y llenó toda la casa donde se hallaban sentados. Y vieron aparecer lenguas como de fuego, que, repartiéndose, se posaban sobre cada uno de ellos. 

    \begin{flushright}
        \emph{Hechos 2, 1-4}
    \end{flushright}    

    1 Paternoster, 10 Avemarías, 1 Gloria, ¡On Jesús mío!... y María, Madre de gracia... (\cpageref{sec:endTenPrayers})

    \rule{\textwidth}{0.5pt}

    \begin{enumerate}
        \item \textbf{\emph{Paternóster}}. Y al cumplirse el día de Pentecostés, estaban todos juntos en el mismo lugar. Y se produjo de súbito desde el cielo un estruendo como de viento 
            que soplaba vehemente, y llenó toda la casa donde se hallaban sentados. \emph{Hch. 2, 1-2}. \textbf{\emph{Avemaría}}.

        \item Y vieron aparecer lenguas como de fuego, que, repartiéndose, se posaban sobre cada uno de ellos. Y se llenaron todos del Espíritu Santo, y comenzaron a hablar en lenguas diferentes, 
            según que el Espíritu Santo les movía a expresarse. \emph{Hch. 2, 3-4}. \textbf{\emph{Avemaría}}.

        \item Hallábanse en Jerusalén judíos allí domiciliados, hombres religiosos de toda nación de las que están debajo del cielo. \emph{Hch. 2, 5}. \textbf{\emph{Avemaría}}.

        \item Puesto de pie Pedro, acompañado de los Once, alzó en voz y les habló en estos términos: «Arrepentíos, dice, y bautícese cada uno de vosotros en el nombre del 
            Jesucristo para remisión de vuestros pecados, y recibiréis el don del Espíritu Santo. Porque para vosotros es esta promesa y para vuestros hijos y para todos los de lejos
            cuantos llamare a sí el Señor, Dios nuestros». \emph{Hch. 2, 14.38-39}. \textbf{\emph{Avemaría}}.

        \item Ellos, pues, acogieron su palabra, fueron bautizados; y fueron agragados en aquel día como unas tres mil almas. \emph{Hch. 2, 41}. \textbf{\emph{Avemaría}}.

        \item Mas la fructificación del Espíritu es: caridad, gozo, paz, longanimidad, benignidad, bondad, fe, mansedumbre, continencia; frente a tales cosas no tiene objeto la ley. 
            \emph{Gál. 5, 22-23}. \textbf{\emph{Avemaría}}.

        \item Mas la sabiduría que viene de arriba primeramente es casta, luego pacífica, condescendiente, que se allana a razones, llena de misericordia y de frutos buenos, 
            no amiga de criticar, no solapada. \emph{Sant. 3, 17}. \textbf{\emph{Avemaría}}.

        \item Porque no recibisteis espíritu de esclavitud para reincidir de nuevo en el temor; antes recibisteis Espíritu de filiación adoptiva, con el cual clamamos: ¡Abba!¡Padre! 
            El Espíritu mismo testifica a una que somos hijos de Dios. \emph{Rom 8, 15-16}. \textbf{\emph{Avemaría}}.

        \item Y, asimismo, también el Espíritu acude en socorro de nuestras flaquezas. Pues qué hemos de orar, según conviene, no lo sabemos; mas el Espíritu mismo interviene a 
            favor nuestro con gemidos inefables. \emph{Rom. 8, 26}. \textbf{\emph{Avemaría}}.

        \item Mas el Paráclito, el Espiritu Santo, que enviará el Padre en mi nombre, Él os enseñará todas las cosas y os recordará todas las cosas que os dije yo. 
            \emph{Jn. 14, 26}. \textbf{\emph{Avemaría}} y \textbf{\emph{Gloria}}.
    \end{enumerate}    

    \rule{\textwidth}{0.5pt}
    ¡On Jesús mío!... y María, Madre de gracia... (\cpageref{sec:endTenPrayers})

    \subsection*{IV Misterio: La Asunción de María Santísima a los cielos}

    Y se abrió el templo de Dios, que está en el cielo, y fué vista el arca de la alianza en el templo, 
    y se produjeron relámpagos, y voces, y truenos, y temblor de tierra, y fuerte granizada. 

    \begin{flushright}
        \emph{Apocalipsis 11, 19}
    \end{flushright}    

    1 Paternoster, 10 Avemarías, 1 Gloria, ¡On Jesús mío!... y María, Madre de gracia... (\cpageref{sec:endTenPrayers})

    \rule{\textwidth}{0.5pt}

    \begin{enumerate}
        \item \textbf{\emph{Paternóster}}. Bendita tú, hija, ante el Dios Altísimo sobre todas las mujeres de la tierra,
            y bendito el Señor Dios, que crió los cielos y la tierra, que enderezó tus pasos para quebrantar la cabeza del jefe 
            de nuestros enemigos. Pues no se apartará etérnamente tu esperanza del corazón de los hombres, que recordarán la fortaleza de Dios. 
            \emph{Jdt. 13, 18-19}. \textbf{\emph{Avemaría}}.

        \item Y esto haga contigo Dios para eterno encubrimiento, que te visite con sus bienes; por cuanto no perdonate a tu vida, 
            lastimada por la humillación de nuestro linaje, antes acudiste en socorro de nuestro abatimiento, 
            caminando derechamente en el acatamiento de Dios. \emph{Jdt. 13, 20}. \textbf{\emph{Avemaría}}.

        \item ¿Qué es eso que sube del desierto como columna de humo sahumado de mirra e incienso y de toda clase de aromas del mercader? 
            He aquí la litera de Salomón. \emph{Cant 3, 6-7a}. \textbf{\emph{Avemaría}}.

        \item Y se abrió el templo de Dios, que está en el cielo, y fué vista el arca de la alianza en el templo, y se produjeron relámpagos, y voces, y truenos, 
            y temblor de tierra, y fuerte granizada. \emph{Ap. 11, 19}. \textbf{\emph{Avemaría}}.

        \item Y entrando a ella, bendijéronla todos a una voz y la dijeron: tú eres enaltecimiento de Jerusalén, tú gloria grande de Isarel, tú grande honor de nuestro linaje. 
            \emph{Jdt 15, 9}. \textbf{\emph{Avemaría}}.

        \item Hiciste todo esto por tu mano, acarreaste bienes a Israel, y se agradó Dios en ellos. Bendita seas en el acatamiento del Señor omnipotente para tiempo sin fin. 
            \emph{Jdt 15, 10}. \textbf{\emph{Avemaría}}.

        \item Del rey la hija toda hermosa entra; vestidos áureos su adorno son; al rey la llevan con recamados; amigas vírgenes van de ella en pos. 
            Entre alborozos y gritos de júbilo van penetrando en la real mansión. \emph{Sal 45, 14-16}. \textbf{\emph{Avemaría}}.

        \item Antes de los siglos, desde el principio me creó y hasta la eternidad no cesaré. He arraigado en pueblo ilustre, en la porción del Señor, heredad suya. 
            \emph{Eci 24, 14.16}. \textbf{\emph{Avemaría}}.

        \item Yo soy la madre de la hermosa dilección, y del temor, y del conocimiento y santa esperanza. \emph{Eci 24, 24}. \textbf{\emph{Avemaría}}.

        \item Entonad a Yahveh cántico nuevo, que portentos ha obrado. Su diestra le ha traído la victoria y aquel su brazo santo. \emph{Sal. 98, 1}. 
            \textbf{\emph{Avemaría}} y \textbf{\emph{Gloria}}
    \end{enumerate}    

    \rule{\textwidth}{0.5pt}
    ¡On Jesús mío!... y María, Madre de gracia... (\cpageref{sec:endTenPrayers})

    \subsection*{V Misterio: La Coronación de la Santísima Virgen María}

    Y una gran señal fué vista en el cielo: una Mujer vestida del sol, y la luna debajo de sus pies, y sobre su cabeza una corona de doce estrellas. 

    \begin{flushright}
        \emph{Apocalipsis 12, 1}
    \end{flushright}    

    1 Paternoster, 10 Avemarías, 1 Gloria, ¡On Jesús mío!... y María, Madre de gracia... (\cpageref{sec:endTenPrayers})

    \rule{\textwidth}{0.5pt}

    \begin{enumerate}
        \item \textbf{\emph{Paternóster}}. Pongo perpétua enemistad entre ti y el mujer. Y entre tu linaje y el suyo. Este te aplastará la cabeza.
            Y tú le acecharas el calcañal. \emph{Gen. 3, 15}. \textbf{\emph{Avemaría}}.
    
        \item Una es mi paloma, mi pura; única es ella de su madre, la preferida de la que la dió a luz. Viéronla las doncellas, 
            y la felicitaron; las reinas y las concubinas, y exclamaron loándola. \emph{Cant. 6, 9}. \textbf{\emph{Avemaría}}.

        \item Yo he salido de la boca del Altísimo. En las alturas he armado mi tienda y mi trono está en columna de nube. El círculo celeste he rodeado sola y en lo profundo 
            del abismo me he paseado; y en la tierra toda, y en todo pueblo y nación he imperado. En mi está toda la gracia del camino y de la verdad, en mi toda esperanza de 
            la vida y de la virtud. Venid a mí cuantos me deseáis y saciaos de mis frutos. Porque recordarme es más dulce que la miel, y poseerme más rico que el panal de miel. 
            \emph{Eci. 24, 5-11.26-27}. \textbf{\emph{Avemaría}}.

        \item Y dijo María: Engrandece mi alma al Señor, y se regocijó mi espíritu em Dios, mi Salvador; porque puso sus ojos en la bajeza de su esclava. Pues he aquí que desde Ahora
            me llamarán dichosa todas las generaciones; porque hizo en mi favor grandes cosas el Poderoso, y cuyo nombre es «Santo»; y su misericordia por generaciones y generaciones
            para con aquellos que le temen. \emph{Lc. 1, 46-50}. \textbf{\emph{Avemaría}}.

        \item ¿Quién es esa que aparece resplandeciente como la aurora, hermosa cual luna, deslumbradora como el sol, imponente como batallones? 
            \emph{Cant 6,10}. \textbf{\emph{Avemaría}}.

        \item Y una gran señal fué vista en el cielo: una Mujer vestida del sol, y la luna debajo de sus pies, y sobre su cabeza una corona de doce estrellas. 
            \emph{Ap 12, 1}. \textbf{\emph{Avemaría}}.

        \item Quién me obedece no se avergonzará y los que obran por no pecarán. Los que me esclarecen tendrán vida eterna. 
            \emph{Eci. 24, 30-31}. \textbf{\emph{Avemaría}}.

        \item Parió un varón que ha de apacentar a todas las naciones con vara de hierro, pero el Hijo fue arrebatado a Dios y a su trono. La mujer huyó
            al desierto, en donde tenía un lugar preparado por Dios [, para que allí la alimentasen durante mil dosciento sesenta días]. \emph{Ap 12, 5}. \textbf{\emph{Avemaría}}.

        \item Ahora, pues, hijos míos, oídme; y felices quienes guardan mis caminos. Escuchad la corrección y sed sabios, y no la rechacéis.  
            Feliz el hombre que me escucha, velando a mis puertas cada día, guardando las jambas de mis entradas. Pues quien me halla, ha hallado la vida y alcanza el favor 
            de Yahveh. Mas quién peca contra mi, se perjudica a si mismo, y cuantos me odian aman la muerte. \emph{Prv. 8, 32-35}. \textbf{\emph{Avemaría}}.

        \item Tiene Él escrito en su vestido y en su manto Rey de reyes y Señor de los que dominan. Está la Reina a su derecha, adornada con oro finísimo. 
            \emph{Ap. 18, 16. Sal. 44, 10}. \textbf{\emph{Avemaría}} y \textbf{\emph{Gloria}}
    \end{enumerate}    

    \rule{\textwidth}{0.5pt}
    ¡On Jesús mío!..., María, Madre de gracia... (\cpageref{sec:endTenPrayers}) y Oraciones finales (\cpageref{sec:final-prayer}).

    \section*{Oraciones finales y letanías lauretanas} 
    \label{sec:final-prayer}

    Gracias os damos, soberana Princesa, por los favores que todos los días recibimos de vuestra benéfica mano; dignaos, Señora, tenernos ahora 
    y siempre bajo vuestra protección y amparo y para más obligados, os saludamos con una Salve (ir a \cpageref{sec:salve})

    \subsection*{Letanías lauretanas}
    \begin{longtable} { p{0.5\textwidth} p{0.5\textwidth} }
        Señor, tened piedad de nosptros & Kýrie, eléison.\\
        Cristo, tened piedad de nosptros & Christe, eléison.\\
        Señor, tened piedad de nosptros. & Kýrie, eléison.\\
        Cristo, oídnos. & Christe, audi nos.\\
        Cristo, escuchadnos. & Christe, exáudi nos.\\
        Dios, Padre celestial, \emph{tened misericordia de nosotros}. & Pater de cælis, Deus, \emph{miserére nobis}.\\
        Dios Hijo, Redentor del mundo. & Fili, Redémptor mundi, Deus.\\
        Dios Espíritu Santo. & Spíritus Sancte, Deus.\\
        Trinidad Santa, un solo Dios. & Sancta Trínitas, unus Deus\\
        Santa María, \emph{rogad por nosotros}. & Sancta Maria, \emph{ora pro nobis}.\\
        Santa Madre de Dios. & Sancta Dei Génetrix.\\
        Santa Virgen de las vírgenes. & Sancta Virgo vírginum.\\
        Madre de Cristo. & Mater Christi.\\
        Madre de la divina gracia. & Mater divínæ gratiæ.\\
        Madre purísima. & Mater puríssima.\\
        Madre castísima. & Mater castíssima.\\
        Madre virginal. & Mater invioláta.\\
        Madre sin corrupción. & Mater intemeráta.\\
        Madre inmaculada. & Mater immaculáta.\\
        Madre amable. & Mater amábilis.\\
        Madre admirable. & Mater admirábilis.\\
        Madre del Buen Consejo. & Mater boni Consílii.\\
        Madre del Creador. & Mater Creatóris.\\
        Madre del Salvador. & Mater Salvatóris.\\
        Virgen pru­den­tísima. & Virgo pru­den­tíssima.\\
        Virgen digna de veneración. & Virgo veneránda.\\
        Virgen digna de alabanza. & Virgo prædicánda.\\
        Virgen poderosa. & Virgo potens.\\
        Virgen clemente. & Virgo clemens.\\
        Virgen fiel. & Virgo fidélis.\\
        Espejo de justicia. & Spéculum iustítiæ.\\
        Trono de sabiduría. & Sedes Sapiéntiæ.\\
        Causa de nuestra alegría. & Causa nostræ lætítiæ.\\
        Vaso espiritual. & Vas spirituále.\\
        Vaso digno de honor. & Vas honorábile.\\
        Vaso insigne de devoción. & Vas insigne devotiónis.\\
        Rosa mística. & Rosa mýstica.\\
        Torre de David. & Turris Davídica.\\
        Torre de marfil. & Turris ebúrnea.\\
        Casa de oro. & Domus áurea.\\
        Arca de la alianza. & Fœderis arca.\\
        Puerta del cielo. & Iánua cæli.\\
        Estrella de la mañana. & Stella matutina.\\
        Salud de los enfermos. & Salus infirmórum.\\
        Refugio de los pecadores. & Refugium peccatórum.\\
        Consuelo de los afligidos. & Consolátrix af­flic­tórum.\\
        Auxilio de los cristianos. & Auxílium chris­tia­nórum.\\
        Reina de los Ángeles. & Regina Angelórum.\\
        Reina de los Patriarcas. & Regina Pa­triar­chárum.\\
        Reina de los Profetas. & Regina Pro­phe­tárum.\\
        Reina de los Apóstoles. & Regina Apos­to­lórum.\\
        Reina de los Mártires. & Regina Mártyrum.\\
        Reina de los Confesores. & Regina Con­fe­ssórum.\\
        Reina de las Vírgenes. & Regina Vírginum.\\
        Reina de todos los Santos. & Regina Sanctórum ómnium.\\
        Reina concebida sin pecado original. & Regina sine labe originali concépta.\\
        Reina elevada al cielo. & Regina in cælum assumpta.\\
        Reina del Santísimo Rosario. & Regina sa­cra­tíssimi Rosárii.\\
        Reina de la paz. & Regina pacis.\\
        Cordero de Dios, que quitas los pecados del mundo, \emph{perdonadnos, Señor.} &
        Agnus Dei, qui tollis peccáta mundi, \emph{parce nobis, Dómine}.\\
        Cordero de Dios, que quitas los pecados del mundo, \emph{escuchadnos, Señor.} &
        Agnus Dei, qui tollis peccáta mundi, \emph{exáudi nos, Dómine}.\\
        Cordero de Dios, que quitas los pecados del mundo, \emph{tened piedad de nosotros} &
        Agnus Dei, qui tollis peccáta mundi, \emph{miserére nobis}.\\\\
        Bajo tu amparo nos acogemos, Santa Madre de Dios: no desprecies las súplicas que te dirigimos en nuestras necesidades, 
        antes bien, líbranos siempre de todos los peligros, Virgen gloriosa y bendita. &
        Sub tuum præsídium confúgimus, Sancta Dei Génetrix, nostras de­pre­ca­tiónes ne despícias in ne­ces­si­tátibus; 
        sed a perículis cunctis líbera nos semper, Virgo gloriósa et benedícta.\\\\
        Ruega por nosotros, Santa Madre de Dios.--- Para que seamos dignos de alcanzar las promesas de nuestro Señor Jesucristo. &
        Ora pro nobis, Sancta Dei Génetrix.--- Ut digni efficiámur pro­mi­ssiónibus Christi.
        
    \end{longtable}

    \chapter*{Ángelus y Regina Cœli}

    \section*{Ángelus}

    \begin{longtable} { p{0.5\textwidth} p{0.5\textwidth} }
        V. El Ángel del Señor anunció a María. & V. Angelus Dómini nuntiávit Maríae.\\
        R. Y ella concibió por obra y gracia del Espíritu Santo. & R. Et concépit de Spíritu Sancto.
    \end{longtable}

    \begin{center}
        \textbf{Avemaría}
    \end{center}

    \begin{longtable} { p{0.5\textwidth} p{0.5\textwidth} }
        V. He aquí la excalva del Señor. & V. Ecce Amcilla Dómini.\\
        R. Hágase en mi según tu palabra. & R. Fiat mihi secúndum verbum tuum.
    \end{longtable}

    \begin{center}
        \textbf{Avemaría}
    \end{center}

    \begin{longtable} { p{0.5\textwidth} p{0.5\textwidth} }
        V. Y el Verbo se hizo carne. & V. Et Verbum caro factum est.\\
        R. Y habitó entre nosotros. & R. Et habitávit in nobis.
    \end{longtable}

    \begin{center}
        \textbf{Avemaría}
    \end{center}

    \begin{longtable} { p{0.5\textwidth} p{0.5\textwidth} }
        V. Rogad por nosotros, Santa Madre de Dios. & V. Ora pro nobis, Sancta Dei Genetrix.\\
        R. Para que seamos dignos de alcanzar las promesas de Cristo. & R. Ut digni efficiámur promissiónibus Christi. Amen.\\\\
        \textbf{Oración} -- Os rogamos, Señor, que infundáis vuestra gracia en nuestras almas para que, 
        habiendo conocido la Encarnación de vuestro Hijo Jesucristo por el Ángel que la anunció, 
        seamos llevados a la gloria de la resurrección, por los méritos de su pasión y cruz santísima. 
        Por el mismo Jesucristo nuestro Señor. Amén. & 
        \textbf{Orémus} -- Grátiam tuam quáesumus. Dómine, méntibus nostris infúnde ut qui Angelo nuntiáte. 
        Christi Filii tui incarnatiónem cognóvimus, per passiónem ejus et crucem ad resurrectiónis glóriam perducámur. 
        Per eúmdem Christum Dóminum nostrum. Amen.
    \end{longtable}

    \section*{Regina Cœli Lœtáre}
    \begin{longtable} { p{0.5\textwidth} p{0.5\textwidth} }

        Reina del Cielo, alegraos, aleluya. & Regina caeli laetáre, allelúia.\\
        Porque Aquel que merecisteis llevar en vuestro seno, aleluya. & Quia quem meruisti portáre, allelúia.\\
        Resucitó, como Él predijo, aleluya. & Resurréxit sicut dixit, allelúia.\\
        Rogad por nosotros a Dios, aleluya. & Ora pro nobis Deum, allelúia.\\
        V. Alegraos y regocijaos, Virgen María, aleluya. & V. Gaudate el laetáre, Virgo María, allelúia.\\
        R. Porque resucitó verdaderamente el Señor, aleluya. & R. Quia surréxit Dóminus vere, allelúia.\\\\
        \textbf{Oración} -- Oh Dios, que, por la resurreción de vuestro Hijo y Señor nuestro Jesucristo, 
        os habéis dignado alegrar el mundo: concedednos por medio de su divina Madre, la Virgen Santísima, 
        que merezcamos obtener los goces de la vida eterna. Por el mismo Cristo, Señor nuestro. Amén. & 
        \textbf{Orémus} -- Deus, qui per resurrectiónem Filii tui Dómini nostri Jesu Christi, 
        mundum laetificáre dignátus es: praesta, quáesumus, ut per ejus Genetricem Vírginem Maríam, 
        perpétuae capiámus gáudia vitae. Per eúmdem Christum Dóminum nostrum. Amen.
    \end{longtable}

\end{document}