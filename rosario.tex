\documentclass[a4paper,11pt,sans]{article}

\usepackage[utf8]{inputenc}
\usepackage[spanish]{babel}
\usepackage{multicol}

\begin{document}
  \section*{\hfil Misterios Gozosos \hfil}
    \subsection*{\hfil La Anunciación de la Santísima Virgen María \hfil}
      
      %\begin{multicols}{2}
      %  \textsuperscript{26}En el sexto mes fué enviado el Ángel Gabriel de parte de Dios a una ciudad de Galilea, llamada Nazaret, \textsuperscript{27}a una doncella desposada con un varón llamada José, de la familia de 
      %  David, y el nombre de la doncella era María.
      %\columnbreak
          
      %\end{multicols}

      % PRIMER
      \textsuperscript{26}En el sexto mes fué enviado el Ángel Gabriel de parte de Dios a una ciudad de Galilea, llamada Nazaret, \textsuperscript{27}a una 
      doncella desposada con un varón llamada José, de la familia de David, y el nombre de la doncella era María.
      
      \begin{center}
        Lc. 1,26- 27           
      \end{center}
      
      %SEGUNDO
      \textsuperscript{28}Y habiendo entrado a ella, dijo: "Dios te salve, llena de gracia, el Señor es contigo, bendita tú entre las mujeres 
      \textsuperscript{42}y levanto la voz con gran clamor y dijo: "Bendita tú entre las mujeres y bendito el fruto de tu vientre"
      
      \begin{center}
        Lc. 1, 28, 42      
      \end{center}
      
      %TERCERO
      \textsuperscript{29}Ella, al oír estas palabras, se turbó, y discurría qué podía ser esta salutación

      \begin{center}
        Lc. 1, 29         
      \end{center}
      
      %CUARTO
      \textsuperscript{30}Y le dijo el ángel: "No temas, María, pues hallaste gracia a los ojos de Dios".

      \begin{center}
        Lc. 1, 30         
      \end{center}
      
      %QUINTO
      \textsuperscript{31}``He aquí que concebirás en tu seno y darás a luz un Hijo, a quién darás por nombre Jesús".

      \begin{center}
        Lc. 1, 31      
      \end{center}

      %SEXTO
      \textsuperscript{32}``Este será grande, y será llamado Hijo del Altísimo, y le dará el Señor Dios el trono de David su padre, \textsuperscript{33}y reinará
      sobre la casa de Jacob etérnamente, y su reinado no tendrá fin"

      \begin{center}
        Lc. 1, 32-33        
      \end{center}
      
      %SEPTIMO
      \textsuperscript{34}DIjo María al ángel: "¿cómo será eso, pues, no conozco varón?".

      \begin{center}
        Lc. 1, 34         
      \end{center}
      
      %OCTAVO
      \textsuperscript{35}Y respondiendo el ángel, le dijo: ``el Espíritu Santo descenderá sobre ti, y el poder del Altísimo te cobijará con su sombra";

      \begin{center}
        Lc. 1, 35       
      \end{center}
      
      %NOVENO
      ``por lo cual también lo que nacerá será llamado santo, Hijo de Dios".

      \begin{center}
        Lc. 1, 35      
      \end{center}      
      
      %DECIMO
      \textsuperscript{38}María dijo: ``he aquí la esclava del Señor; hágase en mí según tu palabra".

      \begin{center}
        Lc. 1, 38      
      \end{center}
            
    \subsection*{\hfil La Visitación de Nuestra Señora \hfil}
      
      % PRIMER
      \textsuperscript{39}Por aquellos días, levantándose María, se dirigió presurosa a la montaña, a un ciudad de Judá, \textsuperscript{40}y entró en la casa
      de Zacarías y saludó a Isabel
      \begin{center}
        Lc. 1, 39-40        
      \end{center}
      
      %SEGUNDO
      \textsuperscript{41}Y aconteció que, al oír Isabel la salutación de María. dió saltos de gozo el niño en su seno, y fué llena Isabel del Espíritu Santo.
      \begin{center}
        Lc. 1, 41        
      \end{center}
      
      %TERCERO
      \textsuperscript{42}y levantó la voz con gran clamor y dijo: "Bendita tu entre las mujeres y bendito esl fruto de tu vientre".

      \begin{center}
        Lc. 1, 42         
      \end{center}
      
      %CUARTO
      \textsuperscript{45}Y dichosa la que creyó que tendrá cumplimiento las cosas que le han sido dichas de parte del Señor
      \begin{center}
        Lc. 1, 45         
      \end{center}
      
      %QUINTO
      \textsuperscript{46}Y dijo María:
      \begin{center}
        Engrandece mi alma al Señor, \\
        \textsuperscript{47}y se regocijó mi espíritu en Dios, mi Salvador; \\
        \textsuperscript{48}porque puso sus ojos en la bajeza de su esclava.
      \end{center}

      \begin{center}
        Lc. 1, 46-48        
      \end{center}

      %SEXTO
      \begin{center}
        Pues he aquí que desde ahora \\
        me llamarán dichosa todas las generaciones;\\
        \textsuperscript{49}porque hizo en mi favor grandes cosas el Poderoso,
        y cuyo nombre es "Santo";
      \end{center}

      \begin{center}
        Lc. 1, 48-49          
      \end{center}
      
      %SEPTIMO
      \begin{center}
        y cuyo nombre es "Santo"; \\
        \textsuperscript{50}y su misericordia por generaciones y generaciones \\
        para con aquellos que le temen
      \end{center}

      \begin{center}
        Lc. 1, 49-50          
      \end{center}
      
      %OCTAVO
      \begin{center}
        \textsuperscript{51}Hizo ostentación de poder con su brazo: \\
        desbarató a los soberbios en los proyectos de su corazón
      \end{center}

      \begin{center}
        Lc. 1, 51        
      \end{center}
      
      %NOVENO
      \begin{center}
        \textsuperscript{52}derrocó de su trono a los potentados \\
        y enalteció a los humildes.
      \end{center}

      \begin{center}
        Lc. 1, 52       
      \end{center}      
      
      %DECIMO
      \begin{center}
        \textsuperscript{53}llenó de bienes a los hambrientos \\
        y despidió vacíos a los ricos.
      \end{center}

      \begin{center}
        Lc. 1, 53        
      \end{center}
            
    \subsection*{\hfil La Natividad de Nuestro Señor Jesucristo \hfil}
      
      % PRIMER
      \begin{center}
        \textsuperscript{6}Y sucedió que estando ellos allí se le complieron a ella los díaas del parto
      \end{center}

      \begin{center}
        Lc. 2,6        
      \end{center}
      
      %SEGUNDO
      \begin{center}
        \textsuperscript{7}Y dió a luz a su hijo primogénito, y le envolvió en pañales...
      \end{center}

      \begin{center}
        Lc. 2, 7          
      \end{center}
      
      %TERCERO
      \begin{center}
        \ldots y le recostó en un pesebre, pues no había para ellos lugar en el mesón
      \end{center}
      
      \begin{center}
        Lc. 2, 7         
      \end{center}
      
      %CUARTO
      \begin{center}
        \textsuperscript{8}Y había unos pastores en aquella misma comarca, que pernoctaban al raso y velaban por turno para guardar su ganado, \textsuperscript{9}y un ángel
        del Señor se presentó ante ellos, y la gloria del Señor los envolvió en sus fulgores, y se atemorizaron con gran temor.
      \end{center}
      \begin{center}
        Lc. 2, 8-9         
      \end{center}
      
      %QUINTO
      \begin{center}
        \textsuperscript{10}Y les dijo el ángel: ``no temáis, pues he aquí que os traigo una buena nueva, que será de grande alegría para todo el pueblo:"
      \end{center}
      \begin{center}
        Lc. 2, 10         
      \end{center}

      %SEXTO
      \begin{center}
        \textsuperscript{11}``que os ha nacido hoy en la ciudad de David un Salvador, que es el Mesías, el Señor"
      \end{center}
      \begin{center}
        Lc. 2, 11       
      \end{center}
      
      %SEPTIMO
      \begin{center}
        \textsuperscript{14}Gloria a Dios en las alturas \\
        y en la tierra a los hombres del [divino] agrado.
      \end{center}

      \begin{center}
        Lc. 2, 14        
      \end{center}
      
      %OCTAVO
      \begin{center}
        \textsuperscript{1}Nacido Jesús en Belén de la Judea en los días de Herodes el rey, he aquí que unos magos venidos de las regiones orientales llegaron a Jerusalén.
        \textsuperscript{11}Y entrando en la casa, vieron al niño con María, su madre;
      \end{center}

      \begin{center}
        Mt. 2, 1, 11        
      \end{center}
      
      %NOVENO
      \begin{center}
        y postrándose en tierra le adoraron, y abrieron sus tesoros le ofrecieron presentes, oro, incienso y mirra.
      \end{center}

      \begin{center}
        Mt. 2, 11         
      \end{center}      
      
      %DECIMO
      \begin{center}
        \textsuperscript{19}Pero María guardaba todas estas palabras confiriéndolas en su corazón
      \end{center}
      \begin{center}
        Lc. 2, 19        
      \end{center}
            
    \subsection*{\hfil La Presentación del Niño Jesús en el Templo \hfil}
      
      % PRIMER
      \begin{center}
        \textsuperscript{22}Y cuando se les cumplieron los días de la purificación según la ley de Moisés (Lev. 12, 6), le subieron a Jerusalén para presentarle al Señor
      \end{center}
      \begin{center}
        Lc. 2, 22         
      \end{center}
      
      %SEGUNDO
      \begin{center}
        \textsuperscript{25}y he aquí había un hombre en Jerusalén por nombre Simeón. Y era este hombre justo y temeroso de Dios que aguardaba la consolación de Israel, 
        y el Espíritu Santo estaba sobre él;
      \end{center}
      \begin{center}
        Lc. 2, 25         
      \end{center}
      
      %TERCERO
      \begin{center}
        \textsuperscript{26}y le había sido revelado por el Espíritu Santo qye no vería la muerte antes de ver al Ungido del Señor.  
      \end{center}
      
      \begin{center}
        Lc. 2, 26       
      \end{center}
      
      %CUARTO
      \begin{center}
        \textsuperscript{27}Y vino al templo impulsado por el Espíritu Santo. Y cuando sus padres intrudujeron al niño Jesús para cumplir las prescipciones usuales
        de la ley tocantes a El, \textsuperscript{28}Simeón le recibió en sus brazos y bendijo a Dios diciendo:
      \end{center}
      \begin{center}
        Lc. 2, 27-28         
      \end{center}
      
      %QUINTO
      \begin{center}
        \textsuperscript{29}Ahora dejas ir a tu siervo, Señor, según tu palabra, en paz;
      \end{center}
      \begin{center}
        Lc. 2, 29        
      \end{center}

      %SEXTO
      \begin{center}
        \textsuperscript{30}pues ya vieron mis ojos tu salud, \textsuperscript{31}que preparaste a la faz de todos los pueblos;
      \end{center}
      \begin{center}
        Lc. 2, 30-31         
      \end{center}
      
      %SEPTIMO
      \begin{center}
        \textsuperscript{31}luz para iluminación de los gentiles, y gloria de tu pueblo Israel.
      \end{center}
      \begin{center}
        Lc. 2, 32        
      \end{center}
      
      %OCTAVO
      \begin{center}
        \textsuperscript{34}Y les bendijo Simeón, y dijo a María, su madre: ``He aquí que éste está puesto para caída y resurgimiento de muchos en Israel, y como
         señal a quien se contradice''
      \end{center}
      \begin{center}
        Lc. 2, 34       
      \end{center}
      
      %NOVENO
      \begin{center}
        \textsuperscript{34}``y a ti misma una espada te traspasará el alma, para que salgan a la luz de muchos corazones los pensamientos''
      \end{center}
      \begin{center}
        Lc. 2, 35         
      \end{center}      
      
      %DECIMO
      \begin{center}
        \textsuperscript{39}Y así que cumplieron todas las cosas ordenadas en la ley del Señor, se volvieron a Galilea, a su ciudad de Nazaret. \textsuperscript{40}EL niño crecía
        y se robustecía llenándose de sabiduría, y la gracia de Dios estaba en Él.
      \end{center}
      \begin{center}
        Lc. 2, 39-40         
      \end{center}
            
    \subsection*{\hfil La Pérdida y Hallazgo del Niño Jesús en el Templo \hfil}
      
      % PRIMER
      \begin{center}
        \textsuperscript{42}Y cuando fué de doce años, habiendo ellos subido, según la costumbre de la fiesta,
      \end{center}
      \begin{center}
        Lc. 2, 42     
      \end{center}
      
      %SEGUNDO
      \begin{center}
        \textsuperscript{43}y acabados los días, al volverse ellos, quedóse el niño Jesús en Jerusalén, sin que lo advirtiesen sus padres
      \end{center}
      \begin{center}
        Lc. 2, 43        
      \end{center}
      
      %TERCERO
      \begin{center}
        \textsuperscript{45}y no hallándole, se tornaron a Jerusalén para burcarlo. \textsuperscript{46}Y sucedió que después de tres días le hallaron en el templo,
        sentado en medio de los maestros, escuchándoles y haciéndoles preguntas;
      \end{center}
      \begin{center}
        Lc. 2, 45-46       
      \end{center}
      
      %CUARTO
      \begin{center}
        sentado en medio de los maestros, escuchándoles y haciéndoles preguntas;
      \end{center}
      \begin{center}
        Lc. 2, 46       
      \end{center}
      
      %QUINTO
      \begin{center}
        \textsuperscript{47}y se pasmaban todos los que le oían de su inteligencia y de sus respuestas.
      \end{center}
      \begin{center}
        Lc. 2, 47     
      \end{center}

      %SEXTO
      \begin{center}
        \textsuperscript{48}Y sus padres, al verle, quedaron sorprendidos; y le dijo su madre: ``hijo, ¿por qué lo hiciste así con nosotros? Mira que tu padre
        y yo, llenos de aflicción, te andábamos buscando''
      \end{center}
      \begin{center}
        Lc. 2, 48     
      \end{center}
      
      %SEPTIMO
      \begin{center}
        \textsuperscript{49}Díjoles Él: ``¿pues por qué me buscabais? ¿No sabíais que había yo de estar en casa de mi padre?''
      \end{center}
      \begin{center}
        Lc. 2, 49        
      \end{center}
      
      %OCTAVO
      \textsuperscript{50}Y ellos no comprendieron lo que les dijo.
      \begin{center}
        Lc. 2, 50        
      \end{center}
      
      %NOVENO
      \textsuperscript{51}Y bajó en su compañía y se fue a Nazaret, y vivía sometido a ellos. Y su madre guardaba todas estas
      cosas en su corazón.
      \begin{center}
        Lc. 2,51       
      \end{center}      
      
      %DECIMO
      \textsuperscript{52}Y Jesús progresaba en sabiduría, en estatura y en gracia delante de Dios y de los hombres.
      \begin{center}
        Lc. 2, 52        
      \end{center}
            
    \newpage
        
  \section*{\hfil Misterios Dolorosos \hfil}
    \subsection*{\hfil La Agonía de Nuestro Señor en el Huerto de los Olivos \hfil}
      \begin{multicols}{2}

      \columnbreak
           
      \end{multicols}
      \begin{center}
        Lc. 1,26- 27           
      \end{center}
    \subsection*{\hfil La flagelación de Nuestro Señor Jesucristo \hfil}
        
      \begin{multicols}{2}

      \columnbreak
           
      \end{multicols}
      \begin{center}
        Lc. 1,26- 27           
      \end{center}
    \subsection*{\hfil La coronación de espinas de Nuestro Señor Jesucristo \hfil}
      \begin{multicols}{2}

      \columnbreak
           
      \end{multicols}
      \begin{center}
         Lc. 1,26- 27           
      \end{center}
    \subsection*{\hfil Jesús con la Cruz a cuestas \hfil}
      \begin{multicols}{2}

      \columnbreak
           
      \end{multicols}
      \begin{center}
        Lc. 1,26- 27           
      \end{center}
    \subsection*{\hfil La Crucifixión y Muerte del Redentor \hfil}
      \begin{multicols}{2}

      \columnbreak
           
      \end{multicols}
      \begin{center}
         Lc. 1,26- 27           
      \end{center}
         
    \newpage
         
  \section*{\hfil Miserios Gloriosos \hfil}
    \subsection*{\hfil La Resurección del Señor \hfil}
      \begin{multicols}{2}

      \columnbreak
                           
      \end{multicols}
      \begin{center}
        Lc. 1,26- 27           
      \end{center}
    \subsection*{\hfil La Ascensión de Jesucristo a los cielos \hfil}
      \begin{multicols}{2}

      \columnbreak
                           
      \end{multicols}         
      \begin{center}
        Lc. 1,26- 27           
      \end{center}
    \subsection*{\hfil La Venida del Espíritu Santo sobre los Apóstoles \hfil}
      \begin{multicols}{2}

      \columnbreak
                           
      \end{multicols}         
      \begin{center}
        Lc. 1,26- 27           
      \end{center}
    \subsection*{\hfil La Asunción de Nuestra Señora a los cielos \hfil}
      \begin{multicols}{2}

      \columnbreak
                           
      \end{multicols}         
      \begin{center}
        Lc. 1,26- 27           
      \end{center}
    \subsection*{\hfil La Coronación de la Santísima Virgen María \hfil}
      \begin{multicols}{2}

      \columnbreak
                           
      \end{multicols}         
      \begin{center}
        Lc. 1,26- 27           
      \end{center}
\end{document}
