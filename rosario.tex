\documentclass[a4paper,11pt,sans]{article}

\usepackage[utf8]{inputenc}
\usepackage[spanish]{babel}
\usepackage{multicol}

\begin{document}
  \section*{\hfil Misterios Gozosos \hfil}
    \subsection*{\hfil La Anunciación de la Santísima Virgen María \hfil}
      
      %\begin{multicols}{2}
      %  \textsuperscript{26}En el sexto mes fué enviado el Ángel Gabriel de parte de Dios a una ciudad de Galilea, llamada Nazaret, \textsuperscript{27}a una doncella desposada con un varón llamada José, de la familia de 
      %  David, y el nombre de la doncella era María.
      %\columnbreak
          
      %\end{multicols}

      % PRIMER
      \textsuperscript{26}En el sexto mes fué enviado el Ángel Gabriel de parte de Dios a una ciudad de Galilea, llamada Nazaret, \textsuperscript{27}a una 
      doncella desposada con un varón llamada José, de la familia de David, y el nombre de la doncella era María.
      
      \begin{center}
        Lc. 1,26- 27           
      \end{center}
      
      %SEGUNDO
      \textsuperscript{28}Y habiendo entrado a ella, dijo: "Dios te salve, llena de gracia, el Señor es contigo, bendita tú entre las mujeres 
      \textsuperscript{42}y levanto la voz con gran clamor y dijo: "Bendita tú entre las mujeres y bendito el fruto de tu vientre"
      
      \begin{center}
        Lc. 1, 28, 42      
      \end{center}
      
      %TERCERO
      \textsuperscript{29}Ella, al oír estas palabras, se turbó, y discurría qué podía ser esta salutación

      \begin{center}
        Lc. 1, 29         
      \end{center}
      
      %CUARTO
      \textsuperscript{30}Y le dijo el ángel: "No temas, María, pues hallaste gracia a los ojos de Dios".

      \begin{center}
        Lc. 1, 30         
      \end{center}
      
      %QUINTO
      \textsuperscript{31}"He aquí que concebirás en tu seno y darás a luz un Hijo, a quién darás por nombre Jesús".

      \begin{center}
        Lc. 1, 31      
      \end{center}

      %SEXTO
      \textsuperscript{32}"Este será grande, y será llamado Hijo del Altísimo, y le dará el Señor Dios el trono de David su padre, \textsuperscript{33}y reinará
      sobre la casa de Jacob etérnamente, y su reinado no tendrá fin"

      \begin{center}
        Lc. 1, 32-33        
      \end{center}
      
      %SEPTIMO
      \textsuperscript{34}DIjo María al ángel: "¿cómo será eso, pues, no conozco varón?".

      \begin{center}
        Lc. 1, 34         
      \end{center}
      
      %OCTAVO
      \textsuperscript{35}Y respondiendo el ángel, le dijo: "el Espíritu Santo descenderá sobre ti, y el poder del Altísimo te cobijará con su sombra";

      \begin{center}
        Lc. 1, 35       
      \end{center}
      
      %NOVENO
      "por lo cual también lo que nacerá será llamado santo, Hijo de Dios".

      \begin{center}
        Lc. 1, 35      
      \end{center}      
      
      %DECIMO
      \textsuperscript{38}María dijo: "he aquí la esclava del Señor; hágase en mí según tu palabra".

      \begin{center}
        Lc. 1, 38      
      \end{center}
            
    \subsection*{\hfil La Visitación de Nuestra Señora \hfil}
      
      % PRIMER
      \textsuperscript{39}Por aquellos días, levantándose María, se dirigió presurosa a la montaña, a un ciudad de Judá, \textsuperscript{40}y entró en la casa
      de Zacarías y saludó a Isabel
      \begin{center}
        Lc. 1, 39-40        
      \end{center}
      
      %SEGUNDO
      \textsuperscript{41}Y aconteció que, al oír Isabel la salutación de María. dió saltos de gozo el niño en su seno, y fué llena Isabel del Espíritu Santo.
      \begin{center}
        Lc. 1, 41        
      \end{center}
      
      %TERCERO
      \textsuperscript{42}y levantó la voz con gran clamor y dijo: "Bendita tu entre las mujeres y bendito esl fruto de tu vientre".

      \begin{center}
        Lc. 1, 42         
      \end{center}
      
      %CUARTO
      \textsuperscript{45}Y dichosa la que creyó que tendrá cumplimiento las cosas que le han sido dichas de parte del Señor
      \begin{center}
        Lc. 1, 45         
      \end{center}
      
      %QUINTO
      \textsuperscript{46}Y dijo María:
      \begin{center}
        Engrandece mi alma al Señor, \\
        \textsuperscript{47}y se regocijó mi espíritu en Dios, mi Salvador; \\
        \textsuperscript{48}porque puso sus ojos en la bajeza de su esclava.
      \end{center}

      \begin{center}
        Lc. 1, 46-48        
      \end{center}

      %SEXTO
      \begin{center}
        Pues he aquí que desde ahora \\
        me llamarán dichosa todas las generaciones;\\
        \textsuperscript{49}porque hizo en mi favor grandes cosas el Poderoso,
        y cuyo nombre es "Santo";
      \end{center}

      \begin{center}
        Lc. 1, 48-49          
      \end{center}
      
      %SEPTIMO
      \begin{center}
        y cuyo nombre es "Santo"; \\
        \textsuperscript{50}y su misericordia por generaciones y generaciones \\
        para con aquellos que le temen
      \end{center}

      \begin{center}
        Lc. 1, 49-50          
      \end{center}
      
      %OCTAVO
      \begin{center}
        \textsuperscript{51}Hizo ostentación de poder con su brazo: \\
        desbarató a los soberbios en los proyectos de su corazón
      \end{center}

      \begin{center}
        Lc. 1, 51        
      \end{center}
      
      %NOVENO
      \begin{center}
        \textsuperscript{52}derrocó de su trono a los potentados \\
        y enalteció a los humildes.
      \end{center}

      \begin{center}
        Lc. 1, 52       
      \end{center}      
      
      %DECIMO
      \begin{center}
        \textsuperscript{53}llenó de bienes a los hambrientos \\
        y despidió vacíos a los ricos.
      \end{center}

      \begin{center}
        Lc. 1, 53        
      \end{center}
            
    \subsection*{\hfil La Natividad de Nuestro Señor Jesucristo \hfil}
      
      % PRIMER
      \begin{multicols}{2}

      \columnbreak
          
      \end{multicols}
      \begin{center}
        Lc. 2,6        
      \end{center}
      
      %SEGUNDO
      \begin{multicols}{2}

      \columnbreak
          
      \end{multicols}
      \begin{center}
        Lc. 2, 7          
      \end{center}
      
      %TERCERO
      \begin{multicols}{2}

      \columnbreak
          
      \end{multicols}
      \begin{center}
        Lc. 2, 7         
      \end{center}
      
      %CUARTO
      \begin{multicols}{2}

      \columnbreak
          
      \end{multicols}
      \begin{center}
        Lc. 2, 8-9         
      \end{center}
      
      %QUINTO
      \begin{multicols}{2}

      \columnbreak
          
      \end{multicols}
      \begin{center}
        Lc. 2, 10         
      \end{center}

      %SEXTO
      \begin{multicols}{2}

      \columnbreak
          
      \end{multicols}
      \begin{center}
        Lc. 2, 11       
      \end{center}
      
      %SEPTIMO
      \begin{multicols}{2}

      \columnbreak
          
      \end{multicols}
      \begin{center}
        Lc. 2, 14        
      \end{center}
      
      %OCTAVO
      \begin{multicols}{2}

      \columnbreak
          
      \end{multicols}
      \begin{center}
        Mt. 2; 1, 11        
      \end{center}
      
      %NOVENO
      \begin{multicols}{2}

      \columnbreak
          
      \end{multicols}
      \begin{center}
        Mt. 2, 11         
      \end{center}      
      
      %DECIMO
      \begin{multicols}{2}

      \columnbreak
          
      \end{multicols}
      \begin{center}
        Lc. 2, 19        
      \end{center}
            
    \subsection*{\hfil La Presentación del Niño Jesús en el Templo \hfil}
      
      % PRIMER
      \begin{multicols}{2}

      \columnbreak
          
      \end{multicols}
      \begin{center}
        Lc. 2, 22         
      \end{center}
      
      %SEGUNDO
      \begin{multicols}{2}

      \columnbreak
          
      \end{multicols}
      \begin{center}
        Lc. 2, 25         
      \end{center}
      
      %TERCERO
      \begin{multicols}{2}

      \columnbreak
          
      \end{multicols}
      \begin{center}
        Lc. 2, 26       
      \end{center}
      
      %CUARTO
      \begin{multicols}{2}

      \columnbreak
          
      \end{multicols}
      \begin{center}
        Lc. 2, 27-28         
      \end{center}
      
      %QUINTO
      \begin{multicols}{2}

      \columnbreak
          
      \end{multicols}
      \begin{center}
        Lc. 2, 29        
      \end{center}

      %SEXTO
      \begin{multicols}{2}

      \columnbreak
          
      \end{multicols}
      \begin{center}
        Lc. 2, 30-31         
      \end{center}
      
      %SEPTIMO
      \begin{multicols}{2}

      \columnbreak
          
      \end{multicols}
      \begin{center}
        Lc. 2, 32        
      \end{center}
      
      %OCTAVO
      \begin{multicols}{2}

      \columnbreak
          
      \end{multicols}
      \begin{center}
        Lc. 2, 34       
      \end{center}
      
      %NOVENO
      \begin{multicols}{2}

      \columnbreak
          
      \end{multicols}
      \begin{center}
        Lc. 2, 35         
      \end{center}      
      
      %DECIMO
      \begin{multicols}{2}

      \columnbreak
          
      \end{multicols}
      \begin{center}
        Lc. 2, 39-40         
      \end{center}
            
    \subsection*{\hfil La Pérdida y Hallazgo del Niño Jesús en el Templo \hfil}
      
      % PRIMER
      \begin{multicols}{2}

      \columnbreak
          
      \end{multicols}
      \begin{center}
        Lc. 2, 42     
      \end{center}
      
      %SEGUNDO
      \begin{multicols}{2}

      \columnbreak
          
      \end{multicols}
      \begin{center}
        Lc. 2, 43        
      \end{center}
      
      %TERCERO
      \begin{multicols}{2}

      \columnbreak
          
      \end{multicols}
      \begin{center}
        Lc. 2, 45-46       
      \end{center}
      
      %CUARTO
      \begin{multicols}{2}

      \columnbreak
          
      \end{multicols}
      \begin{center}
        Lc. 2, 46       
      \end{center}
      
      %QUINTO
      \begin{multicols}{2}

      \columnbreak
          
      \end{multicols}
      \begin{center}
        Lc. 2, 47     
      \end{center}

      %SEXTO
      \begin{multicols}{2}

      \columnbreak
          
      \end{multicols}
      \begin{center}
        Lc. 2, 48     
      \end{center}
      
      %SEPTIMO
      \begin{multicols}{2}

      \columnbreak
          
      \end{multicols}
      \begin{center}
        Lc. 2, 49        
      \end{center}
      
      %OCTAVO
      \begin{multicols}{2}

      \columnbreak
          
      \end{multicols}
      \begin{center}
        Lc. 2, 50        
      \end{center}
      
      %NOVENO
      \begin{multicols}{2}

      \columnbreak
          
      \end{multicols}
      \begin{center}
        Lc. 2,51       
      \end{center}      
      
      %DECIMO
      \begin{multicols}{2}

      \columnbreak
          
      \end{multicols}
      \begin{center}
        Lc. 2, 52        
      \end{center}
            
    \newpage
        
  \section*{\hfil Misterios Dolorosos \hfil}
    \subsection*{\hfil La Agonía de Nuestro Señor en el Huerto de los Olivos \hfil}
      \begin{multicols}{2}

      \columnbreak
           
      \end{multicols}
      \begin{center}
        Lc. 1,26- 27           
      \end{center}
    \subsection*{\hfil La flagelación de Nuestro Señor Jesucristo \hfil}
        
      \begin{multicols}{2}

      \columnbreak
           
      \end{multicols}
      \begin{center}
        Lc. 1,26- 27           
      \end{center}
    \subsection*{\hfil La coronación de espinas de Nuestro Señor Jesucristo \hfil}
      \begin{multicols}{2}

      \columnbreak
           
      \end{multicols}
      \begin{center}
         Lc. 1,26- 27           
      \end{center}
    \subsection*{\hfil Jesús con la Cruz a cuestas \hfil}
      \begin{multicols}{2}

      \columnbreak
           
      \end{multicols}
      \begin{center}
        Lc. 1,26- 27           
      \end{center}
    \subsection*{\hfil La Crucifixión y Muerte del Redentor \hfil}
      \begin{multicols}{2}

      \columnbreak
           
      \end{multicols}
      \begin{center}
         Lc. 1,26- 27           
      \end{center}
         
    \newpage
         
  \section*{\hfil Miserios Gloriosos \hfil}
    \subsection*{\hfil La Resurección del Señor \hfil}
      \begin{multicols}{2}

      \columnbreak
                           
      \end{multicols}
      \begin{center}
        Lc. 1,26- 27           
      \end{center}
    \subsection*{\hfil La Ascensión de Jesucristo a los cielos \hfil}
      \begin{multicols}{2}

      \columnbreak
                           
      \end{multicols}         
      \begin{center}
        Lc. 1,26- 27           
      \end{center}
    \subsection*{\hfil La Venida del Espíritu Santo sobre los Apóstoles \hfil}
      \begin{multicols}{2}

      \columnbreak
                           
      \end{multicols}         
      \begin{center}
        Lc. 1,26- 27           
      \end{center}
    \subsection*{\hfil La Asunción de Nuestra Señora a los cielos \hfil}
      \begin{multicols}{2}

      \columnbreak
                           
      \end{multicols}         
      \begin{center}
        Lc. 1,26- 27           
      \end{center}
    \subsection*{\hfil La Coronación de la Santísima Virgen María \hfil}
      \begin{multicols}{2}

      \columnbreak
                           
      \end{multicols}         
      \begin{center}
        Lc. 1,26- 27           
      \end{center}
\end{document}
