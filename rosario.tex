\documentclass[a4paper,11pt]{article}

\usepackage[utf8]{inputenc}
\usepackage[spanish]{babel}
\usepackage{subfiles}
\usepackage[T1]{fontenc}

\begin{document}
  \section*{\hfil Misterios Gozosos \hfil}
    
    \subsection*{\hfil La Anunciación de la Santísima Virgen María \hfil}

      % PRIMER
      \textsuperscript{26}En el sexto mes fué enviado el Ángel Gabriel de parte de Dios a una ciudad de Galilea, llamada Nazaret, \textsuperscript{27}a una 
      doncella desposada con un varón llamada José, de la familia de David, y el nombre de la doncella era María.
      \begin{flushright}
        Lc. 1,26- 27
      \end{flushright}

      %SEGUNDO
      \textsuperscript{28}Y habiendo entrado a ella, dijo: »Dios te salve, llena de gracia, el Señor es contigo, bendita tú entre las mujeres 
      \textsuperscript{42}y levanto la voz con gran clamor y dijo: »Bendita tú entre las mujeres y bendito el fruto de tu vientre».
      \begin{flushright}
        Lc. 1, 28, 42
      \end{flushright}

      %TERCERO
      \textsuperscript{29}Ella, al oír estas palabras, se turbó, y discurría qué podía ser esta salutación.
      \begin{flushright}
        Lc. 1, 29
      \end{flushright}

      %CUARTO
      \textsuperscript{30}Y le dijo el ángel: »No temas, María, pues hallaste gracia a los ojos de Dios».
      \begin{flushright}
        Lc. 1, 30
      \end{flushright}

      %QUINTO
      \textsuperscript{31}«He aquí que concebirás en tu seno y darás a luz un Hijo, a quién darás por nombre Jesús».
      \begin{flushright}
        Lc. 1, 31
      \end{flushright}

      %SEXTO
      \textsuperscript{32}«Este será grande, y será llamado Hijo del Altísimo, y le dará el Señor Dios el trono de David su padre, \textsuperscript{33}y reinará
      sobre la casa de Jacob etérnamente, y su reinado no tendrá fin».
      \begin{flushright}
        Lc. 1, 32-33
      \end{flushright}

      %SEPTIMO
      \textsuperscript{35}Y respondiendo el ángel, le dijo: «el Espíritu Santo descenderá sobre ti, y el poder del Altísimo te cobijará con su sombra»;
      \begin{flushright}
        Lc. 1, 35
      \end{flushright}
      
      %OCTAVO
      \textsuperscript{26}En el sexto mes fué enviado el Ángel Gabriel de parte de Dios a una ciudad de Galilea, llamada Nazaret, \textsuperscript{27}a una 
      doncella desposada con un varón llamada José, de la familia de David, y el nombre de la doncella era María.
      \begin{flushright}
        Lc. 1,26- 27
      \end{flushright}

      %NOVENO
      «por lo cual también lo que nacerá será llamado santo, Hijo de Dios».
      \begin{flushright}
        Lc. 1, 35
      \end{flushright}

      %DECIMO
      \textsuperscript{38}María dijo: «he aquí la esclava del Señor; hágase en mí según tu palabra».
      \begin{flushright}
        Lc. 1, 38
      \end{flushright}
            
    \subsection*{\hfil La Visitación de Nuestra Señora \hfil}
      
      % PRIMER
      \textsuperscript{39}Por aquellos días, levantándose María, se dirigió presurosa a la montaña, a un ciudad de Judá, \textsuperscript{40}y entró en la casa
      de Zacarías y saludó a Isabel
      \begin{flushright}
        Lc. 1, 39-40        
      \end{flushright}
      
      %SEGUNDO
      \textsuperscript{41}Y aconteció que, al oír Isabel la salutación de María. dió saltos de gozo el niño en su seno, y fué llena Isabel del Espíritu Santo.
      \begin{flushright}
        Lc. 1, 41        
      \end{flushright}
      
      %TERCERO
      \textsuperscript{42}y levantó la voz con gran clamor y dijo: »Bendita tu entre las mujeres y bendito esl fruto de tu vientre».
      \begin{flushright}
        Lc. 1, 42         
      \end{flushright}
      
      %CUARTO
      \textsuperscript{45}Y dichosa la que creyó que tendrá cumplimiento las cosas que le han sido dichas de parte del Señor
      \begin{flushright}
        Lc. 1, 45         
      \end{flushright}
      
      %QUINTO
      \textsuperscript{46}Y dijo María: «Engrandece mi alma al Señor, \textsuperscript{47}y se regocijó mi espíritu en Dios, mi Salvador;
      \textsuperscript{48}porque puso sus ojos en la bajeza de su esclava.»
      \begin{flushright}
        Lc. 1, 46-48        
      \end{flushright}

      %SEXTO
      «Pues he aquí que desde ahora me llamarán dichosa todas las generaciones; \textsuperscript{49}porque hizo en mi favor grandes cosas el Poderoso,
      \begin{flushright}
        Lc. 1, 48-49          
      \end{flushright}
      
      %SEPTIMO
      «y cuyo nombre es Santo; \textsuperscript{50}y su misericordia por generaciones y generaciones para con aquellos que le temen»
      \begin{flushright}
        Lc. 1, 49-50          
      \end{flushright}
      
      %OCTAVO
      \textsuperscript{51}Hizo ostentación de poder con su brazo: desbarató a los soberbios en los proyectos de su corazón
      \begin{flushright}
        Lc. 1, 51        
      \end{flushright}
      
      %NOVENO
      «\textsuperscript{52}derrocó de su trono a los potentados y enalteció a los humildes».
      \begin{flushright}
        Lc. 1, 52       
      \end{flushright}
      
      %DECIMO
      «\textsuperscript{53}llenó de bienes a los hambrientos y despidió vacíos a los ricos».
      \begin{flushright}
        Lc. 1, 53
      \end{flushright}
            
    \subsection*{\hfil La Natividad de Nuestro Señor Jesucristo \hfil}
      
      % PRIMER
      \textsuperscript{6}Y sucedió que estando ellos allí se le complieron a ella los díaas del parto
      \begin{flushright}
        Lc. 2,6        
      \end{flushright}
      
      %SEGUNDO
      \textsuperscript{7}Y dió a luz a su hijo primogénito, y le envolvió en pañales\ldots
      \begin{flushright}
        Lc. 2, 7          
      \end{flushright}
      
      %TERCERO
      \ldots y le recostó en un pesebre, pues no había para ellos lugar en el mesón      
      \begin{flushright}
        Lc. 2, 7         
      \end{flushright}
      
      %CUARTO
      \textsuperscript{8}Y había unos pastores en aquella misma comarca, que pernoctaban al raso y velaban por turno para guardar su ganado, \textsuperscript{9}y un ángel
      del Señor se presentó ante ellos, y la gloria del Señor los envolvió en sus fulgores, y se atemorizaron con gran temor.
      \begin{flushright}
        Lc. 2, 8-9         
      \end{flushright}
      
      %QUINTO
      \textsuperscript{10}Y les dijo el ángel: «no temáis, pues he aquí que os traigo una buena nueva, que será de grande alegría para todo el pueblo:»
      \begin{flushright}
        Lc. 2, 10         
      \end{flushright}

      %SEXTO
      \textsuperscript{11}«que os ha nacido hoy en la ciudad de David un Salvador, que es el Mesías, el Señor»
      \begin{flushright}
        Lc. 2, 11       
      \end{flushright}
      
      %SEPTIMO
      \textsuperscript{14}Gloria a Dios en las alturas y en la tierra a los hombres del [divino] agrado.
      \begin{flushright}
        Lc. 2, 14        
      \end{flushright}

      %OCTAVO
      \textsuperscript{1}Nacido Jesús en Belén de la Judea en los días de Herodes el rey, he aquí que unos magos venidos de las regiones orientales llegaron a Jerusalén.
      \textsuperscript{11}Y entrando en la casa, vieron al niño con María, su madre;
      \begin{flushright}
        Mt. 2, 1, 11        
      \end{flushright}
      
      %NOVENO
      y postrándose en tierra le adoraron, y abrieron sus tesoros le ofrecieron presentes, oro, incienso y mirra.
      \begin{flushright}
        Mt. 2, 11         
      \end{flushright} 
      
      %DECIMO
      \textsuperscript{19}Pero María guardaba todas estas palabras confiriéndolas en su corazón
      \begin{flushright}
        Lc. 2, 19        
      \end{flushright}

    \subsection*{\hfil La Presentación del Niño Jesús en el Templo \hfil}
      
      % PRIMER
      \textsuperscript{22}Y cuando se les cumplieron los días de la purificación según la ley de Moisés (Lev. 12, 6), le subieron a Jerusalén para presentarle al Señor
      \begin{flushright}
        Lc. 2, 22         
      \end{flushright}
      
      %SEGUNDO
      \textsuperscript{25}y he aquí había un hombre en Jerusalén por nombre Simeón. Y era este hombre justo y temeroso de Dios que aguardaba la consolación de Israel, 
      y el Espíritu Santo estaba sobre él;
      \begin{flushright}
        Lc. 2, 25         
      \end{flushright}
      
      %TERCERO
      \textsuperscript{26}y le había sido revelado por el Espíritu Santo qye no vería la muerte antes de ver al Ungido del Señor.
      \begin{flushright}
        Lc. 2, 26       
      \end{flushright}
      
      %CUARTO
      \textsuperscript{27}Y vino al templo impulsado por el Espíritu Santo. Y cuando sus padres intrudujeron al niño Jesús para cumplir las prescipciones usuales
      de la ley tocantes a El, \textsuperscript{28}Simeón le recibió en sus brazos y bendijo a Dios diciendo:
      \begin{flushright}
        Lc. 2, 27-28         
      \end{flushright}
      
      %QUINTO
      \textsuperscript{29}Ahora dejas ir a tu siervo, Señor, según tu palabra, en paz;
      \begin{flushright}
        Lc. 2, 29        
      \end{flushright}

      %SEXTO
      \textsuperscript{30}pues ya vieron mis ojos tu salud, \textsuperscript{31}que preparaste a la faz de todos los pueblos;
      \begin{flushright}
        Lc. 2, 30-31         
      \end{flushright}
      
      %SEPTIMO
      \textsuperscript{31}luz para iluminación de los gentiles, y gloria de tu pueblo Israel.
      \begin{flushright}
        Lc. 2, 32        
      \end{flushright}
      
      %OCTAVO
      \textsuperscript{34}Y les bendijo Simeón, y dijo a María, su madre: «He aquí que éste está puesto para caída y resurgimiento de muchos en Israel, y como
      señal a quien se contradice»
      \begin{flushright}
        Lc. 2, 34       
      \end{flushright}
      
      %NOVENO
      \textsuperscript{34}«y a ti misma una espada te traspasará el alma, para que salgan a la luz de muchos corazones los pensamientos»
      \begin{flushright}
        Lc. 2, 35         
      \end{flushright} 
      
      %DECIMO
      \textsuperscript{39}Y así que cumplieron todas las cosas ordenadas en la ley del Señor, se volvieron a Galilea, a su ciudad de Nazaret. \textsuperscript{40}EL niño crecía
      y se robustecía llenándose de sabiduría, y la gracia de Dios estaba en Él.
      \begin{flushright}
        Lc. 2, 39-40         
      \end{flushright}
            
    \subsection*{\hfil La Pérdida y Hallazgo del Niño Jesús en el Templo \hfil}
      
      % PRIMER
      \textsuperscript{42}Y cuando fué de doce años, habiendo ellos subido, según la costumbre de la fiesta,
      \begin{flushright}
        Lc. 2, 42     
      \end{flushright}
      
      %SEGUNDO
      \textsuperscript{43}y acabados los días, al volverse ellos, quedóse el niño Jesús en Jerusalén, sin que lo advirtiesen sus padres
      \begin{flushright}
        Lc. 2, 43        
      \end{flushright}
      
      %TERCERO
      \textsuperscript{45}y no hallándole, se tornaron a Jerusalén para burcarlo. \textsuperscript{46}Y sucedió que después de tres días le hallaron en el templo,
      sentado en medio de los maestros, escuchándoles y haciéndoles preguntas;
      \begin{flushright}
        Lc. 2, 45-46       
      \end{flushright}
      
      %CUARTO
      sentado en medio de los maestros, escuchándoles y haciéndoles preguntas;
      \begin{flushright}
        Lc. 2, 46       
      \end{flushright}
      
      %QUINTO
      \textsuperscript{47}y se pasmaban todos los que le oían de su inteligencia y de sus respuestas.
      \begin{flushright}
        Lc. 2, 47     
      \end{flushright}

      %SEXTO
      \textsuperscript{48}Y sus padres, al verle, quedaron sorprendidos; y le dijo su madre: «hijo, ¿por qué lo hiciste así con nosotros? Mira que tu padre
      y yo, llenos de aflicción, te andábamos buscando»
      \begin{flushright}
        Lc. 2, 48     
      \end{flushright}
      
      %SEPTIMO
      \textsuperscript{49}Díjoles Él: «¿pues por qué me buscabais? ¿No sabíais que había yo de estar en casa de mi padre?»
      \begin{flushright}
        Lc. 2, 49        
      \end{flushright}
      
      %OCTAVO
      \textsuperscript{50}Y ellos no comprendieron lo que les dijo.
      \begin{flushright}
        Lc. 2, 50        
      \end{flushright}
      
      %NOVENO
      \textsuperscript{51}Y bajó en su compañía y se fue a Nazaret, y vivía sometido a ellos. Y su madre guardaba todas estas
      cosas en su corazón.
      \begin{flushright}
        Lc. 2,51       
      \end{flushright}     
      
      %DECIMO
      \textsuperscript{52}Y Jesús progresaba en sabiduría, en estatura y en gracia delante de Dios y de los hombres.
      \begin{flushright}
        Lc. 2, 52        
      \end{flushright}
            
    \newpage
        
  \section*{\hfil Misterios Dolorosos \hfil}
    
    \subsection*{\hfil La Agonía de Nuestro Señor en el Huerto de los Olivos \hfil}
      
      % UNO
      \subfile{mysteries/painful/painful_01_01}
      \medskip

      % DOS
      \subfile{mysteries/painful/painful_01_02}
      \medskip

      % TRES
      \subfile{mysteries/painful/painful_01_03}
      \medskip

      % CUATRO
      \subfile{mysteries/painful/painful_01_04}
      \medskip

      % CINCO
      \subfile{mysteries/painful/painful_01_05}
      \medskip

      % SEIS
      \subfile{mysteries/painful/painful_01_06}
      \medskip

      % SIETE
      \subfile{mysteries/painful/painful_01_07}
      \medskip

      % OCHO
      \subfile{mysteries/painful/painful_01_08}
      \medskip

      % NUEVE
      \subfile{mysteries/painful/painful_01_09}
      \medskip

      % DIEZ
      \subfile{mysteries/painful/painful_01_10}
      \medskip

    \subsection*{\hfil La flagelación de Nuestro Señor Jesucristo \hfil}
      
      % UNO
      \subfile{mysteries/painful/painful_03_01}
      \medskip

      % DOS
      \subfile{mysteries/painful/painful_03_02}
      \medskip

      % TRES
      \subfile{mysteries/painful/painful_03_03}
      \medskip

      % CUATRO
      \subfile{mysteries/painful/painful_03_04}
      \medskip

      % CINCO
      \subfile{mysteries/painful/painful_03_05}
      \medskip

      % SEIS
      \subfile{mysteries/painful/painful_03_06}
      \medskip

      % SIETE
      \subfile{mysteries/painful/painful_03_07}
      \medskip

      % OCHO
      \subfile{mysteries/painful/painful_03_08}
      \medskip

      % NUEVE
      \subfile{mysteries/painful/painful_03_09}
      \medskip

      % DIEZ
      \subfile{mysteries/painful/painful_03_10}
      \medskip
      
    \subsection*{\hfil La coronación de espinas de Nuestro Señor Jesucristo \hfil}
      
      % UNO
      \subfile{mysteries/painful/painful_04_01}
      \medskip

      % DOS
      \subfile{mysteries/painful/painful_04_02}
      \medskip

      % TRES
      \subfile{mysteries/painful/painful_04_03}
      \medskip

      % CUATRO
      \subfile{mysteries/painful/painful_04_04}
      \medskip

      % CINCO
      \subfile{mysteries/painful/painful_04_05}
      \medskip

      % SEIS
      \subfile{mysteries/painful/painful_04_06}
      \medskip

      % SIETE
      \subfile{mysteries/painful/painful_04_07}
      \medskip

      % OCHO
      \subfile{mysteries/painful/painful_04_08}
      \medskip

      % NUEVE
      \subfile{mysteries/painful/painful_04_09}
      \medskip

      % DIEZ
      \subfile{mysteries/painful/painful_04_10}
      \medskip

    \subsection*{\hfil Jesús con la Cruz a cuestas \hfil}
      
      % UNO
      \subfile{mysteries/painful/painful_05_01}
      \textsuperscript{23}Si alguno quiere venir en pos de Mí, niégese a sí mismo
      \begin{flushright}
        Lc. 9, 23
      \end{flushright}

      % DOS
      \subfile{mysteries/painful/painful_05_02}
      y tome a cuestas su cruz cada día y sígame
      \begin{flushright}
        Lc. 9, 23
      \end{flushright}

      % TRES
      \subfile{mysteries/painful/painful_05_03}
      \textsuperscript{17}y, llevando a cuestas su cruz, salió hacia el lugar llamado el Cráneo, que en hebreo se dice Gólgota.
      \begin{flushright}
        Jn. 19, 17
      \end{flushright}

      % CUATRO
      \subfile{mysteries/painful/painful_05_04}
      \textsuperscript{26}Y como le hubieron sacado, echaron mano de un tal Simón de Cirene que venía del campo, le pusieron en hombros la cruz para que la llevase
      detrás de Jesús.
      \begin{flushright}
        Lc. 23, 26
      \end{flushright}

      % CINCO
      \subfile{mysteries/painful/painful_05_05}
      \textsuperscript{29}Tomad mi yugo sobre vuestros, y aprended de mi, 
      \begin{flushright}
        Mt. 11, 29  
      \end{flushright}

      % SEIS
      \subfile{mysteries/painful/painful_05_06}
      pues soy manso y humilde de Corazón, y hallaréis reposo para vuestras almas.
      \begin{flushright}
        Mt. 11, 29
      \end{flushright}

      % SIETE
      \subfile{mysteries/painful/painful_05_07}
      Porque mi yugo es suave, y mi carga, ligera.
      \begin{flushright}
        Mt. 11, 30
      \end{flushright}

      % OCHO
      \subfile{mysteries/painful/painful_05_08}
      \textsuperscript{27}Seguíanle gran mucheduncbre de pueblo y de mujeres, las cuales le plañían y lamentaban.
      \begin{flushright}
        Lc. 23, 27
      \end{flushright}

      % NUEVE
      \subfile{mysteries/painful/painful_05_09}
      \textsuperscript{28}Volviéndose Jesús a ellas, les dijo: «Hijas de Jerusalén: no lloréis sobre mi, sino llorad má bien sobre vosotras mismas y sobre
      vuestros hijos»
      \begin{flushright}
        Lc. 23, 28
      \end{flushright}

      % DIEZ
      \subfile{mysteries/painful/painful_05_10}
      \textsuperscript{31}«Porque si en el leño verde esto hacen, ¿en el seco qué se hará?».
      \begin{flushright}
        Lc. 23, 31
      \end{flushright}

    \subsection*{\hfil La Crucifixión y Muerte del Redentor \hfil}
      
      % UNO
      \textsuperscript{33}Y cuando hubieron llegado al lugar llamado «Cráneo», allí crucificaron a Él y a los malhechores, uno a la derecha y el otro a la izquierda.
      \begin{flushright}
        Lc. 23, 33
      \end{flushright}

      % DOS
      \textsuperscript{34}Y Jesús decía: «Padre, perdónalos, porque no saben lo que hacen».
      \begin{flushright}
        Lc. 23, 34
      \end{flushright}

      % TRES
      \textsuperscript{39}Uno de los malhechores que estaba colgado \textsuperscript{42}decía a Jesús: «acuérdate de mi cuando vinieres en la gloria de tu realeza».
      \begin{flushright}
        Lc. 23, 39, 42
      \end{flushright}

      % CUATRO
      \textsuperscript{43}Díjole: «en verdad te digo que hoy estarás conmigo en el paraíso»
      \begin{flushright}
        Lc. 23, 43
      \end{flushright}

      % CINCO
      \textsuperscript{26}Jesús, pues, viendo a la Madre, y junto a ella al discípulo a quien amaba,
      \begin{flushright}
        Jn. 19, 26
      \end{flushright}

      % SEIS
      \textsuperscript{27}dice a tu Madre: «mujer, he ahí a tu hijo». \textsuperscript{27}Luego dice al discípulo: «He aquí a tu Madre»
      \begin{flushright}
        Jn. 19, 26-27
      \end{flushright}

      % SIETE
      Y desde aquella hora la tomó el discípulo en su compañía.
      \begin{flushright}
        Jn. 19, 27
      \end{flushright}

      % OCHO
      \textsuperscript{45}habiendo faltado el sol; y se rasgó por medio el velo del santuario.
      \begin{flushright}
        Lc. 23, 45
      \end{flushright}

      % NUEVE
      \textsuperscript{46}Y clamando con voz poderosa, Jesús dijo: «Padre, en tus manos encomiendo mis espíritu».
      \begin{flushright}
        Lc. 23, 46 
      \end{flushright}

      % DIEZ
      \textsuperscript{30}Jesús dijo: «Consumado está». E inclinando la cabeza entregó el espíritu.
      \begin{flushright}
        Jn. 19, 30
      \end{flushright}
 
    \newpage
         
  \section*{\hfil Miserios Gloriosos \hfil}
    \subsection*{\hfil La Resurección del Señor \hfil}
      
    %PRIMERO
      \textsuperscript{20}«En verdad, en verdad os digo que vosotros lloraréis y os lamentaréis, y el mundo se rogocijará;
      vosotros os afligiréis, pero vuestra aflicción está de parto»
      \begin{flushright}
        Jn. 16, 20        
      \end{flushright}

      %SEGUNDO  
      \textsuperscript{22}«Pues así también vosotros, ahora cierto tenéis congoja; mas otra vez os veré, y se gozará vuestro corazon,
      y vuestro gozo nadie os lo quita»
      \begin{flushright}
        Jn. 16, 22       
      \end{flushright}

      %TERCERO
      \textsuperscript{1}Más el primer día de la semana, apenas rayó el alba, se vinieron al monumento llevando consigo los aromas
      que habían preparado.
      \begin{flushright}
        Lc. 24, 1        
      \end{flushright}

      %CUARTO
      \textsuperscript{2}De pronto se produjo un gran temblor de tierra, pues un ángel del Señor, bajado de cielo y acercándose, hizo rodar
      de su sitio la losa, y se sentó sobre ella.
      \begin{flushright}
        Mt. 28, 2        
      \end{flushright}      
      
      %QUINTO
      \textsuperscript{5}Tomando la palabra el ángel, dijo a las mujeres: «no tengáis miedo vosotras, que ya sé que buscáis a Jesús el crucificado;»
      \begin{flushright}
        Mt. 28, 5        
      \end{flushright}

      %SEXTO
      \textsuperscript{6}«no está aquí; resucitó, como dijo. Venid, ved el lugar dode estuvo puesto».
      \begin{flushright}
        Mt. 28, 6       
      \end{flushright}

      %SEPTIMO
      \textsuperscript{7}«Y marchando a toda prisa, decid a sus discípulos que resucitó de entre los muertos, y he aquí que se os adelanta en ir a Galilea:
      allí le veréis. Conque os lo tengo dicho».
      \begin{flushright}
        Mt. 28, 7     
      \end{flushright}

      %OCTAVO
      \textsuperscript{8}Y partiendo a toda prisa del monumento, con temor y grande gozo corrieron a dar la nueva a sus discípulos
      \begin{flushright}
        Mt. 28, 8       
      \end{flushright}

      %NOVENO
      \textsuperscript{25}«Yo soy la resurección y la vida; quien cree en mi, aun cuando muera, vivirá»
      \begin{flushright}
        Jn. 11, 25    
      \end{flushright}

      %DECIMO
      \textsuperscript{26}«y todo el que vive y cree en mi, no morirá para siempre. ¿Crees esto?»
      \begin{flushright}
        Jn. 11,26     
      \end{flushright}
    \subsection*{\hfil La Ascensión de Jesucristo a los cielos \hfil}
      
      %PRIMERO
      \textsuperscript{50}Y los sacó afuera hasta llegar junto a Betania, y alzando sus manos los bendijo.
      \begin{flushright}
        Lc. 24, 50     
      \end{flushright}

      %SEGUNDO
      \textsuperscript{18}Y acercándose Jesús, les habló diciendo: «Me fué dada toda potestad en el cielo y sobre la tierra».
      \begin{flushright}
        Mt. 28, 18      
      \end{flushright}

      %TERCERO
      \textsuperscript{19}«Id, pues, amaestrad a todas las gentes».
      \begin{flushright}
        Mt. 28, 19
      \end{flushright}

      %CUARTO
      \textsuperscript{19}«bautizándoles en el nombre del Padre y del Hijo y del Espíritu Santo».
      \begin{flushright}
        Mt. 28, 19      
      \end{flushright}

      %QUINTO
      \textsuperscript{20}«enseñándoles a guardar todas cuantas cosas os ordené».
      \begin{flushright}
        Mt. 28, 20  
      \end{flushright}

      %SEXTO
      \textsuperscript{16}El que creyere y fuere bautizado, se salvará;
      \begin{flushright}
        Mc. 16, 16
      \end{flushright}

      %SEPTIMO
      \textsuperscript{16}mas el que no creyere, será condenado.
      \begin{flushright}
        Mc. 16, 16
      \end{flushright}

      %OCTAVO
      \textsuperscript{28}«Y sabed que estoy con vosotros todos los días hasta la consumación de los siglos»
      \begin{flushright}
        Mt. 28, 20
      \end{flushright}

      %NOVENO
      \textsuperscript{51}Y aconteció que, mientras los bendecía, se desprendió de ellos, y era llevado en alto al cielo
      \begin{flushright}
        Lc. 24, 51
      \end{flushright}

      %DECIMO
      \textsuperscript{19}Con esto el Señor Jesús, despues de hablarles, fue elevado al cielo y se sentó a la diestra de Dios.
      \begin{flushright}
        Mc. 16, 19
      \end{flushright}
    \subsection*{\hfil La Venida del Espíritu Santo sobre los Apóstoles \hfil}

      %PRIMERO
      \textsuperscript{1}Y al cumplirse el día de Pentecostés, estaban todos juntos en el mismo lugar.
      \begin{flushright}
        Hch. 2, 1
      \end{flushright}

      %SEGUNDO
      \textsuperscript{2}Y se produjo de súbito desde el cielo un estruendo como de viento que soplaba vehemente, u llenó toda la casa
      donde se hallaban sentados.
      \begin{flushright}
        Hch. 2, 2
      \end{flushright}

      %TERCERO
      \textsuperscript{3}Y vieron aparecer lenguas como de fuego, que, repartiéndose, se posaban sobre cada uno de ellos.
      \begin{flushright}
        Hch. 2, 3
      \end{flushright}

      %CUARTO
      \textsuperscript{4}Y se llenaron todos del Espíritu Santo, y comenzaron a hablar en lenguas diferentes, según que el Espíritu Santo les movía
      a expresarse.
      \begin{flushright}
        Hch. 2, 4
      \end{flushright}

      %QUINTO
      \textsuperscript{5}Hallábanse en Jerusalén judíos allí domiciliados, hombres religiosos de toda nación de las que están debajo del cielo.
      \begin{flushright}
        Hch. 2, 5
      \end{flushright}

      %SEXTO
      \textsuperscript{14}Puesto de pie Pedro, acompañado de los Once, alzó en voz y les habló en estos términos
      \begin{flushright}
        Hch. 2, 14
      \end{flushright}

      %SEPTIMO
      \textsuperscript{38}«Arrepentíos, dice, y bautícese cada uno de vosotros en el nombre del Jesu-Cristo para remisión de vuestros pecados, y recibiréis el don
      del Espíritu Santo»
      \begin{flushright}
        Hch. 2, 38
      \end{flushright}

      %OCTAVO
      \textsuperscript{41}Ellos, pues, acogieron su palabra, fueron bautizados; y fueron agragados en aqueñ día como unas tres mil almas.
      \begin{flushright}
        Hch. 2,41
      \end{flushright}

      %NOVENO
      \textsuperscript{30}SI tu Espíritu envías, son creados, y la faz de la tierra así renuevas.
      \begin{flushright}
        Sal. 104, 30
      \end{flushright}

      %DECIMO
      ¡Ven, Espíritu Santo, y desde el Cielo envía rayos de tu virtud!¡!Ven, Padre de los pobres!¡Ven, Dador de tus Dones!¡Ven, de almas luz!
      \begin{flushright}
        Secuencia de Pentecostés
      \end{flushright}

    \subsection*{\hfil La Asunción de Nuestra Señora a los cielos \hfil}

      %PRIMERO
      \textsuperscript{18}Bendita tú, hija, ante el Dios Altísimo sobre todas las mujeres de la tierra...
      \begin{flushright}
        Jdt. 13, 18
      \end{flushright}

      %SEGUNDO
      \textsuperscript{19}Pues no se apartará etérnamente tu esperanza del corazón de los hombre...
      \begin{flushright}
        Jdt. 13, 19
      \end{flushright}

      %TERCERO
      \textsuperscript{20}Y esto haga contigo Dios para eterno encubrimiento, que te visite con sus bienes; por cuanto no perdonate
      a tu vida, lastimada por la humillación de nuestro linaje...
      \begin{flushright}
        Jdt. 13, 20
      \end{flushright}

      %CUARTO
      \textsuperscript{9}...Tú eres enaltecimiento de Jerusalén, tú gloria grande de Israel, tú grande honor de nuestro linaje
      \begin{flushright}
        Jdt. 15, 9
      \end{flushright}

      %QUINTO
      \textsuperscript{11}Oye, hija, mira; tu oído aplica; tu pueblo olvida y la mansión paterna; \textsuperscript{12}deja que tu hermosura
      el rey codicie, que es tu señor. A él le doblega
      \begin{flushright}
        Sal. 45; 11-12
      \end{flushright}

      %SEXTO
      \textsuperscript{19}Y se abrió el templo de Dios, que está en el cielo, y fué vista el arca de la alianza en el templo,
      y se produjeron relámpagos, y voces, y truenos, y temblor de tierra, y fuerte granizada.
      \begin{flushright}
        Ap. 11, 19
      \end{flushright}

      %SEPTIMO
      \textsuperscript{1}Y una gran señal fué vista en el cielo: una Mujer vestida del sol...
      \begin{flushright}
        Ap. 12, 1
      \end{flushright}

      %OCTAVO
      \textsuperscript{1}...y la luna debajo de sus pies, y sobre su cabeza una corona de doce estrellas.
      \begin{flushright}
        Ap. 12, 1
      \end{flushright}

      %NOVENO
      \textsuperscript{14}Del rey la hija toda hermosa entra; vestidos áureos se adorno son.
      \begin{flushright}
        Sal. 45, 14
      \end{flushright}

      %DECIMO
      \textsuperscript{1}Entonad a Yahveh cántico nuevo, que portentos ha obrado. Su diestra le ha traído la victoria y aquel su brazo santo.
      \begin{flushright}
        Sal. 98, 1
      \end{flushright}
    \subsection*{\hfil La Coronación de la Santísima Virgen María \hfil}

      %PRIMERO
      \textsuperscript{10}¿Quién es esa que aparece resplandeciente como la aurora,
      hermosa cual luna, deslumbradora como el sol, imponente como batallones?
      \begin{flushright}
        Cant. 6, 10
      \end{flushright}

      %SEGUNDO
      \textsuperscript{8}Como el arco iris, que se aparece en las nubes; como flor entre el ramaje en días primaverales, como azucena junto
      a la corriente de las aguas, como las flores del Líbano en días de verano; \textsuperscript{9}como el incienso que arde sobre la ofrenda,
      como el vaso de oro fínamente trabajado.
      \begin{flushright}
        Eclo. 50, 8-9
      \end{flushright}

      %TERCERO
      \textsuperscript{24}Yo soy la madre del amor, del temor, de la ciencia y de la santa esperanza.
      \begin{flushright}
        Eclo. 24, 24
      \end{flushright}

      %CUARTO
      \textsuperscript{25}EN mi está toda la gracia del camino y de la verdad, en mi toda esperanza de la vida y de la virtud.
      \begin{flushright}
        Eclo. 24, 25
      \end{flushright}

      %QUINTO
      \textsuperscript{26}Venid a mí cuantos me deseáis y saciaos de mis frutos
      \begin{flushright}
        Eclo. 24, 26
      \end{flushright}

      %SEXTO
      \textsuperscript{27}Porque recordarme es más dulce que la miel, y poseerme más rico que el panal de miel.
      \begin{flushright}
        Eclo. 24, 27
      \end{flushright}

      %SEPTIMO
      \textsuperscript{32}Ahora, pues, hijos míos, oídme; y felices quienes guardan mis caminos. \textsuperscript{33}Escochad la corrección y sed sabios,
      \begin{flushright}
        Prov. 8, 32-33
      \end{flushright}

      %OCTAVO
      \textsuperscript{33}y no la rechacéis. Feliz el hombre que me escucha, velando a mis puertas cada día,
      guardando las jambas de mis entradas
      \begin{flushright}
        Prov. 8, 33-34
      \end{flushright}

      %NOVENO
      \textsuperscript{35}Pues quien me halla, ha hallado la vida y alcanza el favor de Yahveh. Más quien peca contra mi, se perjudica a si mismo,
      y cuantos me odian aman la muerte.
      \begin{flushright}
        Prov. 8, 35
      \end{flushright}

      %DECIMO
      \textsuperscript{16}Tiene Él escrito en su vestido y en su manto Rey de reyes y Señor de los que dominan\textsubscript{Ap. 18, 16}.  \\
      \textsuperscript{10}Está la Reina a su derecha, adornada con oro finísimo\textsubscript{Sal. 44, 10}.
      \begin{flushright}
        Ap. 18, 16. Sal. 44, 10
      \end{flushright}

\end{document}
