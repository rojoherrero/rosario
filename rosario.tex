\documentclass[a4paper,11pt]{article}

\usepackage[utf8]{inputenc}
\usepackage[spanish]{babel}
\usepackage{multicol}
\usepackage{subfiles}

\begin{document}
  \section*{\hfil Misterios Gozosos \hfil}
    \subsection*{\hfil La Anunciación de la Santísima Virgen María \hfil}

      % PRIMER
      \subfile{mysteries/joyful/joyful_01_01}
      \medskip

      %SEGUNDO
      \subfile{mysteries/joyful/joyful_01_02}
      \medskip

      %TERCERO
      \subfile{mysteries/joyful/joyful_01_03}
      \medskip

      %CUARTO
      \subfile{mysteries/joyful/joyful_01_04}
      \medskip

      %QUINTO
      \subfile{mysteries/joyful/joyful_01_05}
      \medskip

      %SEXTO
      \subfile{mysteries/joyful/joyful_01_06}
      \medskip

      %SEPTIMO
      \subfile{mysteries/joyful/joyful_01_07}
      \medskip
      
      %OCTAVO
      \subfile{mysteries/joyful/joyful_01_08}
      \medskip

      %NOVENO
      \subfile{mysteries/joyful/joyful_01_09}
      \medskip

      %DECIMO
      \subfile{mysteries/joyful/joyful_01_10}
      \medskip
            
    \subsection*{\hfil La Visitación de Nuestra Señora \hfil}
      
      % PRIMER
      \subfile{mysteries/joyful/joyful_02_01}
      \medskip
      
      %SEGUNDO
      \subfile{mysteries/joyful/joyful_02_02}
      \medskip
      
      %TERCERO
      \subfile{mysteries/joyful/joyful_02_03}
      \medskip
      
      %CUARTO
      \subfile{mysteries/joyful/joyful_02_04}
      \medskip
      
      %QUINTO
      \subfile{mysteries/joyful/joyful_03_05}
      \medskip

      %SEXTO
      \subfile{mysteries/joyful/joyful_02_06}
      \medskip
      
      %SEPTIMO
      \subfile{mysteries/joyful/joyful_02_07}
      \medskip
      
      %OCTAVO
      \subfile{mysteries/joyful/joyful_02_08}
      \medskip
      
      %NOVENO
      \subfile{mysteries/joyful/joyful_02_09}
      \medskip
      
      %DECIMO
      \subfile{mysteries/joyful/joyful_02_10}
      \medskip
            
    \subsection*{\hfil La Natividad de Nuestro Señor Jesucristo \hfil}
      
      % PRIMER
      \subfile{mysteries/joyful/joyful_03_01}
      \medskip
      
      %SEGUNDO
      \subfile{mysteries/joyful/joyful_03_02}
      \medskip
      
      %TERCERO
      \subfile{mysteries/joyful/joyful_03_03}
      \medskip
      
      %CUARTO
      \subfile{mysteries/joyful/joyful_03_04}
      \medskip
      
      %QUINTO
      \subfile{mysteries/joyful/joyful_03_05}
      \medskip

      %SEXTO
      \subfile{mysteries/joyful/joyful_03_06}
      \medskip
      
      %SEPTIMO
      \subfile{mysteries/joyful/joyful_03_07}
      \medskip

      %OCTAVO
      \subfile{mysteries/joyful/joyful_03_08}
      \medskip
      
      %NOVENO
      \subfile{mysteries/joyful/joyful_03_09}
      \medskip
      
      %DECIMO
      \subfile{mysteries/joyful/joyful_03_10}

            
    \subsection*{\hfil La Presentación del Niño Jesús en el Templo \hfil}
      
      % PRIMER
      \subfile{mysteries/joyful/joyful_04_01}
      \medskip
      
      %SEGUNDO
      \subfile{mysteries/joyful/joyful_04_02}
      \medskip
      
      %TERCERO
      \subfile{mysteries/joyful/joyful_04_03}
      \medskip
      
      %CUARTO
      \subfile{mysteries/joyful/joyful_04_04}
      \medskip
      
      %QUINTO
      \subfile{mysteries/joyful/joyful_04_05}
      \medskip

      %SEXTO
      \subfile{mysteries/joyful/joyful_04_06}
      \medskip
      
      %SEPTIMO
      \subfile{mysteries/joyful/joyful_04_07}
      \medskip
      
      %OCTAVO
      \subfile{mysteries/joyful/joyful_04_08}
      \medskip
      
      %NOVENO
      \subfile{mysteries/joyful/joyful_04_09}
      \medskip
      
      %DECIMO
      \subfile{mysteries/joyful/joyful_04_10}
      \medskip
            
    \subsection*{\hfil La Pérdida y Hallazgo del Niño Jesús en el Templo \hfil}
      
      % PRIMER
      \subfile{mysteries/joyful/joyful_05_01}
      \textsuperscript{42}Y cuando fué de doce años, habiendo ellos subido, según la costumbre de la fiesta,
      \begin{flushright}
        Lc. 2, 42     
      \end{flushright}
      
      %SEGUNDO
      \subfile{mysteries/joyful/joyful_05_02}
      \textsuperscript{43}y acabados los días, al volverse ellos, quedóse el niño Jesús en Jerusalén, sin que lo advirtiesen sus padres
      \begin{flushright}
        Lc. 2, 43        
      \end{flushright}
      
      %TERCERO
      \subfile{mysteries/joyful/joyful_05_03}
      \textsuperscript{45}y no hallándole, se tornaron a Jerusalén para burcarlo. \textsuperscript{46}Y sucedió que después de tres días le hallaron en el templo,
      sentado en medio de los maestros, escuchándoles y haciéndoles preguntas;
      \begin{flushright}
        Lc. 2, 45-46       
      \end{flushright}
      
      %CUARTO
      \subfile{mysteries/joyful/joyful_05_04}
      sentado en medio de los maestros, escuchándoles y haciéndoles preguntas;
      \begin{flushright}
        Lc. 2, 46       
      \end{flushright}
      
      %QUINTO
      \subfile{mysteries/joyful/joyful_05_05}
      \textsuperscript{47}y se pasmaban todos los que le oían de su inteligencia y de sus respuestas.
      \begin{flushright}
        Lc. 2, 47     
      \end{flushright}

      %SEXTO
      \subfile{mysteries/joyful/joyful_05_06}
      \textsuperscript{48}Y sus padres, al verle, quedaron sorprendidos; y le dijo su madre: ``hijo, ¿por qué lo hiciste así con nosotros? Mira que tu padre
      y yo, llenos de aflicción, te andábamos buscando''
      \begin{flushright}
        Lc. 2, 48     
      \end{flushright}
      
      %SEPTIMO
      \subfile{mysteries/joyful/joyful_05_07}
      \textsuperscript{49}Díjoles Él: ``¿pues por qué me buscabais? ¿No sabíais que había yo de estar en casa de mi padre?''
      \begin{flushright}
        Lc. 2, 49        
      \end{flushright}
      
      %OCTAVO
      \subfile{mysteries/joyful/joyful_05_08}
      \textsuperscript{50}Y ellos no comprendieron lo que les dijo.
      \begin{flushright}
        Lc. 2, 50        
      \end{flushright}
      
      %NOVENO
      \subfile{mysteries/joyful/joyful_05_09}
      \textsuperscript{51}Y bajó en su compañía y se fue a Nazaret, y vivía sometido a ellos. Y su madre guardaba todas estas
      cosas en su corazón.
      \begin{flushright}
        Lc. 2,51       
      \end{flushright}      
      
      %DECIMO
      \subfile{mysteries/joyful/joyful_05_10}
      \textsuperscript{52}Y Jesús progresaba en sabiduría, en estatura y en gracia delante de Dios y de los hombres.
      \begin{flushright}
        Lc. 2, 52        
      \end{flushright}
            
    \newpage
        
  \section*{\hfil Misterios Dolorosos \hfil}
    \subsection*{\hfil La Agonía de Nuestro Señor en el Huerto de los Olivos \hfil}
      
      % UNO
      \textsuperscript{36}Entonces llego Jesús con ellos a una granja llamada Getsemaní, y dice a los discípulos: ``Sentaos aquí mientras voy allá para hacer oración''. 
      \textsuperscript{37}Y llevando consigo a Pedro y a los dos hijos de Zebedeo, comenzó a ponerse triste y a sentir abatimiento.
      \begin{flushright}
        Mt. 26; 36-37     
      \end{flushright}

      % DOS
      \textsuperscript{38}Entonces les dice: ``triste em gran manera está mi alma hasta la muerte; quedad aquí y velad conmigo''
      \begin{flushright}
        Mt. 26, 38
      \end{flushright}

      % TRES
      \textsuperscript{35}Y adelantándose un poco, caía sobre la tierra, y rogaba que, a ser posible, pasase de Él aquella hora
      \begin{flushright}
        Mc. 14, 35  
      \end{flushright}

      % CUATRO
      \textsuperscript{42}diciendo: ``Padre, si quieres, traspasa de mi este cáliz; mas no se haga mi voluntad, sino la tuya''.
      \begin{flushright}
        Lc. 22, 42
      \end{flushright}

      % CINCO
      \textsuperscript{43}Y se le apareció un ángel venido del cielo, que le confortaba
      \begin{flushright}
        Lc. 22, 43
      \end{flushright}

      % SEIS
      \textsuperscript{44}Y venido en agonía, oraba más intensamente. 
      \begin{flushright}
        Lc. 22, 44
      \end{flushright}

      % SIETE
      Y se hizo su sudor como grumos de sangre, que caían hasta el suerlo.
      \begin{flushright}
        Lc. 22, 44
      \end{flushright}

      % OCHO
      \textsuperscript{40}Y viene a los discípulos y los halla durmiendo, y dice a Pedro: ``¿así no pudísteis velar una hora conmigo?''
      \begin{flushright}
        Mt. 26, 40
      \end{flushright}

      % NUEVE
      \textsuperscript{41}``Velad y orad, para que no entréis en tentación;''
      \begin{flushright}
        Mt. 26, 41
      \end{flushright}

      % DIEZ
      ``el espíritu sí, está animoso, más la carne es flaca''
      \begin{flushright}
        Mt. 26, 41
      \end{flushright}

    \subsection*{\hfil La flagelación de Nuestro Señor Jesucristo \hfil}
      
      % UNO
      \textsuperscript{1}Y luego al amanecer, después de celebrar consejo, los sumos sacerdotes con los ancianos y los escribas, es decir, todo el sanhedrín, atando a Jesús,
      le llevaron de allí y le entregaron a Pilato. \textsuperscript{2}Y le interrogó Pilato: ``¿Tú eres el Rey de los judíos?''. El le respondió: ``Tú lo dices''.
      \begin{flushright}
        Mc. 15, 1-2
      \end{flushright}

      % DOS
      \textsuperscript{36}Respondió: ``Mi reino no es de este mundo. Si de este mundo fuera mi reino, mis ministros lucharían para que yo no fuera entregado a los judíos
      Más ahora mi reino no es de aquí''.
      \begin{flushright}
        Jn. 18, 36
      \end{flushright}

      % TRES
      \textsuperscript{37}Díjole, pues, Pilato: ``¿luego, rey eres tu?''. Respondión Jesús : ``Tú dices que yo soy rey. Yo para eso he nacido y para esto he venido
      al mundo: para dar testimonio a favor de la verdad. Todo el que es de la verdad, oye mi voz''.
      \begin{flushright}
        Jn. 18, 37
      \end{flushright}

      % CUATRO
      \textsuperscript{4}Pilato dijo a los sumos sacerdotes y a los turbas: ``ningún delito hallo en este hombre. \textsuperscript{16}Le castigaré, pues, y le soltaré''.
      \begin{flushright}
        Lc. 23, 4, 16
      \end{flushright}

      % CINCO
      \textsuperscript{1}Entonces, pues, tomó Pilato a Jesús y le azotó.
      \begin{flushright}
        Jn. 19, 1
      \end{flushright}

      % SEIS
      \textsuperscript{8}De opresión u juicio fué tomado, y a sus contemporáneos, ¿quién tendrá en cuenta?. \textsuperscript{3}Fué despreciado y abandonado de los hombres,
      varón de dolores y familiarizado de los hombres
      \begin{flushright}
        Is. 53, 8, 3
      \end{flushright}

      % SIETE
      \textsuperscript{4}Mas nuestros sufrimientos él los ha llevado, nuestros dolores él los cargó sobre sí,
      \begin{flushright}
        Is. 53, 4
      \end{flushright}

      % OCHO
      \textsuperscript{5}Fué traspasado por causa de nuestros pecados, molido por causa de nuestras iniquidades; 
      \begin{flushright}
        Is. 53, 5
      \end{flushright}

      % NUEVE
      mientras nosotros le tuvimos por azotado, por herido de Dios y abatido
      \begin{flushright}
        Is. 53, 4 
      \end{flushright}

      % DIEZ
      el castigo de nuestra paz cayó sobre Él y por sus verdugones se nos curó.
      \begin{flushright}
        Is. 53, 5
      \end{flushright}
      
    \subsection*{\hfil La coronación de espinas de Nuestro Señor Jesucristo \hfil}
      
      % UNO
      \textsuperscript{16}Los soldados se lo llevaron dentro del palacio, que es el pretorio, y convocan a toda la cohorte (Marcos).
      \begin{flushright}
        Mc. 15, 16
      \end{flushright}
      \textsuperscript{28}Y habiéndole quitado sus vestidos, le envolvieron en uns clámide de grana (Mateo).
      \begin{flushright}
        Mt. 27, 28
      \end{flushright}

      % DOS
      \textsuperscript{29}y trenzando una corna de espinas, la pusieron sobre la cabeza, y una caña en la mano derecha; 
      \begin{flushright}
        Mt. 27, 29
      \end{flushright}

      % TRES
      y doblando la rodilla delante de Él, le mofaban, diciendo: ``Salud, Rey de los judíos''
      \begin{flushright}
        Mt. 27, 29
      \end{flushright}

      % CUATRO
      \textsuperscript{30}Y escupiendo en Él, tomaron la caña y le daba golpes en la cabeza.
      \begin{flushright}
        Mt 27, 30
      \end{flushright}

      % CINCO
      \textsuperscript{4}Salió Pilato otra vez fuera, y les dice: ``Ved, os le traigo para que conozcáis que no hallo en Él delito alguno''.
      \begin{flushright}
        Jn. 19, 4
      \end{flushright}

      % SEIS
      \textsuperscript{5}Salió, pues, Jesús afuera, llevando la corona de espinas y el manto de púrpura. Y les dice: ``ved aquí el hombre''
      \begin{flushright}
        Jn. 19, 5
      \end{flushright}

      % SIETE
      \textsuperscript{5}Y les dice: ``ved aquí el hombre''. \textsuperscript{15}Gritaron, pues, ellos: ``quita, quita; crucifícale''
      \begin{flushright}
        Jn. 19, 5, 15
      \end{flushright}

      % OCHO
      \textsuperscript{14}Pilato, queriendo dar satisfacción a la turba les soltó a Barrabás. Y entregó a Jesús, después de azotarle, para que fuese crucificado.
      \begin{flushright}
        Mc. 15, 14
      \end{flushright}

      % NUEVE
      Díceles Pilato: ``¿A vuestro rey he de crucificar?''. Respondieron los pontífices: ``no tenemos rey, sino César''
      \begin{flushright}
        Jn. 19, 15
      \end{flushright}

      % DIEZ
      \textsuperscript{16}Entonces, pués, se le entregó para que fuera crucificado. Se apoderaron, pues, de Jesús.
      \begin{flushright}
        Jn. 19, 16
      \end{flushright}

    \subsection*{\hfil Jesús con la Cruz a cuestas \hfil}
      
      % UNO
      \textsuperscript{23}Si alguno quiere venir en pos de Mí, niégese a sí mismo
      \begin{flushright}
        Lc. 9, 23
      \end{flushright}

      % DOS
      y tome a cuestas su cruz cada día y sígame
      \begin{flushright}
        Lc. 9, 23
      \end{flushright}

      % TRES
      \textsuperscript{17}y, llevando a cuestas su cruz, salió hacia el lugar llamado el Cráneo, que en hebreo se dice Gólgota.
      \begin{flushright}
        Jn. 19, 17
      \end{flushright}

      % CUATRO
      \textsuperscript{26}Y como le hubieron sacado, echaron mano de un tal Simón de Cirene que venía del campo, le pusieron en hombros la cruz para que la llevase
      detrás de Jesús.
      \begin{flushright}
        Lc. 23, 26
      \end{flushright}

      % CINCO
      \textsuperscript{29}Tomad mi yugo sobre vuestros, y aprended de mi, 
      \begin{flushright}
        Mt. 11, 29  
      \end{flushright}

      % SEIS
      pues soy manso y humilde de Corazón, y hallaréis reposo para vuestras almas.
      \begin{flushright}
        Mt. 11, 29
      \end{flushright}

      % SIETE
      Porque mi yugo es suave, y mi carga, ligera.
      \begin{flushright}
        Mt. 11, 30
      \end{flushright}

      % OCHO
      \textsuperscript{27}Seguíanle gran mucheduncbre de pueblo y de mujeres, las cuales le plañían y lamentaban.
      \begin{flushright}
        Lc. 23, 27
      \end{flushright}

      % NUEVE
      \textsuperscript{28}Volviéndose Jesús a ellas, les dijo: ``Hijas de Jerusalén: no lloréis sobre mi, sino llorad má bien sobre vosotras mismas y sobre
      vuestros hijos''
      \begin{flushright}
        Lc. 23, 28
      \end{flushright}

      % DIEZ
      \textsuperscript{31}``Porque si en el leño verde esto hacen, ¿en el seco qué se hará?''.
      \begin{flushright}
        Lc. 23, 31
      \end{flushright}

    \subsection*{\hfil La Crucifixión y Muerte del Redentor \hfil}
      
      % UNO
      \textsuperscript{33}Y cuando hubieron llegado al lugar llamado ``Cráneo'', allí crucificaron a Él y a los malhechores, uno a la derecha y el otro a la izquierda.
      \begin{flushright}
        Lc. 23, 33
      \end{flushright}

      % DOS
      \textsuperscript{34}Y Jesús decía: ``Padre, perdónalos, porque no saben lo que hacen''.
      \begin{flushright}
        Lc. 23, 34
      \end{flushright}

      % TRES
      \textsuperscript{39}Uno de los malhechores que estaba colgado \textsuperscript{42}decía a Jesús: ``acuérdate de mi cuando vinieres en la gloria de tu realeza''.
      \begin{flushright}
        Lc. 23, 39, 42
      \end{flushright}

      % CUATRO
      \textsuperscript{43}Díjole: ``en verdad te digo que hoy estarás conmigo en el paraíso''
      \begin{flushright}
        Lc. 23, 43
      \end{flushright}

      % CINCO
      \textsuperscript{26}Jesús, pues, viendo a la Madre, y junto a ella al discípulo a quien amaba,
      \begin{flushright}
        Jn. 19, 26
      \end{flushright}

      % SEIS
      \textsuperscript{27}dice a tu Madre: ``mujer, he ahí a tu hijo''. \textsuperscript{27}Luego dice al discípulo: ``He aquí a tu Madre''
      \begin{flushright}
        Jn. 19, 26-27
      \end{flushright}

      % SIETE
      Y desde aquella hora la tomó el discípulo en su compañía.
      \begin{flushright}
        Jn. 19, 27
      \end{flushright}

      % OCHO
      \textsuperscript{45}habiendo faltado el sol; y se rasgó por medio el velo del santuario.
      \begin{flushright}
        Lc. 23, 45
      \end{flushright}

      % NUEVE
      \textsuperscript{46}Y clamando con voz poderosa, Jesús dijo: ``Padre, en tus manos encomiendo mis espíritu''.
      \begin{flushright}
        Lc. 23, 46 
      \end{flushright}

      % DIEZ
      \textsuperscript{30}Jesús dijo: ``Consumado está''. E inclinando la cabeza entregó el espíritu.
      \begin{flushright}
        Jn. 19, 30
      \end{flushright}
 
    \newpage
         
  \section*{\hfil Miserios Gloriosos \hfil}
    \subsection*{\hfil La Resurección del Señor \hfil}
      \begin{multicols}{2}

      \columnbreak
                           
      \end{multicols}
      \begin{flushright}
        Lc. 1,26- 27           
      \end{flushright}
    \subsection*{\hfil La Ascensión de Jesucristo a los cielos \hfil}
      \begin{multicols}{2}

      \columnbreak
                           
      \end{multicols}         
      \begin{flushright}
        Lc. 1,26- 27           
      \end{flushright}
    \subsection*{\hfil La Venida del Espíritu Santo sobre los Apóstoles \hfil}
      \begin{multicols}{2}

      \columnbreak
                           
      \end{multicols}         
      \begin{flushright}
        Lc. 1,26- 27           
      \end{flushright}
    \subsection*{\hfil La Asunción de Nuestra Señora a los cielos \hfil}
      \begin{multicols}{2}

      \columnbreak
                           
      \end{multicols}         
      \begin{flushright}
        Lc. 1,26- 27           
      \end{flushright}
    \subsection*{\hfil La Coronación de la Santísima Virgen María \hfil}
      \begin{multicols}{2}

      \columnbreak
                           
      \end{multicols}         
      \begin{flushright}
        Lc. 1,26- 27           
      \end{flushright}
\end{document}
