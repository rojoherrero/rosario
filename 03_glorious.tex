\documentclass[./rosary.tex]{subfiles}
\newcounter{glorious-counter}

\begin{document}
\section*{Misterios Gloriosos.}
\begin{itemize}
      \item Tradicional: miércoles, sábados y domingos.
      \item Nuevo: miércoles y domingos.
\end{itemize}

\stepcounter{glorious-counter}
\subsection*{\Roman{glorious-counter} Misterio: La Resurrección del Señor}

Tomando la palabra el ángel, dijo a las mujeres: «no tengáis miedo vosotras, que ya sé que buscáis a Jesús el crucificado;
no está aquí; resucitó, como dijo. Venid, ved el lugar donde estuvo puesto. Y marchando a toda prisa,
decid a sus discípulos que resucitó de entre los muertos, y he aquí que se os adelanta en ir a Galilea: allí le veréis.
Conque os lo tengo dicho». Y partiendo a toda prisa del monumento, con temor y grande gozo corrieron a dar la nueva a sus discípulos.

\begin{flushright}
      \emph{Mateo 28, 5-8}
\end{flushright}

Paternóster, diez Avemarías, Gloria y María, Madre de gracia{\ldots}

\bigskip

\stepcounter{glorious-counter}
\subsection*{\Roman{glorious-counter} Misterio: Las Ascensión Jesucristo a los cielos}

[Y les dijo:] «Id al mundo entero y predicad el Evangelio a toda la creación. El que creyere y fuere bautizado,
se salvará, mas el que no creyere, será condenado. Con esto el Señor Jesús, despues de hablarles,
fue elevado al cielo y se sentó a la diestra de Dios. Y ellos, partiéndose de allí,
predicaron por todas partes, cooperando el Señor y confirmando la palabra con las señales que la acompañaban.

\begin{flushright}
      \emph{Marcos 16, 15-16.19-20}
\end{flushright}

Paternóster, diez Avemarías, Gloria y María, Madre de gracia{\ldots}

\bigskip

\stepcounter{glorious-counter}
\subsection*{\Roman{glorious-counter} Misterio glorioso: La Venida del Espíritu Santo sobre los Apóstoles}

Y al cumplirse el día de Pentecostés, estaban todos juntos en el mismo lugar. Y se produjo de súbito desde el cielo un estruendo como de viento que soplaba vehementemente,
y llenó toda la casa donde se hallaban sentados. Y vieron aparecer lenguas como de fuego, que, repartiéndose, se posaban sobre cada uno de ellos.

\begin{flushright}
      \emph{Hechos 2, 1-4}
\end{flushright}

Paternóster, diez Avemarías, Gloria y María, Madre de gracia{\ldots}

\bigskip

\stepcounter{glorious-counter}
\subsection*{\Roman{glorious-counter} Misterio: La Asunción de María Santísima a los cielos}

Y se abrió el templo de Dios, que está en el cielo, y fué vista el arca de la alianza en el templo,
y se produjeron relámpagos, y voces, y truenos, y temblor de tierra, y fuerte granizada.

\begin{flushright}
      \emph{Apocalipsis 11, 19}
\end{flushright}

Paternóster, diez Avemarías, Gloria y María, Madre de gracia{\ldots}

\bigskip

\stepcounter{glorious-counter}
\subsection*{\Roman{glorious-counter} Misterio: La Coronación de la Santísima Virgen María}

Y una gran señal fué vista en el cielo: una Mujer vestida del sol, y la luna debajo de sus pies, y sobre su cabeza una corona de doce estrellas.

\begin{flushright}
      \emph{Apocalipsis 12, 1}
\end{flushright}

Paternóster, diez Avemarías, Gloria y María, Madre de gracia{\ldots} y oraciones finales (\cpageref{sec:final-prayer}).


\end{document}