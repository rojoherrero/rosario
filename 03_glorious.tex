\documentclass[./rosary.tex]{subfiles}
\newcounter{glorious-counter}

\begin{document}
\section*{Misterios Gloriosos.}
\begin{itemize}
      \item Tradicional: miércoles, sábados y domingos.
      \item Nuevo: miércoles y domingos.
\end{itemize}

\stepcounter{glorious-counter}
\subsection*{\Roman{glorious-counter} Misterio: La Resurrección del Señor}

Pasado el sábado, ya para amanecer el día primero de la semana, vino María Magdalena con la otra María al sepulcro. Y sobrevino un gran terremoto, 
pues un ángel del Señor bajó del cielo y acercándose removió la piedra del sepulcro y se sentó sobre ella. Era su aspecto como el relámpago, 
y su vestidura blanca como la nueve. De miedo de él temblaron los guardias y se quedaron como muertos. El ángel, dirigiéndose a las mujeres, dijo:
No temáis vosotras, pues sé que buscáis a Jesús el crucificado. No está aquí; ha resucitado, según lo había dicho. Venid y ved el sitio donde fue puesto.
Id luego y decid a sus discípulos que ha resucitado de entre los muertos y que os precede a Galilea; allí le veréis. Es lo que tenía que deciros. 
\textbf{\emph{Mateo 28, 1-7}}

\begin{center}
      Paternóster, diez Avemarías, Gloria y María, Madre de gracia{\ldots}
\end{center}


\bigskip

\stepcounter{glorious-counter}
\subsection*{\Roman{glorious-counter} Misterio: Las Ascensión Jesucristo a los cielos}

[Y les dijo:] «Id al mundo entero y predicad el Evangelio a toda la creación. El que creyere y fuere bautizado, se salvará, mas el que no creyere, será condenado. 
Con esto el Señor Jesús, despues de hablarles, fue elevado al cielo y se sentó a la diestra de Dios. Y ellos, partiéndose de allí, predicaron por todas partes, 
cooperando el Señor y confirmando la palabra con las señales que la acompañaban. 
\textbf{\emph{Marcos 16, 15-16.19-20}}

\begin{center}
      Paternóster, diez Avemarías, Gloria y María, Madre de gracia{\ldots}
\end{center}

\bigskip

\stepcounter{glorious-counter}
\subsection*{\Roman{glorious-counter} Misterio: La Venida del Espíritu Santo sobre los Apóstoles}

Y comiendo con ellos, los mandó no apartarse de Jerusalén, sino esperar la promesa del Padre, que de mi habéis escuchado: porque Juan bautizó en agua, pero vosotros, pasados no muechos días,
seréis bautizados en el Espíritu Santo. El les dijo: No os toca a vosotros conocer los tiempos ni los momentos que el Padre ha fijado en virtud de su poder soberano; pero recibiréis 
la virtud del Espíritu Santo, que descenderá sobre vosotros, y seréis mis testigos en Jerusalén, en toda Judea, en Samaria y hasta los extremos de la tierra. Y al cumplirse el día de Pentecostés, 
estaban todos juntos en el mismo lugar. Y se produjo de súbito desde el cielo un estruendo como de viento que soplaba vehementemente, y llenó toda la casa donde se hallaban sentados. 
Y vieron aparecer lenguas como de fuego, que, repartiéndose, se posaban sobre cada uno de ellos. \textbf{\emph{Hechos 1, 4-5.7-8; 2, 1-4}}

\begin{center}
      Paternóster, diez Avemarías, Gloria y María, Madre de gracia{\ldots}
\end{center}

\bigskip

\stepcounter{glorious-counter}
\subsection*{\Roman{glorious-counter} Misterio: La Asunción de María Santísima a los cielos}

Mi alma magnifica al Señor y exulta de júbilo mi espíritu al Señor, mi Salvador, porque ha mirado la humildad de su sierva; por eso todas las generaciones
me llamarán bienaventurada, porque ha hecho en mi maravillas el Poderoso, cuyo nombre es santo. | Y se abrió el templo de Dios, que está en el cielo, y 
fué vista el arca de la alianza en el templo, y se produjeron relámpagos, y voces, y truenos, y temblor de tierra, y fuerte granizada. 
\textbf{\emph{Lucas 1, 46-49} | \emph{Apocalipsis 11, 19}}

\begin{center}
      Paternóster, diez Avemarías, Gloria y María, Madre de gracia{\ldots}
\end{center}

\bigskip

\stepcounter{glorious-counter}
\subsection*{\Roman{glorious-counter} Misterio: La Coronación de la Santísima Virgen María}

Y una gran señal fué vista en el cielo: una Mujer vestida del sol, y la luna debajo de sus pies, y sobre su cabeza una corona de doce estrellas. 
\textbf{\emph{Apocalipsis 12, 1}}

\begin{center}
      Paternóster, diez Avemarías, Gloria y María, Madre de gracia{\ldots}

      Oraciones finales (\cpageref{sec:final-prayer}).
\end{center}

\end{document}