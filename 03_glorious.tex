\documentclass[./rosary.tex]{subfiles}

\begin{document}
\section*{Misterios Gloriosos.}
Miércoles, sábados y domingos de Pascua y Pentecostés.

\subsection*{I Misterio: La Resurrección del Señor}

Tomando la palabra el ángel, dijo a las mujeres: «no tengáis miedo vosotras, que ya sé que buscáis a Jesús el crucificado;
no está aquí; resucitó, como dijo. Venid, ved el lugar donde estuvo puesto. Y marchando a toda prisa,
decid a sus discípulos que resucitó de entre los muertos, y he aquí que se os adelanta en ir a Galilea: allí le veréis.
Conque os lo tengo dicho». Y partiendo a toda prisa del monumento, con temor y grande gozo corrieron a dar la nueva a sus discípulos.

\begin{flushright}
    \emph{Mateo 28, 5-8}
\end{flushright}

1 Paternoster, 10 Avemarías, 1 Gloria, ¡On Jesús mío!... y María, Madre de gracia... (\cpageref{endTenPrayers})

\rule{\textwidth}{0.5pt}

\begin{enumerate}
    \item \textbf{\emph{Paternóster}}. Alzase Dios, y al presentarse ante ellos desbándese sus enemigos, y huyen aquellos que le odiaron. \emph{Sal 68, 2}. \textbf{\emph{Avemaría}}.

    \item Y si Cristo no resucitó, vana es nuestra predicación. Vana es nuestra fe. \emph{1Cor. 15, 14}. \textbf{\emph{Avemaría}}.

    \item Bienaventurados los que están afligidos, porque ellos serán consolados. \emph{Mt 5,5}. \textbf{\emph{Avemaría}}.

    \item «Pues así también vosotros, ahora cierto tenéis congoja; mas otra vez os veré, y se gozará vuestro corazon, y vuestro gozo nadie os lo quita». \emph{Jn. 16, 22}. \textbf{\emph{Avemaría}}.

    \item De pronto se produjo un gran temblor de tierra, pues un ángel del Señor, bajado de cielo y acercándose, hizo rodar de su sitio la losa, y se sentó sobre ella.
          Era su aspecto como relámpago, y su vestidura blanca como la nieve. Tomando la palabra el ángel, dijo a las mujeres: «no tengáis miedo vosotras, que ya sé que buscáis
          a Jesús el crucificado; no está aquí; resucitó, como dijo. Venid, ved el lugar donde estuvo puesto». \emph{Mt. 28, 2.5-6}. \textbf{\emph{Avemaría}}.

    \item «Y marchando a toda prisa, decid a sus discípulos que resucitó de entre los muertos, y he aquí que se os adelanta en ir a Galilea: allí le veréis. Conque os lo tengo dicho».
          Y partiendo a toda prisa del monumento, con temor y grande gozo corrieron a dar la nueva a sus discípulos. \emph{Mt. 28, 7-8}. \textbf{\emph{Avemaría}}.

    \item Pues así tambíén vosotros, ahora cierto tenéis congoja; mas otra vez os veré, y se gozará vuestro corazón, y vuestro gozo nadie os lo quitará. \emph{Jn. 16, 22}. \textbf{\emph{Avemaría}}.

    \item Si hemos muerto con Cristo, también viviremos con Él; pues sabemos que Cristo, resucitado de entre los muertos, ya no muere, la muerte no tiene ya dominio sobre Él. Porque
          muriendo, murió al pecado una vez para siempre; pero viviendo, vive para Dios. \emph{Rom. 6, }. \textbf{\emph{Avemaría}}.

    \item Pues de gracia habéis sido salvados por la fe, y esto no os viene de vosotros, es don de Dios; no viene de las obras, para que nadie se gloríe; que hechura suya somos,
          creados en Cristo Jesús, para hacer buenas obras, que Dios de antemano preparó, para que en ella anduviésemos \emph{Ef. 2, 8-10}. \textbf{\emph{Avemaría}}.

    \item Fuése María Magdalena a dar la nueva a los discípulos: «he visto al Señor, y me ha dicho esto y esto». Siendo, pues, tarde aquel día, primero de la semana, y estando cerradas,
          por miedo a los judíos, las puertas de la casa donde estaban los discípulos, vino Jesús y se presentó en medio de ellos y les dice: «la paz sea con vosotros». Y en diciendo esto,
          les mostró las manos y el costado. Se gozaron, pues, los discípulos al ver al Señor. \emph{Jn. 20, 18-20}. \textbf{\emph{Avemaría}} y \textbf{\emph{Gloria}}.
\end{enumerate}

\rule{\textwidth}{0.5pt}
¡On Jesús mío!... y María, Madre de gracia... (\cpageref{endTenPrayers})

\subsection*{II Misterio: Las Ascensión Jesucristo a los cielos}

[Y les dijo:] «Id al mundo entero y predicad el Evangelio a toda la creación. El que creyere y fuere bautizado,
se salvará, mas el que no creyere, será condenado. Con esto el Señor Jesús, despues de hablarles,
fue elevado al cielo y se sentó a la diestra de Dios. Y ellos, partiéndose de allí,
predicaron por todas partes, cooperando el Señor y confirmando la palabra con las señales que la acompañaban.

\begin{flushright}
    \emph{Marcos 16, 15-16.19-20}
\end{flushright}

1 Paternoster, 10 Avemarías, 1 Gloria, ¡On Jesús mío!... y María, Madre de gracia... (\cpageref{endTenPrayers})

\rule{\textwidth}{0.5pt}

\begin{enumerate}
    \item \textbf{\emph{Paternóster}}. Que no entró Cristo en un santuario hecho por mano de hombre, figura del verdadero, sino en el mismo cielo, para compadecer ahora en
          la presencia de Dios a favor nuestro. \emph{Heb. 9, 24}. \textbf{\emph{Avemaría}}.

    \item Y acercándose Jesús, les habló diciendo: «Me fué dada toda potestad en el cielo y sobre la tierra». \emph{Mt. 28,18}. \textbf{\emph{Avemaría}}.

    \item {[Y les dijo:]} «Id al mundo entero y predicad el Evangelio a toda la creación. El que creyere y fuere bautizado,
          se salvará, mas el que no creyere, será condenado». \emph{Mc. 16, 15-16}. \textbf{\emph{Avemaría}}.

    \item «Y a los que hubieren creído les acompañarán estas señales: en mi nombre lanzarán demonios, hablarán lenguas nuevas, en sus manos tomarán serpientes,
          y si bebieren ponzoña mortífera, no les dañará; pondrán sus manos sobre los enfermos, y se hallarán bien». \emph{Mc. 16, 17}. \textbf{\emph{Avemaría}}.

    \item «Enseñándoles a guardar todas cuantas cosas os ordené. Y sabed que estoy con vosotros todos los días hasta la consumación de los siglos».
          \emph{Mt. 28, 20}. \textbf{\emph{Avemaría}}.

    \item Con esto el Señor Jesús, despues de hablarles, fue elevado al cielo y se sentó a la diestra de Dios. \emph{Mc. 16, 19}. \textbf{\emph{Avemaría}}.

    \item «Varones galileos, ¿qué hacéis ahí plantados mirando fíjamente al cielo?; Este mismo Jesucristo, que ha sido quitado de entre vosotros
          para ser elevado al cielo, así vendrá, de la misma manera que le habéis contemplado irse al cielo». \emph{Hch. 1, 11}. \textbf{\emph{Avemaría}}.

    \item Y ellos, partiéndose de allí, predicaron por todas partes, cooperando el Señor y confirmando la palabra con las señales que la acompañaban.
          \emph{Mc. 16,20}. \textbf{\emph{Avemaría}}.

    \item Teniendo, pues, un Potífice grande, que ha penetrado los cielos, Jesús, el Hijo de Dios, mantengamos firme la fe que profesamos. Lleguémonos, pues,
          con segura confianza al trono de la gracia, para que alcancemos misericordia y hallemos gracia en orden a ser socorridos en el tiempo oportuno.
          \emph{Heb 4, 14.16}. \textbf{\emph{Avemaría}}.

    \item {[Jesús dijo:]} Yo soy el camino, la verdad y la vida; nadie viene al Padre sino por mi. \emph{Heb 4,16}. \textbf{\emph{Avemaría}} y \textbf{\emph{Gloria}}.
\end{enumerate}

\rule{\textwidth}{0.5pt}
¡On Jesús mío!... y María, Madre de gracia... (\cpageref{endTenPrayers})

\subsection*{III Misterio glorioso: La Venida del Espíritu Santo sobre los Apóstoles}

Y al cumplirse el día de Pentecostés, estaban todos juntos en el mismo lugar. Y se produjo de súbito desde el cielo un estruendo como de viento que soplaba vehementemente,
y llenó toda la casa donde se hallaban sentados. Y vieron aparecer lenguas como de fuego, que, repartiéndose, se posaban sobre cada uno de ellos.

\begin{flushright}
    \emph{Hechos 2, 1-4}
\end{flushright}

1 Paternoster, 10 Avemarías, 1 Gloria, ¡On Jesús mío!... y María, Madre de gracia... (\cpageref{endTenPrayers})

\rule{\textwidth}{0.5pt}

\begin{enumerate}
    \item \textbf{\emph{Paternóster}}. Y al cumplirse el día de Pentecostés, estaban todos juntos en el mismo lugar. Y se produjo de súbito desde el cielo un estruendo como de viento
          que soplaba vehemente, y llenó toda la casa donde se hallaban sentados. \emph{Hch. 2, 1-2}. \textbf{\emph{Avemaría}}.

    \item Y vieron aparecer lenguas como de fuego, que, repartiéndose, se posaban sobre cada uno de ellos. Y se llenaron todos del Espíritu Santo, y comenzaron a hablar en lenguas diferentes,
          según que el Espíritu Santo les movía a expresarse. \emph{Hch. 2, 3-4}. \textbf{\emph{Avemaría}}.

    \item Hallábanse en Jerusalén judíos allí domiciliados, hombres religiosos de toda nación de las que están debajo del cielo. \emph{Hch. 2, 5}. \textbf{\emph{Avemaría}}.

    \item Puesto de pie Pedro, acompañado de los Once, alzó en voz y les habló en estos términos: «Arrepentíos, dice, y bautícese cada uno de vosotros en el nombre del
          Jesucristo para remisión de vuestros pecados, y recibiréis el don del Espíritu Santo. Porque para vosotros es esta promesa y para vuestros hijos y para todos los de lejos
          cuantos llamare a sí el Señor, Dios nuestros». \emph{Hch. 2, 14.38-39}. \textbf{\emph{Avemaría}}.

    \item Ellos, pues, acogieron su palabra, fueron bautizados; y fueron agragados en aquel día como unas tres mil almas. \emph{Hch. 2, 41}. \textbf{\emph{Avemaría}}.

    \item Mas la fructificación del Espíritu es: caridad, gozo, paz, longanimidad, benignidad, bondad, fe, mansedumbre, continencia; frente a tales cosas no tiene objeto la ley.
          \emph{Gál. 5, 22-23}. \textbf{\emph{Avemaría}}.

    \item Mas la sabiduría que viene de arriba primeramente es casta, luego pacífica, condescendiente, que se allana a razones, llena de misericordia y de frutos buenos,
          no amiga de criticar, no solapada. \emph{Sant. 3, 17}. \textbf{\emph{Avemaría}}.

    \item Porque no recibisteis espíritu de esclavitud para reincidir de nuevo en el temor; antes recibisteis Espíritu de filiación adoptiva, con el cual clamamos: ¡Abba!¡Padre!
          El Espíritu mismo testifica a una que somos hijos de Dios. \emph{Rom 8, 15-16}. \textbf{\emph{Avemaría}}.

    \item Y, asimismo, también el Espíritu acude en socorro de nuestras flaquezas. Pues qué hemos de orar, según conviene, no lo sabemos; mas el Espíritu mismo interviene a
          favor nuestro con gemidos inefables. \emph{Rom. 8, 26}. \textbf{\emph{Avemaría}}.

    \item Mas el Paráclito, el Espiritu Santo, que enviará el Padre en mi nombre, Él os enseñará todas las cosas y os recordará todas las cosas que os dije yo.
          \emph{Jn. 14, 26}. \textbf{\emph{Avemaría}} y \textbf{\emph{Gloria}}.
\end{enumerate}

\rule{\textwidth}{0.5pt}
¡On Jesús mío!... y María, Madre de gracia... (\cpageref{endTenPrayers})

\subsection*{IV Misterio: La Asunción de María Santísima a los cielos}

Y se abrió el templo de Dios, que está en el cielo, y fué vista el arca de la alianza en el templo,
y se produjeron relámpagos, y voces, y truenos, y temblor de tierra, y fuerte granizada.

\begin{flushright}
    \emph{Apocalipsis 11, 19}
\end{flushright}

1 Paternoster, 10 Avemarías, 1 Gloria, ¡On Jesús mío!... y María, Madre de gracia... (\cpageref{endTenPrayers})

\rule{\textwidth}{0.5pt}

\begin{enumerate}
    \item \textbf{\emph{Paternóster}}. Bendita tú, hija, ante el Dios Altísimo sobre todas las mujeres de la tierra,
          y bendito el Señor Dios, que crió los cielos y la tierra, que enderezó tus pasos para quebrantar la cabeza del jefe
          de nuestros enemigos. Pues no se apartará etérnamente tu esperanza del corazón de los hombres, que recordarán la fortaleza de Dios.
          \emph{Jdt. 13, 18-19}. \textbf{\emph{Avemaría}}.

    \item Y esto haga contigo Dios para eterno encubrimiento, que te visite con sus bienes; por cuanto no perdonate a tu vida,
          lastimada por la humillación de nuestro linaje, antes acudiste en socorro de nuestro abatimiento,
          caminando derechamente en el acatamiento de Dios. \emph{Jdt. 13, 20}. \textbf{\emph{Avemaría}}.

    \item ¿Qué es eso que sube del desierto como columna de humo sahumado de mirra e incienso y de toda clase de aromas del mercader?
          He aquí la litera de Salomón. \emph{Cant 3, 6-7a}. \textbf{\emph{Avemaría}}.

    \item Y se abrió el templo de Dios, que está en el cielo, y fué vista el arca de la alianza en el templo, y se produjeron relámpagos, y voces, y truenos,
          y temblor de tierra, y fuerte granizada. \emph{Ap. 11, 19}. \textbf{\emph{Avemaría}}.

    \item Y entrando a ella, bendijéronla todos a una voz y la dijeron: tú eres enaltecimiento de Jerusalén, tú gloria grande de Isarel, tú grande honor de nuestro linaje.
          \emph{Jdt 15, 9}. \textbf{\emph{Avemaría}}.

    \item Hiciste todo esto por tu mano, acarreaste bienes a Israel, y se agradó Dios en ellos. Bendita seas en el acatamiento del Señor omnipotente para tiempo sin fin.
          \emph{Jdt 15, 10}. \textbf{\emph{Avemaría}}.

    \item Del rey la hija toda hermosa entra; vestidos áureos su adorno son; al rey la llevan con recamados; amigas vírgenes van de ella en pos.
          Entre alborozos y gritos de júbilo van penetrando en la real mansión. \emph{Sal 45, 14-16}. \textbf{\emph{Avemaría}}.

    \item Antes de los siglos, desde el principio me creó y hasta la eternidad no cesaré. He arraigado en pueblo ilustre, en la porción del Señor, heredad suya.
          \emph{Eci 24, 14.16}. \textbf{\emph{Avemaría}}.

    \item Yo soy la madre de la hermosa dilección, y del temor, y del conocimiento y santa esperanza. \emph{Eci 24, 24}. \textbf{\emph{Avemaría}}.

    \item Entonad a Yahveh cántico nuevo, que portentos ha obrado. Su diestra le ha traído la victoria y aquel su brazo santo. \emph{Sal. 98, 1}.
          \textbf{\emph{Avemaría}} y \textbf{\emph{Gloria}}
\end{enumerate}

\rule{\textwidth}{0.5pt}
¡On Jesús mío!... y María, Madre de gracia... (\cpageref{endTenPrayers})

\subsection*{V Misterio: La Coronación de la Santísima Virgen María}

Y una gran señal fué vista en el cielo: una Mujer vestida del sol, y la luna debajo de sus pies, y sobre su cabeza una corona de doce estrellas.

\begin{flushright}
    \emph{Apocalipsis 12, 1}
\end{flushright}

1 Paternoster, 10 Avemarías, 1 Gloria, ¡On Jesús mío!... y María, Madre de gracia... (\cpageref{endTenPrayers})

\rule{\textwidth}{0.5pt}

\begin{enumerate}
    \item \textbf{\emph{Paternóster}}. Pongo perpétua enemistad entre ti y el mujer. Y entre tu linaje y el suyo. Este te aplastará la cabeza.
          Y tú le acecharas el calcañal. \emph{Gen. 3, 15}. \textbf{\emph{Avemaría}}.

    \item Una es mi paloma, mi pura; única es ella de su madre, la preferida de la que la dió a luz. Viéronla las doncellas,
          y la felicitaron; las reinas y las concubinas, y exclamaron loándola. \emph{Cant. 6, 9}. \textbf{\emph{Avemaría}}.

    \item Yo he salido de la boca del Altísimo. En las alturas he armado mi tienda y mi trono está en columna de nube. El círculo celeste he rodeado sola y en lo profundo
          del abismo me he paseado; y en la tierra toda, y en todo pueblo y nación he imperado. En mi está toda la gracia del camino y de la verdad, en mi toda esperanza de
          la vida y de la virtud. Venid a mí cuantos me deseáis y saciaos de mis frutos. Porque recordarme es más dulce que la miel, y poseerme más rico que el panal de miel.
          \emph{Eci. 24, 5-11.26-27}. \textbf{\emph{Avemaría}}.

    \item Y dijo María: Engrandece mi alma al Señor, y se regocijó mi espíritu em Dios, mi Salvador; porque puso sus ojos en la bajeza de su esclava. Pues he aquí que desde Ahora
          me llamarán dichosa todas las generaciones; porque hizo en mi favor grandes cosas el Poderoso, y cuyo nombre es «Santo»; y su misericordia por generaciones y generaciones
          para con aquellos que le temen. \emph{Lc. 1, 46-50}. \textbf{\emph{Avemaría}}.

    \item ¿Quién es esa que aparece resplandeciente como la aurora, hermosa cual luna, deslumbradora como el sol, imponente como batallones?
          \emph{Cant 6,10}. \textbf{\emph{Avemaría}}.

    \item Y una gran señal fué vista en el cielo: una Mujer vestida del sol, y la luna debajo de sus pies, y sobre su cabeza una corona de doce estrellas.
          \emph{Ap 12, 1}. \textbf{\emph{Avemaría}}.

    \item Quién me obedece no se avergonzará y los que obran por no pecarán. Los que me esclarecen tendrán vida eterna.
          \emph{Eci. 24, 30-31}. \textbf{\emph{Avemaría}}.

    \item Parió un varón que ha de apacentar a todas las naciones con vara de hierro, pero el Hijo fue arrebatado a Dios y a su trono. La mujer huyó
          al desierto, en donde tenía un lugar preparado por Dios [, para que allí la alimentasen durante mil dosciento sesenta días]. \emph{Ap 12, 5}. \textbf{\emph{Avemaría}}.

    \item Ahora, pues, hijos míos, oídme; y felices quienes guardan mis caminos. Escuchad la corrección y sed sabios, y no la rechacéis.
          Feliz el hombre que me escucha, velando a mis puertas cada día, guardando las jambas de mis entradas. Pues quien me halla, ha hallado la vida y alcanza el favor
          de Yahveh. Mas quién peca contra mi, se perjudica a si mismo, y cuantos me odian aman la muerte. \emph{Prv. 8, 32-35}. \textbf{\emph{Avemaría}}.

    \item Tiene Él escrito en su vestido y en su manto Rey de reyes y Señor de los que dominan. Está la Reina a su derecha, adornada con oro finísimo.
          \emph{Ap. 18, 16. Sal. 44, 10}. \textbf{\emph{Avemaría}} y \textbf{\emph{Gloria}}
\end{enumerate}

\rule{\textwidth}{0.5pt}
¡On Jesús mío!..., María, Madre de gracia... (\cpageref{endTenPrayers}) y Oraciones finales (\cpageref{sec:final-prayer}).
\end{document}