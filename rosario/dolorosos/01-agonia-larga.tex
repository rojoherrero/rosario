\documentclass[../../devocionario.tex]{subfiles}

\begin{document}
    \begin{enumerate}
    
        \item \textbf{\textit{Paternoster}} | Hijo, si te acercares a servir al Señor Dios, prepara tu alma a la 
            tentación Gobierna tu corazón y muéstrate firme y no te apresures en tiempo de evasión. \textit{Eci. 2,1-2} | \textbf{\textit{Ave María}}

        \item Y llegan a una granja, cuyo nombre es Getsemaní, y dice a sus discípulos: «sentaos aquí mientras hago oración». 
            Y lleva consigo a Pedro y a Santiago y a Juan, y comenzó a sentir espanto y abatimiento; 
            y le dice: «triste en gran manera está mi corazón hasta la muerte; quedad aquí y velad». \textit{Mc. 14,32-34} | \textbf{\textit{Ave María}}

        \item Y apartándose un poco, caía sobre tierra, y rogaba que, a ser posible, pasase el Él aquella hora, y decía: 
            «Abba, Padre, todas las cosas te son posibles: traspasa de mi este cáliz; 
            más no se haga lo que yo quiero, sino lo que tú». \textit{Mc. 14,35-36} | \textbf{\textit{Ave María}}

        \item Y viene, y los halla durmiendo, y dice a Pedro: «¡Simón!¿Duermes?¿No pudiste velar una hora? 
            Velad y orad para que no entréis en tentación; el espíritu , sí, está pronto, más la carne es flaca». 
            \textit{Mc. 14,37-38\textit} | \textbf{\textit{Ave María}}

        \item Y de nuevo habiéndse retirado se puso a orar, repitiendo las mismas palabras. 
            Y volviendo los halló otra vez durmiendo, porque estaban sus ojos cargados, 
            y no qué responderle. \textit{Mc. 14,39-40} | \textbf{\textit{Ave María}}

        \item Y viene tercera vez y les dice: «Ya por mi, dormid y descansad... Ya está: llegó la hora; 
            he aquí que es entregado el Hijo del hombre en las manos de los pecadores. Levantaos, vamos; mirad, 
            el que me entrega está aquí cerca». \textit{Mc. 14,41-42} | \textbf{\textit{Ave María}}

        \item Y luego, estando Él hablando todavía, se presenta Judas, uno de los Doce, y con él una turba con espadas y bastones, 
            de parte de los escribas y los ancianos. Y así que llegó, luego acercándose dijo: «Rabí». 
            Y le dió un fuerte beso. Ellos le echaron las manos y le sujetaron. \textit{Mc. 14,43.45-46} | \textbf{\textit{Ave María}} 

        \item En aquella hora dijo Jesús a las turbas: «¡cómo contra un salteador habéis salido con espadas y bastones a prenderme! 
            Cada día en el templo me sentaba para enseñar, y no me predisteis. Mas todo esto ha pasado para que se cumplan las Escrituras de los profetas». 
            Entonces los discípulo todos, abandonándole, huyeron». \textit{Mt. 26, 55-56} | \textbf{\textit{Ave María}}

        \item No volverá atrás la colerá de Yahveh hasta que ejecute y lleve a efecto los dedignios de su corazón; 
            al fin de los tiempos adquiriréis de ello inteligencia. \textit{Jer 23,20\textit} | \textbf{\textit{Ave María}}

        \item {[Y Samuel exclamó:]} «¿Acaso se complace Yahveh tanto en holocaustos y sacrificios cuanto en que se obedezca su voz? 
            He aquí que la obediencia vale más que el sacrificio y la docilidad más que la grosura de carneros». 
            \textit{1 Sam 15,22\textit} | \textbf{\textit{Ave María}} y \textbf{\textit{Gloria}}

    \end{enumerate}
\end{document}