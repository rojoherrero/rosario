\documentclass[../../devocionario.tex]{subfiles}

\begin{document}
    \begin{enumerate}
    
        \item \textbf{\textit{Paternoster}}. Y luego al amanecer, después de celebrar consejo, los sumos sacerdotes con los ancianos y los escribas, 
            es decir, todo el sanhedrín, atando a Jesús, le llevaron de allí y le entregaron a Pilato. 
            Y le interrogó Pilato: «¿Tú eres el Rey de los judíos?». El le respondió: «Tú lo dices». \textit{Mc. 15, 1-2}. \textbf{\textit{Ave María}}.

        \item Respondió Jesús: «Mi reino no es de este mundo. Si de este mundo fuera mi reino, mis ministros lucharían para 
            que yo no fuera entregado a los judíos Más ahora mi reino no es de aquí». \textit{Jn. 18, 36}. \textbf{\textit{Ave María}}.

        \item Díjole, pues, Pilato: «¿luego, rey eres tu?». Respondió Jesús: «Tú dices que yo soy rey. Yo para eso he nacido y 
            para esto he venido al mundo: para dar testimonio a favor de la verdad. 
            Todo el que es de la verdad, oye mi voz». \textit{Jn. 18, 37}. \textbf{\textit{Ave María}}.

        \item Dícele Pilato: «¿Qué es verdad?». Dicho esto, de nuevo, salió a los judíos y les dice: 
            «Yo no hallo en Él delito alguno. Es costumbre vuestra que yo os suelte un preso por la Pascua: ¿queréis, pues, 
            que os suelte al rey de los Judíos?». Gritaron, pues, de nuevo, diciendo: «No, a ése, sino a Barrabás». 
            Era este Barrabás un salteador. \textit{Jn. 18, 38-40}. \textbf{\textit{Ave María}}.

        \item Entonces, pues, tomó Pilato a Jesús y le azotó. \textit{Jn. 19, 1}. \textbf{\textit{Ave María}}.

        \item De opresión u juicio fué tomado, y a sus contemporáneos, ¿quién tendrá en cuenta?. Fué despreciado y abandonado de los hombres, 
            varón de dolores y familiarizado de los hombres. \textit{Is. 53, 8, 3}. \textbf{\textit{Ave María}}.

        \item Hízose para nosotros censura de nuestros criterios: pesado es para nosotros aun el verlo; 
            pues discordante de los otros es su vida, y muy otros sus caminos. \textit{Sab 2,14-15}. \textbf{\textit{Ave María}}.

        \item Fué traspasado por causa de nuestros pecados, molido por causa de nuestra iniquidades; 
            el castigo [precio] de nuestra paz cayó sobre Él y por sus verdugones se nos curó; \textit{Is. 53, 5}. \textbf{\textit{Ave María}}.

        \item ¡Oh vosotros kis que pasáis por el camino, mirad y ved si hay dolor semejante al dolor que me hiere, 
            pues me ha afligido Yahveh en el día del ardor de su cólera! \textit{Lm 1,12}. \textbf{\textit{Ave María}}.

        \item Yo soy el varón que ha visto la aflicción bajo el látigo de su cólera. Me ha guiado y conducido en tiniebla, 
            sin luz; sólo contra mi vuelve y revuelve su mano todo el día. \textit{Lm 3,1-3}. \textbf{\textit{Ave María}} y \textbf{\textit{Gloria}}

    \end{enumerate}
\end{document}