1. *_Paternoster_* | Fue maltratado, mas el se doblegó y no abre la boca; como cordero llevado al matadero y cual oveja ante sus esquiladores enmudecida, y no abre la boca. _Is 53,7_ | *_Ave María_*

2. Me han rodeado de un vallado y no puedo salir, me han aherrojado con pesadas cadena. Aunque grito y pido auxilio, ha cerrado el paso a mi plegaria. Ha obstruido mis caminos con piedras sillares, mis senderos han hecho instransitables. _Lm 3,7-9_ | *_Ave María_*

3. Al cuello tenemos los acosadores, estamos fatigados, no tenemos reposo. Los jovenes han tenido que arrastrar muela, y los muchachos han vacilado y caído, cargados de leña. _Lm 5,5.13_ | *_Ave María_*

4. Vagaron vacilantes como ciegos por las calles, manchados de sangre, sin que nadie pudiera tocar sus vestiduras. «¡Apartaos!¡Un impuro!, les gritaban. ¡Apartaos, apartaos!¡No toquéis!» Cuando huyieron y vagaron vacilantes, decíase entre las naciones: «¡no pueden quedarse [aquí]!». _Lm 4,14-15_ | *_Ave María_*

5. Si alguno quiere venir en pos de Mí, niégese a sí mismo y tome a cuestas su cruz cada día y sígame. _Lc. 9, 23_ | *_Ave María_*

6. Gritaron, pues, ellos: «Quita, quita, crucifícale». Díceles Pilato: «¿A vuestro rey voy he de crucificar?». Respondieron los pontífices: «No tenemos rey, sino César». Entonces, pues, se le entregó para que fuera crucificando. Se apoderaron, pues, de Jesús. _Jn. 19, 15-16_ | *_Ave María_*

7. Y, llevando a cuestas su cruz, salió hacia el lugar llamado el Cráneo, que en hebreo se dice Gólgota. _Jn. 19, 17_ | *_Ave María_*

8. Y como le hubieron sacado, echaron mano de un tal Simón de Cirene que venía del campo, le pusieron en hombros la cruz para que la llevase detrás de Jesús. _Lc. 23, 26_ | *_Ave María_*

9. Tomad mi yugo sobre vuestros, y aprended de mi, pues soy manso y humilde de Corazón, y hallaréis reposo para vuestras almas. _Mt. 11, 29_ | *_Ave María_*

10. Porque mi yugo es suave, y mi carga, ligera. _Mt. 11, 30_  | *_Ave María_* y *_Gloria_*