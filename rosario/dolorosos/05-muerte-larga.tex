\documentclass[../../devocionario.tex]{subfiles}

\begin{document}
    \begin{enumerate}
    
        \item \textbf{\emph{Pater Noster}}. Y respondió Abraham: «Dios proveerá de cordero para el holocausto, hijo mío». \emph{Gn. 22, 8}. \textbf{\emph{Ave María}}.
        
        \item Es que quiso quebrantarle Yahveh con padecimientos. Ofreciendo su vida en sacrificio por el pecado, tendrá posteridad y vivirá largos días, y en sus manos
            prosperará la obra de Yahveh. \emph{Is. 53, 10}. \textbf{\emph{Ave María}}.

        \item Y cuando hubieron llegado al lugar llamado «Cráneo», allí crucificaron a Él y a los malhechores, uno a la derecha y el otro a la izquierda. Y Jesús decía: 7
            «Padre, perdónalos, porque no saben lo que hacen». Y al repartir sus vestidos, echaron suertes. \emph{Lc. 23, 33-34}. \textbf{\emph{Ave María}}.

        \item Había también por encima de Él una inscripción escrita en letras griegas, latinas y hebreas: Este es el Rey de los judios. \emph{Lc. 23, 38}. \textbf{\emph{Ave María}}.

        \item Uno de los malhechores crucificados le insultaba, diciendo: ¿No eres tú el Mesías? Sálvate, pues, a ti mismo y a nosotros. Pero el otro,
            tomando la palabra, le respondió, diciendo: ¿Ni tú, que estás sufriendo el mismo suplicio, temes a Dios? En nosotros se cumple la justicia, pues
            recibimos el justo castigo de nuestras obras; pero éste nada malo ha hecho. Y decía: Jesús, acuérdate de mi cuando llegues a tu reino. El le dijo: En verdad te digo, 
            hoy serás conmigo en el paraíso. \emph{Lc. 23, 39-43}. \textbf{\emph{Ave María}}.

        \item Jesús, pues, viendo a la Madre, y junto a ella al discípulo a quien amaba, dice a tu Madre: «Mujer, he ahí a tu hijo». 
            Luego dice al discípulo: «He aquí a tu Madre». Y desde aquella hora la tomó el discípulo en su compañía. \emph{Jn. 19, 26-27}. \textbf{\emph{Ave María}}.

        \item Y era ya como la hora sexta, y se produjeron tinieblas sobre toda la tierra hasta la hora nona, habiendo faltado el sol; 
            y se rasgó por medio el velo del santuario.Y clamando con voz poderosa, Jesús dijo: «Padre, en tus manos encomiendo mis espíritu». Y dicho esto, expiró. 
            \emph{Lc. 23, 44-46}. \textbf{\emph{Ave María}}.

        \item Mas viendo a Jesús, cuando vinieron, como le vieron ya muerto, no le quebrantaron las piernas, sino que uno de los soldados 
            con una lanza le traspasó el costado, y salió al punto sangre y agua. Pues acontecieron estas cosas para que se cumpliese las
            Escrituras: «No le será quebrantado hueso alguno»\footnote{Ex. 12, 46; Núm. 9, 12}. Y asimismo otra Escritura dice: «Verán al que
            traspasaron»\footnote{Zac. 12, 10}. \emph{Jn 19, 33-34.36}. \textbf{\emph{Ave María}}.

        \item Después de esto, José de Arimatea, que era discípulo de Jesús, si bien oculto por miedo a los judíos, rogó a Pilato le permitiese
            quitar el cuerpo de Jesús. Y se lo permitió Pilato. Vino, pues, y quitó su cuerpo. Vino también Nicodemo, el que la primera vez había
            venido a Él de noche, trayendo una mixtura de mirra y áloe, como cien libras.
            \emph{Jn. 19, 38-39}. \textbf{\emph{Ave María}}.            

        \item Tomaron pues, el cuerpo de Jesús y lo fajaron con bandas y aromas, según es costumbre sepultar entre los judíos. Había cerca
            del sitio donde fue crucificado un huerto, y en el huerto un sepulcro nuevo, en el cual nadie aún había sido depositado.
            Allí, a causa de la Parasceve de los judíos, por estar cerca del monumento, pusieron a Jesús. 
            \emph{Jn. 19, 40-42}. \textbf{\emph{Ave María}} y \textbf{\emph{Gloria}}

    \end{enumerate}
\end{document}