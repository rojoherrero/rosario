\documentclass[../../devocionario.tex]{subfiles}

\begin{document}
    \begin{enumerate}
    
        \item \textbf{\textit{Pater Noster}}. Y respondió Abraham: «Dios proveerá de cordero para el holocausto, hijo mío». 
            \textit{Gn. 22, 8}. \textbf{\textit{Ave María}}.

        \item Y cuando hubieron llegado al lugar llamado «Cráneo», allí crucificaron a Él y a los malhechores, 
            uno a la derecha y el otro a la izquierda. Y Jesús decía: «Padre, perdónalos, porque no saben lo que hacen». 
            Y al repartir sus vestidos, echaron suertes. \textit{Lc. 23, 33-34}. \textbf{\textit{Ave María}}.

        \item Había también por encima de Él una inscripción escrita en letras griegas, latinas y hebreas: 
            Este es el Rey de los judios. \textit{Lc. 23, 38}. \textbf{\textit{Ave María}}.

        \item Uno de los malhechores que estaba colgado decía a Jesús: «acuérdate de mi cuando vinieres en la gloria de tu realeza». 
            Díjole: «en verdad te digo que hoy estarás conmigo en el paraíso». \textit{Lc. 23, 39, 42-43}. \textbf{\textit{Ave María}}.

        \item Jesús, pues, viendo a la Madre, y junto a ella al discípulo a quien amaba, dice a tu Madre: 
            «Mujer, he ahí a tu hijo». Luego dice al discípulo: «He aquí a tu Madre». 
            Y desde aquella hora la tomó el discípulo en su compañía. \textit{Jn. 19, 26-27}. \textbf{\textit{Ave María}}.

        \item Y era ya como la hora sexta, y se produjeron tinieblas sobre toda la tierra hasta la hora nona, 
            habiendo faltado el sol; y se rasgó por medio el velo del santuario. \textit{Lc. 23, 44-45}. \textbf{\textit{Ave María}}.

        \item Y clamando con voz poderosa, Jesús dijo: «Padre, en tus manos encomiendo mis espíritu». Y dicho esto, expiró. 
            \textit{Lc. 23, 46}. \textbf{\textit{Ave María}}.

        \item Mas viendo a Jesús, cuando vinieron, como le vieron ya muerto, no le quebrantaron las piernas, sino que uno de los soldados 
            con una lanza le traspasó el costado, y salió al punto sangre y agua. Porque esto sucedió para que se  cumpliese
            la Escritura: «No romperéis ni uno de sus huesos». Y otra Escritura dice: «Mirarán al que traspasaron».
             \textit{Jn 19,33-34.39}. \textbf{\textit{Ave María}}.

        \item Después de esto rogó a Pilato José de Arimatea, que era discípulo de Jesús, aunque en secreto por temos a los judíos,
            que le permitiese tomar el cuerpo de Jesús, y Pilato se lo permitió. Vino, pues, y tomó su cuerpo. Llegó Nicodemo, el mimos
            que había venido a Él de novhe al principio, y trajo una mezcla de mirra y aloe, como unas cien libras.
            \textit{Jn. 19, 38-39}. \textbf{\textit{Ave María}}.            

        \item Tomaron pues, el cuerpo de Jesús y lo fajaron con bandas y aromas, según es costumbre sepultar entre los judíos. Había cerca
            del sitio donde fue crucificado un huerto, y en el huerto un sepulcro nuevo, en el cual nadie aún había sido depositado.
            Allí, a causa de la Parasceve de los judíos, por estar cerca del monumento, pusieron a Jesús. 
            \textit{Jn. 19, 40-42}. \textbf{\textit{Ave María}} y \textbf{\textit{Gloria}}

    \end{enumerate}
\end{document}