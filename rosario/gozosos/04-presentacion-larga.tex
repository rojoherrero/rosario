\documentclass[../../devocionario.tex]{subfiles}

\begin{document}
    \begin{enumerate}
        \item \textbf{\textit{Paternoster}}. ¡Alzaos, oh vosotras las puertas eternales, para que el Rey de la gloria entre!
            ¿Y quién es este Rey de la gloria? El Señor fuerte, Señor [fuerte] y potente, poderoso en la liza. \textit{Sal 24,7-8} . \textbf{\textit{Ave María}}.

        \item Habló después Yahveh a Moises diciendo: «Conságradme todo primogénito; todo primer nacido entre los hijos de Israel, 
            tanto en hombres como en animales es mío». \textit{Ex. 13,1-2}. \textbf{\textit{Ave María}}.

        \item Cuando se hayan cumplido los días de su purificación; por hijo o po hija, llevará al sacerdote, a la entrada de la tienda de reunión, 
            un cordero añal para holocausto y un pichón o una tórtola en sacrificio por el pecado. \textit{Lev. 12, 6}. \textbf{\textit{Ave María}}.

        \item Y cuando se les cunplieron los días de la purificación según la ley de Moisés, le subieron a Jerusalen para presentarle al Señor, 
            según está escrito en la Ley del Señor que «todo primogénito del sexo masculino será consagrado al Señor», 
            y para ofrecer como sacrificio, según lo que se ordena en la Ley del Señor, 
            «un par de tórtolas o dos palominos \textit{Lc. 2, 22-24}. \textbf{\textit{Ave María}}.

        \item Proseguí viendo en la visión nocturna, y he aquí que en las nubes del cielo venía como un hombre, 
            y llegó hasta el anciano y fué llevado hasta Él. \textit{Dan 7,13}. \textbf{\textit{Ave María}}.

        \item y he aquí había un hombre en Jerusalén por nombre Simeón. Y era este hombre justo y temeroso de Dios que aguardaba la consolación de Israel, 
            y el Espíritu Santo estaba sobre él; \textit{Lc. 2, 25}. \textbf{\textit{Ave María}}.

        \item y le había sido revelado por el Espíritu Santo que no vería la muerte antes de ver al Ungido del Señor. \textit{Lc. 2, 26}. \textbf{\textit{Ave María}}.

        \item Y vino al templo impulsado por el Espíritu Santo. Y cuando sus padres intrudujeron al niño Jesús para cumplir 
            las prescipciones usuales de la ley tocantes a El, 
            Simeón le recibió en sus brazos y bendijo a Dios diciendo. \textit{Lc. 2, 27-28}. \textbf{\textit{Ave María}}.

        \item Ahora dejas ir a tu siervo, Señor, según tu palabra, en paz; pues ya vieron mis ojos tu salud, 
            que preparaste a la faz de todos los pueblos; luz para iluminación de los gentiles, 
            y gloria de tu pueblo Israel. \textit{Lc. 2, 29-32}. \textbf{\textit{Ave María}}.

        \item Y así que cumplieron todas las cosas ordenadas en la ley del Señor, se volvieron a Galilea, a su ciudad de Nazaret. 
            El niño crecía y se robustecía llenándose de sabiduría, 
            y la gracia de Dios estaba en Él. \textit{Lc. 2, 39-40}. \textbf{\textit{Ave María}} y \textbf{\textit{Gloria}}
    
    \end{enumerate}
\end{document}