\documentclass[../../devocionario.tex]{subfiles}

\begin{document}
    \begin{enumerate}
        \item \textbf{\textit{Paternoster}}. Iban sus padres cada año a Jerusalén por la fiesta de la Pascua. 
            Y cuando fué de doce años, habiendo ellos subido, según la costumbre de la fiesta, \textit{Lc. 2, 41-42}. \textbf{\textit{Ave María}}.

        \item Y acabados los días, al volverse ellos, quedóse el niño Jesús en Jerusalén, sin que lo advirtiesen sus padres. 
            Y creyendo ellos que Él andaría en la comitiva, caminaron una jornada; 
            y le buscaban entre los parientes y conocidos. \textit{Lc. 2, 43-44}. \textbf{\textit{Ave María}}.

        \item Y no hallándole, se tornaron a Jerusalén para buscarlo. Y sucedió que después de tres días le hallaron en el templo, 
            sentado en medio de los maestros, escuchándoles y haciéndoles preguntas; 
            y se pasmaban todos los que le oían de su inteligencia y de sus respuestas. \textit{Lc. 2, 45-47}. \textbf{\textit{Ave María}}.

        \item Y sus padres, al verle, quedaron sorprendidos; y le dijo su madre: «hijo, ¿por qué lo hiciste así con nosotros? 
            Mira que tu padre y yo, llenos de aflicción, te andábamos buscando». \textit{Lc. 2, 48}. \textbf{\textit{Ave María}}.

        \item Díjoles Él: «¿pues por qué me buscabais? ¿No sabíais que había yo de estar en casa de mi padre?». 
            Y ellos no comprendieron lo que les dijo. \textit{Lc. 2, 49-50}. \textbf{\textit{Ave María}}.

        \item Y bajó en su compañía y se fue a Nazaret, y vivía sometido a ellos. Y su madre guardaba todas estas cosas en su corazón. 
            Y Jesús progresaba en sabiduría, en estatura y en gracia delante de Dios y de los hombres. \textit{Lc. 2, 51-52}. \textbf{\textit{Ave María}}.

        \item Una cosa [tan sólo] a Yahveh pido, esto ansío obtener: que en su casa los días de mi vida pueda morar con Él. \textit{Sal 27,4}. \textbf{\textit{Ave María}}.

        \item Los judíos, por su parte, bendencían al Señor, que había glorificado su lugar santo; y el templo, momentos antes lleno de terror y de turbación, 
            robosaba ahora gozo y alegría, gracias a la manifestación del Señor omnipotente. \textit{2 Mac 3,30}. \textbf{\textit{Ave María}}.

        \item Grande es Yahveh, laudable por extremo en la ciudad de nuestro [padre] Dios. Su monte santo, altura encantadora, 
            del mundo entero  es la fruición. Monte Dión, remate boreal es la ciudad morada de un gran Rey. \textit{Sal 48,2-3}. \textbf{\textit{Ave María}}.

        \item Y la ciudad no tiene necesidad de sol ni de luna, para que alumbren en ella; 
            porque la gloria de Dios la ilumina y su antorcha es el Cordero. \textit{Ap 21,23}. \textbf{\textit{Ave María}} y \textbf{\textit{Gloria}}
    
    \end{enumerate}
\end{document}