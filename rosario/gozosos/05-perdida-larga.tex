\documentclass[../../devocionario.tex]{subfiles}

\begin{document}
    \begin{enumerate}
        \item Pero aquella misma noche tuvo Natán palabra de Yahveh: «Anda y ve a decir a David, mi siervo: Así habla Yahveh: ¿Vas a edificarme tú una casa
            para que yo habite en ella?» \emph{2 Sam. 7, 4}. \textbf{\emph{Ave María}}.

        \item \textbf{\emph{Pater Noster}}. Iban sus padres cada año a Jerusalén por la fiesta de la Pascua. 
            Y cuando fué de doce años, habiendo ellos subido, según la costumbre de la fiesta, \emph{Lc. 2, 41-42}. \textbf{\emph{Ave María}}.

        \item Y acabados los días, al volverse ellos, quedóse el niño Jesús en Jerusalén, sin que lo advirtiesen sus padres. 
            Y creyendo ellos que Él andaría en la comitiva, caminaron una jornada; y le buscaban entre los parientes y conocidos. 
            \emph{Lc. 2, 43-44}. \textbf{\emph{Ave María}}.

        \item Y no hallándole, se tornaron a Jerusalén para buscarlo. Y sucedió que después de tres días le hallaron en el templo, 
            sentado en medio de los maestros, escuchándoles y haciéndoles preguntas; y se pasmaban todos los que le oían de su inteligencia 
            y de sus respuestas. \emph{Lc. 2, 45-47}. \textbf{\emph{Ave María}}.

        \item Y sus padres, al verle, quedaron sorprendidos; y le dijo su madre: «hijo, ¿por qué lo hiciste así con nosotros? 
            Mira que tu padre y yo, llenos de aflicción, te andábamos buscando». \emph{Lc. 2, 48}. \textbf{\emph{Ave María}}.

        \item Díjoles Él: «¿pues por qué me buscabais? ¿No sabíais que había yo de estar en casa de mi padre?». Y ellos no comprendieron lo que les dijo. 
            \emph{Lc. 2, 49-50}. \textbf{\emph{Ave María}}.

        \item Y bajó en su compañía y se fue a Nazaret, y vivía sometido a ellos. Y su madre guardaba todas estas cosas en su corazón. 
            Y Jesús progresaba en sabiduría, en estatura y en gracia delante de Dios y de los hombres. \emph{Lc. 2, 51-52}. \textbf{\emph{Ave María}}.

        \item Jesús les respondió y dijo: Mi doctrina no es mía, sino del que me ha enviado. Quien quisiere hacer la voluntad de Él conocerá si mi doctrina
            es de Dios o si es mía. \emph{Jn. 7, 16-17}. \textbf{\emph{Ave María}}.

        \item El que de sí mismo habla, busca su propia gloria; pero el que busca la gloria del quen le ha enviado, ése es veraz y no hay en el injusticia.
             \emph{Jn. 7, 18}. \textbf{\emph{Ave María}}.

        \item Y la ciudad no tiene necesidad de sol ni de luna, para que alumbren en ella; porque la gloria de Dios la ilumina y su antorcha es el Cordero. 
            \emph{Ap 21,23}. \textbf{\emph{Ave María}} y \textbf{\emph{Gloria}}
    
    \end{enumerate}
\end{document}