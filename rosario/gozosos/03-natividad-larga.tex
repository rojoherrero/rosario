\documentclass[../../devocionario.tex]{subfiles}

\begin{document}

    \begin{enumerate}
        \item \textbf{\textit{Pater Noster}}. Pero tú, Belén de Efratá, pequeño entre los clanes de Judá, de ti me saldrá quien señoreará en Israel, cuyos
            orígenes serán de antiguo, de días de muy remota antigüedad. \textit{Miq, 4, 2}. \textbf{\textit{Ave María}}.

        \item El pueblo que camina en las tinieblas vió una gran luz; una luz ha resplandecido sobre los que habitaban en la tierra de sombras de muerte.
            Pues un niño nos ha nacido, un hijo se nos ha dado, sobre cuyo hombro está el principado y cuyo nombre se llamará Consejero maravilloso, 
            Dios fuerte, Padre eterno, Principe de la Paz. \textit{Is. 9, 2.5}. \textbf{\textit{Ave María}}.

        \item José subió de Galilea, de la ciudad de Nazaret, a Judea, a la ciudad de David, que se llama Belén, por ser el de la casa y de la familia de David 
            \textit{Lc. 2, 4-5}. \textbf{\textit{Ave María}}.

        \item Y sucedió que estando ellos allí se le complieron a ella los días del parto. Y dió a luz a su hijo primogénito, 
            y le envolvió en pañales y le recostó en un pesebre, pues no había para ellos lugar en el mesón y le recostó en un pesebre, 
            pues no había para ellos lugar en el mesón. \textit{Lc. 2, 6-7}. \textbf{\textit{Ave María}}.

        \item Y había unos pastores en aquella misma comarca, que pernoctaban al raso y velaban por turno para guardar su ganado, 
            y un ángel del Señor se presentó ante ellos, y la gloria del Señor los envolvió en sus fulgores, y se atemorizaron con gran temor. 
            \textit{Lc. 2, 8-9}. \textbf{\textit{Ave María}}.

        \item Y les dijo el ángel: «no temáis, pues he aquí que os traigo una buena nueva, que será de grande alegría para todo el pueblo: 
            que os ha nacido hoy en la ciudad de David un Salvador, que es el Mesías, el Señor». Esto tendréis por señal: encontraréis un 
            niño envuelto en pañales y reclinado en un pesebre. \textit{Lc. 2, 10-12}. \textbf{\textit{Ave María}} y \textbf{\textit{Gloria}}
        
        \item Al instante se juntó con el ángel una multitud del ejército celestial, que alababa a Dios diciendo: «Gloria a Dios en las alturas y paz en
            la tierra a los hombres de buena voluntad» \textit{Lc. 2, 13-14}. \textbf{\textit{Ave María}}.

        \item Así que los ángeles se fueron al cielo, se dijeron los pastores unos a otros: Vamos a Belén a ver esto que el Señor nos ha anunciado. fueron
            con presteza y encontraron a María, a José y al Niño acostado en un pesebre, y viéndole, contaron lo que se les había dicho acerca del
            Niño. \textit{Lc. 2, 15-17}. \textbf{\textit{Ave María}}.   

        \item Nacido, pues, Jesús en Belén de Judá en los días del rey Herodes, llegaron de Oriente a Jerusalén unos magos, diciendo: ¿Dónde está el rey de los judíos
            que acaba de nacer? Porque hemos visto su estrella al oriente y hemos venido a adorarle. \textit{Mt. 2, 1-2}. \textbf{\textit{Ave María}}.            
        
        \item Y, llegando a la casa, vieron al niño con María, su madre, y de hinojos le adoraron y, abriendo sus cofres, le ofrecieron como dones oro, incienso y mirra. 
            \textit{Mt. 2, 11}. \textbf{\textit{Ave María}} y \textbf{\textit{Gloria}}
    
    \end{enumerate}
\end{document}