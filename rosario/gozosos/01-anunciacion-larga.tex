\documentclass[../../devocionario.tex]{subfiles}

\begin{document}
    \begin{enumerate}

        \item \textbf{\textit{Pater Noster}}. Al principio era el Verbo, y el Verbo estaba en Dios y el Verbo era Dios. El estaba al principio en Dios. Y el Verbo se hizo carne, 
                y habitó entre nosotros; y contemplamos su gloria, gloria cual del Unigénito procedente del Padre, lleno de gracia y de verdad. 
                \textit{Jn. 1, 1-2.14}. \textbf{\textit{Ave María}}.

        \item He aquí que se le apareció en sueños un ángel del Señor y le dijo: José, hijo de David, no temas recibir en tu casa a María, tu esposa,
                pues lo concebido en ella es obra del Espiritu Santo \textit{Mt. 1, 20}. \textbf{\textit{Ave María}}.

        \item Dará a luz un hijo, a quien pondrás por nombre Jesús, porque salvará a su pueblo de sus pecados. Todo esto sucedió para que se cumplieran
                lo que el Señor había anunciado por el profeta. \textit{Mt. 1, 21-22}. \textbf{\textit{Ave María}}.
    
        \item Pues bien, el Señor mismo os dará una señal: He aquí que la virgen concebirá y parirá un hijo, 
                a quien ella denominará con el nombre de Emmanuel. Y brotará un retoño del trono de Jesé y retoñará de sus raices un vástago. 
                Sobre el que reposará el espíritu de Yavé, espíritu de sabiduría y de inteligencia, espíritu de consejo y de fortaleza, espíritu 
                de entendimiento y de temor de Yavé. \textit{Is. 7,14; 11, 1-4}. \textbf{\textit{Ave María}}.

        \item Fué enviado el Ángel Gabriel de parte de Dios a una ciudad de Galilea, llamada Nazaret, 
                a una doncella desposada con un varón llamada José, de la familia de David, y el nombre de la doncella era María. \textit{Lc. 1, 26-27}. \textbf{\textit{Ave María}}.

        \item Y habiendo entrado a ella, dijo: «Dios te salve, llena de gracia, el Señor es contigo, bendita tú entre las mujeres». 
                Ella, al oír estas palabras, se turbó, y discurría qué podía ser esta salutación. \textit{Lc. 1, 28-29}. \textbf{\textit{Ave María}}.

        \item Y le dijo el ángel: «No temas, María, pues hallaste gracia a los ojos de Dios. He aquí que concebirás en tu seno y darás a luz un Hijo, 
                a quién darás por nombre Jesús». \textit{Lc. 1, 30-31}. \textbf{\textit{Ave María}}.

        \item «Este será grande, y será llamado Hijo del Altísimo, y le dará el Señor Dios el trono de David su padre, 
                y reinará sobre la casa de Jacob etérnamente, y su reinado no tendrá fin».  \textit{Lc. 1, 32-33}. \textbf{\textit{Ave María}}.

        \item Dijo María al ángel: «¿Cómo será eso, pues, no conozco varón?». Y respondiendo el ángel, le dijo: «el Espíritu Santo descenderá sobre ti, 
                y el poder del Altísimo te cobijará con su sombra; por lo cual también lo que nacerá será llamado santo, Hijo de Dios». 
                \textit{Lc. 1, 35}. \textbf{\textit{Ave María}}.

        \item María dijo: «he aquí la esclava del Señor; hágase en mí según tu palabra». Y se retiró de ella el ángel. \textit{Lc. 1, 38}. 
                \textbf{\textit{Ave María}} y \textbf{\textit{Gloria}}

    \end{enumerate}
\end{document}