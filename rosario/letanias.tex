\documentclass[./devocionario.tex]{subfiles}

\begin{document}

    \begin{longtable} { p{0.5\textwidth} p{0.5\textwidth} }
        Señor, tened piedad de nosptros & Kýrie, eléison.\\
        Cristo, tened piedad de nosptros & Christe, eléison.\\
        Señor, tened piedad de nosptros. & Kýrie, eléison.\\
        Cristo, oídnos. & Christe, audi nos.\\
        Cristo, escuchadnos. & Christe, exáudi nos.\\
        Dios, Padre celestial, \textit{tened misericordia de nosotros}. & Pater de cælis, Deus, \textit{miserére nobis}.\\
        Dios Hijo, Redentor del mundo. & Fili, Redémptor mundi, Deus.\\
        Dios Espíritu Santo. & Spíritus Sancte, Deus.\\
        Trinidad Santa, un solo Dios. & Sancta Trínitas, unus Deus\\
        Santa María, \textit{rogad por nosotros}. & Sancta Maria, \textit{ora pro nobis}.\\
        Santa Madre de Dios. & Sancta Dei Génetrix.\\
        Santa Virgen de las vírgenes. & Sancta Virgo vírginum.\\
        Madre de Cristo. & Mater Christi.\\
        Madre de la Iglesia. & Mater Ecclésiæ.\\
        Madre de la divina gracia. & Mater divínæ gratiæ.\\
        Madre purísima. & Mater puríssima.\\
        Madre castísima. & Mater castíssima.\\
        Madre virginal. & Mater invioláta.\\
        Madre sin corrupción. & Mater intemeráta.\\
        Madre inmaculada. & Mater immaculáta.\\
        Madre amable. & Mater amábilis.\\
        Madre admirable. & Mater admirábilis.\\
        Madre del Buen Consejo. & Mater boni Consílii.\\
        Madre del Creador. & Mater Creatóris.\\
        Madre del Salvador. & Mater Salvatóris.\\
        Virgen pru­den­tísima. & Virgo pru­den­tíssima.\\
        Virgen digna de veneración. & Virgo veneránda.\\
        Virgen digna de alabanza. & Virgo prædicánda.\\
        Virgen poderosa. & Virgo potens.\\
        Virgen clemente. & Virgo clemens.\\
        Virgen fiel. & Virgo fidélis.\\
        Espejo de justicia. & Spéculum iustítiæ.\\
        Trono de sabiduría. & Sedes Sapiéntiæ.\\
        Causa de nuestra alegría. & Causa nostræ lætítiæ.\\
        Vaso espiritual. & Vas spirituále.\\
        Vaso digno de honor. & Vas honorábile.\\
        Vaso insigne de devoción. & Vas insigne devotiónis.\\
        Rosa mística. & Rosa mýstica.\\
        Torre de David. & Turris Davídica.\\
        Torre de marfil. & Turris ebúrnea.\\
        Casa de oro. & Domus áurea.\\
        Arca de la alianza. & Fœderis arca.\\
        Puerta del cielo. & Iánua cæli.\\
        Estrella de la mañana. & Stella matutina.\\
        Salud de los enfermos. & Salus infirmórum.\\
        Refugio de los pecadores. & Refugium peccatórum.\\
        Consuelo de los afligidos. & Consolátrix af­flic­tórum.\\
        Auxilio de los cristianos. & Auxílium chris­tia­nórum.\\
        Reina de los Ángeles. & Regina Angelórum.\\
        Reina de los Patriarcas. & Regina Pa­triar­chárum.\\
        Reina de los Profetas. & Regina Pro­phe­tárum.\\
        Reina de los Apóstoles. & Regina Apos­to­lórum.\\
        Reina de los Mártires. & Regina Mártyrum.\\
        Reina de los Confesores. & Regina Con­fe­ssórum.\\
        Reina de las Vírgenes. & Regina Vírginum.\\
        Reina de todos los Santos. & Regina Sanctórum ómnium.\\
        Reina concebida sin pecado original. & Regina sine labe originali concépta.\\
        Reina elevada al cielo. & Regina in cælum assumpta.\\
        Reina del Santísimo Rosario. & Regina sa­cra­tíssimi Rosárii.\\
        Reina de la familia. & Regina famíliæ.\\
        Reina de la paz. & Regina pacis.\\
        \textbf{Cordero de Dios, que quitas los pecados del mundo}, \textit{perdonadnos, Señor.} 
        &
        \textbf{Agnus Dei, qui tollis peccáta mundi}, \textit{parce nobis, Dómine}.\\
        \textbf{Cordero de Dios, que quitas los pecados del mundo}, \textit{escuchadnos, Señor.}
        &
        \textbf{Agnus Dei, qui tollis peccáta mundi}, \textit{exáudi nos, Dómine}.\\
        \textbf{Cordero de Dios, que quitas los pecados del mundo}, \textit{tened piedad de nosotros}
        &
        \textbf{Agnus Dei, qui tollis peccáta mundi}, \textit{miserére nobis}.\\
        Bajo tu amparo nos acogemos, Santa Madre de Dios: no desprecies las súplicas que te dirigimos en nuestras necesidades, 
        antes bien, líbranos siempre de todos los peligros, Virgen gloriosa y bendita.
        &
        Sub tuum præsídium confúgimus, Sancta Dei Génetrix, nostras de­pre­ca­tiónes ne despícias in ne­ces­si­tátibus; 
        sed a perículis cunctis líbera nos semper, Virgo gloriósa et benedícta.\\
        \textbf{Ruega por nosotros, Santa Madre de Dios}. \textit{Para que seamos dignos de alcanzar las promesas de nuestro Señor Jesucristo.}
        &
        \textbf{Ora pro nobis, Sancta Dei Génetrix}. \textit{Ut digni efficiámur pro­mi­ssiónibus Christi}.
        
    \end{longtable}

\end{document}
