\documentclass[../../devocionario.tex]{subfiles}

\begin{document}
    \begin{enumerate}
    
        \item \textbf{\textit{Pater Noster}}. Alzase Dios, y al presentarse ante ellos desbándese sus enemigos, 
            y huyen aquellos que le odiaron. \textit{Sal 68,2}. \textbf{\textit{Ave María}}.

        \item Bienaventurados los que están afligidos, porque ellos serán consolados. \textit{Mt 5,5}. \textbf{\textit{Ave María}}.

        \item «Pues así también vosotros, ahora cierto tenéis congoja; mas otra vez os veré, y se gozará vuestro corazon, 
            y vuestro gozo nadie os lo quita». \textit{Jn. 16, 22}. \textbf{\textit{Ave María}}.

        \item De pronto se produjo un gran temblor de tierra, pues un ángel del Señor, bajado de cielo y acercándose, 
            hizo rodar de su sitio la losa, y se sentó sobre ella. Era su aspecto como relámpago, 
            y su vestidura blanca como la nieve. \textit{Mt. 28, 2}. \textbf{\textit{Ave María}}.

        \item Tomando la palabra el ángel, dijo a las mujeres: «no tengáis miedo vosotras, que ya sé que buscáis a Jesús el crucificado; 
            no está aquí; resucitó, como dijo. Venid, ved el lugar donde estuvo puesto». \textit{Mt. 28, 5-6}. \textbf{\textit{Ave María}}.

        \item «Y marchando a toda prisa, decid a sus discípulos que resucitó de entre los muertos, 
            y he aquí que se os adelanta en ir a Galilea: allí le veréis. Conque os lo tengo dicho». 
            Y partiendo a toda prisa del monumento, con temor y grande gozo corrieron a dar la nueva a sus discípulos. \textit{Mt. 28, 7-8}. \textbf{\textit{Ave María}}.

        \item Pues así tambíén vosotros, ahora cierto tenéis congoja; mas otra vez os veré, y se gozará vuestro corazón, 
            y vuestro gozo nadie os lo quitará. \textit{Jn. 16,22}. \textbf{\textit{Ave María}}.

        \item Fuése María Magdalena a dar la nueva a los discípulos: «he visto al Señor, 
            y me ha dicho esto y esto». \textit{Jn. 20,18}. \textbf{\textit{Ave María}}.

        \item Siendo, pues, tarde aquel día, primero de la semana, y estando cerradas, por miedo a los judíos, 
            las puertas de la casa donde estaban los discípulos, vino Jesús y se presentó en medio de ellos y 
            les dice: «la paz sea con vosotros». \textit{Jn. 20,19}. \textbf{\textit{Ave María}}.

        \item Y en diciendo esto, les mostró las manos y el costado. Se gozaron, pues, 
            los discípulos al ver al Señor. \textit{Jn. 20,20}. \textbf{\textit{Ave María}} y \textbf{\textit{Gloria}}.

    \end{enumerate}
\end{document}