\documentclass[../../devocionario.tex]{subfiles}

\begin{document}
    \begin{enumerate}

        \item \textbf{\emph{Pater Noster}}. Pongo perpétua enemistad entre ti y el mujer. Y entre tu linaje y el suyo. Este te aplastará la cabeza.
            Y tú le acecharas el calcañal. \emph{Gen. 3, 15}. \textbf{\emph{Ave María}}.
    
        \item Una es mi paloma, mi pura; única es ella de su madre, la preferida de la que la dió a luz. Viéronla las doncellas, 
            y la felicitaron; las reinas y las concubinas, y exclamaron loándola. \emph{Cant. 6, 9}. \textbf{\emph{Ave María}}.

        \item Yo he salido de la boca del Altísimo. En las alturas he armado mi tienda y mi trono está en columna de nube. El círculo celeste he rodeado sola y en lo profundo 
            del abismo me he paseado; y en la tierra toda, y en todo pueblo y nación he imperado. En mi está toda la gracia del camino y de la verdad, en mi toda esperanza de 
            la vida y de la virtud. Venid a mí cuantos me deseáis y saciaos de mis frutos. Porque recordarme es más dulce que la miel, y poseerme más rico que el panal de miel. 
            \emph{Eci. 24, 5-11.26-27}. \textbf{\emph{Ave María}}.

        \item Y dijo María: Engrandece mi alma al Señor, y se regocijó mi espíritu em Dios, mi Salvador; porque puso sus ojos en la bajeza de su esclava. Pues he aquí que desde Ahora
            me llamarán dichosa todas las generaciones; porque hizo en mi favor grandes cosas el Poderoso, y cuyo nombre es «Santo»; y su misericordia por generaciones y generaciones
            para con aquellos que le temen. \emph{Lc. 1, 46-50}. \textbf{\emph{Ave María}}.

        \item ¿Quién es esa que aparece resplandeciente como la aurora, hermosa cual luna, deslumbradora como el sol, imponente como batallones? 
            \emph{Cant 6,10}. \textbf{\emph{Ave María}}.

        \item Y una gran señal fué vista en el cielo: una Mujer vestida del sol, y la luna debajo de sus pies, y sobre su cabeza una corona de doce estrellas. 
            \emph{Ap 12, 1}. \textbf{\emph{Ave María}}.

        \item Quién me obedece no se avergonzará y los que obran por no pecarán. Los que me esclarecen tendrán vida eterna. 
            \emph{Eci. 24, 30-31}. \textbf{\emph{Ave María}}.

        \item Parió un varón que ha de apacentar a todas las naciones con vara de hierro, pero el Hijo fue arrebatado a Dios y a su trono. La mujer huyó
            al desierto, en donde tenía un lugar preparado por Dios [, para que allí la alimentasen durante mil dosciento sesenta días]. \emph{Ap 12, 5}. \textbf{\emph{Ave María}}.

        \item Ahora, pues, hijos míos, oídme; y felices quienes guardan mis caminos. Escuchad la corrección y sed sabios, y no la rechacéis.  
            Feliz el hombre que me escucha, velando a mis puertas cada día, guardando las jambas de mis entradas. Pues quien me halla, ha hallado la vida y alcanza el favor 
            de Yahveh. Mas quién peca contra mi, se perjudica a si mismo, y cuantos me odian aman la muerte. \emph{Prv. 8, 32-35}. \textbf{\emph{Ave María}}.

        \item Tiene Él escrito en su vestido y en su manto Rey de reyes y Señor de los que dominan. Está la Reina a su derecha, adornada con oro finísimo. 
            \emph{Ap. 18, 16. Sal. 44, 10}. \textbf{\emph{Ave María}} y \textbf{\emph{Gloria}}

    \end{enumerate}
\end{document}