\documentclass[../../devocionario.tex]{subfiles}

\begin{document}
    \begin{enumerate}
    
        \item \textbf{\textit{Paternoster}}. Y al cumplirse el día de Pentecostés, estaban todos juntos en el mismo lugar. 
            Y se produjo de súbito desde el cielo un estruendo como de viento que soplaba vehemente, 
            y llenó toda la casa donde se hallaban sentados. \textit{Hch. 2, 1-2}. \textbf{\textit{Ave María}}.

        \item Y vieron aparecer lenguas como de fuego, que, repartiéndose, se posaban sobre cada uno de ellos. 
            Y se llenaron todos del Espíritu Santo, y comenzaron a hablar en lenguas diferentes, 
            según que el Espíritu Santo les movía a expresarse. \textit{Hch. 2, 3-4}. \textbf{\textit{Ave María}}.

        \item Hallábanse en Jerusalén judíos allí domiciliados, hombres religiosos de toda nación de 
            las que están debajo del cielo. \textit{Hch. 2, 5}. \textbf{\textit{Ave María}}.

        \item Puesto de pie Pedro, acompañado de los Once, alzó en voz y les habló en estos términos: 
            «Arrepentíos, dice, y bautícese cada uno de vosotros en el nombre del JesuCristo para remisión de vuestros pecados, 
            y recibiréis el don del Espíritu Santo». \textit{Hch. 2, 14.38}. \textbf{\textit{Ave María}}.

        \item Ellos, pues, acogieron su palabra, fueron bautizados; y fueron agragados 
            en aquel día como unas tres mil almas. \textit{Hch. 2,41}. \textbf{\textit{Ave María}}.

        \item Mas la fructificación del Espíritu es: caridad, gozo, paz, longanimidad, benignidad, 
            bondad, fe, mansedumbre, continencia; frente a tales cosas no tiene objeto la ley. \textit{Gál. 5,22-23}. \textbf{\textit{Ave María}}.

        \item Mas la sabiduría que viene de arriba primeramente es casta, luego pacífica, condescendiente, 
            que se allana a razones, llena de misericordia y de frutos buenos, 
            no amiga de criticar, no solapada. \textit{Sant. 3,17}. \textbf{\textit{Ave María}}.

        \item Porque no recibisteis espíritu de esclavitud para reincidir de nuevo en el temor; 
            antes recibisteis Espíritu de filiación adoptiva, con el cual clamamos: ¡Abba!¡Padre! 
            El Espíritu mismo testifica a una que somos hijos de Dios. \textit{Rom 8,15-16}. \textbf{\textit{Ave María}}.

        \item Y, asimismo, también el Espíritu acude en socorro de nuestras flaquezas. Pues qué hemos de orar, 
            según conviene, no lo sabemos; mas el Espíritu mismo interviene a favor nuestro con gemidos inefables. \textit{Rom. 8,26}. \textbf{\textit{Ave María}}.

        \item Mas el Paráclito, el Espiritu Santo, que enviará el Padre en mi nombre, Él os enseñará todas las cosas y os recordará todas 
            las cosas que os dije yo. \textit{Jn. 14,26}. \textbf{\textit{Ave María}} y \textbf{\textit{Gloria}}.

    \end{enumerate}
\end{document}

