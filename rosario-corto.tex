\documentclass[a4paper,11pt, oneside]{report}

\usepackage[utf8]{inputenc}
\usepackage[spanish]{babel}
\usepackage[T1]{fontenc}
\usepackage{biblatex}
\usepackage{tocloft}
\renewcommand{\cftchapleader}{\cftdotfill{\cftdotsep}}
\renewcommand{\cftchapfont}{\mdseries}
\renewcommand{\cftchappagefont}{\mdseries}

\setcounter{secnumdepth}{-2}

\newcommand\mypart[2]{\part{#1 \\ #2}}


\title{Devocionario}
\author{Sergio Rojo Herrero}
\date{Junio 2019}
\bibliography{books}

\begin{document}
  
  \begin{titlepage}
    \maketitle    
  \end{titlepage}

  \tableofcontents{}

  \newpage

  \chapter{Oraciones}

    \section{La Señal de la Cruz\cite{frater:oraciones}}
      En el Nombre del Padre, y del Hijo, y del Espíritu Santo. Amén.

      \medskip

      \textit{In Nómine Pátris, et Filii, et Spíritus Sancti. Amen.}

    \section{Oración Dominical (Padrenuestro)}
      
      Padre Nuestro que estás en los cielos, santificado sea tu Nombre. Venga a nosotros tu Reino. Hágase tu voluntad, así en la tierra como
      en el cielo. El pan nuestro de cada día dánosle hoy. Y perdónamos nuestras deuda, así como nosotros perdonamos a nuestros deudores.
      Y no nos dejes caer en la tentación: mas líbranos del mal. Amén.
      
      \medskip

      \textit{Pater noster, qui es in c{\oe}lis, sanctifificétur nomen tuum. Advéniat regnum tuum. Fiat voluntas tua, sicut in c{\oe}lo et in terra.
      Panem nóstrum quotidiánum da nobis hódie. Et dimite nosbis debita nostra, sicut et nos dimittimus debitóribus nostris. Et ne nos indúcas
      in tentatiónem: sed libera nos a malo. Amen}.\cite{frater:oraciones}

    \section{Salutación Evangélica (Avemaría)}
      Dios de salve, María, llena eres de gracia, el Señor es contigo; dendita eres entre todas las mujeres, y bendito es el fruto de tu
      vientre, Jeeús. Santa María, Madre de Dios, ruega por nosotros pecadores, ahora u en el hora de nuestra muerte. Amén.
      
      \medskip

      \textit{Ave María, grátia plena, Dóminus tecum; benedicta tu in muliéribus, et benedictum fructus ventris tui, Jesus.
      Sancta Maria, Mater Dei, ora pro nobis peccatóribus, nunc et in hora mortis nostr{\ae}. Amen.}\cite{frater:oraciones}

    \section{Símbolo de los Apóstoles (Credo Apostólico)}

      Creo en Dios, Padre todopoderoso. Creador del cielo y de la tierra. Y en Jesucristo, su único Hijo, Nuestro Señor, que fue concebido por
      obra y gracia del Espíritu Santo; nació de Santa María Vírgen; padeció bajo el poder de Poncio Pilato, fue crucificado, muerto y sepultado;
      descendió a los infiernos; al tercer día resucitó de entre los muertos; subió a los cielos, está sentado a la derecha de Dios Padre todopoderoso;
      desde allí ha de venir a juzgar a vivos y muertos. Creo en el Espíritu Santo, la Santa Iglesia Católica, la comunión de los Santos, el perdón
      de los pecados, la resurección de la carne y la vida eterna. Amén.

      \medskip

      \textit{Credo in Deum, Patrem omnipoténtem. Creatórem c{\oe}li et terr{\ae}. Et in Jesum CHristum, Filium ejus únicum, Dóminum nostrum; qui concéptus
      est de Spíritu Sancto; natus ex María Virgine; passus sub Póntio Pilato, crucifíxus, mortuus et sepúltus: descéndit ad inferos; tértia die resurréxit
      a mórtuis: ascéndit ad c{\oe}los, sedet ad dexteram Dei Patris omnipoténtis; inde ventúrus est judicáre vivos et mórtuos. Credo in Spíritum Sanctum,
      sanctam Ecclésiam cathólicam, Sanctórum communiónem, remisiónem peccatórum, carnis resurrectiónem, vitam {\ae}térnam. Amen.}\cite{frater:oraciones}

    \section{Gloria Patri}

      \noindent Gloria al Padre, al Hijo, y al Espíritu Santo.\\
      R. Como era en el principio, ahora, y siempre, y por los siglos de los siglos. Amén.

      \medskip

      \noindent \textit{Gloria Patri, et Filio, et Spíritui Sancto.\\
      R. Sicut erat in pcincípio et nunc, et semper et in s{\ae}cula s{\ae}culórum, Amen.}\cite{frater:oraciones}

    \section{Confíteor}

      Yo pecador me confieso a Dios todopoderoso, a la bienaventurada siempre Virgen María, al bienaventurado San Miguel Arcángel,
      al bienaventurado San Juan Bautista, a los Santos Apóstoles Pedro y Pablo, y a todos los Santos, porque pequé gravemente de 
      pensamiento, palabra y obra: por mi culpa, por mi culpa, por mi grandísima culpa. Por tanto, ruego a la bienaventurada siempre
      Virgen María, al bienaventurado San Miguel Arcángel, al bienaventurado San Juan Bautista, a los Santos Apóstoles Pedro y Pablo,
      y a todos los Santos, que roguéis por mi a Dios Nuestro Señor.\par\smallskip
      R. El Señor omnipotente tenga piedad de nosotros y, perdonados nuestros pecados, nos lleve a la vida eterna. Amén.\par\smallskip
      V. El Señor omnipotente y misericordioso nos conceda la indulgencia, la absolución y el perdón de nuestros pecados. Amén.

      \medskip

      \textit{Confíteor Deo omnipoténti, beát\ae Marí\ae Virigini, beáto Michaéli Archángelo, beáto Joánni Baptíst\ae, sanctis Apóstolis
      Petro et Paulo, et omníbus Sanctis, quia peccávi nimis, cogitatióne et ópere, mea culpa, mea culpa, mea máxima culpa. Ideo precor
      beátam Maríam semper Virgínem, beátum Michaélem Archángelum, beátum Joánnem Bastístam, sanctis Apóstolos Petrum et Paolum, et omnes
      Sanctos, oráre pro me ad Dóminum Deum nostrum.}\par\smallskip
      \textit{R. Misereátur nostri omnipotens Deus, et dimissis pecátis nostris, perdúcat nos ad vitam {\ae}térnam. Amen}\par\smallskip
      \textit{V. Indulgéntiam, absolutiónem, et remissiónem pecatórum nostrórum, tribuat nobis omnipotens et miséricors Dóminus. Amen.}\cite{frater:oraciones}
      
  \chapter{Santo Rosario}

    \section{Misterios Gozosos (Lunes y Jueves)}
      
      \subsection{La Anunciación de la Santísima Virgen María}
        Y habiendo entrado a ella, dijo: «Dios te salve, llena de gracia, el Señor es contigo, bendita tu entre las mujeres». Ella, al oír estas palabras, se turbó,
        y discurría que podría ser esta salucatión. Y le dijo el ángel: «No temas, María, pues hallaste gracia a los ojos de Dios. He aquí que concebirás en tu seno
        y darás a luz un Hijo, a quien daras por nombre Jesús. Este será grande, y será llamado Hijo del Altísimo, y le dará el Señor Dios el trono de DAvid su padre,
        y reinará sobre la casa de Jacob etérnamete, y su reinado no tendrá fin». Lucas 1, 28-33\cite{bover-cantera}.

      \subsection{La Visitación de Nuestra Señora}
        Y aconteció que, al oir Isabel la salutación de María, dió saltos de gozo el niño en su seno, y fue llena Isabel del Espíritu Santo, y levantó la voz con gran
        clamor y dijo: «Bendita tu entre las mujeres y bendito el fruto de tu vientre. ¿Y de dónde a mí esto que venga la madre de mi Señor a mí?». 
        Lucas 1, 41-43\cite{bover-cantera}
                      
      \subsection{La Natividad de Nuestro Señor Jesucristo}
        Y dió a luz su hijo primogénito, y le envolvió en pañales y le recostó en un pesebre, pues no había para ellos lugar en el mesón.
        Y les dijo el Ángel: «No tenáis, pues he aquí que os traigo una buena nueva, que será de grande alegría para todo el pueblo: que os ha nacido hoy en la ciudad de DAvid
        un Salvador, que es el Mesías, el Señor». Lucas 2, 7.10-11\cite{bover-cantera}.
    
      \subsection{La Presentación del Niño Jesús en el Templo}
        Y cuando se les cunplieron los días de la purificación según la ley de Moisés \footnote{Levítico 12, 6}, le subieron a Jerusalen para presentarle al Señor, según está escrito
        en la Ley del Señor que «todo primogénito del sexo masculino será consagrado al Señor\footnote{Éxodo 13, 2; 12, 15}», y para ofrecer como sacrificio, según lo que 
        se ordena en la Ley del Señor, «un par de tórtolas o dos palominos\footnote{Levítico 12, 8; 5, 11}». Lucas 2, 22-24
              
      \subsection{La Pérdida y Hallazgo del Niño Jesús en el Templo}
        Y no hallándole, se tornaron a Jerusalén para burcarle. Y sucedió que después de tres días le hallaron en el templo, sentado en medio de los maestros,
        escuchándolos y haciéndoles preguntas; y se pasmaban todos los que le oían de su inteligencia y de sus respuestas. Y sus padres padres, al verle, quedaron
        sorprendidos; y le dijo su madre: «Hijo, ¿por qué lo niciste así con nosotros? Mira que tu padre y yo, llenos de aflicción, te andábamos buscando»
        
      \newpage

    \section{Misterios Dolorosos (Martes y Viernes)}
      
      \subsection{La Agonía de Nuestro Señor en el Huerto de los Olivos}
        
      \subsection{La flagelación de Nuestro Señor Jesucristo}
        
      \subsection{La coronación de espinas de Nuestro Señor Jesucristo}
      
      \subsection{Jesús con la Cruz a cuestas}
        
      \subsection{La Crucifixión y Muerte del Redentor}
          
    \section{Miserios Gloriosos {Miércoles, Sábados y Domingos}}
      \subsection{La Resurección del Señor}
        
      \subsection{La Ascensión de Jesucristo a los cielos}
          
      \subsection{La Venida del Espíritu Santo sobre los Apóstoles}

      \subsection{La Asunción de Nuestra Señora a los cielos}

      \subsection{La Coronación de la Santísima Virgen María}

    \section{Letanías lauretanas}

  \chapter{Novenas}

  \printbibliography

\end{document}
