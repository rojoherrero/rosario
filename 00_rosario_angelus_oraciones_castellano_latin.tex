\documentclass[10pt,a4paper,oneside]{book}
\usepackage[utf8]{inputenc}
%\usepackage[spanish,latin]{babel}
\usepackage[spanish]{babel}
\usepackage[T1]{fontenc}
\usepackage[spanish]{cleveref}
\usepackage{geometry}
\geometry{
    a4paper,
    total={170mm,257mm},
    left=15mm,
    right=15mm,
    top=20mm,
}
\usepackage{lettrine}
\usepackage{color}
\usepackage{yfonts}
\usepackage{longtable}
\usepackage{subfiles}
\usepackage{comment}
\usepackage{gregoriotex}
\usepackage{graphicx}
\graphicspath{ {images/} }
\usepackage{fancyhdr} 
\fancyhf{}
\cfoot{\thepage}
\pagestyle{fancy}
\usepackage{afterpage}
\newcommand\blankpage{%
    \null
    \thispagestyle{empty}%
    \addtocounter{page}{-1}%
    \newpage
}

\usepackage{datetime}
\newdateformat{monthyeardate}{%
  \monthname[\THEMONTH] \THEYEAR
}
\setlength\parindent{0pt}
\newcommand{\primeraletragranderoja}[2]{
    \lettrine[lines=2]{\textcolor{red}{#1}}{#2} 
}

\newcommand{\primeraletragrande}[2]{
    \lettrine[lines=2]{#1}{#2}
}

\newcommand{\letraroja}[2]{
    \textcolor{red}{#1}{#2}
}

\newcommand{\versiculo}[1]{
    \textcolor{red}{\gothVbar.} {#1}.
}

\newcommand{\respuesta}[1]{
    \textcolor{red}{\gothRbar.} {#1}.
}

\newcommand{\textopequenorojo}[1]{
    {\textcolor{red}{\scriptsize{#1}}}
}

\newcommand{\versiculorespuesta}[2]{
    \versiculo{#1}\\
    \respuesta{#2}
}

\newcommand{\versiculorespuestaseguido}[2]{
    \versiculo{#1}\respuesta{#2}
}

\newcommand{\redcross}{
    \textcolor{red}{\grecross}
}

\newcommand{\lineahorizontal}[2]{
    \begin{center}
        {\rule{#1cm}{#2pt}}
    \end{center}
}




\title{
    {\Huge \uppercase{Santo Rosario y Ángelus}}\\
    {\includegraphics{foto-04.jpg}}\\
}
\author{Sergio}
\date{Valladolid (España), \monthyeardate\today}

\newcounter{joyful-counter}
\newcounter{sorrowful-counter}
\newcounter{glorious-counter}
\newcounter{lux-counter}

\begin{document}

\maketitle
\afterpage{\blankpage}

\chapter*{Santo Rosario Meditado}

\begin{minipage}[t]{0.475\textwidth}
      En el Nombre del Padre, y del{\redcross}Hijo, y del Espíritu Santo. Amén.\\

      \primeraletragranderoja{S}{eñor} mío Jesucristo, Dios y Hombre verdadero, Creador y Redentor mío: por ser vos quién sois, y porque os amo sobre todas las cosas,
      me pesa de todo corazón de haberos ofendido, propongo firmemente nunca más pecar, y apartarme de todas las ocasiones de ofenderos,
      confesarme, y cumplir la penitencia que me fuere impuesta; ofrézcoos mi vida, obras y trabajos en satisfacción de todos mis pecados;
      y confío en vuestra bondad y misericordia infinita me los perdonaréis por los merecimientos de vuestra preciosísima sangre, pasión y muerte,
      y me daréis gracia para enmendarme y para perseverar en vuestro santo servicio hasta el fin de mi vida. Amén.\\

      \versicleresponse{Abre, Señor, mis labios}{Y mi boca cantará tus alabanzas}\\
      \versicleresponse{Apresútare, Señor, a socorrerme}{Ven, oh Dios, en mi ayuda}\\\\
      \versicleresponse{Gloria al Padre, al Hijo, y al Espíritu Santo}{Como era en el principio, ahora, y siempre, y por los siglos de los siglos. Amén}\\\\
      \textsc{María}, Madre de gracia, Madre de Misericordia. Defendednos del enemigo y amparadnos ahora y en la hora de nuestra muerte. Amén. 
\end{minipage}
\lineaverticalroja{1}
\begin{minipage}[t]{0.475\textwidth}
      In Nómine Pátris, et{\redcross}Filii, et Spíritus Sancti. Amen.\\

      \primeraletragranderoja{C}{onfíteor} Deo omnipoténti, beát{\ae} Marí{\ae} semper Virigini, beáto Michaéli Archángelo, beáto Joánni Baptíst{\ae}, 
      sanctis Apóstolis Petro et Paulo, ómníbus Sanctis, quia peccávi nimis, cogitatióne, verbo et ópere, mea culpa, mea culpa, 
      mea máxima culpa. Ideo precor beátam     Maríam semper Virgínem, beátum Michaélem Archángelum, beátum Joánnem Baptístam, 
      sanctis Apóstolos Petrum et Paulum, omnes Sanctos, oráre pro me ad Dóminum Deum nostrum. Amen.\\
      Misereátur nostri omnipotens Deus, et, dimíssis peccátis nostris, perdúcat nos ad vitam {\ae}térnam. Amen.\\
      Indulgéntiam, absolutionem, et remissiónem peccatórum nostrórum tríbuat nobis omnípotens et miséricors Dóminus. Amen.\\

      \versicleresponse{Dómine, lábia mea apéries}{Et os meum annuntiábit laudem tuam}\\
      \versicleresponse{Deum, in adjutórium meum inténde}{Dómine, ad adjuvándum me festina}\\\\
      \versicleresponse{Glória Patri, et Filio, et Spíritui Sancto}{Sicut erat in princípio et nunc, et semper et in s{\ae}cula s{\ae}culórum. Amen}\\\\
      \textsc{Maria} Mater grati{\ae}, Mater Misericordi{\ae}. Tu nos ab hoste protege. Et mortis hora suspice.
\end{minipage}

\medskip

\begin{center}
      \textorojo{Decimos aquí las intenciones de este Rosario}
\end{center}

%%%%%%%%%%%%%%%%%%%%%%%%%%%%
% INICIO MISTERIOS GOZOSOS %
%%%%%%%%%%%%%%%%%%%%%%%%%%%%
\section*{Misterios Gozosos (Lunes y sábados)}

\stepcounter{joyful-counter}
\subsection*{\Roman{joyful-counter} Misterio: La Anunciación de la Santísima Virgen María}
\begin{flushright}
      {\color{red}Lc 1, 27-33}
\end{flushright}
\primeraletragranderoja{Y}{}habiendo entrado a ella, dijo: \quotes{Dios te salve, llena de gracia, el Señor es contigo, bendita tu entre las mujeres}. Ella, al oír estas palabras, se turbó, 
y discurría que podría ser esta salutación. Y le dijo el ángel: \quotes{No temas, María, pues hallaste gracia a los ojos de Dios. He aquí que concebirás en tu seno y darás a luz un Hijo, 
a quien darás por nombre Jesús. Este será grande, y será llamado Hijo del Altísimo, y le dará el Señor Dios el trono de David su padre, y reinará sobre la casa de Jacob etérnamente, 
y su reinado no tendrá fin}.

\begin{center}
      Paternóster, diez Avemarías, Gloria y María, Madre de gracia.
\end{center}

\stepcounter{joyful-counter}
\subsection*{\Roman{joyful-counter} Misterio: La Visitación de Nuestra Señora}
\begin{flushright}
      {\color{red}Lc 1, 39-45}
\end{flushright}
\primeraletragranderoja{P}{or} aquellos días, levantándose María, se dirigió presurosa a la montaña, a un ciudad de Judá, y entró en la casa de Zacarías y saludó a Isabel. 
Y aconteció que, al oir Isabel la salutación de María, dió saltos de gozo el niño en su seno, y fue llena Isabel del Espíritu Santo, y levantó la voz con gran clamor y dijo: 
Bendita tu entre las mujeres y bendito el fruto de tu vientre. ¿Y de dónde a mí esto que venga la madre de mi Señor a mí? Porque he aquí que, como sonó la voz de tu salutación en mi oídos, 
dió saltos de alborozo el niño en mi seno.

\begin{center}
      Paternóster, diez Avemarías, Gloria y María, Madre de gracia.
\end{center}

\stepcounter{joyful-counter}
\subsection*{\Roman{joyful-counter} Misterio: La Natividad del Nuestro Señor Jesucristo}
\begin{flushright}
      {\color{red}Lc 2, 7-8.10-11}
\end{flushright}
\primeraletragranderoja{Y}{} dió a luz su hijo primogénito, y le envolvió en pañales y le recostó en un pesebre, pues no había para ellos lugar en el mesón. 
Y había unos pastores en aquella misma comarca, que pernoctaban al raso y velaban por turno para guardar su ganado, y un ángel del Señor se presentó ante ellos. 
Y les dijo el Ángel: \quotes{No temáis, pues he aquí que os traigo una buena nueva, que será de grande alegría para todo el pueblo: 
que os ha nacido hoy en la ciudad de David un Salvador, que es el Mesías, el Señor}.

\begin{center}
      Paternóster, diez Avemarías, Gloria y María, Madre de gracia.
\end{center}

\stepcounter{joyful-counter}
\subsection*{\Roman{joyful-counter} Misterio: La Presentación del Niño Jesús en el Templo}
\begin{flushright}
      {\color{red}Lc 2, 22-24}
\end{flushright}
\primeraletragranderoja{Y}{} cuando se les cumplieron los días de la purificación según la ley de Moisés, le subieron a Jerusalén para presentarle al Señor, 
según está escrito en la Ley del Señor que \quotes{todo primogénito del sexo masculino será consagrado al Señor}, y para ofrecer como sacrificio, 
según lo que se ordena en la Ley del Señor, \quotes{un par de tórtolas o dos palominos}.

\begin{center}
      Paternóster, diez Avemarías, Gloria y María, Madre de gracia.
\end{center}

\stepcounter{joyful-counter}
\subsection*{\Roman{joyful-counter} Misterio: La pérdida y hallazgo del Niño Jesús en el Templo}
\begin{flushright}
      {\color{red}Lc 2, 46-48}
\end{flushright}
\primeraletragranderoja{Y}{} no hallándole, se tornaron a Jerusalén para buscarle. Y sucedió que después de tres días le hallaron en el templo, 
sentado en medio de los maestros, escuchándolos y haciéndoles preguntas; y se pasmaban todos los que le oían de su inteligencia y de sus respuestas. 
Y sus padres, al verle, quedaron sorprendidos; y le dijo su madre: \quotes{Hijo, ¿por qué lo hiciste así con nosotros? Mira que tu padre y yo, llenos de aflicción, 
te andábamos buscando}.

\begin{center}
      Paternóster, diez Avemarías, Gloria y María, Madre de gracia.\\
      Oraciones finales (\cpageref{final-prayer}).
\end{center}
%%%%%%%%%%%%%%%%%%%%%%%%%%%
% FINAL MISTERIOS GOZOSOS %
%%%%%%%%%%%%%%%%%%%%%%%%%%%

%%%%%%%%%%%%%%%%%%%%%%%%%%%%%%
% INICIO MISTERIOS LUMINOSOS %
%%%%%%%%%%%%%%%%%%%%%%%%%%%%%%
\section*{Misterios Luminosos (Jueves)}

\stepcounter{lux-counter}
\subsection*{\Roman{lux-counter} Misterio: El Bautismo del Señor en el Jordán}
\begin{flushright}
      {\color{red}Mc 1, 9-11}
\end{flushright}
\primeraletragranderoja{Y}{} aconteció por aquellos días que vino Jesús desde Nazaret de Galilea y fué bautizado en el Jordán por Juan el Bautista. 
Y al punto subiendo del agua, vió rasgarse los cielos y venir sobre Él el Espíritu Santo como paloma; y una voz vino de los cielos: 
\quotes{Tú eres mi Hijo amado, en Ti me agradé}.

\begin{center}
      Paternóster, diez Avemarías, Gloria y María, Madre de gracia.
\end{center}

\stepcounter{lux-counter}
\subsection*{\Roman{lux-counter} Misterio: La auto revelación de Jesús en las Bodas de Caná}
\begin{flushright}
      {\color{red}Jn 2, 1-5}
\end{flushright}
\primeraletragranderoja{A}{l} tercer día hubo una boda en Caná de Galilea, y estaba allí la madre de Jesús. Fue invitado también Jesús con sus discípulos a la boda. 
No tenían vino, porque el vino de la boda se había acabado. En esto dijo la madre de Jesús a este: No tiene vino. Díjole Jesús: Mujer, 
¿qué nos va a mi y a ti? No es aún llegada mi hora. Dijo la madre a los servidores: Haced lo que Él os diga.

\begin{center}
      Paternóster, diez Avemarías, Gloria y María, Madre de gracia.
\end{center}

\stepcounter{lux-counter}
\subsection*{\Roman{lux-counter} Misterio: El Anuncio del Rey de Dios}
\begin{flushright}
      {\color{red}Mc 1, 14-15.21-22}
\end{flushright}
\primeraletragranderoja{D}{espués} que Juan fue preso vino Jesús a Galilea predicando el Evangelio de Dios y diciendo: Cumplido es el tiempo, y el reino de Dios está cercano; 
arrepentíos y creed en el Evangelio. Llegaron a Cafarnaúm, y luego, el día sábado, entrando en la sinagoga, enseñaba. Se maravillaban de su doctrina, 
pues la enseñaba como quien tiene autoridad y no como los escribas.

\begin{center}
      Paternóster, diez Avemarías, Gloria y María, Madre de gracia.
\end{center}

\stepcounter{lux-counter}
\subsection*{\Roman{lux-counter} Misterio: La Transfiguración}
\begin{flushright}
      {\color{red}Mt 17, 1-3.5}
\end{flushright}
\primeraletragranderoja{S}{eis} días después tomó Jesús a Pedro, a Santiago y a Juan, su hermano, y los llevo aparte, a un monte alto. Y se transfiguró ante ellos; 
brilló su rostro como el sol y sus vestidos se volvieron blancos como la luz. Aún estaba el hablando, cuando los cubrió una nube resplandeciente, 
y salió de la nube una voz que decía: Este es mi Hijo amado, en quien tengo mi complacencia; escuchadle.

\begin{center}
      Paternóster, diez Avemarías, Gloria y María, Madre de gracia.
\end{center}

\stepcounter{lux-counter}
\subsection*{\Roman{lux-counter} Misterio: La Institución de la Eucaristía}
\begin{flushright}
      {\color{red}Mt 26, 26-28}
\end{flushright}
\primeraletragranderoja{M}{ientras} comían, Jesús tomó pan, lo bendijo, lo partió y, dándoselo a los discípulos, dijo: Tomad y comed, éste es mi cuerpo. Y tomando un cáliz y dando gracias, 
se lo dió, diciendo: Bebed de él todos, que está es mi sangre del Nuevo Testamento, que será derramada por muchos para remisión de los pecados.

\begin{center}
      Paternóster, diez Avemarías, Gloria y María, Madre de gracia.\\
      Oraciones finales (\cpageref{final-prayer}).
\end{center}
%%%%%%%%%%%%%%%%%%%%%%%%%%%%%
% FINAÑ MISTERIOS LUMINOSOS %
%%%%%%%%%%%%%%%%%%%%%%%%%%%%%

%%%%%%%%%%%%%%%%%%%%%%%%%%%%%%
% INICIO MISTERIOS DOLOROSOS %
%%%%%%%%%%%%%%%%%%%%%%%%%%%%%%

\section*{Misterios Dolorosos (Martes y viernes)}

\stepcounter{sorrowful-counter}
\subsection*{\Roman{sorrowful-counter} Misterio: La Agonía de Nuestro Señor en el Huerto de los Olivos}
\begin{flushright}
      {\color{red}Mc 14, 33-36}
\end{flushright}
\primeraletragranderoja{Y}{} lleva consigo a Pedro y a Santiago y a Juan, y comenzó a sentir espanto y abatimiento; y le dice: \quotes{triste en gran manera está mi corazón hasta la muerte; 
quedad aquí y velad}. Y apartándose un poco, caía sobre tierra, y rogaba que, a ser posible, pasase el Él aquella hora, y decía: \quotes{Abba, Padre, todas las cosas te son posibles: 
traspasa de mi este cáliz; más no se haga lo que yo quiero, sino lo que tú quieres}.

\begin{center}
      Paternóster, diez Avemarías, Gloria y María, Madre de gracia.
\end{center}

\stepcounter{sorrowful-counter}
\subsection*{\Roman{sorrowful-counter} Misterio: La Flagelación de Nuestro Señor Jesucristo}
\begin{flushright}
      {\color{red}Jn 18,38-40; 19, 1}
\end{flushright}
\primeraletragranderoja{{\guillemotleft}Y}{o} no hallo en Él delito alguno. Es costumbre vuestra que yo os suelte un preso por la Pascua: ¿queréis, pues, que os suelte al rey de los Judíos?{\guillemotright}. 
Gritaron, pues, de nuevo, diciendo: \quotes{No, a ése, sino a Barrabás}.Era este Barrabás un salteador. Entonces, pues, tomó Pilato a Jesús y le azotó.

\begin{center}
      Paternóster, diez Avemarías, Gloria y María, Madre de gracia.
\end{center}

\stepcounter{sorrowful-counter}
\subsection*{\Roman{sorrowful-counter} Misterio: La Coronación de espinas de Nuestro Señor Jesucristo}
\begin{flushright}
      {\color{red}Mt 27, 27-30}
\end{flushright}
\primeraletragranderoja{E}{ntonces} los soldados del gobernador, tomando a Jesús y conduciéndole al pretorio, reunieron en torno a Él toda la cohorte. Y habiéndole quitado sus vestidos, 
le envolvieron en una clámide de grana, y trenzando una corona de espinas, la pusieron sobre su cabeza, y una caña en su mano derecha; y doblando la rodilla delante de Él, 
le mofaban, diciendo: \quotes{Salud, Rey de los judíos}. Y escupiendo en Él, tomaron la caña y le daban golpes en la cabeza.

\begin{center}
      Paternóster, diez Avemarías, Gloria y María, Madre de gracia.
\end{center}

\stepcounter{sorrowful-counter}
\subsection*{\Roman{sorrowful-counter} Misterio: El Señor con la cruz a cuestas}
\begin{flushright}
      {\color{red}Jn 19, 16-17; Lc 23, 26}
\end{flushright}
\primeraletragranderoja{E}{ntonces}, pues, se le entregó para que fuera crucificando. Se apoderaron, pues, de Jesús, y llevando a cuestas su cruz, salió hacia el lugar llamado el Cráneo, 
que en hebreo se dice Gólgota. Y como le hubieron sacado, echaron mano de un tal Simón de Cirene que venía del campo, le pusieron en hombros la cruz para que la llevase detrás de Jesús.

\begin{center}
      Paternóster, diez Avemarías, Gloria y María, Madre de gracia.
\end{center}

\stepcounter{sorrowful-counter}
\subsection*{\Roman{sorrowful-counter} Misterio: Crucifixión y Muerte del Redentor}
\begin{flushright}
      {\color{red}Lc 23, 33-34; Jn 19, 19.25-27; Lc 23, 44-46}
\end{flushright}
\primeraletragranderoja{C}{uando} llegaron al lugar llamado Calvario, le crucificaron allí, y a los dos malhechores, uno a la derecha y otro a la izquierda. Jesús decía: 
Padre, perdónalos, porque no saben los que hacen. Dividiendo sus vestidos, echaron suertes sobre ellos. Escribió Pilato un título y lo puso sobre la cruz; 
estaba escrito: \textit{Jesús Nazareno, Rey de los judíos}. Estaba junto a la cruz de Jesús su Madre y la hermana de su Madre, María la debajo Cleofás y María Magdalena. 
Jesús, viendo a su Madre y al discípulo a quien amaba, que estaba allí, dijo a la Madre: Mujer, he ahí a tu hijo. Luego dijo al discípulo: He ahí a tu Madre. 
Y desde aquella hora el discípulo la recibió en su casa. Era ya como la hora de sexta, y las tinieblas cubrieron toda la tierra hasta la hora de nona, 
obscurecióse el sol y el velo del templo se rasgó por medio. Jesús, dando una gran voz, dijo: Padre, en tus manos entrego mi espíritu; y diciendo esto expiró.

\begin{center}
      Paternóster, diez Avemarías, Gloria y María, Madre de gracia.\\
      Oraciones finales (\cpageref{final-prayer}).
\end{center}
%%%%%%%%%%%%%%%%%%%%%%%%%%%%%
% FINAL MISTERIOS DOLOROSOS %
%%%%%%%%%%%%%%%%%%%%%%%%%%%%%

%%%%%%%%%%%%%%%%%%%%%%%%%%%%%%
% INICIO MISTERIOS GLORIOSOS %
%%%%%%%%%%%%%%%%%%%%%%%%%%%%%%
\section*{Misterio Gloriosos (Miércoles y domingos)}
\stepcounter{glorious-counter}
\subsection*{\Roman{glorious-counter} Misterio: La Resurrección del Señor}
\begin{flushright}
      {\color{red}Mt 28, 1-3.5-7}
\end{flushright}
\primeraletragranderoja{P}{asado} el sábado, ya para amanecer el día primero de la semana, vino María Magdalena con la otra María al sepulcro. Y sobrevino un gran terremoto, 
pues un ángel del Señor bajó del cielo y acercándose removió la piedra del sepulcro y se sentó sobre ella. Era su aspecto como el relámpago, y su vestidura blanca como la nueve. 
El ángel, dirigiéndose a las mujeres, dijo: No temáis vosotras, pues sé que buscáis a Jesús el crucificado. No está aquí; ha resucitado, según lo había dicho. 
Venid y ved el sitio donde fue puesto. Id luego y decid a sus discípulos que ha resucitado de entre los muertos y que os precede a Galilea; allí le veréis. 
Es lo que tenía que deciros.

\begin{center}
      Paternóster, diez Avemarías, Gloria y María, Madre de gracia.
\end{center}

\stepcounter{glorious-counter}
\subsection*{\Roman{glorious-counter} Misterio: Las Ascensión Jesucristo a los cielos}
\begin{flushright}
      {\color{red}Lc 24, 50; Mc 16, 15-16.19-20}
\end{flushright}
\primeraletragranderoja{L}{os} llevó hasta cerca de Betania, y levantando sus manos les bendijo. Y les dijo: Id al mundo entero y predicad el Evangelio a toda la creación. 
El que creyere y fuere bautizado, se salvará, mas el que no creyere, será condenado. Con esto el Señor Jesús, después de hablarles, fue elevado al cielo y se sentó a la diestra de Dios. 
Y ellos, partiéndose de allí, predicaron por todas partes, cooperando el Señor y confirmando la palabra con las señales que la acompañaban.

\begin{center}
      Paternóster, diez Avemarías, Gloria y María, Madre de gracia.
\end{center}

\stepcounter{glorious-counter}
\subsection*{\Roman{glorious-counter} Misterio: La Venida del Espíritu Santo sobre los Apóstoles}
\begin{flushright}
      {\color{red}Jn 14, 26; Hch. 2, 1-4}
\end{flushright}
\primeraletragranderoja{M}{as} el Paráclito, el Espíritu Santo, que enviará el Padre en mi nombre, Él os enseñará todas las cosas y os recordará todas las cosas que os dije yo. 
Y al cumplirse el día de Pentecostés, estaban todos juntos en el mismo lugar. Y se produjo de súbito desde el cielo un estruendo como de viento que soplaba vehementemente, 
y llenó toda la casa donde se hallaban sentados. Y vieron aparecer lenguas como de fuego, que, repartiéndose, se posaban sobre cada uno de ellos.

\begin{center}
      Paternóster, diez Avemarías, Gloria y María, Madre de gracia.
\end{center}

\stepcounter{glorious-counter}
\subsection*{\Roman{glorious-counter} Misterio: La Asunción de María Santísima a los cielos}
\begin{flushright}
      {\color{red}Lc 1, 46-49; Ap 11, 19}
\end{flushright}
\primeraletragranderoja{M}{i} alma magnifica al Señor y exulta de júbilo mi espíritu al Señor, mi Salvador, porque ha mirado la humildad de su sierva; 
por eso todas las generaciones me llamarán bienaventurada, porque ha hecho en mi maravillas el Poderoso, cuyo nombre es santo. Y se abrió el templo de Dios, que está en el cielo, 
y fué vista el arca de la alianza en el templo, y se produjeron relámpagos, y voces, y truenos, y temblor de tierra, y fuerte granizada.

\begin{center}
      Paternóster, diez Avemarías, Gloria y María, Madre de gracia.
\end{center}

\stepcounter{glorious-counter}
\subsection*{\Roman{glorious-counter} Misterio: La Coronación de la Santísima Virgen María}
\begin{flushright}
      {\color{red}Cant. 6,10; Ap 12, 1; 18, 16; Sal. 44, 10}
\end{flushright}
\primeraletragranderoja{{?`}E}{uién} es esa que aparece resplandeciente como la aurora, hermosa cual luna, deslumbradora como el sol, imponente como batallones?. 
Y una gran señal fué vista en el cielo: una Mujer vestida del sol, y la luna debajo de sus  pies, y sobre su cabeza una corona de doce estrellas. 
Tiene Él escrito en su vestido y en su manto Rey de reyes y Señor de los que dominan. Está la Reina a su derecha, adornada con oro finísimo.

\begin{center}
      Paternóster, diez Avemarías, Gloria y María, Madre de gracia.
\end{center}
%%%%%%%%%%%%%%%%%%%%%%%%%%%%%
% FINAL MISTERIOS GLORIOSOS %
%%%%%%%%%%%%%%%%%%%%%%%%%%%%%
%%%%%%%%%%%%%%%%%%%%%%%%%%%%%%%%
% ORACIONES FINALES Y LETANIAS %
%%%%%%%%%%%%%%%%%%%%%%%%%%%%%%%%
\bigskip

\label{final-prayer}
\begin{center}
    \textorojo{Terminados los misterios podemos rezar}
\end{center}

\medskip

\begin{minipage}[t]{0.475\textwidth}
    \redletter{D}{ios} te Salve, María, Hija de Dios Padre, Virgen purísima y castísima antes del parto, llena eres de gracia{\ldots}\\\\
    \redletter{D}{ios}te Salve, María, Madre de Dios Hijo, Virgen purísima y castísima en el parto, llena eres de gracia{\ldots}\\\\
    \redletter{D}{ios} te Salve, María, Esposa de Dios Espíritu Santo, Virgen purísima y castísima después del parto, llena eres de gracia{\ldots}\\\\
    \redletter{S}{anta}, templo y sagrario de la Santísima Trinidad. Gloria al Padre{\ldots}
\end{minipage}
\lineaverticalroja{1}
\begin{minipage}[t]{0.475\textwidth}
    \redletter{A}{ve}, Filia Dei Patri, Virgo purissima et castissima ante partum, gratia plena{\ldots}\\\\
    \redletter{A}{ve} Maria, Mater Dei Filii, Virgo purissima et castissima in partu, gratia plena{\ldots}\\\\
    \redletter{A}{ve} Maria, Sponsa Spíritus Sanctii, Virgo purissima et castissima post partum, gratia plena{\ldots}\\\\
    \redletter{S}{ancta} Maria, templum et sacrarium totis Sanctissim{\ae} Trinitatis. Glória Patri{\ldots}
\end{minipage}

\bigskip\medskip


\begin{center}
    \textorojo{Letanías de Nuestra Señora}
\end{center}

\begin{minipage}[t]{0.475\textwidth}
    \primeraletragranderoja{S}{eñor}, ten piedad.\\
    Cristo, ten piedad.\\
    Señor, ten piedad.\\
    Cristo, óyenos.\\
    Cristo, escúchanos.\\
    Dios Padre celestial, ten piedad de nosotros\\
    Dios Hijo Redentor del mundo, ten piedad de nosotros\\
    Dios Espíritu Santo, ten piedad de nosotros\\
    Trinidad Santa, un solo Dios, ten piedad de nosotros\\
    Santa María, ruega por nosotros.\\
    Santa Madre de Dios, ruega.\\
    Santa Virgen de las Vírgenes, ruega.\\
    Madre de Cristo, ruega.\\
    Madre de la divina gracia, ruega.\\
    Madre purísima, ruega.\\
    Madre castíssima, ruega.\\
    Madre virignal, ruega.\\
    Madre sin corrupción, ruega.\\
    Madre inmaculada, ruega.\\
    Madre amable, ruega.\\
    Madre admirable, ruega.
\end{minipage}
\lineaverticalroja{1}
\begin{minipage}[t]{0.475\textwidth}
    \primeraletragranderoja{K}{ýrie}, eléison.\\
    Christe, eléison.\\
    Kýrie, eléison.\\
    Christe, audi nos.\\
    Christe, exáudi nos.\\
    Pater de c{\ae}lis, Deus, miserére nobis.\\
    Fili, Redémptor mundi, Deus, miserére nobis.\\
    Spíritus Sancte, Deus, miserére nobis.\\
    Sancta Trínitas, unus Deus, miserére nobis.\\
    Sancta Maria, ora pro nobis.\\
    Sancta Dei Génetrix, ora.\\
    Sancta Virgo vírginum, ora.\\
    Mater Christi, ora.\\
    Mater divín{\ae} grati{\ae}, ora.\\
    Mater puríssima, ora.\\
    Mater castíssima, ora.\\
    Mater invioláta, ora.\\
    Mater intemeráta, ora.\\
    Mater immaculáta, ora.\\
    Mater amábilis, ora.\\
    Mater admirábilis, ora.
\end{minipage}

\begin{minipage}[t]{0.475\textwidth}
    Madre del buen Consejo, ruega por nosotros.\\
    Madre del Creador, ruega.\\
    Madre del Salvador, ruega.\\
    Virgen prudentísima, ruega.\\
    Virgen digna de veneración, ruega.\\
    Virgen digna de alabanza, ruega.\\
    Virgen poderosa, ruega.\\
    Virgen clemente, ruega.\\
    Virgen fiel, ruega.\\
    Espejo de justicia, ruega.\\
    Sede de la sabiduría, ruega.\\
    Causa de nuestra alegría, ruega.\\
    Vaso espiritual, ruega.\\
    Vaso honorable, ruega.\\
    Vaso insigne de devoción, ruega.\\
    Rosa mística, ruega.\\
    Torre de David, ruega.\\
    Torre de marfil, ruega.\\
    Casa de oro, ruega.\\
    Arca de la alianza, ruega.\\
    Puerta del cielo, ruega.\\
    Estrella de la mañana, ruega.\\
    Salud de los enfermos, ruega.\\
    Refugio de los pecadores, ruega.\\
    Consuelo de la afligidos, ruega.\\
    Auxilio de los cristianos, ruega.\\
    Reina de los Ángeles, ruega.\\
    Reina de los Patriarcas, ruega.\\
    Reina de los Profetas, ruega.\\
    Reina de los Apóstoles, ruega.\\
    Reina de los Mártires, ruega.\\
    Reina de los Confesores, ruega.\\
    Reina de las Vírgenes, ruega.\\
    Reina de todos los Santos, ruega.\\
    Reina concebida sin pecado original, ruega.\\
    Reina asunta al cielo, ruega.\\
    Reina del santísimo Rosario, ruega.\\
    Reina de la paz, ruega.\\
    Cordero de Dios, que quitas los pecados del mundo, perdónanos, Señor.\\
    Cordero de Dios, que quitas los pecados del mundo, escúchanos, Señor.\\
    Cordero de Dios, que quitas los pecados del mundo, ten piedad de nosotros.\\
    \ruegapornosotrossalve\\

    \textbf{Oremos}
    \primeraletragranderoja{T}{e} rogamos nos concedas, Señor Dios nuestro, gozar de continua salud de alma y cuerpo, y por la gloriosa intercesión de la 
    bienaventurada siempre Virgen María, vernos libres de las tristezas de la vida presente y disfrutar de las alegrías eternas.
    Por Nuestro Señor Jesucristo. \response{Amén}\\
\end{minipage}
\lineaverticalroja{1}
\begin{minipage}[t]{0.475\textwidth}
    Mater boni Consílii, ora pro nobis.\\
    Mater Creatóris, ora.\\
    Mater Salvatóris, ora.\\
    Virgo prudentíssima, ora.\\
    Virgo veneránda, ora.\\
    Virgo pr{\ae}dicánda, ora.\\
    Virgo potens, ora.\\
    Virgo clemens, ora.\\
    Virgo fidélis, ora.\\
    Spéculum iustíti{\ae}, ora.\\
    Sedes Sapiénti{\ae}, ora.\\
    Causa nostr{\ae} l{\ae}títi{\ae}, ora.\\
    Vas spirituále, ora.\\
    Vas honorábile, ora.\\
    Vas insigne devotiónis, ora.\\
    Rosa mýstica, ora.\\
    Turris Davídica, ora.\\
    Turris ebúrnea, ora.\\
    Domus áurea, ora.\\
    F{\oe}deris arca, ora.\\
    Iánua c{\ae}li, ora.\\
    Stella matutina, ora.\\
    Salus infirmórum, ora.\\
    Refugium peccatórum, ora.\\
    Consolátrix afflictórum, ora.\\
    Auxílium christianórum, ora.\\
    Regina Angelórum, ora.\\
    Regina Patriarchánum, ora.\\
    Regina Prophetárum, ora.\\
    Regina Apostolórum, ora.\\
    Regina Mártyrum, ora.\\
    Regina Confessórum, ora.\\
    Regina Vírginum, ora.\\
    Regina Sanctórum ómnium, ora.\\
    Regina sine labe originali concépta, ora.\\
    Regina in c{\ae}lum assumpta, ora.\\
    Regina sacratíssimi Rosárii, ora.\\
    Regina pacis, ora.\\
    Ágnus Dei, qui tolli peccáta mundi, Parce nobis, Dómine.\\
    Ágnus Dei, qui tolli peccáta mundi, Exáudi nos, Dómine.\\
    Ágnus Dei, qui tolli peccáta mundi, Miserére nobis.\\
    \orapronobissalve\\

    \textbf{Orémus}
    \primeraletragranderoja{C}{oncéde} nos fámulos tuos, qu{\'\ae}sumus, Dómine Deus, perpétua mentis et córporis sanitáte gaudére: et, gloriosa beát{\ae}
    Marí{\ae} semper Virginis intercessióne, a pr{\ae}sénti liberári tristítia, et {\ae}térnan pérfrui l{\ae}títia. Per Christum, Dóminum nostrum.
    \response{Amen}\\
\end{minipage}

\newpage

\noindent\small{Desde la Purificación hasta Sábado Santo y desde la Santísima Trinidad hasta el I Domingo de Adviento:}\\
\begin{minipage}[t]{0.475\textwidth}
    \ruegapornosotrossalve\\\\
    \textbf{Oremos}
    \primeraletragranderoja{T}{e} rogamos, Señor, que nos concedas a nosotros tus siervos, gozar de perpetua salud de alma y cuerpo, y por la gloriosa intercesión de la bienaventurada siempre Virgen 
    María, seamos librados de la tristeza presente y disfrutemos de la eterna alegría. Por el mismo Jesucristo Nuestro Señor, Tu Hijo, que vive contigo y reina en unidad con el mismo Espíritu
    Santo, Dios, por lo siglos de los siglos.
\end{minipage}
\lineaverticalroja{1}
\begin{minipage}[t]{0.475\textwidth}
    \orapronobissalve\\\\
    \textbf{Orémus} 
    \primeraletragranderoja{C}{oncéde} nos famulos tuos, qu{\'\ae}sumus, Dómine Deus, perpétua mentis et córporis sanitáte gaudére: et gloriósa beát{\ae}\ Marí{\ae} semper Virginis intercessióne
    a pr{\ae}sénti liberári tristítia, et {\ae}térna pérfrui l{\ae}títia. Per eudem Dóminum nostrum Jesum Christium Filium tuum, qui tecum vivit et regnat im unitáte Spiritus Sancti, 
    Deus, per ómnia s{\'\ae}cula s{\'\ae}culórum. Amen.\\
\end{minipage}

\bigskip

\noindent\small{Desde el I Domingo de Adviento hasta Navidad:}\\
\begin{minipage}[t]{0.475\textwidth}
    \versicleresponse{El Ángel del Señor anunció a María}{Y ella concibió por obra y gracia del Espíritu Santo}\\\\
    \textbf{Oremos}
    \primeraletragranderoja{O}{h} Dios, que quisiste que tu Verbo tomase nuestra carne de las entrañas de la Santísima Virgen, al anunciarle el Ángel el misterio: concede a tus siervos que, pues
    la creemos verdadera Madre de Dios, seamos ayudados anti Ti por su intercesión. Por el mismo Jesucristo Nuestro Señor, Tu Hijo, que vive contigo y reina en unidad con el mismo Espíritu
    Santo, Dios, por lo siglos de los siglos.
\end{minipage}
\lineaverticalroja{1}
\begin{minipage}[t]{0.475\textwidth}
    \versicleresponse{Angelus Dómini nuntiávit Marí{\ae}}{Et concépit de Spíritu Sancto}\\\\
    \textbf{Orémus}
    \primeraletragranderoja{D}{eus}, qui de beát{\ae} Marí{\ae} Virginis útero Verbum tuum, Angelo nuntiánte; carnem suscípere voluisti; pr{\ae}ta supplicibus tuis; ut, qui vere eam Genetrícem
    Dei crédimus, ejus apud te intercessiónibus adjuvémur. Per eudem Dóminum nostrum Jesum Christium Filium tuum, qui tecum vivit et regnat im unitáte Spiritus Sancti, Deus, per ómnia
    s{\'\ae}cula s{\'\ae}culórum. Amen.
\end{minipage}

\bigskip

\noindent\small{Desde Navidad hasta la Purificación:}\\
\begin{minipage}[t]{0.475\textwidth}
    \versicleresponse{Después del parto, oh Virgen, has permanecido intacta}{Madre de Dios, intercedes por nosostros}\\\\
    \textbf{Oremos}
    \primeraletragranderoja{O}{h} Dios, que por la fecunda virginidad de la bienaventurada siempre Virgen María has concedido al género humano los bienes de la salvación eterna, haznos sentir
    la intercesión de aquélla, por quien hemos merecido recibir al autor de la vida, Jesucristo, tu hijo y Señor nuestro, que vive y reina en unidad con el Espíritu Santo, Dios, por 
    todos los siglos de los siglos. Amén.
\end{minipage}
\lineaverticalroja{1}
\begin{minipage}[t]{0.475\textwidth}
    \versicleresponse{Post partum Virgo invioláta permansísti}{Dei Génitrix, intercéde pro nobis}\\\\
    \textbf{Orémus} 
    \primeraletragranderoja{D}{eus}, qui salútis {\ae}térn{\ae}. beát{\ae} Marí{\ae} virginitáte f{\oe}cúnda, humáno géneri pr{\ae}mia pr{\ae}stitísti; tríbue, qu{\ae}sumus, ut ipsam pro nobis 
    intercédere sentiámus, per quam merúsmus acutórem vit{\ae} suscípere, Dóminum nostrum Jesum Christum Filium tuum, qui tecum vivit et regnat in unitáte Spíritus Sacnti, Deus, per ómnia
    s{\'\ae}cula s{\'\ae}culórum. Amen.
\end{minipage}

\bigskip

\noindent\small{Domingo de Páscua hasta la Santísima Trinidad:}\\
\begin{minipage}[t]{0.475\textwidth}
    \versicleresponse{Reina del cielo, alégrate, aleluya}{Porque el que mereciste llevar en tu seno, aleluya}\\
    \versicleresponse{Resucitó, como Él predijo, aleluya}{Rogad por nosotros a Dios, aleluya}\\ 
    \versicleresponse{Alegraos y regocijaos, Virgen María, aleluya}{Porque resucitó verdaderamente el Señor, aleluya}\\\\
    \textbf{Oremos}
    \primeraletragranderoja{O}{h} Dios, que, por la resurreción de vuestro Hijo y Señor nuestro Jesucristo,
    os habéis dignado alegrar el mundo: concedednos por medio de su divina Madre, la Virgen Santísima,
    que merezcamos obtener los goces de la vida eterna. Por el mismo Cristo, Señor nuestro. Amén. 
\end{minipage}
\lineaverticalroja{1}
\begin{minipage}[t]{0.475\textwidth}
    \versicleresponse{Regina caeli l{\ae}táre, allelúia}{Quia quem meruisti portáre, allelúia}\\
    \versicleresponse{Resurréxit sicut dixit, allelúia}{Ora pro nobis Deum, allelúia}\\ 
    \versicleresponse{Gaudate el l{\ae}táre, Virgo María, allelúia}{Quia surréxit Dóminus vere, allelúia}\\\\
    \textbf{Orémus}
    \primeraletragranderoja{D}{eus}, qui per resurrectiónem Filii tui Dómini nostri Jesu Christi,
    mundum l{\ae}tificáre dignátus es: pr{\ae}sta, qu{\'\ae}sumus, ut per ejus Genetricem Vírginem Maríam,
    perpétu{\ae} capiámus gáudia vit{\ae}. Per eúmdem Christum Dóminum nostrum. Amen.
\end{minipage}

\bigskip

\textbf{Por el Santo Padre y sus intenciones}:\\
\hspace*{10mm}Paternóster, Avemaría y Gloria.\\
\textbf{Por la Hispanidad}:\\ 
\hspace*{10mm}Paternóster, Avemaría, Gloria.\\
\textbf{Por el nuestro Obispo, sacerdotes, seminaristas, monjas\ldots}:\\ 
\hspace*{10mm}Paternóster, Avemaría\\
\hspace*{10mm}Enviad, Señor, sacerdotes santos a vuestra Iglesia.\\
\hspace*{10mm}{!`}Oh Jesús, Salvador del mundo! santificad a vuestros sacerdotes y seminaristas.\\
\hspace*{10mm}Oh María, Reina del sacerdocio, alcanzadnos santos y numerosos sacerdotes.\\
\hspace*{10mm}Gloria.\\
\textbf{Por las benditas almas de Purgatorio}:\\
\hspace*{10mm}Paternóster, Avemaría, Descansen en paz. Amén. (requiéscant in pace. Amen)

\bigskip
{\label{hailMaryQueen}}Una Salve al Inmaculado Corazón de María:\\
\smallskip
\begin{minipage}[t]{0.475\textwidth}
    \primeraletragranderoja{D}{ios te salve}, Reina y Madre de misericordia, vida, dulzura y esperanza nuestra; Dios te salve.
    A ti llamamos los desterrados hijos de Eva; a ti suspiramos, gimiendo y llorando en este valle de lágrimas.
    Ea, pues, Señora, abogada nuestra, vuelve a nosotros esos tus ojos misericordiosos. Y después de este destierro, muéstranos a Jesús,
    fruto bendito de tu vientre. {!`}Oh clementísima, oh piadosa, oh dulce siempre Virgen María!\\
    \ruegapornosotrossalve
\end{minipage}
\lineaverticalroja{1}
\begin{minipage}[t]{0.475\textwidth}
    \primeraletragranderoja{S}{alve}, Regina, Mater misericórdi{\ae}, vita, dulcédo et spes nostra, salve. Ad te clamámus, éxsules fílii Hev{\ae}.
    Ad te suspirámus geméntes et flentes in hac lacrimárum valle. Éia ergo, advocáta nostra, illos tuos misericórdes óculos ad nos convérte.
    Et Iesum benedíctum fructum ventris tui, nobis, post hoc exsílium, osténde. O clemens, o pia, o dulcis Virgo Maria!\\
    \orapronobissalve
\end{minipage}

\bigskip
{\label{creed-apostles}}
Un Credo al Sagrado Corazón de Jesús:\\
\smallskip
\begin{minipage}[t]{0.475\textwidth}
    \primeraletragranderoja{C}{reo en Dios}, Padre todopoderoso. Creador del cielo y de la tierra. Y en Jesucristo, su único Hijo, Nuestro Señor,
    que fue concebido por obra y gracia del Espíritu Santo nació de Santa María Vírgen; padeció bajo el poder de Poncio Pilato,
    fue crucificado, muerto y sepultado; descendió a los infiernos; al tercer día resucitó de entre los muertos; subió a los cielos,
    está sentado a la derecha de Dios Padre todopoderoso; desde allí ha de venir a juzgar a vivos y muertos.
    Creo en el Espíritu Santo, la Santa Iglesia Católica, la comunión de los Santos, el perdón de los pecados,
    la resurección de la carne y la vida eterna. \response{Amén}\\

    \primeraletragranderoja{A}{ vos}, bienaventurado San José, acudimos en nuestra tribulación, y después de invocar el auxilio de vuestra Santísima Esposa, solicitamos
    también confiadamente vuestro patrocinio. Por aquella Caridad que con la Inmaculada Virgen María, Madre de Dios, os tuvo unido y, por el paterno amor 
    con que abrazasteis al Niño Jesús, humildemente os suplicamos volváis benignolos ojos a la herencia que con su Sangre adquirió Jesucristo, y con vuestro 
    poder y auxilio socorráis nuestras necesidades.Proteged, oh providentísimo Custodio de la Sagrada Familia la escogida descendencia de Jesucristo; 
    apartad de nosotros toda mancha de error y corrupción; asistidnos propicio, desde el Cielo, fortísimo libertador nuestro en esta lucha con el poder de
    las tinieblas; y como en otro tiempo librasteis al Niño Jesús del inminente peligro de su vida, así, ahora, defended la Iglesia Santa de Dios de las 
    asechanzas de sus enemigos y de toda adversidad, y a cada uno de nosotros protegednos con perpetuo patrocinio, para que, a ejemplo vuestro y sostenidos 
    por vuestro auxilio, podamos santamente vivir, piadosamente morir, y alcanzar en el Cielo laeterna bienaventuranza. Amén.\\

    \primeraletragranderoja{A}{rcángel San Miguel}, defiéndenos en la batalla; sé nuestro amparo contra la perversidad y asechanzas del demonio. Reprímale Dios, pedimos
    suplicantes; y tú, Príncupe de la milicia celestial, lanza al infierno con el divino poder a Satanás y a otros espíritus malignos que andan dispersos por el
    el mundo para la perdición de las almas. \response{Amén}\\

    \versicleresponse{Corazón sacratísimo de Jesús}{Ten misericordia de nosotros (3)}\\

    \versicleresponse{Ave María purísima}{Sin pecado concebida}{Sine labe originali concépta}\\

    En el Nombre del Padre, y del{\redcross}Hijo, y del Espíritu Santo.
    \response{Amén}
\end{minipage}
\lineaverticalroja{1}
\begin{minipage}[t]{0.475\textwidth}
    \primeraletragranderoja{C}{redo in Deum}, Patrem omnipoténtem. Creatórem c{\oe}li et terr{\ae}. Et in Jesum Christum, Filium ejus únicum, Dóminum nostrum;
    \textsc{qui concéptus est de Spíritu Sancto; natus ex María Virgine}; passus sub Póntio Pilato, crucifíxus, mortuus et sepúltus:
    descéndit ad inferos; tértia die resurréxit a mórtuis: ascéndit ad c{\oe}los, sedet ad dexteram Dei Patris omnipoténtis;
    inde ventúrus est judicáre vivos et mórtuos. Credo in Spíritum Sanctum, sanctam Ecclésiam cathólicam, Sanctórum communiónem,
    remissiónem peccatórum, carnis resurrectiónem, vitam {\ae}térnam. \response{Amen}\\

    \primeraletragranderoja{A}{d} te beáte Joseph, in tribulatióne nostra confúgimus, atque, imploráto Spons{\ae} tu{\ae} sanctíssim{\ae} auxílio, patrocínium 
    quoque tuum fidenter expóscimus. Per eam, qu{\ae}sumus, qu{\ae} te cum immaculáta Vírgine Dei Genitríce coniúnxit, Caritátem, perque patérnum, 
    quoPúerum Iesum ampléxus es, amórem, súpplices deprecámur, ut ad hereditátem, quam Iesus Christus acquisívit Sánguine suo, benígnus respícias, 
    ac necessitátibus nostris tua virtúte et ope succúrras. Tuére, o Custos providentíssime divín{\ae} Famíli{\ae}, Iesu Christi sóbolem eléctam; próhibe a nobis, 
    amantíssime Pater, omnem errórum ac corruptelárum luem; propítius nobis, sospítator noster fortíssime, in hoc cum potestáte tenebrárum certámine e 
    c{\ae}lo adésto; et sicut olim Púerum Iesum e summo eripuísti vitre discrímine, ita nunc Ecclesiam sanctam Dei ab hostílibus insídiis atque ab omni 
    adversitáte défende: nosque síngulos perpétuo tege patrocínio, ut ad tui exémplar et ope tua suffúlti, sancte vívere, pie émori, sempiternámque in 
    c{\ae}lis beatitúdinem ássequi possímus. Amen.\\

    \hspace{4.5mm}

    \primeraletragranderoja{S}{ancte Michaël Archángele}, defénde nos in pr{\ae}lio: contra nequítian et insídias diáboli esto pr{\ae}sidium. Imperet illi Deus, 
    súpplices deprecámur: tuque, Prínceps militi{\ae} c{\oe}léstis, Sátanam aliósque spíritus malignos, qui ad perditiónem animarum pervagántur in mundo,
    divina virtúte in inférnum detrude. \response{Amen}\\

    \versicleresponse{Cor Iesus sacratíssimum}{Miserére nobis (3)}\\

    \versicleresponse{Ave María purísima}{Sine labe originali concépta}\\

    In Nómine Pátris, et{\redcross}Filii, et Spíritus Sancti.
    \response{Amen}
\end{minipage}
%%%%%%%%%%%%%%%%%%%%%%%%%%%%%%%%
% ORACIONES FINALES Y LETANIAS %
%%%%%%%%%%%%%%%%%%%%%%%%%%%%%%%%
%%%%%%%%%%%%%%%%%%%%%%%%%%%
% ANGELUS Y REGINA COELIS %
%%%%%%%%%%%%%%%%%%%%%%%%%%%
\chapter*{Ángelus y Regina C{\oe}li}

\section*{Ángelus}

\begin{minipage}[t]{0.475\textwidth}
    \versicleresponse{El Ángel del Señor anunció a María}{Y concibió por obra y gracia del Espíritu Santo}\\
    \textit{Dios te salve, María}\ldots\\
    \versicleresponse{He aquí la excalva del Señor}{Hágase en mi según tu palabra}\\
    \textit{Dios te salve, María}\ldots\\
    \versicleresponse{Y el Verbo se hizo carne}{Y habitó entre nosotros}\\
    \textit{Dios te salve, María}\ldots\\
    \ruegapornosotrossalve\\\\
    \textbf{Oremos}
    \primeraletragranderoja{O}{s} rogamos, Señor, que infundáis vuestra gracia en nuestras almas para que,
    habiendo conocido la Encarnación de vuestro Hijo Jesucristo por el Ángel que la anunció,
    seamos llevados a la gloria de la resurrección, por los méritos de su pasión y cruz santísima.
    Por el mismo Jesucristo nuestro Señor.\\
    \response{Amén}
\end{minipage}
\lineaverticalroja{1}
\begin{minipage}[t]{0.475\textwidth}
    \versicleresponse{Angelus Dómini nuntiávit Marí{\ae}}{Et concépit de Spíritu Sancto}\\
    \textit{Ave Maria}\ldots\\
    \versicleresponse{Ecce Amcilla Dómini}{Fiat mihi secúndum verbum tuum}\\
    \textit{Ave Maria}\ldots\\
    \versicleresponse{Et Verbum caro factum est}{Et habitávit in nobis}\\
    \textit{Ave Maria}\ldots\\
    \orapronobissalve\\\\
    \textbf{Orémus}
    \primeraletragranderoja{G}{rátiam} tuam qu{\'\ae}sumus. Dómine, méntibus nostris infúnde ut qui Angelo nuntiáte.
    Christi Filii tui incarnatiónem cognóvimus, per passiónem ejus et crucem ad resurrectiónis glóriam perducámur.
    Per eúmdem Christum Dóminum nostrum.\\
    \response{Amen}
\end{minipage}

\section*{Regina C{\oe}li L{\ae}táre (Tiempo Pascual)}

\begin{minipage}[t]{0.475\textwidth}
    \versicleresponse{Reina del cielo, alégrate, aleluya}{Porque el que mereciste llevar en tu seno, aleluya}\\
    \versicleresponse{Resucitó, como Él predijo, aleluya}{Rogad por nosotros a Dios, aleluya}\\
    \versicleresponse{Alegraos y regocijaos, Virgen María, aleluya}{Porque resucitó verdaderamente el Señor, aleluya}\\\\
    \textbf{Oremos}\\
    Oh Dios, que, por la resurreción de vuestro Hijo y Señor nuestro Jesucristo,\\
    os habéis dignado alegrar el mundo: concedednos por medio de su divina Madre, la Virgen Santísima,
    que merezcamos obtener los goces de la vida eterna. Por el mismo Cristo, Señor nuestro. \response{Amén} 
\end{minipage}
\lineaverticalroja{1}
\begin{minipage}[t]{0.475\textwidth}
    \versicleresponse{Regina caeli l{\ae}táre, allelúia}{Quia quem meruisti portáre, allelúia}\\
    \versicleresponse{Resurréxit sicut dixit, allelúia}{Ora pro nobis Deum, allelúia}\\
    \versicleresponse{Gaudate el l{\ae}táre, Virgo María, allelúia}{Quia surréxit Dóminus vere, allelúia}\\\\
    \textbf{Orémus}\\
    Deus, qui per resurrectiónem Filii tui Dómini nostri Jesu Christi,
    mundum l{\ae}tificáre dignátus es: pr{\ae}sta, qu{\'\ae}sumus, ut per ejus Genetricem Vírginem Maríam,
    perpétu{\ae} capiámus gáudia vit{\ae}. Per eúmdem Christum Dóminum nostrum. \response{Amen}
\end{minipage}
%%%%%%%%%%%%%%%%%%%%%%%%%%%
% ANGELUS Y REGINA COELIS %
%%%%%%%%%%%%%%%%%%%%%%%%%%%
%%%%%%%%%%%%%
% ORACIONES %
%%%%%%%%%%%%%
\chapter*{Oraciones}

\begin{minipage}[t]{0.475\textwidth}
    \input{oraciones/padrenuestro_castellano.tex}\\

    \input{oraciones/gloria_castellano.tex}\\

    \primeraletragranderoja{D}{ios te salve}, María, llena eres de gracia, el Señor es contigo; bendita eres entre todas las mujeres,
y bendito es el fruto de tu vientre, Jesús. Santa María, Madre de Dios, ruega por nosotros pecadores,
ahora u en el hora de nuestra muerte. Amén.\\

    \textit{Dios te salve, María}\ldots\\
    \ruegapornosotrossalve\\

    \input{oraciones/acordaos_castellano.tex}\\

    \primeraletragranderoja{B}{ajo} tu amparo nos acogemos, Santa Madre de Dios; no deseches las súplicas que te dirigimos en nuestras necesidades;
antes bien, líbranos siempre de todo peligro, {!`}Oh Virgen gloriosa y bendita!
\end{minipage}
\lineaverticalroja{1}
\begin{minipage}[t]{0.475\textwidth}
    \input{oraciones/padrenuestro_latin.tex}\\

    \input{oraciones/gloria_latin.tex}\\

    \input{oraciones/avemaria_latin.tex}\\

    \input{oraciones/salve_latin.tex}\\
    \orapronobissalve\\

    \primeraletragranderoja{M}{emorare}, O piissima Virgo Maria, non esse auditum a s{\ae}culo, quemquam ad tua currentem pr{\ae}sidia, tua implorantem auxilia, 
tua petentem suffragia, esse derelictum. Ego tali animatus confidentia, ad te, Virgo Virginum, Mater, curro, ad te venio, coram te gemens 
peccator assisto. Noli, Mater Verbi, verba mea despicere; sed audi propitia et exaudi. Amen.\\

    \input{oraciones/bajo_tu_amaparo_latin.tex}

\end{minipage}

\bigskip

\textorojo{Actos de contrición}
\primeraletragranderoja{S}{eñor mío Jesucristo,}Dios y Hombre verdadero, Creador y Redentor mío: por ser vos quién sois, y porque os amo sobre todas las cosas,
me pesa de todo corazón de haberos ofendido, propongo firmemente nunca más pecar, y apartarme de todas las ocasiones de ofenderos,
confesarme, y cumplir la penitencia que me fuere impuesta; ofrézcoos mi vida, obras y trabajos en satisfacción de todos mis pecados;
y confío en vuestra bondad y misericordia infinita me los perdonaréis por los merecimientos de vuestra preciosísima sangre, pasión y muerte,
y me daréis gracia para enmendarme y para perseverar en vuestro santo servicio hasta el fin de mi vida. Amén.

\medskip

\begin{minipage}[t]{0.475\textwidth}
    \primeraletragranderoja{Y}{o} pecador me confieso a Dios todopoderoso, a la bienaventurada siempre Virgen María, al bienaventurado San Miguel Arcángel,
al bienaventurado San Juan Bautista, a los Santos Apóstoles Pedro y Pablo, a todos los Santos, que pequé mucho
de pensamiento, palabra y obra: por mi culpa, por mi culpa, por mi grandísima culpa. Por tanto, ruego a la bienaventurada
siempre Virgen María, al bienaventurado San Miguel Arcángel, al bienaventurado San Juan Bautista, a los Santos Apóstoles
Pedro y Pablo, a todos los Santos, que roguéis por mi a Dios Nuestro Señor. Amén.\\
Dios todopoderoso tenga misericordia de nosotros, y, perdonados nuestros pecados, nos lleve a la vida eterna. Amén.\\
El Señor omnipotente y misericordioso nos conceda indulgencia, absolución y perdón de nuestros pecados. Amén.\\

    \input{oraciones/dios_mio_castellano.tex}\\

    \input{oraciones/credo_niceno_castellano.tex}\\

    \primeraletragranderoja{A}{rcángel San Miguel}, defiéndenos en la batalla; sé nuestro amparo contra la perversidad y asechanzas del demonio. Reprímale Dios, pedimos
suplicantes; y tú, Príncupe de la milicia celestial, lanza al infierno con el divino poder a Satanás y a otros espíritus malignos que andan dispersos por el
el mundo para la perdición de las almas. Amén.

\end{minipage}
\lineaverticalroja{1}
\begin{minipage}[t]{0.475\textwidth}
    \primeraletragranderoja{C}{onfíteor}Deo omnipoténti, beát{\ae} Marí{\ae} semper Virigini, beáto Michaéli Archángelo, beáto Joánni Baptíst{\ae}, sanctis 
Apóstolis Petro et Paulo, ómníbus Sanctis, quia peccávi nimis, cogitatióne, verbo et ópere, mea culpa, mea culpa, mea máxima culpa. Ideo precor beátam
Maríam semper Virgínem, beátum Michaélem Archángelum, beátum Joánnem Baptístam, sanctis Apóstolos Petrum et Paulum, omnes Sanctos,
oráre pro me ad Dóminum Deum nostrum. Amen.\\
Misereátur nostri omnipotens Deus, et, dimíssis peccátis nostris, perdúcat nos ad vitam {\ae}térnam. Amen.\\
Indulgéntiam, absolutionem, et remissiónem peccatórum nostrórum tríbuat nobis omnípotens et miséricors Dóminus. Amen.\\

    \input{oraciones/dios_mio_latin.tex}\\

    \input{oraciones/credo_niceno_latin.tex}\\
    
    \input{oraciones/san_miguel_latin.tex}
\end{minipage}

\end{document}