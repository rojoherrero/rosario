\documentclass[./rosary.tex]{subfiles}
\newcounter{sorrowful-counter}

\begin{document}
\section*{Misterios Dolorosos}
\begin{itemize}
      \item Tradicional: martes y viernes.
      \item Nuevo: martes y viernes.
\end{itemize}

\stepcounter{sorrowful-counter}
\subsection*{\Roman{sorrowful-counter} Misterio: La Agonía de Nuestro Señor en el Huerto de los Olivos}

Y lleva consigo a Pedro y a Santiago y a Juan, y comenzó a sentir espanto y abatimiento; y le dice:
«triste en gran manera está mi corazón hasta la muerte; quedad aquí y velad». Y apartándose un poco,
caía sobre tierra, y rogaba que, a ser posible, pasase el Él aquella hora, y decía: «Abba, Padre, todas las cosas te son posibles:
traspasa de mi este cáliz; más no se haga lo que yo quiero, sino lo que tú quieres» (\textbf{\emph{Marcos 14, 33-36}}).

\begin{center}
      Paternóster, diez Avemarías, Gloria y María, Madre de gracia.
\end{center}

\stepcounter{sorrowful-counter}
\subsection*{\Roman{sorrowful-counter} Misterio: La Flagelación de Nuestro Señor Jesucristo}

«Yo no hallo en Él delito alguno. Es costumbre vuestra que yo os suelte un preso por la Pascua:
¿queréis, pues, que os suelte al rey de los Judíos?». Gritaron, pues, de nuevo, diciendo: «No, a ése, sino a Barrabás».
Era este Barrabás un salteador. Entonces, pues, tomó Pilato a Jesús y le azotó (\textbf{\emph{Juan 18,38-40;19,1}}).

\begin{center}
      Paternóster, diez Avemarías, Gloria y María, Madre de gracia.
\end{center}

\stepcounter{sorrowful-counter}
\subsection*{\Roman{sorrowful-counter} Misterio: La Coronación de espinas de Nuestro Señor Jesucristo}

Entonces los soldados del gobernador, tomando a Jesús y conduciéndole al pretorio,
reunieron en torno a Él toda la cohorte. Y habiéndole quitado sus vestidos, le envolvieron en una clámide de grana,
y trenzando una corona de espinas, la pusieron sobre su cabeza, y una caña en su mano derecha;
y doblando la rodilla delante de Él, le mofaban, diciendo: «Salud, Rey de los judíos».
Y escupiendo en Él, tomaron la caña y le daban golpes en la cabeza (\textbf{\emph{Mateo 27, 27-30}}).

\begin{center}
      Paternóster, diez Avemarías, Gloria y María, Madre de gracia.
\end{center}

\stepcounter{sorrowful-counter}
\subsection*{\Roman{sorrowful-counter} Misterio: El Señor con la cruz a cuestas}

Entonces, pues, se le entregó para que fuera crucificando. Se apoderaron, pues, de Jesús,
y llevando a cuestas su cruz, salió hacia el lugar llamado el Cráneo, que en hebreo se dice Gólgota (\textbf{\emph{Juan 19, 16-17}}).

\begin{center}
      Paternóster, diez Avemarías, Gloria y María, Madre de gracia.
\end{center}

\stepcounter{sorrowful-counter}
\subsection*{\Roman{sorrowful-counter} Misterio: Crucifixión y Muerte del Redentor}

Y era ya como la hora sexta, y se produjeron tinieblas sobre toda la tierra hasta la hora nona,
habiendo faltado el sol; y se rasgó por medio el velo del santuario Y clamando con voz poderosa,
Jesús dijo: «Padre, en tus manos encomiendo mi espíritu». Y dicho esto, expiró (\textbf{\emph{Lucas 23, 44-46}}).

\begin{center}
      Paternóster, diez Avemarías, Gloria y María, Madre de gracia.
      
      Oraciones finales (\cpageref{sec:final-prayer}).
\end{center}

\end{document}