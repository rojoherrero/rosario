\documentclass[./rosary.tex]{subfiles}

\begin{document}

\section*{Oraciones finales y letanías lauretanas}
\label{sec:final-prayer}

\begin{longtable} { p{0.5\textwidth} p{0.5\textwidth} }
    Padre Nuestro{\ldots}

    Dios te Salve, María, Hija de Dios Padre, Virgen purísima y castísima antes del parto, [llena eres de gracia{\ldots}]

    Dios te Salve, María, Madre de Dios Hijo, Virgen purísima y castísima en el parto, [llena eres de gracia{\ldots}]

    Dios te Salve, María, Esposa de Dios Espíritu Santo, Virgen purísima y castísima después del parto, [llena eres de gracia{\ldots}]

    Dios te salve, María, templo y sagrario de la Santísima Trinidad. [Gloria al Padre{\ldots}]

     &

    Pater Noster{\ldots}

    Ave Maria, Filia Dei Patri, Virgo purissima et castissima ante partum, [gratia plena{\ldots}]

    Ave Maria, Mater Dei Filii, Virgo purissima et castissima in partu, [gratia plena{\ldots}]

    Ave Maria, Sponsa Spíritus Sanctii, Virgo purissima et castissima post partum, [gratia plena{\ldots}]

    Ave Maria, templum et sacrarium totis Sanctissimæ Trinitatis. [Gloria Patris{\ldots}]
\end{longtable}

\noindent\textbf{Por el Santo Padre y sus intenciones}:\\
{\indent}Paternóster, Avemaría y Gloria.\\
\noindent\textbf{Por la Hispanidad}:\\ 
{\indent}Paternóster, Avemaría, Gloria.\\
\noindent\textbf{Por el nuestro Obispo, sacerdotes, seminaristas, monjas...}:\\ 
{\indent}Paternóster, Avemaría\\
{\indent}Enviad, Señor, sacerdotes santos a vuestra Iglesia.\\
{\indent}¡Oh Jesús, Salvador del mundo! santificad a vuestros sacerdotes y seminaristas.\\
{\indent}Oh María, Reina del sacerdocio, alcanzadnos santos y numerosos sacerdotes.\\
{\indent}Gloria.\\
\noindent\textbf{Por las benditas almas de Purgatorio}:\\
{\indent}Paternóster, Avemaría, Descansen en paz. Amén. (requiéscant in pace. Amen)

\subsection*{Letanías lauretanas}

\begin{multicols}{2}
    Kýrie, eléison.\\
    Christe, eléison.\\
    Kýrie, eléison.\\
    Christe, audi nos.\\
    Christe, exáudi nos.\\
    Pater de cælis, Deus, \emph{miserére nobis}.\\
    Fili, Redémptor mundi, Deus.\\
    Spíritus Sancte, Deus.\\
    Sancta Trínitas, unus Deus.\\
    Sancta Maria, \emph{ora pro nobis}.\\
    Sancta Dei Génetrix.\\
    Sancta Virgo vírginum.\\
    Mater Christi.\\
    Mater divínæ gratiæ.\\
    Mater puríssima.\\
    Mater castíssima.\\
    Mater invioláta.\\
    Mater intemeráta.\\
    Mater immaculáta.\\
    Mater amábilis.\\
    Mater admirábilis.\\
    Mater boni Consílii.\\
    Mater Creatóris.\\
    Mater Salvatóris.\\
    Virgo pru­den­tíssima.
    Virgo veneránda.\\
    Virgo prædicánda.\\
    Virgo potens.\\
    Virgo clemens.\\
    Virgo fidélis.\\
    Spéculum iustítiæ.\\
    Sedes Sapiéntiæ.\\
    Causa nostræ lætítiæ.\\
    Vas spirituále.\\
    Vas honorábile.\\
    Vas insigne devotiónis.\\
    Rosa mýstica.\\
    Turris Davídica.\\
    Turris ebúrnea.\\
    Domus áurea.\\
    Fœderis arca.\\
    Iánua cæli.\\
    Stella matutina.\\
    Salus infirmórum.\\
    Refugium peccatórum.\\
    Consolátrix af­flic­tórum.\\
    Auxílium chris­tia­nórum.\\
    Regina Angelórum.\\
    Regina Pa­triar­chárum.\\
    Regina Pro­phe­tárum.\\
    Regina Apos­to­lórum.\\
    Regina Mártyrum.\\
    Regina Con­fe­ssórum.\\
    Regina Vírginum.\\
    Regina Sanctórum ómnium.\\
    Regina sine labe originali concépta.\\
    Regina in cælum assumpta.\\
    Regina sa­cra­tíssimi Rosárii.\\
    Regina pacis.\\
    Agnus Dei, qui tollis peccáta mundi, \emph{parce nobis, Dómine}.\\
    Agnus Dei, qui tollis peccáta mundi, \emph{exáudi nos, Dómine}.\\
    Agnus Dei, qui tollis peccáta mundi, \emph{miserére nobis}.\\
    Sub tuum præsídium confúgimus, Sancta Dei Génetrix, nostras de­pre­ca­tiónes ne despícias in ne­ces­si­tátibus;\\
    sed a perículis cunctis líbera nos semper, Virgo gloriósa et benedícta.\\
    Ora pro nobis, Sancta Dei Génetrix. -- Ut digni efficiámur pro­mi­ssiónibus Christi.
\end{multicols}

Gracias os damos, soberana Princesa, por los favores que todos los días recibimos de vuestra benéfica mano; dignaos, Señora, tenernos ahora
y siempre bajo vuestra protección y amparo y para más obligados, os saludamos con una Salve:

\label{hailMaryQueen}
\begin{longtable} { p{0.5\textwidth} p{0.5\textwidth} }
    \textbf{DIOS TE SALVE}, Reina y Madre de mi­se­ri­cordia, vida, dulzura y esperanza nuestra; Dios te salve.
    A ti llamamos los desterrados hijos de Eva; a ti suspiramos, gimiendo y llorando en este valle de lágrimas.
    Ea, pues, Señora, abogada nuestra, vuelve a nosotros esos tus ojos mi­se­ri­cordiosos. Y después de este destierro, muéstranos a Jesús,
    fruto bendito de tu vientre. ¡Oh cle­men­tísima, oh piadosa, oh dulce siempre Virgen María!

     &

    \textbf{SALVE}, Regina, Mater mi­se­ri­córdiæ, vita, dulcédo et spes nostra, salve. Ad te clamámus, éxsules fílii Hevæ.
    Ad te suspirámus geméntes et flentes in hac lacrimárum valle. Éia ergo, advocáta nostra, illos tuos mi­se­ri­córdes óculos ad nos convérte.
    Et Iesum benedíctum fructum ventris tui, nobis, post hoc exsílium, osténde. O clemens, o pia, o dulcis Virgo Maria!            \\

    Ruega por nosotros, Santa Madre de Dios. -- Para que seamos dignos de alcanzar las promesas de nuestro Señor Jesucristo. Amén.                                           
    
     & 
    
    Ora pro nobis, Sancta Dei Génetrix. -- Ut digni efficiámur pro­mi­ssiónibus Christi. Amen.
\end{longtable}

\textbf{ACORDAOS}, ¡oh piadosísima Virgen María!, que jamás se ha oído decir que ninguno de los que han acudido a vuestra protección, 
implorando vuestro auxilio, haya sido desamparado. Animado por esta confianza, a Vos acudo, Madre, Virgen de las vírgenes, y gimiendo 
bajo el peso de mis pecados me atrevo a comparecer ante Vos. Madre de Dios, no desechéis mis súplicas, antes bien, escuchadlas y 
acogedlas benignamente. Amén. (en latín: \cpageref{sec:memorare}).

\label{creed}
\begin{longtable} { p{0.5\textwidth} p{0.5\textwidth} }
    \textbf{CREO} en Dios, Padre todopoderoso. Creador del cielo y de la tierra. Y en Jesucristo, su único Hijo, Nuestro Señor,
    que fue concebido por obra y gracia del Espíritu Santo; nació de Santa María Vírgen; padeció bajo el poder de Poncio Pilato,
    fue crucificado, muerto y sepultado; descendió a los infiernos; al tercer día resucitó de entre los muertos; subió a los cielos,
    está sentado a la derecha de Dios Padre todopoderoso; desde allí ha de venir a juzgar a vivos y muertos.
    Creo en el Espíritu Santo, la Santa Iglesia Católica, la comunión de los Santos, el perdón de los pecados,
    la resurección de la carne y la vida eterna. Amén.
    
     &

    \textbf{CREDO} in Deum, Patrem omnipoténtem. Creatórem cœli et terræ. Et in Jesum Christum, Filium ejus únicum, Dóminum nostrum;
    qui concéptus est de Spíritu Sancto; natus ex María Virgine; passus sub Póntio Pilato, crucifíxus, mortuus et sepúltus:
    descéndit ad inferos; tértia die resurréxit a mórtuis: ascéndit ad cœlos, sedet ad dexteram Dei Patris omnipoténtis;
    inde ventúrus est judicáre vivos et mórtuos. Credo in Spíritum Sanctum, sanctam Ecclésiam cathólicam, Sanctórum communiónem,
    remissiónem peccatórum, carnis resurrectiónem, vitam ætérnam. Amen.
\end{longtable}
\smallskip
Credo Niceo en \cpageref{sec:creed-nicene}.
\smallskip

\begin{longtable} { p{0.5\textwidth} p{0.5\textwidth} }
    \textbf{ARCÁNGEL SAN MIGUEL}, defiéndenos en la batalla; sé nuestro amparo contra la perversidad y asechanzas del demonio. \textbf{REPRÍMALE DIOS}, pedimos
    suplicantes; y tú, Príncupe de la milicia celestial, lanza al infierno con el divino poder a Satanás y a otros espíritus malignos que andan dispersos por el
    el mundo para la perdición de las almas. Amén.
    
     &

    \textbf{SANCTE MICHAËL ARCHÁNGELE}, defénde nos in praelio: contra nequítian et insídias diáboli esto praesidium. \textbf{IMPERET ILLI DEUS}, 
    súpplices deprecámur: tuque, Prínceps militiae coeléstis, Sátanam aliósque spíritus malignos, qui ad perditiónem animarum pervagántur in mundo,
    divina virtúte in inférnum detrude. Amen.
\end{longtable}

\begin{longtable} { p{0.5\textwidth} p{0.5\textwidth} }
    Corazón sacratísimo de Jesús. -- Ten misericordia de nosotros (3)
    
     &

    Cor Iesus sacratíssimum. -- Miserére nobis (3)
\end{longtable}

\begin{center}
    Ave María purísima. Sin pecado concebida.
\end{center}

\begin{longtable} { p{0.5\textwidth} p{0.5\textwidth} }
    En el Nombre del Padre, y del Hijo, y del Espíritu Santo. Amén.

     &

    In Nómine Pátris, et Filii, et Spíritus Sancti. Amen.
\end{longtable}

\end{document}