\documentclass[10pt,twoside]{book}
\usepackage[utf8]{inputenc}
\usepackage[latin,spanish]{babel}
\usepackage[T1]{fontenc}
\usepackage{fontspec}
\usepackage{blindtext}
\usepackage{hyperref}
\hypersetup{
    hidelinks,
    pdftitle={Santo Rosario y \'Angelus},
    pdfauthor={Sergio},
    pdfkeywords={Cat\'olico, Santa Mar\'ia, Santo Rosario},
    pdfcreator={VS Code + LuaTeX}
}
\usepackage[spanish]{cleveref}
\usepackage{geometry}
\usepackage{setspace}
\geometry{
    a4paper,
    inner=15mm,
    outer=10mm,
    top=15mm,
    bottom=15mm
}
\usepackage{paracol}
\usepackage{lettrine}
\usepackage{xcolor}
\usepackage{gregoriotex}
\usepackage{graphicx}
\graphicspath{ {images/} }
\usepackage{fancyhdr}
\pagestyle{fancy}
\fancyhf{}
\fancyfoot[LE,RO]{\thepage}
\renewcommand{\headrulewidth}{0pt}
\renewcommand{\footrulewidth}{0pt}

\setlength{\columnseprule}{0.4pt}
\colseprulecolor{red}


\usepackage{datetime}
\newdateformat{monthyeardate}{%
  \monthname[\THEMONTH] \THEYEAR
}
\newcommand{\primeraletragranderoja}[2]{
    \lettrine[lines=2]{\textcolor{red}{#1}}{#2}
}

\newcommand{\primeraletragranderojasola}[1]{
    \lettrine[lines=2]{\textcolor{red}{#1}}{}
}

\newcommand{\primeraletragrande}[2]{
    \lettrine[lines=2]{#1}#2
}

\newcommand{\letraroja}[2]{
    \textcolor{red}{#1}#2
}

\newcommand{\versiculo}[1]{
    \textcolor{red}{\Vbar.} #1.
}

\newcommand{\respuesta}[1]{
    \textcolor{red}{\Rbar.} #1.
}

\newcommand{\textopequenorojo}[1]{
    {\textcolor{red}{\small{#1}}}
}

\newcommand{\versiculorespuesta}[2]{
    \versiculo{#1}\\
    \respuesta{#2}
}

\newcommand{\versiculorespuestaseguido}[2]{
    \versiculo{#1}\respuesta{#2}
}

\newcommand{\redcross}{
    \textcolor{red}{\grecross}
}

\newcommand{\lineahorizontal}[2]{
    \begin{center}
        {\rule{#1cm}{#2pt}}
    \end{center}
}

\newcommand{\lineahorizontalroja}[2]{
    \begin{center}
        \textcolor{red}{\rule{#1cm}{#2pt}}
    \end{center}
}

\newcommand{\latinderecha}[1]{
    \begin{otherlanguage}{latin}
        \begin{rightcolumn}
            \input{#1}    
        \end{rightcolumn}
    \end{otherlanguage}
}

\newcommand{\castellanoizquierda}[1]{
    \begin{leftcolumn}
        \input{#1}
    \end{leftcolumn}
}

\newcommand{\castellanoizquierdasincro}[1]{
    \begin{leftcolumn*}
        \input{#1}
    \end{leftcolumn*}
}

\newcommand{\castellanoizquierdasincronota}[2]{
    \begin{leftcolumn*}[#1]
        \input{#2}
    \end{leftcolumn*}
}

\newcommand{\filacastellanolatin}[2]{
    {    
        \begin{leftcolumn}
            \input{#1}
        \end{leftcolumn}
        \begin{otherlanguage}{latin}
            \begin{rightcolumn}
                \input{#2}    
            \end{rightcolumn}
        \end{otherlanguage}
    }
}

\newcommand{\filacastellanolatinsincro}[2]{
    {    
        \begin{leftcolumn*}
            \input{#1}
        \end{leftcolumn*}
        \begin{otherlanguage}{latin}
            \begin{rightcolumn}
                \input{#2}    
            \end{rightcolumn}
        \end{otherlanguage}
    }
}

\newcommand{\filacastellanolatinsincronota}[3]{
    \castellanoizquierdasincronota{#1}{#2}
    \latinderecha{#3}
}

\newcommand{\titulomisterios}[2]{
    {    
        \begin{minipage}[t]{0.6\textwidth}
            \subsection*{#1}
        \end{minipage}\begin{minipage}[t]{0.395\textwidth}
            \begin{flushright}
                \textcolor{red}{#2}
            \end{flushright}
        \end{minipage}
    }
}

\newcommand{\titulomisterio}[2]{
    {
        \begin{minipage}[t]{0.6\textwidth}
            \section*{#1}
        \end{minipage}\begin{minipage}[t]{0.395\textwidth}
            \begin{flushright}
                \textcolor{red}{#2}
            \end{flushright}
        \end{minipage}
    }
}

\newcommand{\iralfinal}{
    \begin{center}
        \textcolor{red}{Una vez terminamos nos vamos a la \cpageref{final-prayer} para las oraciones finales.}
    \end{center}
}





\begin{document}

\begin{titlepage}
      \begin{center}

            \vspace*{10em}

            {\Huge \uppercase{Santo Rosario y Ángelus}}

            \vspace{1.5em}

            {\includegraphics[scale=0.95]{foto-04.jpg}}

            \vspace{0.5em}

            Sergio

            Valladolid (España), \monthyeardate\today
      \end{center}
\end{titlepage}
\twosided{c}
\footnotelayout{m}

\chapter*{\centering\textcolor{red}{S}anto \textcolor{red}{R}osario \textcolor{red}{M}editado}

\begin{paracol}{2}

      \ensurevspace{20mm}

      \begin{leftcolumn}
            En el Nombre del Padre, y del{\redcross}Hijo, y del Espíritu Santo. Amén.
      \end{leftcolumn}
      \begin{otherlanguage}{latin}
            \begin{rightcolumn}
                  \primeraletragranderoja{S}{ancte Michaël Archángele,}defénde nos in pr{\ae}lio: contra nequítian et insídias diáboli esto pr{\ae}sidium. Imperet illi Deus, 
súpplices deprecámur: tuque, Prínceps militi{\ae} c{\oe}léstis, Sátanam aliósque spíritus malignos, qui ad perditiónem animarum pervagántur in mundo,
divina virtúte in inférnum detrude. Amen.    
            \end{rightcolumn}
      \end{otherlanguage}

      \definecolumnpreamble{0}{\vspace{0.5em}}
      \definecolumnpreamble{1}{\vspace{0.5em}}

      \begin{leftcolumn*}
            \input{oraciones/contricion/senor_mio_jesucristo.tex}\footnote{\label{contricion}En vez del \textit{Señor mio, Jesucristo} o del \textit{Confíteor}, podemos rezar:\\ \indent\input{oraciones/contricion/dios_mio/castellano_sencilloa.tex}\\ \indent En latín:\\ \indent\versiculorespuestaseguido{Regina caeli l{\ae}táre, allelúja}{Quia quem meruisti portáre, allelúja}

\vspace{1mm}

\versiculorespuestaseguido{Resurréxit sicut dixit, allelúja}{Ora pro nobis Deum, allelúja}

\vspace{1mm}

\versiculorespuestaseguido{Gaude et l{\ae}táre, Virgo María, allelúja}{Quia surréxit Dóminus vere, allelúja}

\vspace{1mm}

\textbf{Orémus}.-- \textcolor{red}{D}eus, qui per resurrectiónem Filii tui Dómini nostri Jesu Christi,
mundum l{\ae}tificáre dignátus es: pr{\ae}sta, qu{\'\ae}sumus, ut per ejus Genetricem Vírginem Maríam,
perpétu{\ae} capiámus gáudia vit{\ae}. Per eúmdem Christum Dóminum nostrum. \respuesta{Amen} 
}
      \end{leftcolumn*}
      \begin{otherlanguage}{latin}
            \begin{rightcolumn}
                \primeraletragranderoja{S}{ancte Michaël Archángele,}defénde nos in pr{\ae}lio: contra nequítian et insídias diáboli esto pr{\ae}sidium. Imperet illi Deus, 
súpplices deprecámur: tuque, Prínceps militi{\ae} c{\oe}léstis, Sátanam aliósque spíritus malignos, qui ad perditiónem animarum pervagántur in mundo,
divina virtúte in inférnum detrude. Amen.(\cref{contricion})
            \end{rightcolumn}
      \end{otherlanguage}

      \begin{leftcolumn*}
            \versiculorespuestaseguido{Abre, Señor, mis labios}{Y mi boca cantará tus alabanzas}
      \end{leftcolumn*}
      \begin{otherlanguage}{latin}
            \begin{rightcolumn}
                \versiculorespuestaseguido{Deum, in adjutórium meum inténde}{Dómine, ad adjuvándum me festina}
            \end{rightcolumn}
      \end{otherlanguage}

      \begin{leftcolumn*}
            \versiculorespuestaseguido{Apresútare, Señor, a socorrerme}{Ven, oh Dios, en mi ayuda}
      \end{leftcolumn*}
      \begin{otherlanguage}{latin}
            \begin{rightcolumn}
                \versiculorespuestaseguido{Deum, in adjutórium meum inténde}{Dómine, ad adjuvándum me festina}
            \end{rightcolumn}
      \end{otherlanguage}

      \begin{leftcolumn*}
            \versiculorespuestaseguido{María, Madre de gracia, Madre de Misericordia}{Defendednos del enemigo y amparadnos ahora y en la hora de nuestra muerte. Amén}
      \end{leftcolumn*}
      \begin{otherlanguage}{latin}
            \begin{rightcolumn}
                \versiculorespuestaseguido{Deum, in adjutórium meum inténde}{Dómine, ad adjuvándum me festina}
            \end{rightcolumn}
      \end{otherlanguage}

      \begin{leftcolumn*}
            \textbf{Oremos por los fieles difuntos;}\\
Padre Nuestro, que estás ne los cielo\ldots\\
\textit{Dios te salve, María}\ldots\\[1mm]
\versiculorespuesta{El Señor les conceda el descanso eterno}{Y brille para ellos laa luz perpétua}\\[1mm]
\versiculorespuesta{Descansen en paz}{Amén}
      \end{leftcolumn*}
      \begin{otherlanguage}{latin}
            \begin{rightcolumn}
                \primeraletragranderoja{S}{ancte Michaël Archángele,}defénde nos in pr{\ae}lio: contra nequítian et insídias diáboli esto pr{\ae}sidium. Imperet illi Deus, 
súpplices deprecámur: tuque, Prínceps militi{\ae} c{\oe}léstis, Sátanam aliósque spíritus malignos, qui ad perditiónem animarum pervagántur in mundo,
divina virtúte in inférnum detrude. Amen.
            \end{rightcolumn}
      \end{otherlanguage}
      \definecolumnpreamble{0}{\vspace{0em}}
      \definecolumnpreamble{1}{\vspace{0em}}
\end{paracol}

\vspace{0.5em}

\begin{center}
      \textcolor{red}{Decimos aquí las intenciones de este Rosario}
\end{center}

\section*{\centering Dominica (Domingo)}

\noindent\subsection*{Misterio Gloriosos}

\noindent\subsubsection*{I La Resurrección del Señor}

\vspace{-0.5em}

\hfill\textcolor{red}{Mt 28, 1-3.5-7}

\lettrine[lines=2]{\textcolor{red}{P}}asado el sábado, ya para amanecer el día primero de la semana, vino María Magdalena con la otra María al sepulcro. Y sobrevino un gran terremoto,
pues un ángel del Señor bajó del cielo y acercándose removió la piedra del sepulcro y se sentó sobre ella. Era su aspecto como el relámpago, y su vestidura blanca como la nueve.
El ángel, dirigiéndose a las mujeres, dijo: No temáis vosotras, pues sé que buscáis a Jesús el crucificado. No está aquí; ha resucitado, según lo había dicho.
Venid y ved el sitio donde fue puesto. Id luego y decid a sus discípulos que ha resucitado de entre los muertos y que os precede a Galilea; allí le veréis.
Es lo que tenía que deciros.

\vspace{0.5em}

\begin{paracol}{2}
    \begin{leftcolumn}
        \input{oraciones/padrenuestro/castellano_vr.tex}
    \end{leftcolumn}
    \begin{otherlanguage}{latin}
            \begin{rightcolumn}
                \versiculorespuestaseguido{Ave María, grátia plena, Dóminus tecum; benedicta tu in muliéribus, et benedíctus fructus ventris tui,
Jesus}{Sancta Maria, Mater Dei, ora pro nobis peccatóribus, nunc et in hora mortis nostr{\ae}. Amen.} 
            \end{rightcolumn}
    \end{otherlanguage}
\end{paracol}

\vspace{0.5em}

\begin{center}
    \textcolor{red}{El avemaría se repite 10 veces}
\end{center}
\vspace{0.2em}
\begin{paracol}{2}
    \begin{leftcolumn}
        \input{oraciones/avemaria/castellano_vr.tex}
    \end{leftcolumn}
    \begin{otherlanguage}{latin}
            \begin{rightcolumn}
                \versiculorespuestaseguido{Ave María, grátia plena, Dóminus tecum; benedicta tu in muliéribus, et benedíctus fructus ventris tui,
Jesus}{Sancta Maria, Mater Dei, ora pro nobis peccatóribus, nunc et in hora mortis nostr{\ae}. Amen.}
            \end{rightcolumn}
    \end{otherlanguage}

    \definecolumnpreamble{0}{\vspace{0.5em}}
    \definecolumnpreamble{1}{\vspace{0.5em}}

    \begin{leftcolumn*}
        \versiculorespuestaseguido{María, Madre de gracia, Madre de Misericordia}{Defendednos del enemigo y amparadnos ahora y en la hora de nuestra muerte. Amén}
    \end{leftcolumn*}
    \begin{otherlanguage}{latin}
            \begin{rightcolumn}
                \versiculorespuestaseguido{Deum, in adjutórium meum inténde}{Dómine, ad adjuvándum me festina}
            \end{rightcolumn}
    \end{otherlanguage}

    \begin{leftcolumn*}
        \textbf{Oremos por los fieles difuntos;}\\
Padre Nuestro, que estás ne los cielo\ldots\\
\textit{Dios te salve, María}\ldots\\[1mm]
\versiculorespuesta{El Señor les conceda el descanso eterno}{Y brille para ellos laa luz perpétua}\\[1mm]
\versiculorespuesta{Descansen en paz}{Amén}
    \end{leftcolumn*}
    \begin{otherlanguage}{latin}
            \begin{rightcolumn}
                \primeraletragranderoja{S}{ancte Michaël Archángele,}defénde nos in pr{\ae}lio: contra nequítian et insídias diáboli esto pr{\ae}sidium. Imperet illi Deus, 
súpplices deprecámur: tuque, Prínceps militi{\ae} c{\oe}léstis, Sátanam aliósque spíritus malignos, qui ad perditiónem animarum pervagántur in mundo,
divina virtúte in inférnum detrude. Amen.
            \end{rightcolumn}
    \end{otherlanguage}

    \definecolumnpreamble{0}{\vspace{0em}}
    \definecolumnpreamble{1}{\vspace{0em}}
\end{paracol}

\vspace{1.5em}

\noindent\subsubsection*{II Las Ascensión Jesucristo a los cielos}

\vspace{-0.5em}

\hfill\textcolor{red}{Lc 24, 50; Mc 16, 15-16.19-20}

\lettrine[lines=2]{\textcolor{red}{L}}os llevó hasta cerca de Betania, y levantando sus manos les bendijo. Y les dijo: Id al mundo entero y predicad el Evangelio a toda la creación.
El que creyere y fuere bautizado, se salvará, mas el que no creyere, será condenado. Con esto el Señor Jesús, después de hablarles, fue elevado al cielo y se sentó a la diestra de Dios.
Y ellos, partiéndose de allí, predicaron por todas partes, cooperando el Señor y confirmando la palabra con las señales que la acompañaban.

\vspace{0.5em}

Paternóster, diez Avemarías, Gloria

\vspace{2mm}

\begin{paracol}{2}
    \filacastellanolatin{oraciones/maria_madre/castellano_seguido.tex}{oraciones/maria_madre/latin_seguido.tex}
\end{paracol}

\vspace{1.5em}

\noindent\subsubsection*{III La Venida del Espíritu Santo sobre los Apóstoles}

\vspace{-0.5em}

\hfill\textcolor{red}{Jn 14, 26; Hch. 2, 1-4}

\lettrine[lines=2]{\textcolor{red}{M}}as el Paráclito, el Espíritu Santo, que enviará el Padre en mi nombre, Él os enseñará todas las cosas y os recordará todas las cosas que os dije yo.
Y al cumplirse el día de Pentecostés, estaban todos juntos en el mismo lugar. Y se produjo de súbito desde el cielo un estruendo como de viento que soplaba vehementemente,
y llenó toda la casa donde se hallaban sentados. Y vieron aparecer lenguas como de fuego, que, repartiéndose, se posaban sobre cada uno de ellos.

\vspace{0.5em}

Paternóster, diez Avemarías, Gloria

\vspace{2mm}

\begin{paracol}{2}
    \filacastellanolatin{oraciones/maria_madre/castellano_seguido.tex}{oraciones/maria_madre/latin_seguido.tex}
\end{paracol}

\vspace{1.5em}

\noindent\subsubsection*{IV La Asunción de María Santísima a los cielos}

\vspace{-0.5em}

\hfill\textcolor{red}{Lc 1, 46-49; Ap 11, 19}

\lettrine[lines=2]{\textcolor{red}{M}}i alma magnifica al Señor y exulta de júbilo mi espíritu al Señor, mi Salvador, porque ha mirado la humildad de su sierva;
por eso todas las generaciones me llamarán bienaventurada, porque ha hecho en mi maravillas el Poderoso, cuyo nombre es santo. Y se abrió el templo de Dios, que está en el cielo,
y fué vista el arca de la alianza en el templo, y se produjeron relámpagos, y voces, y truenos, y temblor de tierra, y fuerte granizada.

\vspace{0.5em}

Paternóster, diez Avemarías, Gloria

\vspace{2mm}

\begin{paracol}{2}
    \filacastellanolatin{oraciones/maria_madre/castellano_seguido.tex}{oraciones/maria_madre/latin_seguido.tex}
\end{paracol}

\vspace{1.5em}

\noindent\subsubsection*{V La Coronación de la Santísima Virgen María}

\vspace{-0.5em}

\hfill\textcolor{red}{Cant 6,10; Ap 12, 1; 18, 16; Sal 44, 10}

\lettrine[lines=2, ante={?`}]{\textcolor{red}{Q}}uién es esa que aparece resplandeciente como la aurora, hermosa cual luna, deslumbradora como el sol, imponente como batallones?.
Y una gran señal fué vista en el cielo: una Mujer vestida del sol, y la luna debajo de sus  pies, y sobre su cabeza una corona de doce estrellas.
Tiene Él escrito en su vestido y en su manto Rey de reyes y Señor de los que dominan. Está la Reina a su derecha, adornada con oro finísimo.

\vspace{0.5em}

Paternóster, diez Avemarías, Gloria

\vspace{2mm}

\begin{paracol}{2}
    \filacastellanolatin{oraciones/maria_madre/castellano_seguido.tex}{oraciones/maria_madre/latin_seguido.tex}
\end{paracol}

\iralfinal

\vspace{0.5em}

\begin{center}
      \begin{spacing}{0.25}
            {\rule{20em}{0.4pt}}\\
            {\rule{20em}{0.4pt}}
      \end{spacing}
\end{center}

%%%%%%%%%%%
% DOMINGO %
%---------%
%  LUNES  %
%%%%%%%%%%%

\section*{\centering Feria Secunda (Lunes)}

\noindent\subsection*{Misterios Gozosos}

\noindent\subsubsection*{I La Anunciación de la Santísima Virgen María}

\vspace{-0.5em}

\hfill\textcolor{red}{Lc 1, 26-33}

\lettrine[lines=2]{\textcolor{red}{F}}ue enviado el ángel Gabriel de parte de Dios a una ciudad de Galilea, llamada Nazaret, 
a una doncella desposada con un varón llamado José, de la familia de David, y el nombre de la doncella era María. 
Y habiendo entrado a ella, dijo: Dios te salve, llena de gracia, el Señor es contigo, bendita tu entre las mujeres.
Ella, al oír estas palabras, se turbó, y discurría que podría ser esta salutación. Y le dijo el ángel: No temas, María, 
pues hallaste gracia a los ojos de Dios. He aquí que concebirás en tu seno y darás a luz un Hijo, a quien darás por nombre Jesús. 
Este será grande, y será llamado Hijo del Altísimo, y le dará el Señor Dios el trono de David su padre, y reinará sobre la 
casa de Jacob etérnamente, y su reinado no tendrá fin.

\vspace{0.5em}

\begin{paracol}{2}
    \begin{leftcolumn}
        \input{oraciones/padrenuestro/castellano_vr.tex}
    \end{leftcolumn}
    \begin{otherlanguage}{latin}
            \begin{rightcolumn}
                \versiculorespuestaseguido{Ave María, grátia plena, Dóminus tecum; benedicta tu in muliéribus, et benedíctus fructus ventris tui,
Jesus}{Sancta Maria, Mater Dei, ora pro nobis peccatóribus, nunc et in hora mortis nostr{\ae}. Amen.} 
            \end{rightcolumn}
    \end{otherlanguage}
\end{paracol}

\vspace{0.5em}

\begin{center}
    \textcolor{red}{El avemaría se repite 10 veces}
\end{center}
\vspace{0.2em}
\begin{paracol}{2}
    \begin{leftcolumn}
        \input{oraciones/avemaria/castellano_vr.tex}
    \end{leftcolumn}
    \begin{otherlanguage}{latin}
            \begin{rightcolumn}
                \versiculorespuestaseguido{Ave María, grátia plena, Dóminus tecum; benedicta tu in muliéribus, et benedíctus fructus ventris tui,
Jesus}{Sancta Maria, Mater Dei, ora pro nobis peccatóribus, nunc et in hora mortis nostr{\ae}. Amen.}
            \end{rightcolumn}
    \end{otherlanguage}

    \definecolumnpreamble{0}{\vspace{0.5em}}
    \definecolumnpreamble{1}{\vspace{0.5em}}

    \begin{leftcolumn*}
        \versiculorespuestaseguido{María, Madre de gracia, Madre de Misericordia}{Defendednos del enemigo y amparadnos ahora y en la hora de nuestra muerte. Amén}
    \end{leftcolumn*}
    \begin{otherlanguage}{latin}
            \begin{rightcolumn}
                \versiculorespuestaseguido{Deum, in adjutórium meum inténde}{Dómine, ad adjuvándum me festina}
            \end{rightcolumn}
    \end{otherlanguage}

    \begin{leftcolumn*}
        \textbf{Oremos por los fieles difuntos;}\\
Padre Nuestro, que estás ne los cielo\ldots\\
\textit{Dios te salve, María}\ldots\\[1mm]
\versiculorespuesta{El Señor les conceda el descanso eterno}{Y brille para ellos laa luz perpétua}\\[1mm]
\versiculorespuesta{Descansen en paz}{Amén}
    \end{leftcolumn*}
    \begin{otherlanguage}{latin}
            \begin{rightcolumn}
                \primeraletragranderoja{S}{ancte Michaël Archángele,}defénde nos in pr{\ae}lio: contra nequítian et insídias diáboli esto pr{\ae}sidium. Imperet illi Deus, 
súpplices deprecámur: tuque, Prínceps militi{\ae} c{\oe}léstis, Sátanam aliósque spíritus malignos, qui ad perditiónem animarum pervagántur in mundo,
divina virtúte in inférnum detrude. Amen.
            \end{rightcolumn}
    \end{otherlanguage}

    \definecolumnpreamble{0}{\vspace{0em}}
    \definecolumnpreamble{1}{\vspace{0em}}
\end{paracol}

\vspace{1.5em}

\noindent\subsubsection*{II La Visitación de Nuestra Señora}

\vspace{-0.5em}

\hfill\textcolor{red}{Lc 1, 39-41}

\lettrine[lines=2]{\textcolor{red}{P}}or aquellos días, levantándose María, se dirigió presurosa a la montaña, a un ciudad de Judá, 
y entró en la casa de Zacarías y saludó a Isabel. Y aconteció que, al oir Isabel la salutación de María, dió saltos de 
gozo el niño en su seno, y fue llena Isabel del Espíritu Santo,

\vspace{0.5em}

Paternóster, diez Avemarías, Gloria

\vspace{2mm}

\begin{paracol}{2}
    \filacastellanolatin{oraciones/maria_madre/castellano_seguido.tex}{oraciones/maria_madre/latin_seguido.tex}
\end{paracol}

\vspace{1.5em}

\noindent\subsubsection*{III La Natividad de Nuestro Señor Jesucristo}

\vspace{-0.5em}

\hfill\textcolor{red}{Lc 2, 4.7-8.10-11}

\lettrine[lines=2]{\textcolor{red}{S}}ubió también José desde la Galilea, de la ciudad de Nazaret, a la Judea, a la ciudad de David que se llama Belén, por ser él del linaje y familia de David.
Y sucedió que estando ellos allí se le cumplieron a ella los días del parto, y dió a luz su hijo primogénito, y le envolvió en pañales y le recostó en un pesebre, 
pues no había para ellos lugar en el mesón. Y había unos pastores en aquella misma comarca, que pernoctaban al raso y velaban por turno para guardar su ganado, 
y un ángel del Señor se presentó ante ellos. Y les dijo el Ángel: No temáis, pues he aquí que os traigo una buena nueva, que será de grande alegría para todo el pueblo: 
que os ha nacido hoy en la ciudad de David un Salvador, que es el Mesías, el Señor.

\vspace{0.5em}

Paternóster, diez Avemarías, Gloria

\vspace{2mm}

\begin{paracol}{2}
    \filacastellanolatin{oraciones/maria_madre/castellano_seguido.tex}{oraciones/maria_madre/latin_seguido.tex}
\end{paracol}

\vspace{1.5em}

\noindent\subsubsection*{IV La Presentación del Niño Jesús en el Templo}

\vspace{-0.5em}

\hfill\textcolor{red}{Lc 2, 22-24}

\lettrine[lines=2]{\textcolor{red}{Y}}\space{} cuando se les cumplieron los días de la purificación según la ley de Moisés, le subieron a Jerusalén para presentarle al Señor,
según está escrito en la Ley del Señor que <<todo primogénito del sexo masculino será consagrado al Señor>>, y para ofrecer como sacrificio,
según lo que se ordena en la Ley del Señor, <<un par de tórtolas o dos palominos>>.

\vspace{0.5em}

Paternóster, diez Avemarías, Gloria

\vspace{2mm}

\begin{paracol}{2}
    \filacastellanolatin{oraciones/maria_madre/castellano_seguido.tex}{oraciones/maria_madre/latin_seguido.tex}
\end{paracol}

\noindent\subsubsection*{V La pérdida y hallazgo del Niño Jesús en el Templo}

\vspace{-0.5em}

\hfill\textcolor{red}{Lc 2, 43-48}

\lettrine[lines=2]{\textcolor{red}{I}}ban sus padres cada año a Jerusalén por la fiesta de la Pascua. Y cuando fué de doce años, habiendo ellos subido, según la costumbre de la fiesta,
y acabados los días, al volverse ellos, quedóse el niño Jesús en Jerusalén sin que lo advirtiesen sus padres. Y creyendo ellos que El andaría en la comitiva caminaron una jornada; y le
buscaban entre los parientes y conocidos; y no hallándole, se tornaron a Jerusalén para buscarle. Y sucedió que después de tres días le hallaron en el templo,
sentado en medio de los maestros, escuchándolos y haciéndoles preguntas; y se pasmaban todos los que le oían de su inteligencia y de sus respuestas.
Y sus padres, al verle, quedaron sorprendidos; y le dijo su madre: Hijo, {?`}por qué lo hiciste así con nosotros? Mira que tu padre y yo, llenos de aflicción, 
te andábamos buscando.

\vspace{0.5em}

{Paternóster, diez Avemarías, Gloria

\vspace{2mm}

\begin{paracol}{2}
    \filacastellanolatin{oraciones/maria_madre/castellano_seguido.tex}{oraciones/maria_madre/latin_seguido.tex}
\end{paracol}}

\vspace{0.5em}

\iralfinal

\vspace{0.5em}

\begin{center}
      \begin{spacing}{0.25}
            {\rule{20em}{0.4pt}}\\
            {\rule{20em}{0.4pt}}
      \end{spacing}
\end{center}

%%%%%%%%%%%
%  LUNES  %
%---------%
%  MARTES %
%%%%%%%%%%%

\section*{\centering Feria Tertia (Martes)}

\noindent\subsection*{Misterios Dolorosos}

\noindent\subsubsection*{I La oración en el Huerto de los Olivos}

\vspace{-0.5em}

\hfill\textcolor{red}{Mc 14, 33-36}

\lettrine[lines=2]{\textcolor{red}{Y}}\space lleva consigo a Pedro y a Santiago y a Juan, y comenzó a sentir espanto y abatimiento; y le dice: <<triste en gran manera está mi corazón hasta la muerte;
quedad aquí y velad>>. Y apartándose un poco, caía sobre tierra, y rogaba que, a ser posible, pasase el Él aquella hora, y decía: <<Abba, Padre, todas las cosas te son posibles:
traspasa de mi este cáliz; más no se haga lo que yo quiero, sino lo que tú quieres>>.

\vspace{0.5em}

\begin{paracol}{2}
    \begin{leftcolumn}
        \input{oraciones/padrenuestro/castellano_vr.tex}
    \end{leftcolumn}
    \begin{otherlanguage}{latin}
            \begin{rightcolumn}
                \versiculorespuestaseguido{Ave María, grátia plena, Dóminus tecum; benedicta tu in muliéribus, et benedíctus fructus ventris tui,
Jesus}{Sancta Maria, Mater Dei, ora pro nobis peccatóribus, nunc et in hora mortis nostr{\ae}. Amen.} 
            \end{rightcolumn}
    \end{otherlanguage}
\end{paracol}

\vspace{0.5em}

\begin{center}
    \textcolor{red}{El avemaría se repite 10 veces}
\end{center}
\vspace{0.2em}
\begin{paracol}{2}
    \begin{leftcolumn}
        \input{oraciones/avemaria/castellano_vr.tex}
    \end{leftcolumn}
    \begin{otherlanguage}{latin}
            \begin{rightcolumn}
                \versiculorespuestaseguido{Ave María, grátia plena, Dóminus tecum; benedicta tu in muliéribus, et benedíctus fructus ventris tui,
Jesus}{Sancta Maria, Mater Dei, ora pro nobis peccatóribus, nunc et in hora mortis nostr{\ae}. Amen.}
            \end{rightcolumn}
    \end{otherlanguage}

    \definecolumnpreamble{0}{\vspace{0.5em}}
    \definecolumnpreamble{1}{\vspace{0.5em}}

    \begin{leftcolumn*}
        \versiculorespuestaseguido{María, Madre de gracia, Madre de Misericordia}{Defendednos del enemigo y amparadnos ahora y en la hora de nuestra muerte. Amén}
    \end{leftcolumn*}
    \begin{otherlanguage}{latin}
            \begin{rightcolumn}
                \versiculorespuestaseguido{Deum, in adjutórium meum inténde}{Dómine, ad adjuvándum me festina}
            \end{rightcolumn}
    \end{otherlanguage}

    \begin{leftcolumn*}
        \textbf{Oremos por los fieles difuntos;}\\
Padre Nuestro, que estás ne los cielo\ldots\\
\textit{Dios te salve, María}\ldots\\[1mm]
\versiculorespuesta{El Señor les conceda el descanso eterno}{Y brille para ellos laa luz perpétua}\\[1mm]
\versiculorespuesta{Descansen en paz}{Amén}
    \end{leftcolumn*}
    \begin{otherlanguage}{latin}
            \begin{rightcolumn}
                \primeraletragranderoja{S}{ancte Michaël Archángele,}defénde nos in pr{\ae}lio: contra nequítian et insídias diáboli esto pr{\ae}sidium. Imperet illi Deus, 
súpplices deprecámur: tuque, Prínceps militi{\ae} c{\oe}léstis, Sátanam aliósque spíritus malignos, qui ad perditiónem animarum pervagántur in mundo,
divina virtúte in inférnum detrude. Amen.
            \end{rightcolumn}
    \end{otherlanguage}

    \definecolumnpreamble{0}{\vspace{0em}}
    \definecolumnpreamble{1}{\vspace{0em}}
\end{paracol}

\vspace{1.5em}

\noindent\subsubsection*{II La Flagelación de Nuestro Señor Jesucristo}

\vspace{-0.5em}

\hfill\textcolor{red}{Jn 18,38-40; 19, 1}

\lettrine[lines=2, ante=\guillemotleft]{\textcolor{red}{Y}}o no hallo en Él delito alguno. Es costumbre vuestra que yo os suelte un preso por la Pascua: {?`}queréis, 
pues, que os suelte al rey de los Judíos?\guillemotright. Gritaron, pues, de nuevo, diciendo: <<No, a ése, sino a Barrabás>>. 
Era este Barrabás un salteador. Entonces, pues, tomó Pilato a Jesús y le azotó.

\vspace{0.5em}

Paternóster, diez Avemarías, Gloria

\vspace{2mm}

\begin{paracol}{2}
    \filacastellanolatin{oraciones/maria_madre/castellano_seguido.tex}{oraciones/maria_madre/latin_seguido.tex}
\end{paracol}

\vspace{1.5em}

\noindent\subsubsection*{III La Coronación de espinas de Nuestro Señor Jesucristo}

\vspace{-0.5em}

\hfill\textcolor{red}{Mt 27, 27-30}

\lettrine[lines=2]{\textcolor{red}{E}}ntonces los soldados del gobernador, tomando a Jesús y conduciéndole al pretorio, reunieron en torno a Él toda la cohorte. 
Y habiéndole quitado sus vestidos, le envolvieron en una clámide de grana, y trenzando una corona de espinas, la pusieron sobre su cabeza, 
y una caña en su mano derecha; y doblando la rodilla delante de Él, le mofaban, diciendo: <<Salud, Rey de los judíos>>. Y escupiendo en Él, 
tomaron la caña y le daban golpes en la cabeza.

\vspace{0.5em}

Paternóster, diez Avemarías, Gloria

\vspace{2mm}

\begin{paracol}{2}
    \filacastellanolatin{oraciones/maria_madre/castellano_seguido.tex}{oraciones/maria_madre/latin_seguido.tex}
\end{paracol}

\vspace{1.5em}

\noindent\subsubsection*{IV El Señor con la Cruz a cuestas}

\vspace{-0.5em}

\hfill\textcolor{red}{Jn 19, 16-17; Lc 23, 26}

\lettrine[lines=2]{\textcolor{red}{E}}ntonces, pues, se le entregó para que fuera crucificando. Se apoderaron, pues, de Jesús, y llevando a cuestas su cruz, 
salió hacia el lugar llamado el Cráneo, que en hebreo se dice Gólgota. Y como le hubieron sacado, echaron mano de un tal Simón de Cirene que venía del campo, 
le pusieron en hombros la cruz para que la llevase detrás de Jesús.

\vspace{0.5em}

Paternóster, diez Avemarías, Gloria

\vspace{2mm}

\begin{paracol}{2}
    \filacastellanolatin{oraciones/maria_madre/castellano_seguido.tex}{oraciones/maria_madre/latin_seguido.tex}
\end{paracol}

\vspace{1.5em}

\noindent\subsubsection*{V El Señor muere en la Cruz}

\vspace{-0.5em}

\hfill\textcolor{red}{Lc 23, 33-34; Jn 19, 19.25-27; Lc 23, 44-46}

\lettrine[lines=2]{\textcolor{red}{C}}uando llegaron al lugar llamado Calvario, le crucificaron allí, y a los dos malhechores, uno a la derecha y otro a la izquierda. 
Jesús decía: Padre, perdónalos, porque no saben los que hacen. Dividiendo sus vestidos, echaron suertes sobre ellos. Escribió Pilato un título y lo puso sobre la cruz;
estaba escrito: \textit{Jesús Nazareno, Rey de los judíos}. Estaba junto a la cruz de Jesús su Madre y la hermana de su Madre, María la debajo Cleofás y María Magdalena.
Jesús, viendo a su Madre y al discípulo a quien amaba, que estaba allí, dijo a la Madre: Mujer, he ahí a tu hijo. Luego dijo al discípulo: He ahí a tu Madre.
Y desde aquella hora el discípulo la recibió en su casa. Era ya como la hora de sexta, y las tinieblas cubrieron toda la tierra hasta la hora de nona,
obscurecióse el sol y el velo del templo se rasgó por medio. Jesús, dando una gran voz, dijo: Padre, en tus manos entrego mi espíritu; y diciendo esto expiró.

\vspace{0.5em}

{Paternóster, diez Avemarías, Gloria

\vspace{2mm}

\begin{paracol}{2}
    \filacastellanolatin{oraciones/maria_madre/castellano_seguido.tex}{oraciones/maria_madre/latin_seguido.tex}
\end{paracol}}

\iralfinal

\vspace{0.5em}

\begin{center}
      \begin{spacing}{0.25}
            {\rule{20em}{0.4pt}}\\
            {\rule{20em}{0.4pt}}
      \end{spacing}
\end{center}

%%%%%%%%%%%%%%
%  MARTES    %
%------------%
%  MIERCOLES %
%%%%%%%%%%%%%%

\section*{\centering Feria Quarta (Miércoles)}

\noindent\subsection*{Misterios Gloriosos}

\noindent\subsubsection*{I La oración en el Huerto de los Olivos}

\vspace{-0.5em}

\hfill\textcolor{red}{Mc 14, 33-36}

\lettrine[lines=2]{\textcolor{red}{Y}}\space lleva consigo a Pedro y a Santiago y a Juan, y comenzó a sentir espanto y abatimiento; y le dice: <<triste en gran manera está mi corazón hasta la muerte;
quedad aquí y velad>>. Y apartándose un poco, caía sobre tierra, y rogaba que, a ser posible, pasase el Él aquella hora, y decía: <<Abba, Padre, todas las cosas te son posibles:
traspasa de mi este cáliz; más no se haga lo que yo quiero, sino lo que tú quieres>>.

\vspace{0.5em}

\begin{paracol}{2}
    \begin{leftcolumn}
        \input{oraciones/padrenuestro/castellano_vr.tex}
    \end{leftcolumn}
    \begin{otherlanguage}{latin}
            \begin{rightcolumn}
                \versiculorespuestaseguido{Ave María, grátia plena, Dóminus tecum; benedicta tu in muliéribus, et benedíctus fructus ventris tui,
Jesus}{Sancta Maria, Mater Dei, ora pro nobis peccatóribus, nunc et in hora mortis nostr{\ae}. Amen.} 
            \end{rightcolumn}
    \end{otherlanguage}
\end{paracol}

\vspace{0.5em}

\begin{center}
    \textcolor{red}{El avemaría se repite 10 veces}
\end{center}
\vspace{0.2em}
\begin{paracol}{2}
    \begin{leftcolumn}
        \input{oraciones/avemaria/castellano_vr.tex}
    \end{leftcolumn}
    \begin{otherlanguage}{latin}
            \begin{rightcolumn}
                \versiculorespuestaseguido{Ave María, grátia plena, Dóminus tecum; benedicta tu in muliéribus, et benedíctus fructus ventris tui,
Jesus}{Sancta Maria, Mater Dei, ora pro nobis peccatóribus, nunc et in hora mortis nostr{\ae}. Amen.}
            \end{rightcolumn}
    \end{otherlanguage}

    \definecolumnpreamble{0}{\vspace{0.5em}}
    \definecolumnpreamble{1}{\vspace{0.5em}}

    \begin{leftcolumn*}
        \versiculorespuestaseguido{María, Madre de gracia, Madre de Misericordia}{Defendednos del enemigo y amparadnos ahora y en la hora de nuestra muerte. Amén}
    \end{leftcolumn*}
    \begin{otherlanguage}{latin}
            \begin{rightcolumn}
                \versiculorespuestaseguido{Deum, in adjutórium meum inténde}{Dómine, ad adjuvándum me festina}
            \end{rightcolumn}
    \end{otherlanguage}

    \begin{leftcolumn*}
        \textbf{Oremos por los fieles difuntos;}\\
Padre Nuestro, que estás ne los cielo\ldots\\
\textit{Dios te salve, María}\ldots\\[1mm]
\versiculorespuesta{El Señor les conceda el descanso eterno}{Y brille para ellos laa luz perpétua}\\[1mm]
\versiculorespuesta{Descansen en paz}{Amén}
    \end{leftcolumn*}
    \begin{otherlanguage}{latin}
            \begin{rightcolumn}
                \primeraletragranderoja{S}{ancte Michaël Archángele,}defénde nos in pr{\ae}lio: contra nequítian et insídias diáboli esto pr{\ae}sidium. Imperet illi Deus, 
súpplices deprecámur: tuque, Prínceps militi{\ae} c{\oe}léstis, Sátanam aliósque spíritus malignos, qui ad perditiónem animarum pervagántur in mundo,
divina virtúte in inférnum detrude. Amen.
            \end{rightcolumn}
    \end{otherlanguage}

    \definecolumnpreamble{0}{\vspace{0em}}
    \definecolumnpreamble{1}{\vspace{0em}}
\end{paracol}

\noindent\subsubsection*{II La Flagelación de Nuestro Señor Jesucristo}

\vspace{-0.5em}

\hfill\textcolor{red}{Jn 18,38-40; 19, 1}

\lettrine[lines=2, ante=\guillemotleft]{\textcolor{red}{Y}}o no hallo en Él delito alguno. Es costumbre vuestra que yo os suelte un preso por la Pascua: {?`}queréis, 
pues, que os suelte al rey de los Judíos?\guillemotright. Gritaron, pues, de nuevo, diciendo: <<No, a ése, sino a Barrabás>>. 
Era este Barrabás un salteador. Entonces, pues, tomó Pilato a Jesús y le azotó.

\vspace{0.5em}

Paternóster, diez Avemarías, Gloria

\vspace{2mm}

\begin{paracol}{2}
    \filacastellanolatin{oraciones/maria_madre/castellano_seguido.tex}{oraciones/maria_madre/latin_seguido.tex}
\end{paracol}

\noindent\subsubsection*{III La Coronación de espinas de Nuestro Señor Jesucristo}

\vspace{-0.5em}

\hfill\textcolor{red}{Mt 27, 27-30}

\lettrine[lines=2]{\textcolor{red}{E}}ntonces los soldados del gobernador, tomando a Jesús y conduciéndole al pretorio, reunieron en torno a Él toda la cohorte. 
Y habiéndole quitado sus vestidos, le envolvieron en una clámide de grana, y trenzando una corona de espinas, la pusieron sobre su cabeza, 
y una caña en su mano derecha; y doblando la rodilla delante de Él, le mofaban, diciendo: <<Salud, Rey de los judíos>>. Y escupiendo en Él, 
tomaron la caña y le daban golpes en la cabeza.

\vspace{0.5em}

Paternóster, diez Avemarías, Gloria

\vspace{2mm}

\begin{paracol}{2}
    \filacastellanolatin{oraciones/maria_madre/castellano_seguido.tex}{oraciones/maria_madre/latin_seguido.tex}
\end{paracol}

\noindent\subsubsection*{IV El Señor con la Cruz a cuestas}

\vspace{-0.5em}

\hfill\textcolor{red}{Jn 19, 16-17; Lc 23, 26}

\lettrine[lines=2]{\textcolor{red}{E}}ntonces, pues, se le entregó para que fuera crucificando. Se apoderaron, pues, de Jesús, y llevando a cuestas su cruz, 
salió hacia el lugar llamado el Cráneo, que en hebreo se dice Gólgota. Y como le hubieron sacado, echaron mano de un tal Simón de Cirene que venía del campo, 
le pusieron en hombros la cruz para que la llevase detrás de Jesús.

\vspace{0.5em}

Paternóster, diez Avemarías, Gloria

\vspace{2mm}

\begin{paracol}{2}
    \filacastellanolatin{oraciones/maria_madre/castellano_seguido.tex}{oraciones/maria_madre/latin_seguido.tex}
\end{paracol}

\noindent\subsubsection*{V El Señor muere en la Cruz}

\vspace{-0.5em}

\hfill\textcolor{red}{Lc 23, 33-34; Jn 19, 19.25-27; Lc 23, 44-46}

\lettrine[lines=2]{\textcolor{red}{C}}uando llegaron al lugar llamado Calvario, le crucificaron allí, y a los dos malhechores, uno a la derecha y otro a la izquierda. 
Jesús decía: Padre, perdónalos, porque no saben los que hacen. Dividiendo sus vestidos, echaron suertes sobre ellos. Escribió Pilato un título y lo puso sobre la cruz;
estaba escrito: \textit{Jesús Nazareno, Rey de los judíos}. Estaba junto a la cruz de Jesús su Madre y la hermana de su Madre, María la debajo Cleofás y María Magdalena.
Jesús, viendo a su Madre y al discípulo a quien amaba, que estaba allí, dijo a la Madre: Mujer, he ahí a tu hijo. Luego dijo al discípulo: He ahí a tu Madre.
Y desde aquella hora el discípulo la recibió en su casa. Era ya como la hora de sexta, y las tinieblas cubrieron toda la tierra hasta la hora de nona,
obscurecióse el sol y el velo del templo se rasgó por medio. Jesús, dando una gran voz, dijo: Padre, en tus manos entrego mi espíritu; y diciendo esto expiró.

\vspace{0.5em}

{Paternóster, diez Avemarías, Gloria

\vspace{2mm}

\begin{paracol}{2}
    \filacastellanolatin{oraciones/maria_madre/castellano_seguido.tex}{oraciones/maria_madre/latin_seguido.tex}
\end{paracol}}

\iralfinal

\vspace{0.5em}

\begin{center}
      \begin{spacing}{0.25}
            {\rule{20em}{0.4pt}}\\
            {\rule{20em}{0.4pt}}
      \end{spacing}
\end{center}

%%%%%%%%%%%%%%
%  MIERCOLES %
%------------%
%   JUEVES   %
%%%%%%%%%%%%%%

\section*{\centering Feria Quinta (Jueves)}

\begin{paracol}{2}
      \begin{leftcolumn}
            \begin{center}
                  \textcolor{red}{Forma Nueva}
                  \subsection*{Misterios Luminosos}
            \end{center}

            \noindent\subsubsection*{I El Bautismo del Señor en el Jordán}

            \vspace{-0.5em}

            \hfill\textcolor{red}{Mc 1, 9-11}

            \lettrine[lines=2]{\textcolor{red}{Y}}\space aconteció por aquellos días que vino Jesús desde Nazaret de Galilea y fué bautizado en el Jordán por Juan [el Bautista].
            Y al punto subiendo del agua, vió rasgarse los cielos y venir sobre Él el Espíritu Santo como paloma; y una voz vino de los cielos: 
            <<Tú eres mi Hijo amado, en Ti me agradé>>.
      \end{leftcolumn}
      \begin{rightcolumn}
            \begin{center}
                  \textcolor{red}{Forma Tradicional}
                  \subsection*{Misterios Gozosos }
            \end{center}

            \noindent\subsubsection*{I La Anunciación de la Santísima Virgen María}

            \vspace{-0.5em}

            \hfill\textcolor{red}{Lc 1, 26-33}

            \lettrine[lines=2]{\textcolor{red}{F}}ue enviado el ángel Gabriel de parte de Dios a una ciudad de Galilea, llamada Nazaret, 
            a una doncella desposada con un varón llamado José, de la familia de David, y el nombre de la doncella era María. 
            Y habiendo entrado a ella, dijo: Dios te salve, llena de gracia, el Señor es contigo, bendita tu entre las mujeres.
            Ella, al oír estas palabras, se turbó, y discurría que podría ser esta salutación. Y le dijo el ángel: No temas, María, 
            pues hallaste gracia a los ojos de Dios. He aquí que concebirás en tu seno y darás a luz un Hijo, a quien darás por nombre Jesús. 
            Este será grande, y será llamado Hijo del Altísimo, y le dará el Señor Dios el trono de David su padre, y reinará sobre la 
            casa de Jacob etérnamente, y su reinado no tendrá fin.
      \end{rightcolumn}
\end{paracol}

\begin{center}
      \rule{25em}{0.4pt}
\end{center}

\begin{paracol}{2}
    \begin{leftcolumn}
        \input{oraciones/padrenuestro/castellano_vr.tex}
    \end{leftcolumn}
    \begin{otherlanguage}{latin}
            \begin{rightcolumn}
                \versiculorespuestaseguido{Ave María, grátia plena, Dóminus tecum; benedicta tu in muliéribus, et benedíctus fructus ventris tui,
Jesus}{Sancta Maria, Mater Dei, ora pro nobis peccatóribus, nunc et in hora mortis nostr{\ae}. Amen.} 
            \end{rightcolumn}
    \end{otherlanguage}
\end{paracol}

\vspace{0.5em}

\begin{center}
    \textcolor{red}{El avemaría se repite 10 veces}
\end{center}
\vspace{0.2em}
\begin{paracol}{2}
    \begin{leftcolumn}
        \input{oraciones/avemaria/castellano_vr.tex}
    \end{leftcolumn}
    \begin{otherlanguage}{latin}
            \begin{rightcolumn}
                \versiculorespuestaseguido{Ave María, grátia plena, Dóminus tecum; benedicta tu in muliéribus, et benedíctus fructus ventris tui,
Jesus}{Sancta Maria, Mater Dei, ora pro nobis peccatóribus, nunc et in hora mortis nostr{\ae}. Amen.}
            \end{rightcolumn}
    \end{otherlanguage}

    \definecolumnpreamble{0}{\vspace{0.5em}}
    \definecolumnpreamble{1}{\vspace{0.5em}}

    \begin{leftcolumn*}
        \versiculorespuestaseguido{María, Madre de gracia, Madre de Misericordia}{Defendednos del enemigo y amparadnos ahora y en la hora de nuestra muerte. Amén}
    \end{leftcolumn*}
    \begin{otherlanguage}{latin}
            \begin{rightcolumn}
                \versiculorespuestaseguido{Deum, in adjutórium meum inténde}{Dómine, ad adjuvándum me festina}
            \end{rightcolumn}
    \end{otherlanguage}

    \begin{leftcolumn*}
        \textbf{Oremos por los fieles difuntos;}\\
Padre Nuestro, que estás ne los cielo\ldots\\
\textit{Dios te salve, María}\ldots\\[1mm]
\versiculorespuesta{El Señor les conceda el descanso eterno}{Y brille para ellos laa luz perpétua}\\[1mm]
\versiculorespuesta{Descansen en paz}{Amén}
    \end{leftcolumn*}
    \begin{otherlanguage}{latin}
            \begin{rightcolumn}
                \primeraletragranderoja{S}{ancte Michaël Archángele,}defénde nos in pr{\ae}lio: contra nequítian et insídias diáboli esto pr{\ae}sidium. Imperet illi Deus, 
súpplices deprecámur: tuque, Prínceps militi{\ae} c{\oe}léstis, Sátanam aliósque spíritus malignos, qui ad perditiónem animarum pervagántur in mundo,
divina virtúte in inférnum detrude. Amen.
            \end{rightcolumn}
    \end{otherlanguage}

    \definecolumnpreamble{0}{\vspace{0em}}
    \definecolumnpreamble{1}{\vspace{0em}}
\end{paracol}

\begin{center}
      \rule{25em}{0.4pt}
\end{center}

\begin{paracol}{2}
      \begin{leftcolumn}
            \noindent\subsubsection*{II Las Bodas de Caná}
            
            \vspace{-0.5em}

            \hfill\textcolor{red}{Jn 2, 1-5}

            \lettrine[lines=2]{\textcolor{red}{Y}}\space al día tercero se celebraron unas bodas en Caná de Galilea, y estaba allí la madre de Jesús. Fueron también invitados a las bodas
            Jesús y sus discípulos. Y como faltase el vino, dice su madre a Jesús: no tienen vino. Y le dice Jesús: {?`}Qué tenemos que ver tú y yo, mujer? Todavía no ha llegado mi hora.
            DIce su madre a los que servían: Todo cuanto Él os diga, hacedlo.
      \end{leftcolumn}
      \begin{rightcolumn}
            \noindent\subsubsection*{II La Visitación de Nuestra Señora}

            \vspace{-0.5em}

            \hfill\textcolor{red}{Lc 1, 39-45}

            \lettrine[lines=2]{\textcolor{red}{P}}or aquellos días, levantándose María, se dirigió presurosa a la montaña, a un ciudad de Judá, 
            y entró en la casa de Zacarías y saludó a Isabel. Y aconteció que, al oir Isabel la salutación de María, dió saltos de 
            gozo el niño en su seno, y fue llena Isabel del Espíritu Santo, y levantó la voz con gran clamor y dijo: Bendita tu entre 
            las mujeres y bendito el fruto de tu vientre. {?`}Y de dónde a mí esto que venga la madre de mi Señor a mí? Porque he aquí que, 
            como sonó la voz de tu salutación en mi oídos, dió saltos de alborozo el niño en mi seno.
      \end{rightcolumn}
\end{paracol}

Paternóster, diez Avemarías, Gloria

\vspace{2mm}

\begin{paracol}{2}
    \filacastellanolatin{oraciones/maria_madre/castellano_seguido.tex}{oraciones/maria_madre/latin_seguido.tex}
\end{paracol}

\begin{paracol}{2}
      \begin{leftcolumn}

            \noindent\subsubsection*{III El Anuncio del Reino de Dios}
            
            \vspace{-0.5em}

            \hfill\textcolor{red}{Mc 1, 14-15.21-22}

            \lettrine[lines=2]{\textcolor{red}{Y}}\space después que Juan [el Bautista] hubo sido entregado, vino Jesús a Galilea, y allí predicaba el Evangelio de Dios, y decía que <<Se ha cumplido
            el tiempo y está cerca el reino de Dios: arrepentíos y creed en el Evangelio>>. Y entran en Cafarnaúm; y luego que fué sábado enseñaba en la sinagoga. Y se asombraban de su
            enseñanza, porque les estaba enseñando como quien tiene autoridad, y no como los escribas.

      \end{leftcolumn}
      \begin{rightcolumn}
            \noindent\subsubsection*{III La Natividad de Nuestro Señor Jesucristo}

            \vspace{-0.5em}

            \hfill\textcolor{red}{Lc 2, 4.7-8.10-11}

            \lettrine[lines=2]{\textcolor{red}{S}}ubió también José desde la Galilea, de la ciudad de Nazaret, a la Judea, a la ciudad de David que se llama Belén, por ser él del linaje y familia de David.
            Y sucedió que estando ellos allí se le cumplieron a ella los días del parto, y dió a luz su hijo primogénito, y le envolvió en pañales y le recostó en un pesebre, 
            pues no había para ellos lugar en el mesón. Y había unos pastores en aquella misma comarca, que pernoctaban al raso y velaban por turno para guardar su ganado, 
            y un ángel del Señor se presentó ante ellos. Y les dijo el Ángel: No temáis, pues he aquí que os traigo una buena nueva, que será de grande alegría para todo el pueblo: 
            que os ha nacido hoy en la ciudad de David un Salvador, que es el Mesías, el Señor.
      \end{rightcolumn}
\end{paracol}

Paternóster, diez Avemarías, Gloria

\vspace{2mm}

\begin{paracol}{2}
    \filacastellanolatin{oraciones/maria_madre/castellano_seguido.tex}{oraciones/maria_madre/latin_seguido.tex}
\end{paracol}

\begin{paracol}{2}
      \begin{leftcolumn}

            \noindent\subsubsection*{IV La Transfiguración}
            
            \vspace{-0.5em}

            \hfill\textcolor{red}{Mt 17, 1-5}

            \lettrine[lines=2]{\textcolor{red}{Y}}\space seis días después toma Jesús consigo a Pedro, a Santiago y a Juan, su hermano, y sube con ellos a un monte elevado a solas. Y se transfiguró
            en presencia de ellos, y comenzó a relumbar su faz como el sol, y sus vestiduras se pararon blancas como la luz. Y de pronto aparecieron a su vista Moisés y Elías, conversando con El.
            Tomando Pedro la palabra, dijo a Jesús: Señor, linda cosa es estarnos aquí; si quieres, haré aquí tres tiendas: una para ti, otra para Moisés y otra para Elías. Estando aún él hablando,
            de pronto una nube luminosa los cubrió. Y he aquí una voz salida de la nube, que decía: Este es mi Hijo querido, en quien me agradé; escuchadle.
      \end{leftcolumn}
      \begin{rightcolumn}

            \noindent\subsubsection*{IV La Presentación del Niño Jesús en el Templo}

            \vspace{-0.5em}

            \hfill\textcolor{red}{Lc 2, 22-24}

            \lettrine[lines=2]{\textcolor{red}{Y}}\space{} cuando se les cumplieron los días de la purificación según la ley de Moisés, le subieron a Jerusalén para presentarle al Señor,
            según está escrito en la Ley del Señor que <<todo primogénito del sexo masculino será consagrado al Señor>>, y para ofrecer como sacrificio,
            según lo que se ordena en la Ley del Señor, <<un par de tórtolas o dos palominos>>.
      \end{rightcolumn}
\end{paracol}

Paternóster, diez Avemarías, Gloria

\vspace{2mm}

\begin{paracol}{2}
    \filacastellanolatin{oraciones/maria_madre/castellano_seguido.tex}{oraciones/maria_madre/latin_seguido.tex}
\end{paracol}

\begin{paracol}{2}
      \begin{leftcolumn}

            \noindent\subsubsection*{V La Institución de la Eucaristía}
            
            \vspace{-0.5em}

            \hfill\textcolor{red}{Mt 26, 26-28}
            
            \lettrine[lines=2]{\textcolor{red}{E}}stando ellos comiendo, tomando Jesús un pan, y habiendo pronunciado la bendición, lo partió, y dándolo a los discípulos dijo: <<Tomad, comed: éste es mi cuerpo>>.
            Y habiendo tomado un cáliz, y habiendo dado gracias, se lo dió, diciendo: <<Bebed de él todos, porque ésta es mi sangre de la alianza, que por muchos es derramada para la remisión de los pecados>>.
      \end{leftcolumn}
      \begin{rightcolumn}

            \noindent\subsubsection*{V La pérdida y hallazgo del Niño Jesús en el Templo}

            \vspace{-0.5em}

            \hfill\textcolor{red}{Lc 2, 43-48}

            \lettrine[lines=2]{\textcolor{red}{I}}ban sus padres cada año a Jerusalén por la fiesta de la Pascua. Y cuando fué de doce años, habiendo ellos subido, según la costumbre de la fiesta,
            y acabados los días, al volverse ellos, quedóse el niño Jesús en Jerusalén sin que lo advirtiesen sus padres. Y creyendo ellos que El andaría en la comitiva caminaron una jornada; y le
            buscaban entre los parientes y conocidos; y no hallándole, se tornaron a Jerusalén para buscarle. Y sucedió que después de tres días le hallaron en el templo,
            sentado en medio de los maestros, escuchándolos y haciéndoles preguntas; y se pasmaban todos los que le oían de su inteligencia y de sus respuestas.
            Y sus padres, al verle, quedaron sorprendidos; y le dijo su madre: Hijo, {?`}por qué lo hiciste así con nosotros? Mira que tu padre y yo, llenos de aflicción, 
            te andábamos buscando.
        \end{rightcolumn}
\end{paracol}

Paternóster, diez Avemarías, Gloria

\vspace{2mm}

\begin{paracol}{2}
    \filacastellanolatin{oraciones/maria_madre/castellano_seguido.tex}{oraciones/maria_madre/latin_seguido.tex}
\end{paracol}

\iralfinal

\vspace{0.5em}

\begin{center}
      \begin{spacing}{0.25}
            {\rule{20em}{0.4pt}}\\
            {\rule{20em}{0.4pt}}
      \end{spacing}
\end{center}

%%%%%%%%%%%%%%
%   JUEVES   %
%------------%
%   VIERNES  %
%%%%%%%%%%%%%%

\section*{\centering Feria Sexta (Viernes)}

\noindent\subsection*{Misterios Dolorosos}

\vspace{1.5em}

\noindent\titulomisterios{I La oración en el Huerto de los Olivos}{Mc 14, 33-36}
\lettrine[lines=2]{\textcolor{red}{Y}}\space lleva consigo a Pedro y a Santiago y a Juan, y comenzó a sentir espanto y abatimiento; y le dice: <<triste en gran manera está mi corazón hasta la muerte;
quedad aquí y velad>>. Y apartándose un poco, caía sobre tierra, y rogaba que, a ser posible, pasase el Él aquella hora, y decía: <<Abba, Padre, todas las cosas te son posibles:
traspasa de mi este cáliz; más no se haga lo que yo quiero, sino lo que tú quieres>>.

\vspace{0.5em}

\begin{paracol}{2}
    \begin{leftcolumn}
        \input{oraciones/padrenuestro/castellano_vr.tex}
    \end{leftcolumn}
    \begin{otherlanguage}{latin}
            \begin{rightcolumn}
                \versiculorespuestaseguido{Ave María, grátia plena, Dóminus tecum; benedicta tu in muliéribus, et benedíctus fructus ventris tui,
Jesus}{Sancta Maria, Mater Dei, ora pro nobis peccatóribus, nunc et in hora mortis nostr{\ae}. Amen.} 
            \end{rightcolumn}
    \end{otherlanguage}
\end{paracol}

\vspace{0.5em}

\begin{center}
    \textcolor{red}{El avemaría se repite 10 veces}
\end{center}
\vspace{0.2em}
\begin{paracol}{2}
    \begin{leftcolumn}
        \input{oraciones/avemaria/castellano_vr.tex}
    \end{leftcolumn}
    \begin{otherlanguage}{latin}
            \begin{rightcolumn}
                \versiculorespuestaseguido{Ave María, grátia plena, Dóminus tecum; benedicta tu in muliéribus, et benedíctus fructus ventris tui,
Jesus}{Sancta Maria, Mater Dei, ora pro nobis peccatóribus, nunc et in hora mortis nostr{\ae}. Amen.}
            \end{rightcolumn}
    \end{otherlanguage}

    \definecolumnpreamble{0}{\vspace{0.5em}}
    \definecolumnpreamble{1}{\vspace{0.5em}}

    \begin{leftcolumn*}
        \versiculorespuestaseguido{María, Madre de gracia, Madre de Misericordia}{Defendednos del enemigo y amparadnos ahora y en la hora de nuestra muerte. Amén}
    \end{leftcolumn*}
    \begin{otherlanguage}{latin}
            \begin{rightcolumn}
                \versiculorespuestaseguido{Deum, in adjutórium meum inténde}{Dómine, ad adjuvándum me festina}
            \end{rightcolumn}
    \end{otherlanguage}

    \begin{leftcolumn*}
        \textbf{Oremos por los fieles difuntos;}\\
Padre Nuestro, que estás ne los cielo\ldots\\
\textit{Dios te salve, María}\ldots\\[1mm]
\versiculorespuesta{El Señor les conceda el descanso eterno}{Y brille para ellos laa luz perpétua}\\[1mm]
\versiculorespuesta{Descansen en paz}{Amén}
    \end{leftcolumn*}
    \begin{otherlanguage}{latin}
            \begin{rightcolumn}
                \primeraletragranderoja{S}{ancte Michaël Archángele,}defénde nos in pr{\ae}lio: contra nequítian et insídias diáboli esto pr{\ae}sidium. Imperet illi Deus, 
súpplices deprecámur: tuque, Prínceps militi{\ae} c{\oe}léstis, Sátanam aliósque spíritus malignos, qui ad perditiónem animarum pervagántur in mundo,
divina virtúte in inférnum detrude. Amen.
            \end{rightcolumn}
    \end{otherlanguage}

    \definecolumnpreamble{0}{\vspace{0em}}
    \definecolumnpreamble{1}{\vspace{0em}}
\end{paracol}

\vspace{1.5em}

\noindent\titulomisterios{II La Flagelación de Nuestro Señor Jesucristo}{Jn 18,38-40; 19, 1}
\lettrine[lines=2, ante=\guillemotleft]{\textcolor{red}{Y}}o no hallo en Él delito alguno. Es costumbre vuestra que yo os suelte un preso por la Pascua: {?`}queréis, 
pues, que os suelte al rey de los Judíos?\guillemotright. Gritaron, pues, de nuevo, diciendo: <<No, a ése, sino a Barrabás>>. 
Era este Barrabás un salteador. Entonces, pues, tomó Pilato a Jesús y le azotó.

\vspace{0.5em}

Paternóster, diez Avemarías, Gloria

\vspace{2mm}

\begin{paracol}{2}
    \filacastellanolatin{oraciones/maria_madre/castellano_seguido.tex}{oraciones/maria_madre/latin_seguido.tex}
\end{paracol}

\vspace{1.5em}

\noindent\titulomisterios{III La Coronación de espinas de Nuestro Señor Jesucristo}{Mt 27, 27-30}
\lettrine[lines=2]{\textcolor{red}{E}}ntonces los soldados del gobernador, tomando a Jesús y conduciéndole al pretorio, reunieron en torno a Él toda la cohorte. 
Y habiéndole quitado sus vestidos, le envolvieron en una clámide de grana, y trenzando una corona de espinas, la pusieron sobre su cabeza, 
y una caña en su mano derecha; y doblando la rodilla delante de Él, le mofaban, diciendo: <<Salud, Rey de los judíos>>. Y escupiendo en Él, 
tomaron la caña y le daban golpes en la cabeza.

\vspace{0.5em}

Paternóster, diez Avemarías, Gloria

\vspace{2mm}

\begin{paracol}{2}
    \filacastellanolatin{oraciones/maria_madre/castellano_seguido.tex}{oraciones/maria_madre/latin_seguido.tex}
\end{paracol}

\vspace{1.5em}

\noindent\titulomisterios{IV El Señor con la Cruz a cuestas}{Jn 19, 16-17; Lc 23, 26}
\lettrine[lines=2]{\textcolor{red}{E}}ntonces, pues, se le entregó para que fuera crucificando. Se apoderaron, pues, de Jesús, y llevando a cuestas su cruz, 
salió hacia el lugar llamado el Cráneo, que en hebreo se dice Gólgota. Y como le hubieron sacado, echaron mano de un tal Simón de Cirene que venía del campo, 
le pusieron en hombros la cruz para que la llevase detrás de Jesús.

\vspace{0.5em}

Paternóster, diez Avemarías, Gloria

\vspace{2mm}

\begin{paracol}{2}
    \filacastellanolatin{oraciones/maria_madre/castellano_seguido.tex}{oraciones/maria_madre/latin_seguido.tex}
\end{paracol}

\vspace{1.5em}

\noindent\titulomisterios{V El Señor muere en la Cruz}{Lc 23, 33-34; Jn 19, 19.25-27;\\Lc 23, 44-46}
\lettrine[lines=2]{\textcolor{red}{C}}uando llegaron al lugar llamado Calvario, le crucificaron allí, y a los dos malhechores, uno a la derecha y otro a la izquierda. 
Jesús decía: Padre, perdónalos, porque no saben los que hacen. Dividiendo sus vestidos, echaron suertes sobre ellos. Escribió Pilato un título y lo puso sobre la cruz;
estaba escrito: \textit{Jesús Nazareno, Rey de los judíos}. Estaba junto a la cruz de Jesús su Madre y la hermana de su Madre, María la debajo Cleofás y María Magdalena.
Jesús, viendo a su Madre y al discípulo a quien amaba, que estaba allí, dijo a la Madre: Mujer, he ahí a tu hijo. Luego dijo al discípulo: He ahí a tu Madre.
Y desde aquella hora el discípulo la recibió en su casa. Era ya como la hora de sexta, y las tinieblas cubrieron toda la tierra hasta la hora de nona,
obscurecióse el sol y el velo del templo se rasgó por medio. Jesús, dando una gran voz, dijo: Padre, en tus manos entrego mi espíritu; y diciendo esto expiró.

\vspace{0.5em}

{Paternóster, diez Avemarías, Gloria

\vspace{2mm}

\begin{paracol}{2}
    \filacastellanolatin{oraciones/maria_madre/castellano_seguido.tex}{oraciones/maria_madre/latin_seguido.tex}
\end{paracol}}

\iralfinal

\vspace{0.5em}

\begin{center}
      \begin{spacing}{0.25}
            {\rule{20em}{0.4pt}}\\
            {\rule{20em}{0.4pt}}
      \end{spacing}
\end{center}


%%%%%%%%%%%%%%
%   VIERNES  %
%------------%
%   SABADO   %
%%%%%%%%%%%%%%

\section*{\centering Sabatto (Sábado)}

\begin{paracol}{2}
      \begin{leftcolumn}
            \begin{center}
                  \textcolor{red}{Forma Nueva}
                  \subsection*{Misterios Misterios Gozosos}
            \end{center}

            \noindent\subsubsection*{I La Anunciación}

            \vspace{-0.5em}

            \hfill\textcolor{red}{Jn 1, 1.14}

            \lettrine[lines=2]{\textcolor{red}{E}}n el principio existía el Verbo, y el Verbo estaba en Dios, y el Verbo era Dios. Y el Verbo se hizo
            carne, y habitó entre nosotros; y contemplamos su gloria, gloria cual Unigénito procedente del Padre: lleno de gracia y de verdad

      \end{leftcolumn}
      \begin{rightcolumn}
            \begin{center}
                  \textcolor{red}{Forma Tradicional}
                  \subsection*{Misterios Gloriosos}
            \end{center}

            \noindent\subsubsection*{I La Resurección del Señor}

            \vspace{-0.5em}

            \hfill\textcolor{red}{Lc 24, 5-8}

            \lettrine[lines=2, ante={?`}]{\textcolor{red}{A}} qué buscáis al vivo entre los muertos? No está aquí, sino que resucitó. Recordad cómo
            os habló cuando estaba aún en Galilea, diciendo que el Hijo del hombre había de ser entregado en manos de hombres pecadores y ser crucificado
            y al tercer día resucitar. Y se acordaron de sus palabras. Y vueltas del monumento, anunciaron todas estas cosas a los Once y a todos los demás.

      \end{rightcolumn}
\end{paracol}

\begin{center}
      \rule{25em}{0.4pt}
\end{center}

\begin{paracol}{2}
    \begin{leftcolumn}
        \input{oraciones/padrenuestro/castellano_vr.tex}
    \end{leftcolumn}
    \begin{otherlanguage}{latin}
            \begin{rightcolumn}
                \versiculorespuestaseguido{Ave María, grátia plena, Dóminus tecum; benedicta tu in muliéribus, et benedíctus fructus ventris tui,
Jesus}{Sancta Maria, Mater Dei, ora pro nobis peccatóribus, nunc et in hora mortis nostr{\ae}. Amen.} 
            \end{rightcolumn}
    \end{otherlanguage}
\end{paracol}

\vspace{0.5em}

\begin{center}
    \textcolor{red}{El avemaría se repite 10 veces}
\end{center}
\vspace{0.2em}
\begin{paracol}{2}
    \begin{leftcolumn}
        \input{oraciones/avemaria/castellano_vr.tex}
    \end{leftcolumn}
    \begin{otherlanguage}{latin}
            \begin{rightcolumn}
                \versiculorespuestaseguido{Ave María, grátia plena, Dóminus tecum; benedicta tu in muliéribus, et benedíctus fructus ventris tui,
Jesus}{Sancta Maria, Mater Dei, ora pro nobis peccatóribus, nunc et in hora mortis nostr{\ae}. Amen.}
            \end{rightcolumn}
    \end{otherlanguage}

    \definecolumnpreamble{0}{\vspace{0.5em}}
    \definecolumnpreamble{1}{\vspace{0.5em}}

    \begin{leftcolumn*}
        \versiculorespuestaseguido{María, Madre de gracia, Madre de Misericordia}{Defendednos del enemigo y amparadnos ahora y en la hora de nuestra muerte. Amén}
    \end{leftcolumn*}
    \begin{otherlanguage}{latin}
            \begin{rightcolumn}
                \versiculorespuestaseguido{Deum, in adjutórium meum inténde}{Dómine, ad adjuvándum me festina}
            \end{rightcolumn}
    \end{otherlanguage}

    \begin{leftcolumn*}
        \textbf{Oremos por los fieles difuntos;}\\
Padre Nuestro, que estás ne los cielo\ldots\\
\textit{Dios te salve, María}\ldots\\[1mm]
\versiculorespuesta{El Señor les conceda el descanso eterno}{Y brille para ellos laa luz perpétua}\\[1mm]
\versiculorespuesta{Descansen en paz}{Amén}
    \end{leftcolumn*}
    \begin{otherlanguage}{latin}
            \begin{rightcolumn}
                \primeraletragranderoja{S}{ancte Michaël Archángele,}defénde nos in pr{\ae}lio: contra nequítian et insídias diáboli esto pr{\ae}sidium. Imperet illi Deus, 
súpplices deprecámur: tuque, Prínceps militi{\ae} c{\oe}léstis, Sátanam aliósque spíritus malignos, qui ad perditiónem animarum pervagántur in mundo,
divina virtúte in inférnum detrude. Amen.
            \end{rightcolumn}
    \end{otherlanguage}

    \definecolumnpreamble{0}{\vspace{0em}}
    \definecolumnpreamble{1}{\vspace{0em}}
\end{paracol}

\begin{center}
      \rule{25em}{0.4pt}
\end{center}

\begin{paracol}{2}
      \begin{leftcolumn}
            \noindent\subsubsection*{II La Visitación de la Virgen a su prima Santa Isabel}

            \vspace{-0.5em}

            \hfill\textcolor{red}{Lc 42-45}

            \lettrine[lines=2]{\textcolor{red}{Y}} levantó la voz con gran clamor y dijo: Bendita tu entre 
            las mujeres y bendito el fruto de tu vientre. {?`}Y de dónde a mí esto que venga la madre de mi Señor a mí? Porque he aquí que, 
            como sonó la voz de tu salutación en mi oídos, dió saltos de alborozo el niño en mi seno. Y dichosa la que creyó que tendrán
            cumplimiento las cosas que le han sido dichas de parte del Señor.
      \end{leftcolumn}
      \begin{rightcolumn}
            \noindent\subsubsection*{II La Ascensión de Nuestro Señor a los Cielos}

            \vspace{-0.5em}

            \hfill\textcolor{red}{Jn 1, 1.14}

            \lettrine[lines=2]{\textcolor{red}{E}}n el principio existía el Verbo, y el Verbo estaba en Dios, y el Verbo era Dios. Y el Verbo se hizo
            carne, y habitó entre nosotros; y contemplamos su gloria, gloria cual Unigénito procedente del Padre: lleno de gracia y de verdad
      \end{rightcolumn}
\end{paracol}

Paternóster, diez Avemarías, Gloria

\vspace{2mm}

\begin{paracol}{2}
    \filacastellanolatin{oraciones/maria_madre/castellano_seguido.tex}{oraciones/maria_madre/latin_seguido.tex}
\end{paracol}

\begin{paracol}{2}
      \begin{leftcolumn}
            \noindent\subsubsection*{III La Natividad del Señor}

            \vspace{-0.5em}

            \hfill\textcolor{red}{Mt 1, 22-25}

            \lettrine[lines=2]{\textcolor{red}{T}}odo esto ha acaecido a fin de que se cumpliese lo que dijo el Señor por el profeta: He aquí que una virgen concebirá y parirá
            un hijo, y llamarán su nombre Emmanuel, que traducido quiere decir \textit{Dios con nosotros}
      \end{leftcolumn}
      \begin{rightcolumn}
            \noindent\subsubsection*{III La Venida del Espíritu Santo sobre los Apóstoles}

            \vspace{-0.5em}

            \hfill\textcolor{red}{Rm 8, 12-16}

            \lettrine[lines=2]{\textcolor{red}{A}}sí, pues, hermanos, deudores somos nos a la carne: de vivir según la carne. Porque si según la carne vivís, habréis de morir; mas
            si con el Espíritu hacéis morir la fechorías del cuerpo viviréis. Pues cuantos son llevados por el Espíritu Santo de Dios, éstos son hijos de Dios. Porque no recibisteis
            espíritu de esclavitud para reincidir de nuevo en el temor; antes recibisteis Espíritu de filiación adoptiva, con el cual clamamos: {!`}Abba!{!`}Padre! El Espíritu mismo
            testifica a una nuestro espíritu que somos hijos de Dios.
      \end{rightcolumn}
\end{paracol}

%%%%%%%%%%%%%%%%%%%%%%%%%%%%%%%%
% ORACIONES FINALES Y LETANIAS %
%%%%%%%%%%%%%%%%%%%%%%%%%%%%%%%%
\label{final-prayer}
\begin{center}
      \textcolor{red}{Terminados los misterios podemos rezar}
\end{center}

\begin{paracol}{2}

      \begin{leftcolumn}
            \versiculorespuestaseguido{Dios te Salve, María, Hija de Dios Padre, Virgen purísima y castísima antes del parto, llena eres de gracia, el Señor es contigo; 
bendita eres entre todas las mujeres, y bendito es el fruto de tu vientre, Jesús}{Santa María, Madre de Dios, ruega por nosotros pecadores,
ahora u en el hora de nuestra muerte. Amén}
      \end{leftcolumn}
      \begin{otherlanguage}{latin}
            \begin{rightcolumn}
                  \versiculorespuestaseguido{Ave María, Fília Dei Patri, Virgo purissima et castissima ante partum, gratia plena, Dóminus tecum; benedicta tu in muliéribus, 
et benedíctus fructus ventris tui, Jesus}{Sancta Maria, Mater Dei, ora pro nobis peccatóribus, nunc et in hora mortis nostr{\ae}. Amen}  
            \end{rightcolumn}
      \end{otherlanguage}

      \definecolumnpreamble{0}{\vspace{0.5em}}
      \definecolumnpreamble{1}{\vspace{0.5em}}

      \begin{leftcolumn*}
            \textbf{Oremos por los fieles difuntos;}\\
Padre Nuestro, que estás ne los cielo\ldots\\
\textit{Dios te salve, María}\ldots\\[1mm]
\versiculorespuesta{El Señor les conceda el descanso eterno}{Y brille para ellos laa luz perpétua}\\[1mm]
\versiculorespuesta{Descansen en paz}{Amén}
      \end{leftcolumn*}
      \begin{otherlanguage}{latin}
            \begin{rightcolumn}
                  \primeraletragranderoja{S}{ancte Michaël Archángele,}defénde nos in pr{\ae}lio: contra nequítian et insídias diáboli esto pr{\ae}sidium. Imperet illi Deus, 
súpplices deprecámur: tuque, Prínceps militi{\ae} c{\oe}léstis, Sátanam aliósque spíritus malignos, qui ad perditiónem animarum pervagántur in mundo,
divina virtúte in inférnum detrude. Amen.
            \end{rightcolumn}
      \end{otherlanguage}

      \begin{leftcolumn*}
            \letraroja{D}{ios}te Salve, María, Esposa de Dios Espíritu Santo, Virgen purísima y castísima después del parto, llena eres de gracia, 
el Señor es contigo; bendita eres entre todas las mujeres, y bendito es el fruto de tu vientre, Jesús. Santa María, Madre de Dios, 
ruega por nosotros pecadores, ahora u en el hora de nuestra muerte. Amén.
      \end{leftcolumn*}
      \begin{otherlanguage}{latin}
            \begin{rightcolumn}
                  \primeraletragranderoja{S}{ancte Michaël Archángele,}defénde nos in pr{\ae}lio: contra nequítian et insídias diáboli esto pr{\ae}sidium. Imperet illi Deus, 
súpplices deprecámur: tuque, Prínceps militi{\ae} c{\oe}léstis, Sátanam aliósque spíritus malignos, qui ad perditiónem animarum pervagántur in mundo,
divina virtúte in inférnum detrude. Amen.
            \end{rightcolumn}
      \end{otherlanguage}

      \begin{leftcolumn*}
            \letraroja{S}{anta}María, templo y sagrario de la Santísima Trinidad. Gloria al Padre, al Hijo, y al Espíritu Santo. Como era en el principio, 
ahora, y siempre, y por los siglos de los siglos. Amén.
      \end{leftcolumn*}
      \begin{otherlanguage}{latin}
            \begin{rightcolumn}
                  \letraroja{S}{ancta}María, templum et sacrarium totis Sanctissim{\ae} Trinitatis. \versiculorespuestaseguido{Deum, in adjutórium meum inténde}{Dómine, ad adjuvándum me festina}
            \end{rightcolumn}
      \end{otherlanguage}

      \definecolumnpreamble{0}{\vspace{0em}}
      \definecolumnpreamble{1}{\vspace{0em}}

      \begin{leftcolumn*}[\begin{center}\subsection*{Letanias de Nuestra Señora}\end{center}]
            \primeraletragranderoja{S}{eñor,}ten piedad,\hfill Señor, ten piedad.\\
Cristo, ten piedad,\hfill Cristo, ten piedad.\\
Señor, ten piedad,\hfill Señor, ten piedad.\\
Cristo, óyenos,\hfill Cristo, óyenos.\\
Cristo, escúchanos,\hfill Cristo, escúchanos.\\
Dios Padre celestial,\hfill ten piedad de nosotros.\\
Dios Hijo Redentor del mundo,\hfill ten piedad de nosotros.\\
Dios Espíritu Santo,\hfill ten piedad de nosotros.\\
Trinidad Santa, un solo Dios,\hfill ten piedad de nosotros.\\
Santa María,\hfill ruega por nosotros.\\
Santa Madre de Dios,\hfill ruega.\\
Santa Virgen de las Vírgenes,\hfill ruega.\\
Madre de Cristo,\hfill ruega.\\
Madre de la divina gracia,\hfill ruega.\\
Madre purísima,\hfill ruega.\\
Madre castíssima,\hfill ruega.\\
Madre virignal,\hfill ruega.\\
Madre sin corrupción,\hfill ruega.\\
Madre inmaculada,\hfill ruega.\\
Madre amable,\hfill ruega.\\
Madre admirable,\hfill ruega.\\
Madre del buen Consejo,\hfill ruega.\\
Madre del Creador,\hfill ruega.\\
Madre del Salvador,\hfill ruega.\\
Virgen prudentísima,\hfill ruega.\\
Virgen digna de veneración,\hfill ruega.\\
Virgen digna de alabanza,\hfill ruega.\\
Virgen poderosa,\hfill ruega.\\
Virgen clemente,\hfill ruega.\\
Virgen fiel,\hfill ruega.\\
Espejo de justicia,\hfill ruega.\\
Sede de la sabiduría,\hfill ruega.\\
Causa de nuestra alegría,\hfill ruega.\\
Vaso espiritual,\hfill ruega.\\
Vaso honorable,\hfill ruega.\\
Vaso insigne de devoción,\hfill ruega.\\
Rosa mística,\hfill ruega.\\
Torre de David,\hfill ruega.\\
Torre de marfil,\hfill ruega.\\
Casa de oro,\hfill ruega.\\
Arca de la alianza,\hfill ruega.\\
Puerta del cielo,\hfill ruega.\\
Estrella de la mañana,\hfill ruega.\\
Salud de los enfermos,\hfill ruega.\\
Refugio de los pecadores,\hfill ruega.\\
Consuelo de la afligidos,\hfill ruega.\\
Auxilio de los cristianos,\hfill ruega por nosotros.\\
Reina de los Ángeles,\hfill ruega.\\
Reina de los Patriarcas,\hfill ruega.\\
Reina de los Profetas,\hfill ruega.\\
Reina de los Apóstoles,\hfill ruega.\\
Reina de los Mártires,\hfill ruega.\\
Reina de los Confesores,\hfill ruega.\\
Reina de las Vírgenes,\hfill ruega.\\
Reina de todos los Santos,\hfill ruega.\\
Reina concebida sin pecado original,\hfill ruega.\\
Reina asunta al cielo,\hfill ruega.\\
Reina del santísimo Rosario,\hfill ruega.\\
Reina de la paz,\hfill ruega.\\
Cordero de Dios, que quitas los pecados del mundo,

\hfill perdónanos, Señor.

\noindent Cordero de Dios, que quitas los pecados del mundo,

\hfill escúchanos, Señor.

\noindent Cordero de Dios, que quitas los pecados del mundo,

\hfill ten piedad de nosotros.

      \end{leftcolumn*}
      \begin{otherlanguage}{latin}
            \begin{rightcolumn}
                  \primeraletragranderoja{K}{ýrie,}eléison,\hfill Kýrie, eléison.\\
Christe, eléison, \hfill Christe, eléison\\
Kýrie, eléison,\hfill Kýrie, eléison.\\
Christe, audi nos,\hfill Christe, audi nos.\\
Christe, exáudi nos,\hfill Christe, exáudi nos.\\
Pater de c{\ae}lis, Deus,\hfill miserére nobis.\\
Fili, Redémptor mundi, Deus,\hfill miserére nobis.\\
Spíritus Sancte, Deus,\hfill miserére nobis.\\
Sancta Trínitas, unus Deus,\hfill miserére nobis.\\
Sancta Maria,\hfill ora pro nobis.\\
Sancta Dei Génetrix,\hfill ora.\\
Sancta Virgo vírginum,\hfill ora.\\
Mater Christi.\hfill ora.\\
Mater divín{\ae} grati{\ae},\hfill ora.\\
Mater puríssima,\hfill ora.\\
Mater castíssima,\hfill ora.\\
Mater invioláta,\hfill ora.\\
Mater intemeráta.\hfill ora.\\
Mater immaculáta,\hfill ora.\\
Mater amábilis,\hfill ora.\\
Mater admirábilis,\hfill ora.\\
Mater boni Consílii,\hfill ora.\\
Mater Creatóris,\hfill ora.\\
Mater Salvatóris,\hfill ora.\\
Virgo prudentíssima,\hfill ora.\\
Virgo veneránda,\hfill ora.\\
Virgo pr{\ae}dicánda,\hfill ora.\\
Virgo potens,\hfill ora.\\
Virgo clemens,\hfill ora.\\
Virgo fidélis,\hfill ora.\\
Spéculum iustíti{\ae},\hfill ora.\\
Sedes Sapiénti{\ae},\hfill ora.\\
Causa nostr{\ae} l{\ae}títi{\ae},\hfill ora.\\
Vas spirituále,\hfill ora.\\
Vas honorábile,\hfill ora.\\
Vas insigne devotiónis,\hfill ora.\\
Rosa mýstica,\hfill ora.\\
Turris Davídica,\hfill ora.\\
Turris ebúrnea,\hfill ora.\\
Domus áurea,\hfill ora.\\
F{\oe}deris arca,\hfill ora.\\
Iánua c{\ae}li,\hfill ora.\\
Stella matutina,\hfill ora.\\
Salus infirmórum,\hfill ora.\\
Refugium peccatórum,\hfill ora.\\
Consolátrix afflictórum,\hfill ora.\\
Auxílium christianórum,\hfill ora pro nobis.\\
Regina Angelórum,\hfill ora.\\
Regina Patriarchánum,\hfill ora.\\
Regina Prophetárum,\hfill ora.\\
Regina Apostolórum,\hfill ora.\\
Regina Mártyrum,\hfill ora.\\
Regina Confessórum,\hfill ora.\\
Regina Vírginum,\hfill ora.\\
Regina Sanctórum ómnium,\hfill ora.\\
Regina sine labe originali concépta,\hfill ora.\\
Regina in c{\ae}lum assumpta,\hfill ora.\\
Regina sacratíssimi Rosárii,\hfill ora.\\
Regina pacis,\hfill ora.\\
Ágnus Dei, qui tolli peccáta mundi,

\hfill parce nobis, Dómine.

Ágnus Dei, qui tolli peccáta mundi, 

\hfill exáudi nos, Dómine.

Ágnus Dei, qui tolli peccáta mundi, 

\hfill miserére nobis.

            \end{rightcolumn}
      \end{otherlanguage}

      \definecolumnpreamble{0}{\vspace{0.5em}}
      \definecolumnpreamble{1}{\vspace{0.5em}}

      \begin{leftcolumn*}
            \input{oraciones/bajo_tu_amparo/castellano_sencillo.tex}
      \end{leftcolumn*}
      \begin{otherlanguage}{latin}
            \begin{rightcolumn}
                  \versiculorespuestaseguido{Regina caeli l{\ae}táre, allelúja}{Quia quem meruisti portáre, allelúja}

\vspace{1mm}

\versiculorespuestaseguido{Resurréxit sicut dixit, allelúja}{Ora pro nobis Deum, allelúja}

\vspace{1mm}

\versiculorespuestaseguido{Gaude et l{\ae}táre, Virgo María, allelúja}{Quia surréxit Dóminus vere, allelúja}

\vspace{1mm}

\textbf{Orémus}.-- \textcolor{red}{D}eus, qui per resurrectiónem Filii tui Dómini nostri Jesu Christi,
mundum l{\ae}tificáre dignátus es: pr{\ae}sta, qu{\'\ae}sumus, ut per ejus Genetricem Vírginem Maríam,
perpétu{\ae} capiámus gáudia vit{\ae}. Per eúmdem Christum Dóminum nostrum. \respuesta{Amen} 

            \end{rightcolumn}
      \end{otherlanguage}
      \definecolumnpreamble{0}{\vspace{0em}}
      \definecolumnpreamble{1}{\vspace{0em}}
\end{paracol}

\vspace{0.5em}

\textcolor{red}{Desde la Purificación hasta Sábado Santo y desde la Santísima Trinidad hasta el I Domingo de Adviento:}
\vspace{0.2em}
\begin{paracol}{2}
      \begin{leftcolumn}
            \versiculorespuestaseguido{Ruega por nos, Santa Madre de Dios}{Para que seamos dignos de alcanzar la promesas Cristo. Amén}
      \end{leftcolumn}
      \begin{otherlanguage}{latin}
            \begin{rightcolumn}
                  \versiculorespuestaseguido{Ora pro nobis, Sancta Dei Génetrix}{Ut digni efficiámur promissiónibus Christi. Amen}
            \end{rightcolumn}
      \end{otherlanguage}

      \definecolumnpreamble{0}{\vspace{0.5em}}
      \definecolumnpreamble{1}{\vspace{0.5em}}

      \begin{leftcolumn*}
            \ruegapornosotrossalve\\[2mm]
\textbf{Oremos}
\primeraletragranderoja{T}{e} rogamos, Señor, que nos concedas a nosotros tus siervos, gozar de perpetua salud de alma y cuerpo, y por la gloriosa intercesión de la bienaventurada siempre Virgen
María, seamos librados de la tristeza presente y disfrutemos de la eterna alegría. Por el mismo Jesucristo Nuestro Señor, Tu Hijo, que vive contigo y reina en unidad con el mismo Espíritu
Santo, Dios, por lo siglos de los siglos.
      \end{leftcolumn*}
      \begin{otherlanguage}{latin}
            \begin{rightcolumn}
                  \orapronobissalve\\[2mm]
\textbf{Orémus}
\primeraletragranderoja{C}{oncéde} nos famulos tuos, qu{\'\ae}sumus, Dómine Deus, perpétua mentis et córporis sanitáte gaudére: et gloriósa beát{\ae}\ Marí{\ae} semper Virginis intercessióne
a pr{\ae}sénti liberári tristítia, et {\ae}térna pérfrui l{\ae}títia. Per eudem Dóminum nostrum Jesum Christium Filium tuum, qui tecum vivit et regnat im unitáte Spiritus Sancti,
Deus, per ómnia s{\'\ae}cula s{\'\ae}culórum. Amen.
            \end{rightcolumn}
      \end{otherlanguage}
      \definecolumnpreamble{0}{\vspace{0em}}
      \definecolumnpreamble{1}{\vspace{0em}}
\end{paracol}

\vspace{0.5em}

\textcolor{red}{Desde el I Domingo de Adviento hasta Navidad:}
\vspace{0.2em}
\begin{paracol}{2}
      \begin{leftcolumn}
            \versiculorespuestaseguido{El Ángel del Señor anunció a María}{Y concibió por obra y gracia del Espíritu Santo}

      \end{leftcolumn}
      \begin{otherlanguage}{latin}
            \begin{rightcolumn}
                  \versiculorespuesta{Angelus Dómini nuntiávit Marí{\ae}}{Et concépit de Spíritu Sancto}

            \end{rightcolumn}
      \end{otherlanguage}

      \definecolumnpreamble{0}{\vspace{0.5em}}
      \definecolumnpreamble{1}{\vspace{0.5em}}

      \begin{leftcolumn*}
            \textbf{Oremos}
\primeraletragranderoja{O}{h}Dios, que quisiste que tu Verbo tomase nuestra carne de las entrañas de la Santísima Virgen, al anunciarle el Ángel el misterio: concede a tus siervos que, pues
la creemos verdadera Madre de Dios, seamos ayudados anti Ti por su intercesión. Por el mismo Jesucristo Nuestro Señor, Tu Hijo, que vive contigo y reina en unidad con el mismo Espíritu
Santo, Dios, por lo siglos de los siglos.
      \end{leftcolumn*}
      \begin{otherlanguage}{latin}
            \begin{rightcolumn}
                  \versiculorespuesta{Angelus Dómini nuntiávit Marí{\ae}}{Et concépit de Spíritu Sancto}\\[2mm]
\textbf{Orémus}
\primeraletragranderoja{D}{eus}, qui de beát{\ae} Marí{\ae} Virginis útero Verbum tuum, Angelo nuntiánte; carnem suscípere voluisti; pr{\ae}ta supplicibus tuis; ut, qui vere eam Genetrícem
Dei crédimus, ejus apud te intercessiónibus adjuvémur. Per eudem Dóminum nostrum Jesum Christium Filium tuum, qui tecum vivit et regnat im unitáte Spiritus Sancti, Deus, per ómnia
s{\'\ae}cula s{\'\ae}culórum. Amen.
            \end{rightcolumn}
      \end{otherlanguage}
      \definecolumnpreamble{0}{\vspace{0em}}
      \definecolumnpreamble{1}{\vspace{0em}}
\end{paracol}

\vspace{0.5em}

\textcolor{red}{Desde Navidad hasta la Purificación:}
\vspace{0.2em}
\begin{paracol}{2}
      \begin{leftcolumn}
            \versiculorespuesta{Después del parto, oh Virgen, has permanecido intacta}{Madre de Dios, intercedes por nosostros}
      \end{leftcolumn}
      \begin{otherlanguage}{latin}
            \begin{rightcolumn}
                  \versiculorespuesta{Post partum Virgo invioláta permansísti}{Dei Génitrix, intercéde pro nobis}
            \end{rightcolumn}
      \end{otherlanguage}

      \definecolumnpreamble{0}{\vspace{0.5em}}
      \definecolumnpreamble{1}{\vspace{0.5em}}

      \begin{leftcolumn*}
            \textbf{Oremos}
\primeraletragranderoja{O}{h}Dios, que por la fecunda virginidad de la bienaventurada siempre Virgen María has concedido al género humano los bienes de la salvación eterna, haznos sentir
la intercesión de aquélla, por quien hemos merecido recibir al autor de la vida, Jesucristo, tu hijo y Señor nuestro, que vive y reina en unidad con el Espíritu Santo, Dios, por
todos los siglos de los siglos. Amén.
      \end{leftcolumn*}
      \begin{otherlanguage}{latin}
            \begin{rightcolumn}
                  \textbf{Orémus}
\primeraletragranderoja{D}{eus,}qui salútis {\ae}térn{\ae}. beát{\ae} Marí{\ae} virginitáte f{\oe}cúnda, humáno géneri pr{\ae}mia pr{\ae}stitísti; tríbue, qu{\ae}sumus, ut ipsam pro nobis
intercédere sentiámus, per quam merúsmus acutórem vit{\ae} suscípere, Dóminum nostrum Jesum Christum Filium tuum, qui tecum vivit et regnat in unitáte Spíritus Sacnti, Deus, per ómnia
s{\'\ae}cula s{\'\ae}culórum. Amen.
            \end{rightcolumn}
      \end{otherlanguage}
      \definecolumnpreamble{0}{\vspace{0em}}
      \definecolumnpreamble{1}{\vspace{0em}}
\end{paracol}

\vspace{0.5em}

\textcolor{red}{Desde Domingo de Resurrección hasta la Santísima Trinidad:}
\vspace{0.2em}
\begin{paracol}{2}
      \begin{leftcolumn*}
            \versiculorespuestaseguido{Reina del cielo, alégrate, aleluya}{Porque el que mereciste llevar en tu seno, aleluya}\\[1mm]
\versiculorespuestaseguido{Resucitó, como Él predijo, aleluya}{Rogad por nosotros a Dios, aleluya}\\[1mm]
\versiculorespuestaseguido{Alegraos y regocijaos, Virgen María, aleluya}{Porque resucitó verdaderamente el Señor, aleluya}
      \end{leftcolumn*}
      \begin{otherlanguage}{latin}
            \begin{rightcolumn}
                  \versiculorespuesta{Regina caeli l{\ae}táre, allelúja}{Quia quem meruisti portáre, allelúja}\\[1mm]
\versiculorespuesta{Resurréxit sicut dixit, allelúja}{Ora pro nobis Deum, allelúja}\\[1mm]
\versiculorespuesta{Gaude et l{\ae}táre, Virgo María, allelúja}{Quia surréxit Dóminus vere, allelúja} 
            \end{rightcolumn}
      \end{otherlanguage}

      \definecolumnpreamble{0}{\vspace{0.5em}}
      \definecolumnpreamble{1}{\vspace{0.5em}}

      \begin{leftcolumn*}
            \versiculorespuesta{Reina del cielo, alégrate, aleluya}{Porque el que mereciste llevar en tu seno, aleluya}\\[1mm]
\versiculorespuesta{Resucitó, como Él predijo, aleluya}{Rogad por nosotros a Dios, aleluya}\\[1mm]
\versiculorespuesta{Alegraos y regocijaos, Virgen María, aleluya}{Porque resucitó verdaderamente el Señor, aleluya}\\[2mm]
\textbf{Oremos}
\primeraletragranderoja{O}{h} Dios, que, por la resurrección de vuestro Hijo y Señor nuestro Jesucristo,
os habéis dignado alegrar el mundo: concedednos por medio de su divina Madre, la Virgen Santísima,
que merezcamos obtener los goces de la vida eterna. Por el mismo Cristo, Señor nuestro. Amén.
      \end{leftcolumn*}
      \begin{otherlanguage}{latin}
            \begin{rightcolumn}
                  \versiculorespuesta{Regina caeli l{\ae}táre, allelúia}{Quia quem meruisti portáre, allelúia}\\[1mm]
\versiculorespuesta{Resurréxit sicut dixit, allelúia}{Ora pro nobis Deum, allelúia}\\[1mm]
\versiculorespuesta{Gaudate el l{\ae}táre, Virgo María, allelúia}{Quia surréxit Dóminus vere, allelúia}\\[2mm]
\textbf{Orémus}
\primeraletragranderoja{D}{eus}, qui per resurrectiónem Filii tui Dómini nostri Jesu Christi,
mundum l{\ae}tificáre dignátus es: pr{\ae}sta, qu{\'\ae}sumus, ut per ejus Genetricem Vírginem Maríam,
perpétu{\ae} capiámus gáudia vit{\ae}. Per eúmdem Christum Dóminum nostrum. Amen.
            \end{rightcolumn}
      \end{otherlanguage}
      \definecolumnpreamble{0}{\vspace{0em}}
      \definecolumnpreamble{1}{\vspace{0em}}
\end{paracol}

\vspace{0.5em}

\textcolor{red}{Rezamos ahora por el Santo Padre, nuestro Obispo y los fieles difuntos:}
\vspace{0.2em}
\begin{paracol}{2}
      \begin{leftcolumn}
            \versiculorespuestaseguido{El Señor sea con vosotros}{Y con tu espíritu}
      \end{leftcolumn}
      \begin{otherlanguage}{latin}
            \begin{rightcolumn}
                  \versiculorespuesta{Dóminus vobíscum}{Et cum spíritu tuo}
            \end{rightcolumn}
      \end{otherlanguage}

      \definecolumnpreamble{0}{\vspace{0.5em}}
      \definecolumnpreamble{1}{\vspace{0.5em}}

      \begin{leftcolumn*}
            <<<<<<< HEAD
\versiculorespuestaseguido{Oremos por nuestro beatísimo Papa \textcolor{red}{N}}
    {El Señor le conserve y vivifique, y le haga feliz en la tierra, y no le entregue an las manos de sus enemigos}
=======
\textbf{Oremos por nuestro Papa \textcolor{red}{N}:}\\
Padre Nuestro, que estás ne los cielo\ldots\\
\textit{Dios te salve, María}\ldots\\
Gloria al Padre\ldots
>>>>>>> master

      \end{leftcolumn*}
      \begin{otherlanguage}{latin}
            \begin{rightcolumn}
                  \textbf{Orémus pro beatíssimo Papa nostro \textcolor{red}{N}:}\\
Pater noster, qui es in c{\ae}lis\ldots\\
\input{oraciones/angelus/salve_latin.tex}\ldots\\
Glória Patri,\ldots
            \end{rightcolumn}
      \end{otherlanguage}

      \begin{leftcolumn*}
            \input{oraciones/padrenuestro/castellano_vr.tex}
      \end{leftcolumn*}
      \begin{otherlanguage}{latin}
            \begin{rightcolumn}
                  \versiculorespuestaseguido{Ave María, grátia plena, Dóminus tecum; benedicta tu in muliéribus, et benedíctus fructus ventris tui,
Jesus}{Sancta Maria, Mater Dei, ora pro nobis peccatóribus, nunc et in hora mortis nostr{\ae}. Amen.}
            \end{rightcolumn}
      \end{otherlanguage}

      \begin{leftcolumn*}
            \input{oraciones/avemaria/castellano_vr.tex}
      \end{leftcolumn*}
      \begin{otherlanguage}{latin}
            \begin{rightcolumn}
                  \versiculorespuestaseguido{Ave María, grátia plena, Dóminus tecum; benedicta tu in muliéribus, et benedíctus fructus ventris tui,
Jesus}{Sancta Maria, Mater Dei, ora pro nobis peccatóribus, nunc et in hora mortis nostr{\ae}. Amen.}
            \end{rightcolumn}
      \end{otherlanguage}

      \begin{leftcolumn*}
            \versiculorespuestaseguido{María, Madre de gracia, Madre de Misericordia}{Defendednos del enemigo y amparadnos ahora y en la hora de nuestra muerte. Amén}
      \end{leftcolumn*}
      \begin{otherlanguage}{latin}
            \begin{rightcolumn}
                  \versiculorespuestaseguido{Deum, in adjutórium meum inténde}{Dómine, ad adjuvándum me festina}
            \end{rightcolumn}
      \end{otherlanguage}

      \begin{leftcolumn*}
            \versiculorespuestaseguido{Oremos por nuestro Obispo \textcolor{red}{N}}
    {Consérvese, Señor, en la sublimidad de vuestro nom­bre, y apaciente con vuestra for­taleza.}

      \end{leftcolumn*}
      \begin{otherlanguage}{latin}
            \begin{rightcolumn}
                  \versiculorespuestaseguido{Orémus et pro Antístite nostro \textcolor{red}{N}}
    {Stet et pascat in fortitídine tua, Dómine, in sublimitáte nóminis tui}
            \end{rightcolumn}
      \end{otherlanguage}

      \begin{leftcolumn*}
            Padre nuestro, Ave María y Gloria.
      \end{leftcolumn*}
      \begin{otherlanguage}{latin}
            \begin{rightcolumn}
                  Pater noster, Ave María et Glória.
            \end{rightcolumn}
      \end{otherlanguage}

      \begin{leftcolumn*}
            \versiculorespuestaseguido{Oremos por los fieles difuntos}
    {Concedeles el descanso eterno y brille para ellos laa luz perpétua}
      \end{leftcolumn*}
      \begin{otherlanguage}{latin}
            \begin{rightcolumn}
                  \versiculorespuestaseguido{Orémus pro fidélibus defúnctis}
    {Réquiem {\ae}térnam dona eis, Dómine, et lux perpétua lúceat eis}
            \end{rightcolumn}
      \end{otherlanguage}

      \begin{leftcolumn*}
            Padre nuestro y Ave María.
      \end{leftcolumn*}      
      \begin{otherlanguage}{latin}
            \begin{rightcolumn}
                  Pater noster et Ave María.
            \end{rightcolumn}
      \end{otherlanguage}

      \begin{leftcolumn*}
            \versiculorespuestaseguido{Descansen en paz}{Amén}
      \end{leftcolumn*}
      \begin{otherlanguage}{latin}
            \begin{rightcolumn}
                  \versiculorespuestaseguido{Requiéscant in pace}{Amen}
            \end{rightcolumn}
      \end{otherlanguage}

      \begin{leftcolumn*}
            \versiculorespuestaseguido{Escucha, Señor, mi oración}{Y llegue a ti mi clamor}
      \end{leftcolumn*}
      \begin{otherlanguage}{latin}
            \begin{rightcolumn}
                  \versiculorespuesta{Dómine, exáudi oratiónem meam}{Et clamor meus ad te véniat}
            \end{rightcolumn}
      \end{otherlanguage}
      \definecolumnpreamble{0}{\vspace{0em}}
      \definecolumnpreamble{1}{\vspace{0em}}
\end{paracol}

\vspace{0.5em}

\textcolor{red}{Credo y salve a los Sagrados Corazones de Jesús y María.}
\vspace{0.2em}
\footnotelayout{c}
\begin{paracol}{2}
      \begin{leftcolumn}
            \input{oraciones/credo/apostoles_castellano.tex}\footnote{Credo Niceno en castellano:\\ \indent\input{oraciones/credo/niceno_castellano_sencillo.tex}}
      \end{leftcolumn}
      \begin{otherlanguage}{latin}
            \begin{rightcolumn}
                  \input{oraciones/credo/apostoles_latin.tex}\footnote{Credo Niceno en latín:\\ \indent\input{oraciones/credo/niceno_latin_sencillo.tex}}
            \end{rightcolumn}
      \end{otherlanguage}

      \definecolumnpreamble{0}{\vspace{0.5em}}
      \definecolumnpreamble{1}{\vspace{0.5em}}

      \begin{leftcolumn*}
            \input{oraciones/salve/castellano_sin.tex}
      \end{leftcolumn*}
      \begin{otherlanguage}{latin}
            \begin{rightcolumn}
                  \lettrine[lines=2]{\textcolor{red}{S}}alve, Regina, Mater misericórdi{\ae}, vita, dulcédo et spes nostra, salve. Ad te clamámus, éxsules fílii Hev{\ae}.
Ad te suspirámus geméntes et flentes in hac lacrimárum valle. Éia ergo, advocáta nostra, illos tuos misericórdes óculos ad nos convérte.
Et Iesum benedíctum fructum ventris tui, nobis, post hoc exsílium, osténde. O clemens, o pia, o dulcis Virgo Maria!

            \end{rightcolumn}
      \end{otherlanguage}

      \begin{leftcolumn*}
            \versiculorespuestaseguido{Ruega por nos, Santa Madre de Dios}{Para que seamos dignos de alcanzar la promesas Cristo. Amén}
      \end{leftcolumn*}
      \begin{otherlanguage}{latin}
            \begin{rightcolumn}
                  \versiculorespuestaseguido{Ora pro nobis, Sancta Dei Génetrix}{Ut digni efficiámur promissiónibus Christi. Amen}
            \end{rightcolumn}
      \end{otherlanguage}

      \begin{leftcolumn*}
            \lettrine[lines=2]{\textcolor{red}{A}} vos, bienaventurado San José, acudimos en nuestra tribulación, y después de invocar el auxilio de vuestra Santísima Esposa, solicitamos
también confiadamente vuestro patrocinio. Por aquella Caridad que con la Inmaculada Virgen María, Madre de Dios, os tuvo unido y, por el paterno amor 
con que abrazasteis al Niño Jesús, humildemente os suplicamos volváis benignolos ojos a la herencia que con su Sangre adquirió Jesucristo, y con vuestro 
poder y auxilio socorráis nuestras necesidades.Proteged, oh providentísimo Custodio de la Sagrada Familia la escogida descendencia de Jesucristo; 
apartad de nosotros toda mancha de error y corrupción; asistidnos propicio, desde el Cielo, fortísimo libertador nuestro en esta lucha con el poder de
las tinieblas; y como en otro tiempo librasteis al Niño Jesús del inminente peligro de su vida, así, ahora, defended la Iglesia Santa de Dios de las 
asechanzas de sus enemigos y de toda adversidad, y a cada uno de nosotros protegednos con perpetuo patrocinio, para que, a ejemplo vuestro y sostenidos 
por vuestro auxilio, podamos santamente vivir, piadosamente morir, y alcanzar en el Cielo laeterna bienaventuranza. Amén.
      \end{leftcolumn*}
      \begin{otherlanguage}{latin}
            \begin{rightcolumn}
                  \primeraletragranderoja{A}{d}te beáte Joseph, in tribulatióne nostra confúgimus, atque, imploráto Spons{\ae} tu{\ae} sanctíssim{\ae} auxílio, patrocínium 
quoque tuum fidenter expóscimus. Per eam, qu{\ae}sumus, qu{\ae} te cum immaculáta Vírgine Dei Genitríce coniúnxit, Caritátem, perque patérnum, 
quo Púerum Iesum ampléxus es, amórem, súpplices deprecámur, ut ad hereditátem, quam Iesus Christus acquisívit Sánguine suo, benígnus respícias, 
ac necessitátibus nostris tua virtúte et ope succúrras. Tuére, o Custos providentíssime divín{\ae} Famíli{\ae}, Iesu Christi sóbolem eléctam; próhibe a nobis, 
amantíssime Pater, omnem errórum ac corruptelárum luem; propítius nobis, sospítator noster fortíssime, in hoc cum potestáte tenebrárum certámine e 
c{\ae}lo adésto; et sicut olim Púerum Iesum e summo eripuísti vitre discrímine, ita nunc Ecclesiam sanctam Dei ab hostílibus insídiis atque ab omni 
adversitáte défende: nosque síngulos perpétuo tege patrocínio, ut ad tui exémplar et ope tua suffúlti, sancte vívere, pie émori, sempiternámque in 
c{\ae}lis beatitúdinem ássequi possímus. Amen.
            \end{rightcolumn}
      \end{otherlanguage}

      \begin{leftcolumn*}
            \textbf{Oremos por los fieles difuntos;}\\
Padre Nuestro, que estás ne los cielo\ldots\\
\textit{Dios te salve, María}\ldots\\[1mm]
\versiculorespuesta{El Señor les conceda el descanso eterno}{Y brille para ellos laa luz perpétua}\\[1mm]
\versiculorespuesta{Descansen en paz}{Amén}
      \end{leftcolumn*}
      \begin{otherlanguage}{latin}
            \begin{rightcolumn}
                  \primeraletragranderoja{S}{ancte Michaël Archángele,}defénde nos in pr{\ae}lio: contra nequítian et insídias diáboli esto pr{\ae}sidium. Imperet illi Deus, 
súpplices deprecámur: tuque, Prínceps militi{\ae} c{\oe}léstis, Sátanam aliósque spíritus malignos, qui ad perditiónem animarum pervagántur in mundo,
divina virtúte in inférnum detrude. Amen.
            \end{rightcolumn}
      \end{otherlanguage}

      \begin{leftcolumn*}
            \versiculorespuestaseguido{Corazón sacratísimo de Jesús}{Ten misericordia de nosotros}
      \end{leftcolumn*}
      \begin{otherlanguage}{latin}
            \begin{rightcolumn}
                  \versiculorespuesta{Cor Iesus sacratíssimum}{Miserére nobis}
            \end{rightcolumn}
      \end{otherlanguage}

      \begin{leftcolumn*}
            \versiculorespuestaseguido{San José}{Ruega por nosotros}
      \end{leftcolumn*}
      \begin{otherlanguage}{latin}
            \begin{rightcolumn}
                  \versiculorespuestaseguido{Sacnte Joseph}{Ora pro nobis}
            \end{rightcolumn}
      \end{otherlanguage}

      \begin{leftcolumn*}
            \versiculorespuestaseguido{Ave María purísima}{Sin pecado concebida}
      \end{leftcolumn*}
      \begin{otherlanguage}{latin}
            \begin{rightcolumn}
                  \versiculorespuestaseguido{Ave María puríssima}{Sine labe originali concépta}
            \end{rightcolumn}
      \end{otherlanguage}

      \begin{leftcolumn*}
            En el Nombre del Padre, y del{\redcross}Hijo, y del Espíritu Santo. Amén.
      \end{leftcolumn*}
      \begin{otherlanguage}{latin}
            \begin{rightcolumn}
                  \primeraletragranderoja{S}{ancte Michaël Archángele,}defénde nos in pr{\ae}lio: contra nequítian et insídias diáboli esto pr{\ae}sidium. Imperet illi Deus, 
súpplices deprecámur: tuque, Prínceps militi{\ae} c{\oe}léstis, Sátanam aliósque spíritus malignos, qui ad perditiónem animarum pervagántur in mundo,
divina virtúte in inférnum detrude. Amen.
            \end{rightcolumn}
      \end{otherlanguage}
      \definecolumnpreamble{0}{\vspace{0em}}
      \definecolumnpreamble{1}{\vspace{0em}}
\end{paracol}
%%%%%%%%%%%%%%%%%%%%%%%%%%%%%%%%
% ORACIONES FINALES Y LETANIAS %
%%%%%%%%%%%%%%%%%%%%%%%%%%%%%%%%

%%%%%%%%%%%%%%%%%%%%%%%%%%%
% ANGELUS Y REGINA COELIS %
%%%%%%%%%%%%%%%%%%%%%%%%%%%

\footnotelayout{m}
\chapter*{Ángelus y Regina C{\oe}li\footnote{En todo tiempo del año se pueden rezar cinco Avemarías ene vez del Ángelus y del Regina C{\oe}li,
y se ganan las mismas indulgencias}}

\section*{Ángelus}
\begin{paracol}{2}
      \begin{leftcolumn}
            \versiculorespuestaseguido{El Ángel del Señor anunció a María}{Y concibió por obra y gracia del Espíritu Santo}

      \end{leftcolumn}
      \begin{otherlanguage}{latin}
            \begin{rightcolumn}
                  \versiculorespuesta{Angelus Dómini nuntiávit Marí{\ae}}{Et concépit de Spíritu Sancto}

            \end{rightcolumn}
      \end{otherlanguage}

      \definecolumnpreamble{0}{\vspace{0.5em}}
      \definecolumnpreamble{1}{\vspace{0.5em}}

      \begin{leftcolumn*}
            \input{oraciones/avemaria/castellano_vr.tex}
      \end{leftcolumn*}
      \begin{otherlanguage}{latin}
            \begin{rightcolumn}
                  \versiculorespuestaseguido{Ave María, grátia plena, Dóminus tecum; benedicta tu in muliéribus, et benedíctus fructus ventris tui,
Jesus}{Sancta Maria, Mater Dei, ora pro nobis peccatóribus, nunc et in hora mortis nostr{\ae}. Amen.}
            \end{rightcolumn}
      \end{otherlanguage}

      \begin{leftcolumn*}
            \versiculorespuestaseguido{He aquí la excalva del Señor}{Hágase en mi según tu palabra}

      \end{leftcolumn*}
      \begin{otherlanguage}{latin}
            \begin{rightcolumn}
                  \versiculorespuesta{Ecce Amcilla Dómini}{Fiat mihi secúndum verbum tuum}
            \end{rightcolumn}
      \end{otherlanguage}

      \begin{leftcolumn*}
            \textit{Dios te salve\ldots}
      \end{leftcolumn*}
      \begin{otherlanguage}{latin}
            \begin{rightcolumn}
                  \textit{Ave María\ldots}
            \end{rightcolumn}
      \end{otherlanguage}

      \begin{leftcolumn*}
            \versiculorespuestaseguido{Y el Verbo se hizo carne}{Y habitó entre nosotros}

      \end{leftcolumn*}
      \begin{otherlanguage}{latin}
            \begin{rightcolumn}
                  \versiculorespuestaseguido{Et Verbum caro factum est}{Et habitávit in nobis}
            \end{rightcolumn}
      \end{otherlanguage}

      \begin{leftcolumn*}
            \textit{Dios te salve\ldots}
      \end{leftcolumn*}
      \begin{otherlanguage}{latin}
            \begin{rightcolumn}
                  \textit{Ave María\ldots}
            \end{rightcolumn}
      \end{otherlanguage}

      \begin{leftcolumn*}
            \versiculorespuestaseguido{Ruega por nos, Santa Madre de Dios}{Para que seamos dignos de alcanzar la promesas Cristo. Amén}
      \end{leftcolumn*}
      \begin{otherlanguage}{latin}
            \begin{rightcolumn}
                  \versiculorespuestaseguido{Ora pro nobis, Sancta Dei Génetrix}{Ut digni efficiámur promissiónibus Christi. Amen}
            \end{rightcolumn}
      \end{otherlanguage}

      \begin{leftcolumn*}
            \textbf{Oremos}
\primeraletragranderoja{O}{s} rogamos, Señor, que infundáis vuestra gracia en nuestras almas para que,
habiendo conocido la Encarnación de vuestro Hijo Jesucristo por el Ángel que la anunció,
seamos llevados a la gloria de la resurrección, por los méritos de su pasión y cruz santísima.
Por el mismo Jesucristo nuestro Señor. \respuesta{Amén}
      \end{leftcolumn*}
      \begin{otherlanguage}{latin}
            \begin{rightcolumn}
                  \textbf{Orémus}
\primeraletragranderoja{G}{rátiam} tuam qu{\'\ae}sumus. Dómine, méntibus nostris infúnde: ut qui Angelo nuntiáte,
Christi Filii tui incarnatiónem cognóvimus, per passiónem ejus et crucem ad resurrectiónis glóriam perducámur.
Per eúmdem Christum Dóminum nostrum. \respuesta{Amen} 
            \end{rightcolumn}
      \end{otherlanguage}
      \definecolumnpreamble{0}{\vspace{0em}}
      \definecolumnpreamble{1}{\vspace{0em}} 
\end{paracol}

\section*{Regina C{\oe}lis (Tiempo Pascual)}
\vspace{0.2em}
\begin{paracol}{2}
      \begin{leftcolumn}
            \versiculorespuestaseguido{Reina del cielo, alégrate, aleluya}{Porque el que mereciste llevar en tu seno, aleluya}\\[1mm]
\versiculorespuestaseguido{Resucitó, como Él predijo, aleluya}{Rogad por nosotros a Dios, aleluya}\\[1mm]
\versiculorespuestaseguido{Alegraos y regocijaos, Virgen María, aleluya}{Porque resucitó verdaderamente el Señor, aleluya}
      \end{leftcolumn}
      \begin{otherlanguage}{latin}
            \begin{rightcolumn}
                  \versiculorespuesta{Regina caeli l{\ae}táre, allelúja}{Quia quem meruisti portáre, allelúja}\\[1mm]
\versiculorespuesta{Resurréxit sicut dixit, allelúja}{Ora pro nobis Deum, allelúja}\\[1mm]
\versiculorespuesta{Gaude et l{\ae}táre, Virgo María, allelúja}{Quia surréxit Dóminus vere, allelúja} 
            \end{rightcolumn}
      \end{otherlanguage}

      \definecolumnpreamble{0}{\vspace{0.5em}}
      \definecolumnpreamble{1}{\vspace{0.5em}}

      \begin{leftcolumn*}
            \textbf{Oremos}\\
Oh Dios, que, por la resurreción de vuestro Hijo y Señor nuestro Jesucristo, os habéis dignado alegrar el mundo: 
concedednos por medio de su divina Madre, la Virgen Santísima, que merezcamos obtener los goces de la vida eterna. 
Por el mismo Cristo, Señor nuestro.\\[2mm]
\respuesta{Amén}
      \end{leftcolumn*}
      \begin{otherlanguage}{latin}
            \begin{rightcolumn}
                  \textbf{Orémus}\\
Deus, qui per resurrectiónem Filii tui Dómini nostri Jesu Christi,
mundum l{\ae}tificáre dignátus es: pr{\ae}sta, qu{\'\ae}sumus, ut per ejus Genetricem Vírginem Maríam,
perpétu{\ae} capiámus gáudia vit{\ae}. Per eúmdem Christum Dóminum nostrum.\\[2mm]
\respuesta{Amen}   
            \end{rightcolumn}
      \end{otherlanguage}
      \definecolumnpreamble{0}{\vspace{0em}}
      \definecolumnpreamble{1}{\vspace{0em}} 
\end{paracol}
%%%%%%%%%%%%%%%%%%%%%%%%%%%
% ANGELUS Y REGINA COELIS %
%%%%%%%%%%%%%%%%%%%%%%%%%%%

%%%%%%%%%%%%%%%%%%%%%%%%%%
% FUENTES Y BIBLIOGRAFÍA %
%%%%%%%%%%%%%%%%%%%%%%%%%%

\chapter*{Bibliografía}

\setlength{\parindent}{0ex}
\textsc{\textbf{Sagrada Biblia}}. Versión Crítica sobre los textos hebreo y griego. P. José María Bover, S.I. y Francisco Cantera Burgos. Biblioteca de Autores Cristianos, Madrid, 1947.

\textsc{\textbf{Breviarium Romanum}}. Typis Societatis S. Joannis Evangelisti\ae. Desclée \& Soch, 1948

\textsc{\textbf{Devocionario Completo, 19 edición}}. P. Remigio Vilariño, S.I. El Mensajero del Corazón de Jesús, Bilbao, 1956.

\textsc{\textbf{Misal Diario}}. Fraternidad Sacerdotal San Pío X. Angelus Press, Kansas City (Missouri, EEUU), 2013.

\textsc{\textbf{Manual del Seminarista, 3 edición}}. P. Daniel Sola. S.I. Apostolado de la Prensa, Madrid, 1921.

\textsc{\textbf{Catecismo de la Iglesia Católica. Compendio}}. Librería Editrice Vaticana y Asociación de Editores del Catecismo. España, 2005.

\textsc{\textbf{Misal Diario Latino-Español y Devocionario, 19 edición}}. P. Luis Ribera CMF. Editorial Regina. Barcelona, 1962.

\end{document}