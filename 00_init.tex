\documentclass[./rosary.tex]{subfiles}

\begin{document}
\chapter*{Rosario\footnote{Los que lleven consigo el Rosario debidamente bendecido y lo besaren devotamente diciento las palabras:
        Dios de salve, María, llena eres de gracia, el Señor es contigo; bendita eres entre todas las mujeres,
        y bendito es el fruto de tu vientre, Jesús. (Ave María, grátia plena, Dóminus tecum; benedicta tu in muliéribus, et benedíctus fructus ventris tui,
        Iesus.) Su Santidad el Papa Pío XII concede perpetuamente 500 días de indulgencia una vez el día}}

\label{crossSignal}
\begin{longtable} { p{0.5\textwidth} p{0.5\textwidth} }
    \textbf{POR LA SEÑAL DE LA SANTA CRUZ}, de nuestros enemigos libranos Señor, Dios nuestro. En el Nombre del Padre,
    y del Hijo, y del Espíritu Santo. Amén.

     &

    \textbf{PER SIGNUM CRUCIS} de inimícis nostris libera nos, Deus noster. In Nómine Pátris, et Filii, et Spíritus Sancti. Amen.
\end{longtable}

\label{iConfess}
\begin{longtable} { p{0.5\textwidth} p{0.5\textwidth} }
    \textbf{YO PECADOR} me confieso a Dios todopoderoso, a la bienaventurada siempre Virgen María, al bienaventurado San Miguel Arcángel,
    al bienaventurado San Juan Bautista, a los Santos Apóstoles Pedro y Pablo, a todos los Santos y a vos, Padre, que pequé mucho
    de pensamiento, palabra y obra: por mi culpa, por mi culpa, por mi grandísima culpa. Por tanto, ruego a la bienaventurada
    siempre Virgen María, al bienaventurado San Miguel Arcángel, al bienaventurado San Juan Bautista, a los Santos Apóstoles
    Pedro y Pablo, a todos los Santos, y a vos, Padre, que roguéis por mi a Dios Nuestro Señor. Amén.
     &
    \textbf{CONFÍTEOR} Deo omnipoténti, beátæ Maríæ semper Virigini, beáto Michaéli Archángelo, beáto Joánni Baptístæ, sanctis Apóstolis Petro et Paulo,
    ómníbus Sanctis (et tibi, pater), quia peccávi nimis, cogitatióne, verbo et ópere, mea culpa, mea culpa, mea máxima culpa. Ideo precor beátam
    Maríam semper Virgínem, beátum Michaélem Archángelum, beátum Joánnem Baptístam, sanctis Apóstolos Petrum et Paulum, omnes Sanctos (et te, pater),
    oráre pro me ad Dóminum Deum nostrum. Amen.
\end{longtable}

\label{contrition}
\textbf{SEÑOR MÍO JESUCRISTO}, Dios y Hombre verdadero, Creador y Redentor mío: por ser vos quién sois, y porque os amo sobre todas las cosas,
me pesa de todo corazón de haberos ofendido, propongo firmemente nunca más pecar, y apartarme de todas las ocasiones de ofenderos,
confesarme, y cumplir la penitencia que me fuere impuesta; ofrézcoos mi vida, obras y trabajos en satisfacción de todos mis pecados;
y confío en vuestra bondad y misericordia infinita me los perdonaréis por los merecimientos de vuestra preciosísima sangre, pasión y muerte,
y me daréis gracia para enmendarme y para perseverar en vuestro santo servicio hasta el fin de mi vida. Amén.

\label{creed}
\begin{longtable} { p{0.5\textwidth} p{0.5\textwidth} }
    \textbf{CREO} en Dios, Padre todopoderoso. Creador del cielo y de la tierra. Y en Jesucristo, su único Hijo, Nuestro Señor,
    que fue concebido por obra y gracia del Espíritu Santo; nació de Santa María Vírgen; padeció bajo el poder de Poncio Pilato,
    fue crucificado, muerto y sepultado; descendió a los infiernos; al tercer día resucitó de entre los muertos; subió a los cielos,
    está sentado a la derecha de Dios Padre todopoderoso; desde allí ha de venir a juzgar a vivos y muertos.
    Creo en el Espíritu Santo, la Santa Iglesia Católica, la comunión de los Santos, el perdón de los pecados,
    la resurección de la carne y la vida eterna. Amén.
     &
    \textbf{CREDO} in Deum, Patrem omnipoténtem. Creatórem cœli et terræ. Et in Jesum Christum, Filium ejus únicum, Dóminum nostrum;
    qui concéptus est de Spíritu Sancto; natus ex María Virgine; passus sub Póntio Pilato, crucifíxus, mortuus et sepúltus:
    descéndit ad inferos; tértia die resurréxit a mórtuis: ascéndit ad cœlos, sedet ad dexteram Dei Patris omnipoténtis;
    inde ventúrus est judicáre vivos et mórtuos. Credo in Spíritum Sanctum, sanctam Ecclésiam cathólicam, Sanctórum communiónem,
    remissiónem peccatórum, carnis resurrectiónem, vitam ætérnam. Amen.
\end{longtable}

\begin{longtable} {p{0.45\textwidth} p{0.45\textwidth} }
    V. Abre, Señor, mis labios                                                           & V. Dómine, lábia mea apéries                                                \\
    R. Y mi boca cantará tus alabanzas                                                   & R. Et os meum annuntiábit laudem tuam                                       \\
    V. Apresútare, Señor, a socorrerme                                                   & V. Deum, in adjutórium meum inténde                                         \\
    R. Ven, oh Dios, en mi ayuda                                                         & R. Dómine, ad adjuvándum me festina                                         \\
    V. Gloria al Padre, al Hijo, y al Espíritu Santo.\label{sec:glory}                   & V. Gloria Patri, et Filio, et Spíritui Sancto.                              \\
    R. Como era en el principio, ahora, y siempre, y por los siglos de los siglos. Amén. & R. Sicut erat in princípio et nunc, et semper et in sæcula sæculórum, Amen.
\end{longtable}

\begin{center}
    \label{endTenPrayers}
    \textbf{MARÍA}, Madre de gracia, Madre de Misericordia. Defendednos del enemigo y amparadnos ahora y en la hora de nuestra muerte. Amén.

    \textbf{¡OH JESÚS MÍO!} Perdonadnos. Libradnos del fuego enterno del infierno. Llevad al Cielo a todas las almas, y socorred especialmente
    a las más necesitadas
\end{center}

\label{sec:ourFather}
\begin{longtable} { p{0.5\textwidth} p{0.5\textwidth} }
    \textbf{PADRE NUESTRO} que estás en los cielos, santificado sea tu Nombre. Venga a nosotros tu Reino.
    Hágase tu voluntad, así en la tierra como en el cielo. El pan nuestro de cada día dánosle hoy.
    Y perdónamos nuestras deuda, así como nosotros perdonamos a nuestros deudores.
    Y no nos dejes caer en la tentación: mas líbranos del mal. Amén.

     &

    \textbf{PATER NOSTER}, qui es in cælis, sanctificétur nomen tuum. Advéniat regnum tuum.
    Fiat volúntas tua, sicut in cœlo et in terra. Panem nóstrum quotidiánum da nobis hódie.
    Et dimitte nobis debita nostra, sicut et nos dimittimus debitóribus nostris.
    Et ne nos indúcas in tentatiónem: sed libera nos a malo. Amen.
\end{longtable}

\label{sec:hailMary}
\begin{longtable} { p{0.5\textwidth} p{0.5\textwidth} }
    \textbf{DIOS TE SALVE}, María, llena eres de gracia, el Señor es contigo; bendita eres entre todas las mujeres,
    y bendito es el fruto de tu vientre, Jesús. Santa María, Madre de Dios, ruega por nosotros pecadores,
    ahora u en el hora de nuestra muerte. Amén.

     &

    \textbf{AVE MARIA}, grátia plena, Dóminus tecum; benedicta tu in muliéribus, et benedíctus fructus ventris tui,
    Iesus. Sancta Maria, Mater Dei, ora pro nobis peccatóribus, nunc et in hora mortis nostræ. Amen.
\end{longtable}

\begin{flushright}
    (3 veces)
\end{flushright}


\begin{longtable} { p{0.5\textwidth} p{0.5\textwidth} }
    V. Gloria al Padre{\ldots} & V. Gloria Patri{\ldots} \\
    R. Como era{\ldots}        & R. Sicut erat{\ldots}   \\
\end{longtable}
\end{document}