\documentclass[a4paper,11pt, oneside]{report}

\usepackage[utf8]{inputenc}
\usepackage[spanish]{babel}
\usepackage[T1]{fontenc}
\usepackage{tocloft}

\title{Santo Rosario}
\author{Sergio Rojo Herrero}
\date{Junio 2019}

\begin{document}
  
  \begin{titlepage}
    \maketitle    
  \end{titlepage}

  \newpage

  \section*{La Señal de la Cruz}
    En el Nombre del Padre, y del Hijo, y del Espíritu Santo. Amén.

    \medskip

    \textit{In Nómine Pátris, et Filii, et Spíritus Sancti. Amen.}

  \section*{Oración Dominical (Padrenuestro)}
    
    Padre Nuestro que estás en los cielos, santificado sea tu Nombre. Venga a nosotros tu Reino. Hágase tu voluntad, así en la tierra como
    en el cielo. El pan nuestro de cada día dánosle hoy. Y perdónamos nuestras deuda, así como nosotros perdonamos a nuestros deudores.
    Y no nos dejes caer en la tentación: mas líbranos del mal. Amén.
    
    \medskip

    \textit{Pater noster, qui es in c{\oe}lis, sanctifificétur nomen tuum. Advéniat regnum tuum. Fiat voluntas tua, sicut in c{\oe}lo et in terra.
    Panem nóstrum quotidiánum da nobis hódie. Et dimite nosbis debita nostra, sicut et nos dimittimus debitóribus nostris. Et ne nos indúcas
    in tentatiónem: sed libera nos a malo. Amen}.

  \section*{Salutación Evangélica (Avemaría)}
    Dios de salve, María, llena eres de gracia, el Señor es contigo; dendita eres entre todas las mujeres, y bendito es el fruto de tu
    vientre, Jeeús. Santa María, Madre de Dios, ruega por nosotros pecadores, ahora u en el hora de nuestra muerte. Amén.
    
    \medskip

    \textit{Ave María, grátia plena, Dóminus tecum; benedicta tu in muliéribus, et benedictum fructus ventris tui, Jesus.
    Sancta Maria, Mater Dei, ora pro nobis peccatóribus, nunc et in hora mortis nostr{\ae}. Amen.}

  \section*{Símbolo de los Apóstoles (Credo Apostólico)}

    Creo en Dios, Padre todopoderoso. Creador del cielo y de la tierra. Y en Jesucristo, su único Hijo, Nuestro Señor, que fue concebido por
    obra y gracia del Espíritu Santo; nació de Santa María Vírgen; padeció bajo el poder de Poncio Pilato, fue crucificado, muerto y sepultado;
    descendió a los infiernos; al tercer día resucitó de entre los muertos; subió a los cielos, está sentado a la derecha de Dios Padre todopoderoso;
    desde allí ha de venir a juzgar a vivos y muertos. Creo en el Espíritu Santo, la Santa Iglesia Católica, la comunión de los Santos, el perdón
    de los pecados, la resurección de la carne y la vida eterna. Amén.

    \medskip

    \textit{Credo in Deum, Patrem omnipoténtem. Creatórem c{\oe}li et terr{\ae}. Et in Jesum CHristum, Filium ejus únicum, Dóminum nostrum; qui concéptus
    est de Spíritu Sancto; natus ex María Virgine; passus sub Póntio Pilato, crucifíxus, mortuus et sepúltus: descéndit ad inferos; tértia die resurréxit
    a mórtuis: ascéndit ad c{\oe}los, sedet ad dexteram Dei Patris omnipoténtis; inde ventúrus est judicáre vivos et mórtuos. Credo in Spíritum Sanctum,
    sanctam Ecclésiam cathólicam, Sanctórum communiónem, remisiónem peccatórum, carnis resurrectiónem, vitam {\ae}térnam. Amen.}

  \section*{Gloria Patri}

    V. Gloria al Padre, al Hijo, y al Espíritu Santo.\\
    \indent R. Como era en el principio, ahora, y siempre, y por los siglos de los siglos. Amén.

    \medskip

    \textit{V. Gloria Patri, et Filio, et Spíritui Sancto.}\\
    \indent \textit{R. Sicut erat in pcincípio et nunc, et semper et in s{\ae}cula s{\ae}culórum, Amen.}

  \section*{Confíteor}

    Yo pecador me confieso a Dios todopoderoso, a la bienaventurada siempre Virgen María, al bienaventurado San Miguel Arcángel,
    al bienaventurado San Juan Bautista, a los Santos Apóstoles Pedro y Pablo, y a todos los Santos, porque pequé gravemente de 
    pensamiento, palabra y obra: por mi culpa, por mi culpa, por mi grandísima culpa. Por tanto, ruego a la bienaventurada siempre
    Virgen María, al bienaventurado San Miguel Arcángel, al bienaventurado San Juan Bautista, a los Santos Apóstoles Pedro y Pablo,
    y a todos los Santos, que roguéis por mi a Dios Nuestro Señor.\par\smallbreak{}
    R. El Señor omnipotente tenga piedad de nosotros y, perdonados nuestros pecados, nos lleve a la vida eterna. Amén.\par\smallbreak{}
    V. El Señor omnipotente y misericordioso nos conceda la indulgencia, la absolución y el perdón de nuestros pecados. Amén.

    \medskip

    \textit{Confíteor Deo omnipoténti, beát\ae Marí\ae Virigini, beáto Michaéli Archángelo, beáto Joánni Baptíst\ae, sanctis Apóstolis
    Petro et Paulo, et omníbus Sanctis, quia peccávi nimis, cogitatióne et ópere, mea culpa, mea culpa, mea máxima culpa. Ideo precor
    beátam Maríam semper Virgínem, beátum Michaélem Archángelum, beátum Joánnem Bastístam, sanctis Apóstolos Petrum et Paolum, et omnes
    Sanctos, oráre pro me ad Dóminum Deum nostrum.}\par\smallbreak{}
    \textit{R. Misereátur nostri omnipotens Deus, et dimissis pecátis nostris, perdúcat nos ad vitam {\ae}térnam. Amen}\par\smallbreak{}
    \textit{V. Indulgéntiam, absolutiónem, et remissiónem pecatórum nostrórum, tribuat nobis omnipotens et miséricors Dóminus. Amen.}
    


  \section*{Misterios Gozosos (Lunes y Jueves)}
    
    \subsection*{La Anunciación de la Santísima Virgen María}
      Y habiendo entrado a ella, dijo: «Dios te salve, llena de gracia, el Señor es contigo, bendita tu entre las mujeres». Ella, al oír estas palabras, se turbó,
      y discurría que podría ser esta salucatión. Y le dijo el ángel: «No temas, María, pues hallaste gracia a los ojos de Dios. He aquí que concebirás en tu seno
      y darás a luz un Hijo, a quien daras por nombre Jesús. Este será grande, y será llamado Hijo del Altísimo, y le dará el Señor Dios el trono de DAvid su padre,
      y reinará sobre la casa de Jacob etérnamete, y su reinado no tendrá fin». Lucas 1, 28-33.
      
      \medskip
      1 Paternoster, 10 Avemarías y 1 Gloria
      
      \medskip
      ¡Oh Jesús mío! Perdonadnos. Libradnos del fuego enterno del infierno. Llevad al Cielo a todas las almas, y socorred especialmente a las más 
      necesitadas

    \subsection*{La Visitación de Nuestra Señora}
      Y aconteció que, al oir Isabel la salutación de María, dió saltos de gozo el niño en su seno, y fue llena Isabel del Espíritu Santo, y levantó la voz con gran
      clamor y dijo: «Bendita tu entre las mujeres y bendito el fruto de tu vientre. ¿Y de dónde a mí esto que venga la madre de mi Señor a mí?». 
      Lucas 1, 41-43.
      
      \medskip
      1 Paternoster, 10 Avemarías y 1 Gloria
      
      \medskip
      ¡Oh Jesús mío! Perdonadnos. Libradnos del fuego enterno del infierno. Llevad al Cielo a todas las almas, y socorred especialmente a las más 
      necesitadas
                    
    \subsection*{La Natividad de Nuestro Señor Jesucristo}
      Y dió a luz su hijo primogénito, y le envolvió en pañales y le recostó en un pesebre, pues no había para ellos lugar en el mesón.
      Y les dijo el Ángel: «No tenáis, pues he aquí que os traigo una buena nueva, que será de grande alegría para todo el pueblo: que os ha nacido hoy en la ciudad de DAvid
      un Salvador, que es el Mesías, el Señor». Lucas 2, 7.10-11.
      
      \medskip
      1 Paternoster, 10 Avemarías y 1 Gloria
      
      \medskip
      ¡Oh Jesús mío! Perdonadnos. Libradnos del fuego enterno del infierno. Llevad al Cielo a todas las almas, y socorred especialmente a las más 
      necesitadas
  
    \subsection*{La Presentación del Niño Jesús en el Templo}
      Y cuando se les cunplieron los días de la purificación según la ley de Moisés\footnote{Levítico 12, 6}, le subieron a Jerusalen para presentarle al Señor, según está escrito
      en la Ley del Señor que «todo primogénito del sexo masculino será consagrado al Señor\footnote{Éxodo 13, 2; 12, 15}», y para ofrecer como sacrificio, según lo que 
      se ordena en la Ley del Señor, «un par de tórtolas o dos palominos\footnote{Levítico 12, 8; 5, 11}». Lucas 2, 22-24.
      
      \medskip
      1 Paternoster, 10 Avemarías y 1 Gloria
      
      \medskip
      ¡Oh Jesús mío! Perdonadnos. Libradnos del fuego enterno del infierno. Llevad al Cielo a todas las almas, y socorred especialmente a las más 
      necesitadas
            
    \subsection*{La Pérdida y Hallazgo del Niño Jesús en el Templo}
      Y no hallándole, se tornaron a Jerusalén para burcarle. Y sucedió que después de tres días le hallaron en el templo, sentado en medio de los maestros,
      escuchándolos y haciéndoles preguntas; y se pasmaban todos los que le oían de su inteligencia y de sus respuestas. Y sus padres padres, al verle, quedaron
      sorprendidos; y le dijo su madre: «Hijo, ¿por qué lo niciste así con nosotros? Mira que tu padre y yo, llenos de aflicción, te andábamos buscando»
      
      \medskip
      1 Paternoster, 10 Avemarías y 1 Gloria
      
      \medskip
      ¡Oh Jesús mío! Perdonadnos. Libradnos del fuego enterno del infierno. Llevad al Cielo a todas las almas, y socorred especialmente a las más 
      necesitadas

  \section*{Misterios Dolorosos (Martes y Viernes)}
    
    \subsection*{La Agonía de Nuestro Señor en el Huerto de los Olivos}
      Y saliendo de allí, se dirigió, según costrumbre, al monte de los Olivos; y le siguieron también los discípulos.
      Y Él, arrancándose de ellos, se apartó a la distancia como de un tiro de piedra, y puesto de rodillas oraba. 
      Y venido en agonía, oraba más intensamente. Y se hizo su sudor como grumos de sangre, que caían hasta el suelo
      Estando Él hablando todavía, he aquí una turba, y el que se llamaba Judas, uno de los Doce, iba delante de ellos. Y se llegó a Jesús para besarle.
      Mas Jesús le dijo: «¡Judas! ¿Con un beso entregas al Hijo del hombre?». Lucas 22, 39. 41. 44. 47-48.

      \medskip
      1 Paternoster, 10 Avemarías y 1 Gloria
      
      \medskip
      ¡Oh Jesús mío! Perdonadnos. Libradnos del fuego enterno del infierno. Llevad al Cielo a todas las almas, y socorred especialmente a las más 
      necesitadas
    
    \subsection*{La flagelación de Nuestro Señor Jesucristo}
      Entonces, pues, tomó Pilato a Jesús y le azotó. Juan 19, 1.
      
      \medskip
      Fue despreciado y abandonado de los hombres, varón de dolores y familiarizado con el sufrimiento, y como uno ante el cual
      se oculta el rostro, le despreciamos y no le estimamos. Isaías 53, 5.
      
      \medskip
      1 Paternoster, 10 Avemarías y 1 Gloria
      
      \medskip
      ¡Oh Jesús mío! Perdonadnos. Libradnos del fuego enterno del infierno. Llevad al Cielo a todas las almas, y socorred especialmente a las más 
      necesitadas
    
    \subsection*{La coronación de espinas de Nuestro Señor Jesucristo}

      Y habiéndole quitado sus vestidos, le envolvieron en una clámide de grana, y trenzando una corona de espinas, la pusieron sobre su cabeza, y una
      caña en su mano derecha; y doblando la rodilla delante de Él, le mofaban, diciendo: «Salud, Rey de los judíos». Mateo 27, 28-31.
      
      \medskip
      1 Paternoster, 10 Avemarías y 1 Gloria
      
      \medskip
      ¡Oh Jesús mío! Perdonadnos. Libradnos del fuego enterno del infierno. Llevad al Cielo a todas las almas, y socorred especialmente a las más 
      necesitadas
  
    \subsection*{Jesús con la Cruz a cuestas}
      Gritaron, pues, ellos: «Quita, quita, crucifícale». Díceles Pilato: «¿A vuestro rey voy he de crucificar?». Respondieron los pontífices: «No tenemos rey,
      sino César». Entonces, pues, se le entregó para que fuera crucificando. Se apoderaron, pues, de Jesús, y llevándo a cuestas su cruz, salió hacia el lugar
      llamado el Cráneo, que en hebreo se dice Gólgota\ldots Juan 19, 15-17.

      \medskip
      \noindent 1 Paternoster, 10 Avemarías y 1 Gloria
      
      \medskip
      ¡Oh Jesús mío! Perdonadnos. Libradnos del fuego enterno del infierno. Llevad al Cielo a todas las almas, y socorred especialmente a las más 
      necesitadas
    
    \subsection*{La Crucifixión y Muerte del Redentor}
      {\ldots}en donde le crucificaron, y con Él otros dos, a una mano y a otra, y en medio Jesús. Juan 19, 18-19.

      \medskip
      Y era ya como la hora sexta, y se produjeron tinieblas sobre toda la tierra hasta la hora nona, habiendo faltado el sol; y se rasgó por medio 
      el velo del santuario. Y clamando con voz poderosa, Jesús dijo: «Padre, en tus manos encomiendo mi espíritu\footnote{Sal. 30, 6}». 
      Y, dicho esto, expiró. Lucas 23, 44-46.
      
      \medskip
      1 Paternoster, 10 Avemarías y 1 Gloria
      
      \medskip
      ¡Oh Jesús mío! Perdonadnos. Libradnos del fuego enterno del infierno. Llevad al Cielo a todas las almas, y socorred especialmente a las más 
      necesitadas
      
  \section*{Miserios Gloriosos (Miércoles, Sábados y Domingos)}
    \subsection*{La Resurección del Señor}
      Y muy de madrugada, el primer día de la semana vienen al monumento, salido ya el sol. Y mirando atentamente, observan que la losa había
      sido corrida a un lado; porque era enormemente grande. Y entrando en el monumento, vieron un joven sentado a la derecha, vestido de un largo
      ropaje blanco, y quedaron espantadas. El les dice: «No os espantéis. A Jesús buscáis el Nazareno, el crucificado; resucitó, no está aquí. Mirad
      el lugar donde le pusieron. Pero id, decid a sus discípulos, y a Pedro, que va delante de vosotros a Galilea; allí le veŕéis, conforme os dijo».
      Marcos 16, 2.4-7.

      \medskip
      Y eran María Magdalena, y Juana, y María la de Santiago; y las demás que iban con ellas dijeron esto mismo a los apóstoles. 
      Lucas 24, 10.

      \medskip
      1 Paternoster, 10 Avemarías y 1 Gloria
      
      \medskip
      ¡Oh Jesús mío! Perdonadnos. Libradnos del fuego enterno del infierno. Llevad al Cielo a todas las almas, y socorred especialmente a las más 
      necesitadas
    
    \subsection*{La Ascensión de Jesucristo a los cielos}
      Y los sacó afuera hasta llegar junto a Betania, y alzando sus manos los bendijo. Y aconteció que, mientras los bendecía, se desprendió de ellos,
      y era llevado en alto al cielo. Y ellos, habiéndole adorado, se tornaron a Jerusalén con grande gozo, y estaban continuamente en el templo
      adorando a Dios. Lucas 24, 50-52.

      \medskip
      1 Paternoster, 10 Avemarías y 1 Gloria
      
      \medskip
      ¡Oh Jesús mío! Perdonadnos. Libradnos del fuego enterno del infierno. Llevad al Cielo a todas las almas, y socorred especialmente a las más 
      necesitadas
      
    \subsection*{La Venida del Espíritu Santo sobre los Apóstoles}
      Y al cumplirse el día de Pentecostés, estaban todos juntos en el mismo lugar. Y se produjo de súbito desde el cielo un estruendo como de viento
      que soplaba vehementemente, y llenó toda la casa donde se hallaban sentados. Y vieron aparecer lenguas como de fuego, que, repartiéndose, se 
      posaban sobre cada uno de ellos. Hechos 2, 1-4.

      \medskip
      1 Paternoster, 10 Avemarías y 1 Gloria
      
      \medskip
      ¡Oh Jesús mío! Perdonadnos. Libradnos del fuego enterno del infierno. Llevad al Cielo a todas las almas, y socorred especialmente a las más 
      necesitadas

    \subsection*{La Asunción de Nuestra Señora a los cielos}
      Antes de los siglos, desde el principio me creó y hasta la eternidad no cesaré. En la tienda santa, ante Él he ejercido ministerio
      He arraigado en pueblo ilustre, en la porción del Señor, heredad suya. Yo soy la madre de la hermosa dilección, y del temor, y del
      conocimiento y santa esperanza. Eclesiástico 24, 14.16.24.

      \medskip
      1 Paternoster, 10 Avemarías y 1 Gloria
      
      \medskip
      ¡Oh Jesús mío! Perdonadnos. Libradnos del fuego enterno del infierno. Llevad al Cielo a todas las almas, y socorred especialmente a las más 
      necesitadas

    \subsection*{La Coronación de la Santísima Virgen María}
      Y una gran señal fué vista en el cielo: una Mujer vestida del sol, y la luna debajo de sus pies, y sobre su cabeza una corona de doce estrellas.
      Apocalipsis 12, 1.

      \medskip
      Doy con todas estas cosas los bienes eternos a mis hijos y a los por Él designados. Quién me obedece no se avergonzará y los que obran por
      no pecarán. Los que me esclarecen tendrán vida eterna. Eclesiástico 24, 25.30-31.

      \medskip
      1 Paternoster, 10 Avemarías y 1 Gloria
      
      \medskip
      ¡Oh Jesús mío! Perdonadnos. Libradnos del fuego enterno del infierno. Llevad al Cielo a todas las almas, y socorred especialmente a las más 
      necesitadas
      
\end{document}
