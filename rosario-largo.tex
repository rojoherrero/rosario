\documentclass[a4paper,11pt]{article}

\usepackage[utf8]{inputenc}
\usepackage[spanish]{babel}
\usepackage[T1]{fontenc}

\begin{document}
  \section*{\hfil Misterios Gozosos \hfil}
    
    \subsection*{\hfil La Anunciación de la Santísima Virgen María \hfil}

      \begin{enumerate}
        %PRIMERO
        \item En el sexto mes fué enviado el Ángel Gabriel de parte de Dios a una ciudad de Galilea, llamada Nazaret, a una 
        doncella desposada con un varón llamada José, de la familia de David, y el nombre de la doncella era María. (Lc. 1,26- 27)

        %SEGUNDO
        \item Y habiendo entrado a ella, dijo: »Dios te salve, llena de gracia, el Señor es contigo, bendita tú entre las mujeres 
        y levanto la voz con gran clamor y dijo: »Bendita tú entre las mujeres y bendito el fruto de tu vientre». (Lc. 1, 28, 42)

        %TERCERO
        \item Ella, al oír estas palabras, se turbó, y discurría qué podía ser esta salutación. (Lc. 1, 29)

        %CUARTO
        \item Y le dijo el ángel: «No temas, María, pues hallaste gracia a los ojos de Dios». (Lc. 1, 30)

        %QUINTO
        \item «He aquí que concebirás en tu seno y darás a luz un Hijo, a quién darás por nombre Jesús». (Lc. 1, 31)

        %SEXTO
        \item «Este será grande, y será llamado Hijo del Altísimo, y le dará el Señor Dios el trono de David su padre, 
        y reinará sobre la casa de Jacob etérnamente, y su reinado no tendrá fin». (Lc. 1, 32-33)

        %SEPTIMO
        \item Y respondiendo el ángel, le dijo: «el Espíritu Santo descenderá sobre ti, 
        y el poder del Altísimo te cobijará con su sombra»; (Lc. 1, 35)
        
        %OCTAVO
        \item En el sexto mes fué enviado el Ángel Gabriel de parte de Dios a una ciudad de Galilea, llamada Nazaret, a una 
        doncella desposada con un varón llamada José, de la familia de David, y el nombre de la doncella era María. (Lc. 1,26- 27)

        %NOVENO
        \item «por lo cual también lo que nacerá será llamado santo, Hijo de Dios». (Lc. 1, 35)

        %DECIMO
        \item María dijo: «he aquí la esclava del Señor; hágase en mí según tu palabra». (Lc. 1, 38)

      \end{enumerate}
            
    \subsection*{\hfil La Visitación de Nuestra Señora \hfil}
      
      \begin{enumerate}
        
        %PRIMERO
        \item Por aquellos días, levantándose María, se dirigió presurosa a la montaña, a un ciudad de Judá, y entró en la casa
        de Zacarías y saludó a Isabel. (Lc. 1, 39-40)
        
        %SEGUNDO
        \item Y aconteció que, al oír Isabel la salutación de María. dió saltos de gozo el niño en su seno, y fué llena Isabel del Espíritu Santo. (Lc. 1, 41)
        
        %TERCERO
        \item y levantó la voz con gran clamor y dijo: »Bendita tu entre las mujeres y bendito esl fruto de tu vientre». (Lc. 1, 42)
        
        %CUARTO
        \item Y dichosa la que creyó que tendrá cumplimiento las cosas que le han sido dichas de parte del Señor. (Lc. 1, 45)
        
        %QUINTO
        \item Y dijo María: «Engrandece mi alma al Señor, y se regocijó mi espíritu en Dios, mi Salvador;
        porque puso sus ojos en la bajeza de su esclava». (Lc. 1, 46-48)

        %SEXTO
        \item «Pues he aquí que desde ahora me llamarán dichosa todas las generaciones; 
        porque hizo en mi favor grandes cosas el Poderoso, (Lc. 1, 48-49)
        
        %SEPTIMO
        \item «y cuyo nombre es Santo; y su misericordia por generaciones y generaciones para con aquellos que le temen». (Lc. 1, 49-50)
        
        %OCTAVO
        \item Hizo ostentación de poder con su brazo: desbarató a los soberbios en los proyectos de su corazón. (Lc. 1, 51)
        
        %NOVENO
        \item «derrocó de su trono a los potentados y enalteció a los humildes». (Lc. 1, 52)
        
        %DECIMO
        \item «llenó de bienes a los hambrientos y despidió vacíos a los ricos». (Lc. 1, 53)

      \end{enumerate}
            
    \subsection*{\hfil La Natividad de Nuestro Señor Jesucristo \hfil}
      
      \begin{enumerate}
        
        %PRIMERO
        \item Y sucedió que estando ellos allí se le complieron a ella los díaas del parto. (Lc. 2,6)
        
        %SEGUNDO
        \item Y dió a luz a su hijo primogénito, y le envolvió en pañales\ldots (Lc. 2, 7)
        
        %TERCERO
        \item \ldots y le recostó en un pesebre, pues no había para ellos lugar en el mesón. (Lc. 2, 7)
        
        %CUARTO
        \item Y había unos pastores en aquella misma comarca, que pernoctaban al raso y velaban por turno para guardar su ganado, y un ángel
        del Señor se presentó ante ellos, y la gloria del Señor los envolvió en sus fulgores, y se atemorizaron con gran temor. (Lc. 2, 8-9)
        
        %QUINTO
        \item Y les dijo el ángel: «no temáis, pues he aquí que os traigo una buena nueva, que será de grande alegría para todo el pueblo:» (Lc. 2, 10)

        %SEXTO
        \item «que os ha nacido hoy en la ciudad de David un Salvador, que es el Mesías, el Señor» (Lc. 2, 11)
        
        %SEPTIMO
        \item Gloria a Dios en las alturas y en la tierra a los hombres del [divino] agrado. (Lc. 2, 14)

        %OCTAVO
        \item Nacido Jesús en Belén de la Judea en los días de Herodes el rey, he aquí que unos magos venidos de las regiones orientales llegaron a Jerusalén.
        Y entrando en la casa, vieron al niño con María, su madre; (Mt. 2, 1, 11)
        
        %NOVENO
        \item y postrándose en tierra le adoraron, y abrieron sus tesoros le ofrecieron presentes, oro, incienso y mirra. (Mt. 2, 11)
        
        %DECIMO
        \item Pero María guardaba todas estas palabras confiriéndolas en su corazón. (Lc. 2, 19)

      \end{enumerate}

    \subsection*{\hfil La Presentación del Niño Jesús en el Templo \hfil}
      
      \begin{enumerate}
        
        %PRIMERO
        \item Y cuando se les cumplieron los días de la purificación según la ley de Moisés (Lev. 12, 6), 
        le subieron a Jerusalén para presentarle al Señor. (Lc. 2, 22)
        
        %SEGUNDO
        \item y he aquí había un hombre en Jerusalén por nombre Simeón. Y era este hombre justo y temeroso de Dios que aguardaba la consolación de Israel, 
        y el Espíritu Santo estaba sobre él; (Lc. 2, 25)
        
        %TERCERO
        \item y le había sido revelado por el Espíritu Santo qye no vería la muerte antes de ver al Ungido del Señor. (Lc. 2, 26)
        
        %CUARTO
        \item Y vino al templo impulsado por el Espíritu Santo. Y cuando sus padres intrudujeron al niño Jesús para cumplir las prescipciones usuales
        de la ley tocantes a El, Simeón le recibió en sus brazos y bendijo a Dios diciendo: (Lc. 2, 27-28)
        
        %QUINTO
        \item Ahora dejas ir a tu siervo, Señor, según tu palabra, en paz; (Lc. 2, 29)

        %SEXTO
        \item pues ya vieron mis ojos tu salud, que preparaste a la faz de todos los pueblos; (Lc. 2, 30-31)
        
        %SEPTIMO
        \item luz para iluminación de los gentiles, y gloria de tu pueblo Israel. (Lc. 2, 32)
        
        %OCTAVO
        \item Y les bendijo Simeón, y dijo a María, su madre: «He aquí que éste está puesto para caída y resurgimiento de muchos en Israel, y como
        señal a quien se contradice». (Lc. 2, 34)
        
        %NOVENO
        \item «y a ti misma una espada te traspasará el alma, para que salgan a la luz de muchos corazones los pensamientos». (Lc. 2, 35)
        
        %DECIMO
        \item Y así que cumplieron todas las cosas ordenadas en la ley del Señor, se volvieron a Galilea, a su ciudad de Nazaret. El niño crecía
        y se robustecía llenándose de sabiduría, y la gracia de Dios estaba en Él. (Lc. 2, 39-40)

      \end{enumerate}
            
    \subsection*{\hfil La Pérdida y Hallazgo del Niño Jesús en el Templo \hfil}
      
      \begin{enumerate}
        %PRIMERO
        \item Y cuando fué de doce años, habiendo ellos subido, según la costumbre de la fiesta, (Lc. 2, 42)
        
        %SEGUNDO
        \item y acabados los días, al volverse ellos, quedóse el niño Jesús en Jerusalén, sin que lo advirtiesen sus padres. (Lc. 2, 43)
        
        %TERCERO
        \item y no hallándole, se tornaron a Jerusalén para buscarlo. Y sucedió que después de tres días le hallaron en el templo,
        sentado en medio de los maestros, escuchándoles y haciéndoles preguntas; (Lc. 2, 45-46)
        
        %CUARTO
        \item sentado en medio de los maestros, escuchándoles y haciéndoles preguntas; (Lc. 2, 46)
        
        %QUINTO
        \item y se pasmaban todos los que le oían de su inteligencia y de sus respuestas. (Lc. 2, 47)

        %SEXTO
        \item Y sus padres, al verle, quedaron sorprendidos; y le dijo su madre: «hijo, ¿por qué lo hiciste así con nosotros? Mira que tu padre
        y yo, llenos de aflicción, te andábamos buscando» (Lc. 2, 48)
        
        %SEPTIMO
        \item Díjoles Él: «¿pues por qué me buscabais? ¿No sabíais que había yo de estar en casa de mi padre?» (Lc. 2, 49)
        
        %OCTAVO
        \item Y ellos no comprendieron lo que les dijo. (Lc. 2, 50)
        
        %NOVENO
        \item Y bajó en su compañía y se fue a Nazaret, y vivía sometido a ellos. Y su madre guardaba todas estas
        cosas en su corazón. (Lc. 2, 51)
        
        %DECIMO
        \item Y Jesús progresaba en sabiduría, en estatura y en gracia delante de Dios y de los hombres. (Lc. 2, 52)
      \end{enumerate}

    \newpage
        
  \section*{\hfil Misterios Dolorosos \hfil}
    
    \subsection*{\hfil La Agonía de Nuestro Señor en el Huerto de los Olivos \hfil}
      
      \begin{enumerate}

        %PRIMERO
        \item Entonces llego Jesús con ellos a una granja llamada Getsemaní, y dice a los discípulos: «Sentaos aquí mientras voy allá para hacer oración». 
        Y llevando consigo a Pedro y a los dos hijos de Zebedeo, comenzó a ponerse triste y a sentir abatimiento. (Mt. 26, 36-37)

        %SEGUNDO
        \item Entonces les dice: «triste em gran manera está mi alma hasta la muerte; quedad aquí y velad conmigo». (Mt. 26, 38)

        %TERCERO
        \item Y adelantándose un poco, caía sobre la tierra, y rogaba que, a ser posible, pasase de Él aquella hora. (Mc. 14, 35)

        %CUARTO
        \item diciendo: «Padre, si quieres, traspasa de mi este cáliz; mas no se haga mi voluntad, sino la tuya». (Lc. 22, 42)

        %QUINTO
        \item Y se le apareció un ángel venido del cielo, que le confortaba. (Lc. 22, 43)

        %SEXTO
        \item Y venido en agonía, oraba más intensamente. (Lc. 22, 44)

        %SEPTIMO
        \item Y se hizo su sudor como grumos de sangre, que caían hasta el suerlo. (Lc. 22, 44)

        %OCTAVO
        \item Y viene a los discípulos y los halla durmiendo, y dice a Pedro: «¿así no pudísteis velar una hora conmigo?». (Mt. 26, 40)

        %NOVENO
        \item «Velad y orad, para que no entréis en tentación;». (Mt. 26, 41)

        %DECIMO
        \item «el espíritu sí, está animoso, más la carne es flaca». (Mt. 26, 41)

      \end{enumerate}

    \subsection*{\hfil La flagelación de Nuestro Señor Jesucristo \hfil}
      
      \begin{enumerate}
        
        %PRIMERO
        \item Y luego al amanecer, después de celebrar consejo, los sumos sacerdotes con los ancianos y los escribas, es decir, todo el sanhedrín, atando a Jesús,
        le llevaron de allí y le entregaron a Pilato. Y le interrogó Pilato: «¿Tú eres el Rey de los judíos?». El le respondió: «Tú lo dices». (Mc. 15, 1-2)

        %SEGUNDO
        \item Respondió: «Mi reino no es de este mundo. Si de este mundo fuera mi reino, mis ministros lucharían para que yo no fuera entregado a los judíos
        Más ahora mi reino no es de aquí». (Jn. 18, 36)

        %TERCERO
        \item Díjole, pues, Pilato: «¿luego, rey eres tu?». Respondión Jesús : «Tú dices que yo soy rey. Yo para eso he nacido y para esto he venido
        al mundo: para dar testimonio a favor de la verdad. Todo el que es de la verdad, oye mi voz». (Jn. 18, 37)

        %CUARTO
        \item Pilato dijo a los sumos sacerdotes y a los turbas: «ningún delito hallo en este hombre. 
        Le castigaré, pues, y le soltaré». (Lc. 23, 4, 16)

        %QUINTO
        \item Entonces, pues, tomó Pilato a Jesús y le azotó. (Jn. 19, 1)

        %SEXTO
        \item De opresión u juicio fué tomado, y a sus contemporáneos, ¿quién tendrá en cuenta?. Fué despreciado y abandonado de los hombres,
        varón de dolores y familiarizado de los hombres. (Is. 53, 8, 3)

        %SEPTIMO
        \item Mas nuestros sufrimientos él los ha llevado, nuestros dolores él los cargó sobre sí, (Is. 53, 4)

        %OCTAVO
        \item Fué traspasado por causa de nuestros pecados, molido por causa de nuestras iniquidades; (Is. 53, 5)

        %NOVENO
        \item mientras nosotros le tuvimos por azotado, por herido de Dios y abatido. (Is. 53, 4)

        %DECIMO
        \item el castigo de nuestra paz cayó sobre Él y por sus verdugones se nos curó. (Is. 53, 5)

      \end{enumerate}
      
    \subsection*{\hfil La coronación de espinas de Nuestro Señor Jesucristo \hfil}
      
      \begin{enumerate}

        %PRIMERO
        \item Los soldados se lo llevaron dentro del palacio, que es el pretorio, y convocan a toda la cohorte (Mc. 15, 16).
        Y habiéndole quitado sus vestidos, le envolvieron en una clámide de grana (Mt. 27, 28).

        %SEGUNDO
        \item y trenzando una corna de espinas, la pusieron sobre la cabeza, y una caña en la mano derecha; (Mt. 27, 29)

        %TERCERO
        \item y doblando la rodilla delante de Él, le mofaban, diciendo: «Salud, Rey de los judíos». (Mt. 27, 29)

        %CUARTO
        \item Y escupiendo en Él, tomaron la caña y le daba golpes en la cabeza. (Mt 27, 30)

        %QUINTO
        \item Salió Pilato otra vez fuera, y les dice: «Ved, os le traigo para que conozcáis que no hallo en Él delito alguno». (Jn. 19, 4)

        %SEXTO
        \item Salió, pues, Jesús afuera, llevando la corona de espinas y el manto de púrpura. Y les dice: «ved aquí el hombre». (Jn. 19, 5)

        %SEPTIMO
        \item Y les dice: «ved aquí el hombre». \textsuperscript{15}Gritaron, pues, ellos: «quita, quita; crucifícale». (Jn. 19, 5, 15)

        %OCTAVO
        \item Pilato, queriendo dar satisfacción a la turba les soltó a Barrabás. Y entregó a Jesús, después de azotarle, para que fuese crucificado. (Mc. 15, 14)

        %NOVENO
        \item Díceles Pilato: «¿A vuestro rey he de crucificar?». Respondieron los pontífices: «no tenemos rey, sino César». (Jn. 19, 15)

        %DECIMO
        \item Entonces, pués, se le entregó para que fuera crucificado. Se apoderaron, pues, de Jesús. (Jn. 19, 16)

      \end{enumerate}

    \subsection*{\hfil Jesús con la Cruz a cuestas \hfil}
      
      \begin{enumerate}
        
        %PRIMERO
        \item Si alguno quiere venir en pos de Mí, niégese a sí mismo. (Lc. 9, 23)

        %SEGUNDO
        \item y tome a cuestas su cruz cada día y sígame. (Lc. 9, 23)

        %TERCERO
        \item y, llevando a cuestas su cruz, salió hacia el lugar llamado el Cráneo, que en hebreo se dice Gólgota. (Jn. 19, 17)

        %CUARTO
        \item Y como le hubieron sacado, echaron mano de un tal Simón de Cirene que venía del campo, le pusieron en hombros la cruz para que la llevase
        detrás de Jesús. (Lc. 23, 26)

        %QUINTO
        \item Tomad mi yugo sobre vuestros, y aprended de mi, (Mt. 11, 29)

        %SEXTO
        \item pues soy manso y humilde de Corazón, y hallaréis reposo para vuestras almas. (Mt. 11, 29)

        %SEPTIMO
        \item Porque mi yugo es suave, y mi carga, ligera. (Mt. 11, 30)

        %OCTAVO
        \item Seguíanle gran mucheduncbre de pueblo y de mujeres, las cuales le plañían y lamentaban. (Lc. 23, 27)

        %NOVENO
        \item Volviéndose Jesús a ellas, les dijo: «Hijas de Jerusalén: no lloréis sobre mi, sino llorad má bien sobre vosotras mismas y sobre
        vuestros hijos». (Lc. 23, 28)

        %DECIMO
        \item «Porque si en el leño verde esto hacen, ¿en el seco qué se hará?». (Lc. 23, 31)

      \end{enumerate}

    \subsection*{\hfil La Crucifixión y Muerte del Redentor \hfil}
      
      \begin{enumerate}
        %PRIMERO
        \item Y cuando hubieron llegado al lugar llamado «Cráneo», allí crucificaron a Él y a los malhechores, uno a la derecha y el otro a la izquierda. (Lc. 23, 33)

        %SEGUNDO
        \item Y Jesús decía: «Padre, perdónalos, porque no saben lo que hacen». (Lc. 23, 34)

        %TERCERO
        \item Uno de los malhechores que estaba colgado decía a Jesús: «acuérdate de mi cuando vinieres en la gloria de tu realeza». (Lc. 23, 39, 42)

        %CUARTO
        \item Díjole: «en verdad te digo que hoy estarás conmigo en el paraíso». (Lc. 23, 43)

        %QUINTO
        \item Jesús, pues, viendo a la Madre, y junto a ella al discípulo a quien amaba, (Jn. 19, 26)

        %SEXTO
        \item dice a tu Madre: «mujer, he ahí a tu hijo». \item Luego dice al discípulo: «He aquí a tu Madre». (Jn. 19, 26-27)

        %SEPTIMO
        \item Y desde aquella hora la tomó el discípulo en su compañía. (Jn. 19, 27)

        %OCTAVO
        \item habiendo faltado el sol; y se rasgó por medio el velo del santuario. (Lc. 23, 45)

        %NOVENO
        \item Y clamando con voz poderosa, Jesús dijo: «Padre, en tus manos encomiendo mis espíritu». (Lc. 23, 46 )

        %DECIMO
        \item Jesús dijo: «Consumado está». E inclinando la cabeza entregó el espíritu. (Jn. 19, 30)

      \end{enumerate}

    \newpage
         
  \section*{\hfil Miserios Gloriosos \hfil}
    \subsection*{\hfil La Resurección del Señor \hfil}
      
      \begin{enumerate}
        
        %PRIMERO
        \item «En verdad, en verdad os digo que vosotros lloraréis y os lamentaréis, y el mundo se rogocijará;
        vosotros os afligiréis, pero vuestra aflicción está de parto». (Jn. 16, 20)

        %SEGUNDO  
        \item «Pues así también vosotros, ahora cierto tenéis congoja; mas otra vez os veré, y se gozará vuestro corazon,
        y vuestro gozo nadie os lo quita». (Jn. 16, 22)

        %TERCERO
        \item Más el primer día de la semana, apenas rayó el alba, se vinieron al monumento llevando consigo los aromas
        que habían preparado. (Lc. 24, 1)

        %CUARTO
        \item De pronto se produjo un gran temblor de tierra, pues un ángel del Señor, bajado de cielo y acercándose, hizo rodar
        de su sitio la losa, y se sentó sobre ella. (Mt. 28, 2)
        
        %QUINTO
        \item Tomando la palabra el ángel, dijo a las mujeres: «no tengáis miedo vosotras, que ya sé que buscáis a Jesús el crucificado». (Mt. 28, 5)

        %SEXTO
        \item «no está aquí; resucitó, como dijo. Venid, ved el lugar dode estuvo puesto». (Mt. 28, 6)

        %SEPTIMO
        \item «Y marchando a toda prisa, decid a sus discípulos que resucitó de entre los muertos, y he aquí que se os adelanta en ir a Galilea:
        allí le veréis. Conque os lo tengo dicho». (Mt. 28, 7)

        %OCTAVO
        \item Y partiendo a toda prisa del monumento, con temor y grande gozo corrieron a dar la nueva a sus discípulos. (Mt. 28, 8)

        %NOVENO
        \item «Yo soy la resurección y la vida; quien cree en mi, aun cuando muera, vivirá». (Jn. 11, 25)

        %DECIMO
        \item «y todo el que vive y cree en mi, no morirá para siempre. ¿Crees esto?». (Jn. 11,26)
        
      \end{enumerate}

    \subsection*{\hfil La Ascensión de Jesucristo a los cielos \hfil}
        
      \begin{enumerate} 
        %PRIMERO
        \item Y los sacó afuera hasta llegar junto a Betania, y alzando sus manos los bendijo. (Lc. 24, 50)

        %SEGUNDO
        \item Y acercándose Jesús, les habló diciendo: «Me fué dada toda potestad en el cielo y sobre la tierra». (Mt. 28, 18)

        %TERCERO
        \item «Id, pues, amaestrad a todas las gentes». (Mt. 28, 19)

        %CUARTO
        \item «bautizándoles en el nombre del Padre y del Hijo y del Espíritu Santo». (Mt. 28, 19)

        %QUINTO
        \item «enseñándoles a guardar todas cuantas cosas os ordené». (Mt. 28, 20)

        %SEXTO
        \item El que creyere y fuere bautizado, se salvará; (Mc. 16, 16)

        %SEPTIMO
        \item mas el que no creyere, será condenado. (Mc. 16, 16)

        %OCTAVO
        \item «Y sabed que estoy con vosotros todos los días hasta la consumación de los siglos» (Mt. 28, 20)

        %NOVENO
        \item Y aconteció que, mientras los bendecía, se desprendió de ellos, y era llevado en alto al cielo. (Lc. 24, 51)

        %DECIMO
        \textsuperscript{19}Con esto el Señor Jesús, despues de hablarles, fue elevado al cielo y se sentó a la diestra de Dios. (Mc. 16, 19)

      \end{enumerate}

    \subsection*{\hfil La Venida del Espíritu Santo sobre los Apóstoles \hfil}

      \begin{enumerate}
        %PRIMERO
        \item Y al cumplirse el día de Pentecostés, estaban todos juntos en el mismo lugar. (Hch. 2, 1)

        %SEGUNDO
        \item Y se produjo de súbito desde el cielo un estruendo como de viento que soplaba vehemente, u llenó toda la casa
        donde se hallaban sentados. (Hch. 2, 2)

        %TERCERO
        \item Y vieron aparecer lenguas como de fuego, que, repartiéndose, se posaban sobre cada uno de ellos. (Hch. 2, 3)

        %CUARTO
        \item Y se llenaron todos del Espíritu Santo, y comenzaron a hablar en lenguas diferentes, según que el Espíritu Santo les movía
        a expresarse. (Hch. 2, 4)

        %QUINTO
        \item Hallábanse en Jerusalén judíos allí domiciliados, hombres religiosos de toda nación de las que están debajo del cielo. (Hch. 2, 5)

        %SEXTO
        \item Puesto de pie Pedro, acompañado de los Once, alzó en voz y les habló en estos términos. (Hch. 2, 14)

        %SEPTIMO
        \item «Arrepentíos, dice, y bautícese cada uno de vosotros en el nombre del Jesu-Cristo para remisión de vuestros pecados, y recibiréis el don
        del Espíritu Santo». (Hch. 2, 38)

        %OCTAVO
        \item Ellos, pues, acogieron su palabra, fueron bautizados; y fueron agragados en aqueñ día como unas tres mil almas. (Hch. 2,41)

        %NOVENO
        \item SI tu Espíritu envías, son creados, y la faz de la tierra así renuevas. (Sal. 104, 30)

        %DECIMO
        \item ¡Ven, Espíritu Santo, y desde el Cielo envía rayos de tu virtud!¡!Ven, Padre de los pobres!¡Ven, Dador de tus Dones!¡Ven, de almas luz!
        
      \end{enumerate}

    \subsection*{\hfil La Asunción de Nuestra Señora a los cielos \hfil}

      \begin{enumerate}

        %PRIMERO
        \item Bendita tú, hija, ante el Dios Altísimo sobre todas las mujeres de la tierra\ldots (Jdt. 13, 18)

        %SEGUNDO
        \item Pues no se apartará etérnamente tu esperanza del corazón de los hombre\ldots (Jdt. 13, 19)

        %TERCERO
        \item Y esto haga contigo Dios para eterno encubrimiento, que te visite con sus bienes; por cuanto no perdonate
        a tu vida, lastimada por la humillación de nuestro linaje\ldots (Jdt. 13, 20)

        %CUARTO
        \item {\ldots}Tú eres enaltecimiento de Jerusalén, tú gloria grande de Israel, tú grande honor de nuestro linaje. (Jdt. 15, 9)

        %QUINTO
        \item Oye, hija, mira; tu oído aplica; tu pueblo olvida y la mansión paterna; deja que tu hermosura
        el rey codicie, que es tu señor. A él le doblega. (Sal. 45; 11-12)

        %SEXTO
        \item Y se abrió el templo de Dios, que está en el cielo, y fué vista el arca de la alianza en el templo,
        y se produjeron relámpagos, y voces, y truenos, y temblor de tierra, y fuerte granizada. (Ap. 11, 19)

        %SEPTIMO
        \item Y una gran señal fué vista en el cielo: una Mujer vestida del sol\ldots (Ap. 12, 1)

        %OCTAVO
        \item {\ldots}y la luna debajo de sus pies, y sobre su cabeza una corona de doce estrellas. (Ap. 12, 1)

        %NOVENO
        \item Del rey la hija toda hermosa entra; vestidos áureos se adorno son. (Sal. 45, 14)

        %DECIMO
        \item Entonad a Yahveh cántico nuevo, que portentos ha obrado. Su diestra le ha traído la victoria y aquel su brazo santo. (Sal. 98, 1)

      \end{enumerate}

    \subsection*{\hfil La Coronación de la Santísima Virgen María \hfil}

      \begin{enumerate}

        %PRIMERO
        \item ¿Quién es esa que aparece resplandeciente como la aurora,
        hermosa cual luna, deslumbradora como el sol, imponente como batallones? (Cant. 6, 10)

        %SEGUNDO
        \item Como el arco iris, que se aparece en las nubes; como flor entre el ramaje en días primaverales, como azucena junto
        a la corriente de las aguas, como las flores del Líbano en días de verano; como el incienso que arde sobre la ofrenda,
        como el vaso de oro fínamente trabajado. (Eclo. 50, 8-9)

        %TERCERO
        \item Yo soy la madre del amor, del temor, de la ciencia y de la santa esperanza. (Eclo. 24, 24)

        %CUARTO
        \item EN mi está toda la gracia del camino y de la verdad, en mi toda esperanza de la vida y de la virtud. (Eclo. 24, 25)

        %QUINTO
        \item Venid a mí cuantos me deseáis y saciaos de mis frutos. (Eclo. 24, 26)

        %SEXTO
        \item Porque recordarme es más dulce que la miel, y poseerme más rico que el panal de miel. (Eclo. 24, 27)

        %SEPTIMO
        \item Ahora, pues, hijos míos, oídme; y felices quienes guardan mis caminos. Escochad la corrección y sed sabios, (Prov. 8, 32-33)

        %OCTAVO
        \item y no la rechacéis. Feliz el hombre que me escucha, velando a mis puertas cada día,
        guardando las jambas de mis entradas. (Prov. 8, 33-34)

        %NOVENO
        \item Pues quien me halla, ha hallado la vida y alcanza el favor de Yahveh. Más quien peca contra mi, se perjudica a si mismo,
        y cuantos me odian aman la muerte. (Prov. 8, 35)

        %DECIMO
        \item Tiene Él escrito en su vestido y en su manto Rey de reyes y Señor de los que dominan (Ap. 18, 16).
        Está la Reina a su derecha, adornada con oro finísimo (Sal. 44, 10).
      \end{enumerate}

\end{document}
