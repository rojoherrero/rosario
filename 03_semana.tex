\documentclass[10pt]{article}
\usepackage[utf8]{inputenc}
\usepackage[latin,spanish]{babel}
\usepackage[T1]{fontenc}
\usepackage[a4paper, margin=10mm]{geometry}
\usepackage{lettrine}
\usepackage{xcolor}
\usepackage{yfonts}
\usepackage{gregoriotex}
\usepackage{multicol}
\pagenumbering{gobble}

\setlength{\columnseprule}{0.4pt}
\def\columnseprulecolor{\color{red}}

\newcommand{\primeraletragranderoja}[2]{
    \lettrine[lines=2]{\textcolor{red}{#1}}{#2}
}

\newcommand{\primeraletragranderojasola}[1]{
    \lettrine[lines=2]{\textcolor{red}{#1}}{}
}

\newcommand{\primeraletragrande}[2]{
    \lettrine[lines=2]{#1}#2
}

\newcommand{\letraroja}[2]{
    \textcolor{red}{#1}#2
}

\newcommand{\versiculo}[1]{
    \textcolor{red}{\Vbar.} #1.
}

\newcommand{\respuesta}[1]{
    \textcolor{red}{\Rbar.} #1.
}

\newcommand{\textopequenorojo}[1]{
    {\textcolor{red}{\small{#1}}}
}

\newcommand{\versiculorespuesta}[2]{
    \versiculo{#1}\\
    \respuesta{#2}
}

\newcommand{\versiculorespuestaseguido}[2]{
    \versiculo{#1}\respuesta{#2}
}

\newcommand{\redcross}{
    \textcolor{red}{\grecross}
}

\newcommand{\lineahorizontal}[2]{
    \begin{center}
        {\rule{#1cm}{#2pt}}
    \end{center}
}

\newcommand{\lineahorizontalroja}[2]{
    \begin{center}
        \textcolor{red}{\rule{#1cm}{#2pt}}
    \end{center}
}

\newcommand{\latinderecha}[1]{
    \begin{otherlanguage}{latin}
        \begin{rightcolumn}
            \input{#1}    
        \end{rightcolumn}
    \end{otherlanguage}
}

\newcommand{\castellanoizquierda}[1]{
    \begin{leftcolumn}
        \input{#1}
    \end{leftcolumn}
}

\newcommand{\castellanoizquierdasincro}[1]{
    \begin{leftcolumn*}
        \input{#1}
    \end{leftcolumn*}
}

\newcommand{\castellanoizquierdasincronota}[2]{
    \begin{leftcolumn*}[#1]
        \input{#2}
    \end{leftcolumn*}
}

\newcommand{\filacastellanolatin}[2]{
    {    
        \begin{leftcolumn}
            \input{#1}
        \end{leftcolumn}
        \begin{otherlanguage}{latin}
            \begin{rightcolumn}
                \input{#2}    
            \end{rightcolumn}
        \end{otherlanguage}
    }
}

\newcommand{\filacastellanolatinsincro}[2]{
    {    
        \begin{leftcolumn*}
            \input{#1}
        \end{leftcolumn*}
        \begin{otherlanguage}{latin}
            \begin{rightcolumn}
                \input{#2}    
            \end{rightcolumn}
        \end{otherlanguage}
    }
}

\newcommand{\filacastellanolatinsincronota}[3]{
    \castellanoizquierdasincronota{#1}{#2}
    \latinderecha{#3}
}

\newcommand{\titulomisterios}[2]{
    {    
        \begin{minipage}[t]{0.595\textwidth}
            \subsection*{#1}
        \end{minipage}\begin{minipage}[t]{0.395\textwidth}
            \begin{flushright}
                \textcolor{red}{#2}
            \end{flushright}
        \end{minipage}
    }
}

\newcommand{\titulomisterio}[2]{
    {
        \begin{minipage}[t]{0.595\textwidth}
            \section*{#1}
        \end{minipage}\begin{minipage}[t]{0.395\textwidth}
            \begin{flushright}
                \textcolor{red}{#2}
            \end{flushright}
        \end{minipage}
    }
}

\newcommand{\iralfinal}{
    \begin{center}
        \textcolor{red}{Una vez terminamos nos vamos a la \cpageref{final-prayer} para las oraciones finales.}
    \end{center}
}




\begin{document}

\begin{center}
      \textsc{\Large La Semana Santificada}
\end{center}

\begin{multicols}{2}

      %%%%%%%%%
      % LUNES %
      %%%%%%%%%

      \begin{center}
            \textsc{\textcolor{red}{Feria II (Lunes)}\\ {\large A las Benditas almas del Purgatorio}}
      \end{center}

      \hfill\textcolor{red}{Apoc 14, 13}

      \primeraletragranderoja{E}{n}aquellos días: Oí la voz del cielo, que me decía: Escribe: Bienaventurados los muertos que mueren en el Señor. Ya desde ahora dice el Espíritu que descansen
      de sus trabajos, puesto que sus obras los van acompañando.

      \vspace{2mm}

      \hfill\textcolor{red}{Ioan 6, 37-40}

      \primeraletragranderoja{E}{n}aquel tiempo: Dijo Jesús a las turbas de los judíos: Todos los que el Padre me da, vendrán a mi; y al que viniere a mi, no le desecharé; porque he descendido
      del cielo, no para hacer mi voluntad, sino la de aquel que me ha enviado. Y la voluntad de mi Padre, que me ha enviado, es que yo no pierda ninguno de los que
      me ha dado, sino que los resucite a todos en el último día. Por tanto, la voluntad de mi Padre, que me ha enviado, es que todo el que ve al Hijo, y cree en él,
      tenga vida eterna, y yo le resucitaré en el último día.

      \vspace{2mm}

      \begin{otherlanguage}{latin}
            \textcolor{red}{P}ater noster, qui es in c{\ae}lis, sanctificétur nomen tuum. Advéniat regnum tuum.
Fiat volúntas tua, sicut in c{\oe}lo et in terra. Panem nóstrum quotidiánum da nobis hódie.
Et dimitte nobis debita nostra, sicut et nos dimittimus debitóribus nostris.
Et ne nos indúcas in tentatiónem: sed libera nos a malo. Amen.

            \vspace{1mm}

            \textcolor{red}{A}ve María, grátia plena, Dóminus tecum; benedicta tu in muliéribus, et benedíctus fructus ventris tui,
Iesus. Sancta Maria, Mater Dei, ora pro nobis peccatóribus, nunc et in hora mortis nostr{\ae}. Amen.
      \end{otherlanguage}

      \vspace{2mm}

      \primeraletragrande{O}{h}Dios, dador del perdón y que deseáis la salvación del hombre: rogamos a vuestra clemencia que a las almas de todos lso fieles, que de este mundo salieron,
      les concedáis por intercesión de la bienaventurada siempre Virgen María y de todos sus Santos, llegar a la participación de la eterna felicidad. Por nuestro
      Señor Jesucristo. Amén.
      
      \begin{otherlanguage}{latin}
            \versiculorespuestaseguido{Réquiem {\ae}térnam dona eis, Dómine}
    {Et lux perpétua lúceat eis}

            \vspace{1mm}

            \versiculorespuestaseguido{Requiéscant in pace}{Amen}
      \end{otherlanguage}

      %%%%%%%%%%
      % MARTES %
      %%%%%%%%%%

      \begin{center}
            \textsc{\textcolor{red}{Feria III (Martes)}\\ {\large Al Ángel de la Guarda}}
      \end{center}

      \begin{otherlanguage*}{latin}
            \primeraletragranderoja{A}{ngele} Dei, qui custos es mei, me tibi commissum pietate superna hac die et semper illumina, custodi, rege et guberna. Amén
      \end{otherlanguage*}

      \vspace{2mm}

      \hfill\textcolor{red}{Ps 90, 10-12}

      \primeraletragranderoja{N}{o}se te acercará mal alguno, y no se allegará a tu morada la desgracia. Pues dio a sus ángeles órdenes acerca de ti, para que te guarden en todos tus pasos.
      Te llevarán en las palmas de sus manos, para que tu pie no tropiece en alguna piedra.
      
      \vspace{2mm}

      \hfill\textcolor{red}{Ex 23, 20-22}
      
      \primeraletragranderoja{Y}{o}mandaré a un ángel ante ti, para que te defienda en el camino y te haga llegar al lugar que te he dispuesto. Acátale, y escucha su voz,
      no le resista, porque no perdonará vuestras rebeliones y porque lleva mi nombre. Pero si le escuchas y haces cuanto él te diga, yo seré
      enemigo de tus enemigos y afligiré a los que te aflijan.
      
      \vspace{2mm}

      \versiculorespuestaseguido{Dios mío, os cantaré en la presencia de vuestros Ángeles}{Os adoraré en vuestro santo templo y cantaré a vuestro nombre}
      
      \vspace{2mm}

      \begin{otherlanguage}{latin}
            \textcolor{red}{P}ater noster, qui es in c{\ae}lis, sanctificétur nomen tuum. Advéniat regnum tuum.
Fiat volúntas tua, sicut in c{\oe}lo et in terra. Panem nóstrum quotidiánum da nobis hódie.
Et dimitte nobis debita nostra, sicut et nos dimittimus debitóribus nostris.
Et ne nos indúcas in tentatiónem: sed libera nos a malo. Amen.

            \vspace{1mm}

            \textcolor{red}{A}ve María, grátia plena, Dóminus tecum; benedicta tu in muliéribus, et benedíctus fructus ventris tui,
Iesus. Sancta Maria, Mater Dei, ora pro nobis peccatóribus, nunc et in hora mortis nostr{\ae}. Amen.

            \vspace{1mm}

            \versiculorespuestaseguido{Glória Patri, et Filio, et Spirítui Sancto}{Sicut erat in princípio et nunc, et semper et in s{\ae}cula s{\ae}culórum. Amen}
      \end{otherlanguage}

      \vspace{2mm}

      \primeraletragrande{O}{h}Dios, que con admirable providencia os habéis dignado enviarnos a vuestro santos Ángeles para que nos guarden; concedenos, humildemente os pedimos,
      que seamos siempre defendidos con su protección y gocemos un día eternamente de su compañía en el Cielo. Por Jesucristo nuestro Señor. Amén.

      %%%%%%%%%%%%%
      % MIERCOLES %
      %%%%%%%%%%%%%

      \begin{center}
            \textsc{\textcolor{red}{Feria IV (Miércoles)}\\ {\large Al Glorioso Patriarca San José}}
      \end{center}

      \hfill\textcolor{red}{Lc 3, 23}

      \primeraletragranderoja{J}{esús,}al empezar, tenía unos treinta años, y era, según se creía, hijo de José.

      \vspace{2mm}

      \hfill\textcolor{red}{Mt 1, 18-21}

      \primeraletragranderoja{E}{stando}desposada la Madre de Jesús, María, con José, sin que antes hubiesen estado juntos, se halló que había concebido en su seno por obra del Espíritu Santo. Mas José, su esposo,
      siendo como era justo, y no queriendo infamarla, deliberó dejarla secretamente. Estando él en este pensamiento, he aquí que un ángel del Señor se le apareció en suerños, diciendo:
      José, hijo de David, no tengas recelos en recibir a María tu esposa, porque lo que se ha engendrado en su seno, es obra del Espíritu Santo. Así que dará a luz un hijo, a quien pondrás
      el nombre de Jesús, pues él es el que ha de salvar a su pueblo de sus pecados.
      \vspace{2mm}

      \begin{otherlanguage}{latin}
            \textcolor{red}{P}ater noster, qui es in c{\ae}lis, sanctificétur nomen tuum. Advéniat regnum tuum.
Fiat volúntas tua, sicut in c{\oe}lo et in terra. Panem nóstrum quotidiánum da nobis hódie.
Et dimitte nobis debita nostra, sicut et nos dimittimus debitóribus nostris.
Et ne nos indúcas in tentatiónem: sed libera nos a malo. Amen.

            \vspace{1mm}

            \textcolor{red}{A}ve María, grátia plena, Dóminus tecum; benedicta tu in muliéribus, et benedíctus fructus ventris tui,
Iesus. Sancta Maria, Mater Dei, ora pro nobis peccatóribus, nunc et in hora mortis nostr{\ae}. Amen.

            \vspace{1mm}

            \versiculorespuestaseguido{Glória Patri, et Filio, et Spirítui Sancto}{Sicut erat in princípio et nunc, et semper et in s{\ae}cula s{\ae}culórum. Amen}

            \versiculorespuestaseguido{Ora pro nobis, sancte Joseph}{Ut digni efficiamur promissionisbus Christi. Amen}
      \end{otherlanguage}

      \vspace{2mm}

      \primeraletragrande{O}{h}José! Que los coros celestiales celebren vuestras grandezas; que los cantos de todos los cristianos hagan resonar vuestras alabanzas. Glorioso ya por vuestros méritos, os
      unisteis por una casta alianza a la augusta Virgen.

      Cuando dominado por la duda y la ansiedad, os asombráis del estado en que se halla vuestra esposa, un Ángel viene a deciros que el hijo que ella ha concebido es obra del Espíritu Santo.

      El Señor ha nacido, y le estrecháis en vuestros brazos; partís con él hacia las lejanas playas de Egipto; después de haberle perdido en Jerusalén, le encontráis de nuevos; así vuestros gozos
      van mezclados con lágrimas.

      Otros son glorificados después de una santa muerte, y los que han merecido la palma son recibidos en el seno de la gloria; por vos, por un admirable destino, semejante a los Santos, y aun
      más dichoso, disfrutáis ya en esta vida de la presencia de Dios.

      Oh Trinidad soberana, oíd nuestras preces, concedednos el perdón; que los méritos de José nos ayuden a subir al cielo, para siempre el cántico de acción de gracias y de felicidad. Amén.

      %%%%%%%%%%
      % JUEVES %
      %%%%%%%%%%

      \begin{center}
            \textsc{\textcolor{red}{Feria V (Jueves)}\\ {\large Al Santísimo Sacramento del Altar}}
      \end{center}

      \hfill\textcolor{red}{1 Cor 11, 23-29}

      \primeraletragranderoja{H}{ermanos:}Yo recibí del Señor lo que también os tengo enseñado, y es que el Señor Jesús la noche misma en que había de ser entregado, tomó el pan, y dando gracias, lo partió, y dijo:
      Tomad y comed; este es mi cuerpo, que por vosotros será entregado; haced esto en memoria mía. Y de la misma manera tomó también el cáliz después de haber cenado, diciendo: Este es el
      Nuevo Testamento en mi sangre. Haced esto, cuantas veces le bebiereis, en memoria mía. Pues todas las veces que comiereis este pan, y bebiereis este cáliz, anunciaréis la muerte del
      Señor hasta que venga. De manera que cualquiera que comiere este pan, o bebiere el cáliz del Señor indignamente, reo será del Cuerpo y la Sangre del Señor. Por lo tanto, examínese a
      sí mismo el hombre; y de esta suerte, coma del aquel pan y beba de aquel cáliz; porque quien le come y bebe indignamente, se traga y bebe su propia condenación, no haciendo el debido
      discernimiento del cuerpo del Señor.

      \vspace{2mm}

      \hfill\textcolor{red}{Ioan 6, 56-59}

      \primeraletragranderoja{E}{n}aquel tiempo: Dijo Jesús a las turbas de los judíos: Mi carne verdaderamente es comida, y mi Sangre es verdaderamente bebida. Quien como mi carne y bebe mi Sangre, en mi mora y yo en él.
      Así como el Padre, que me ha enviado, vive, y yo vivo por mi Padre; así quien me come, también él vivirá por mi. Este es el Pan que ha bajado del Cielo. No sucederá como a vuestros padres,
      que comieron el maná y, no obstante esto, murieron. Quien come este pan vivirá eternamente.

      \vspace{2mm}

      \begin{otherlanguage}{latin}
            \textcolor{red}{P}ater noster, qui es in c{\ae}lis, sanctificétur nomen tuum. Advéniat regnum tuum.
Fiat volúntas tua, sicut in c{\oe}lo et in terra. Panem nóstrum quotidiánum da nobis hódie.
Et dimitte nobis debita nostra, sicut et nos dimittimus debitóribus nostris.
Et ne nos indúcas in tentatiónem: sed libera nos a malo. Amen.

            \vspace{1mm}

            \textcolor{red}{A}ve María, grátia plena, Dóminus tecum; benedicta tu in muliéribus, et benedíctus fructus ventris tui,
Iesus. Sancta Maria, Mater Dei, ora pro nobis peccatóribus, nunc et in hora mortis nostr{\ae}. Amen.

            \vspace{1mm}

            \versiculorespuestaseguido{Glória Patri, et Filio, et Spirítui Sancto}{Sicut erat in princípio et nunc, et semper et in s{\ae}cula s{\ae}culórum. Amen}
      \end{otherlanguage}

      \vspace{2mm}

      \primeraletragrande{O}{h}Dios, que dejasteis la memoria de vuestra Pasión en ese Sacramento admirable: concedednos que de tal suerte veneremos los sagrados misterios de vuestro Cuerpo 
      y vuestra Sangre, que experimentemos continuamente en nuestras almas el fruto de vuestra redención: Vos que vivís y reináis con Dios Padre, en la unidad del Espíritu Santo Dios, 
      por los siglos de los siglos, Amén.

      %%%%%%%%%%%
      % VIERNES %
      %%%%%%%%%%%

      \begin{center}
            \textsc{\textcolor{red}{Feria VI (Viernes)}\\ {\large A la Pasión de N.S. Jesucristo (Santo Vía-Crucis\footnote{Tomado del Misal del P. Valentín Sánchez Ruiz, SI. 4a edición, Madrid 1945})}}
      \end{center}

      Por la señal de la Santa Cruz{\redcross}, de nuestros enemigos{\redcross}líbranos, Señor Dios nuestro{\redcross}. En el Nombre del Padre, y del{\redcross}Hijo, y del Espíritu Santo. Amén.

      \vspace{2mm}

      \textcolor{red}{S}eñor mío Jesucristo, Dios y Hombre verdadero, Creador y Redentor mío: por ser vos quién sois, y porque os amo sobre todas las cosas,
me pesa de todo corazón de haberos ofendido, propongo firmemente nunca más pecar, y apartarme de todas las ocasiones de ofenderos,
confesarme, y cumplir la penitencia que me fuere impuesta; ofrézcoos mi vida, obras y trabajos en satisfacción de todos mis pecados;
y confío en vuestra bondad y misericordia infinita me los perdonaréis por los merecimientos de vuestra preciosísima sangre, pasión y muerte,
y me daréis gracia para enmendarme y para perseverar en vuestro santo servicio hasta el fin de mi vida. Amén.

      \vspace{2mm}

      \textbf{Primera Estación:} \textit{Jesús es condenado a muerte}.--- {!`}Oh Señor mío Jesucristo, que quisiste ser condenado a muerte por mis pecados, 
      para que yo fuese perdonado de ellos!, te suplico me perdones en vida mis culpas, y en el día del juicio me absuelvas de las penas eternas

      \vspace{2mm}

      \begin{otherlanguage}{latin}
            \textcolor{red}{P}ater noster, qui es in c{\ae}lis, sanctificétur nomen tuum. Advéniat regnum tuum.
Fiat volúntas tua, sicut in c{\oe}lo et in terra. Panem nóstrum quotidiánum da nobis hódie.
Et dimitte nobis debita nostra, sicut et nos dimittimus debitóribus nostris.
Et ne nos indúcas in tentatiónem: sed libera nos a malo. Amen.

            \vspace{1mm}

            \textcolor{red}{A}ve María, grátia plena, Dóminus tecum; benedicta tu in muliéribus, et benedíctus fructus ventris tui,
Iesus. Sancta Maria, Mater Dei, ora pro nobis peccatóribus, nunc et in hora mortis nostr{\ae}. Amen.

            \vspace{1mm}

            \versiculorespuestaseguido{Glória Patri, et Filio, et Spirítui Sancto}{Sicut erat in princípio et nunc, et semper et in s{\ae}cula s{\ae}culórum. Amen}
      \end{otherlanguage}

      \vspace{1mm}

      {!`}Señor, pequé! Tened misericordia de mí.

      Bendita y alabada sea la Pasión y muerte de nuestro Señor Jesucristo y los Dolores de us Madre María Santísima.

      \vspace{2mm}

      \textbf{Segunda Estación:} \textit{Jesucristo toma la Cruz}.--- {!`}Oh Señor mío Jesucristo, que con tanto ánimo tomaste en tus hombros la cruz de mis pecados!, 
      te suplico me concedas resignación y ánimo para llevar la merecida cruz de mis trabajos por tu amparo

      \vspace{2mm}

      Padre nuestro, Ave María y Gloria.

      {!`}Señor, pequé! Tened misericordia de mí.

      Bendita y alabada sea la Pasión y muerte de nuestro Señor Jesucristo y los Dolores de us Madre María Santísima.

      \vspace{2mm}

      \textbf{Tercera Estación:} \textit{Jesucristo cae por primera vez}.--- {!`}Oh Señor mío Jesucristo!, cuando yo caiga desfallecido y sin ánimo para cumplir mi deber,
      te suplico me levantes y reanimes con tu gracia, para seguir con mi cruz cumpliendo hasta morir tu santa voluntad

      \vspace{2mm}

      Padre nuestro, Ave María y Gloria.

      {!`}Señor, pequé! Tened misericordia de mí.

      Bendita y alabada sea la Pasión y muerte de nuestro Señor Jesucristo y los Dolores de us Madre María Santísima.

      \vspace{2mm}

      \textbf{Cuarta Estación:} \textit{Jesucristo encuentra a su Santísima Madre}.--- {!`}Oh Señor mío Jesucristo!, no sólo a Ti, sino también a tu Madre, fuí causa de dolor. En la calle de
      Amargura de mi vida, envíame el consuelo de encontrar a tu Madre, que con su presencia tendré más ánimo. Y vos, {!`}oh Virgen Dolorosa, Madre mía!, perdonadme y no os apartéis jamas de mí,

      \vspace{2mm}

      Padre nuestro, Ave María y Gloria.

      {!`}Señor, pequé! Tened misericordia de mí.

      Bendita y alabada sea la Pasión y muerte de nuestro Señor Jesucristo y los Dolores de us Madre María Santísima.

      \vspace{2mm}

      \textbf{Quinta Estación:} \textit{La ayuda del Cirineo}.--- {!`}Oh Señor mío Jesucristo!, te suplico me des la gracia de que yo sea tu Cirineo cooperando a la
      salvación de los hombres, que sea el Cirineo de los afligidos, pobres y necesitados, aliviando sus penas, y que tú seas nuestro Cirineo para que perseveremos hasta el fin.

      \vspace{2mm}

      Padre nuestro, Ave María y Gloria.

      {!`}Señor, pequé! Tened misericordia de mí.

      Bendita y alabada sea la Pasión y muerte de nuestro Señor Jesucristo y los Dolores de us Madre María Santísima.

      \vspace{2mm}

      \textbf{Sexta Estación:} \textit{Jesucristo encuentra a la Verónica}.--- {!`}Oh Señor mío Jesucristo!, te suplico que grabes en mi corazón aquella imagen que dejaste a la Verónica
      en el lienzo con que enjugó tu rostro, para que teniendo presente lo que Tú sufriste por mí, me anime a sufrir cualquier cosa por Tí.

      \vspace{2mm}

      Padre nuestro, Ave María y Gloria.

      {!`}Señor, pequé! Tened misericordia de mí.

      Bendita y alabada sea la Pasión y muerte de nuestro Señor Jesucristo y los Dolores de us Madre María Santísima.

      \vspace{2mm}

      \textbf{Séptima Estación:} \textit{Jesucristo cae por segunda vez}.--- {!`}Oh Señor mío Jesucristo!, te suplico que, aun cuando yo caiga segunda vez y muchas veces en mi camino, no me dejes,
      no me abandones caído; {!`}ten paciencia conmigo! Levántame, anímame, ayúdame, para que siga adelante con tu cruz a tu lado.

      \vspace{2mm}

      Padre nuestro, Ave María y Gloria.

      {!`}Señor, pequé! Tened misericordia de mí.

      Bendita y alabada sea la Pasión y muerte de nuestro Señor Jesucristo y los Dolores de us Madre María Santísima.

      \vspace{2mm}

      \textbf{Octava Estación:} \textit{Jesucristo habla a las hijas de Jerusalén}.--- {!`}Oh Señor mío Jesucristo, que, a pesa de ser árbol florido y fructuoso, tan duramente fuiste castigado
      por mis culpas!, dame tu santo amor, temor y humilde resignación, para que, pues soy tronco árido y leño seco, sufra lo que tu providencia me envía, que es mucho menos de lo que yo 
      merezco y, sin comparación, menos de los que padeciste Tú por mí. 

      \vspace{2mm}

      Padre nuestro, Ave María y Gloria.

      {!`}Señor, pequé! Tened misericordia de mí.

      Bendita y alabada sea la Pasión y muerte de nuestro Señor Jesucristo y los Dolores de us Madre María Santísima.

      \vspace{2mm}

      \textbf{Novena Estación:} \textit{Jesucristo cae por tercera vez}.--- {!`}Oh Señor mío Jesucristo!, yp te suplico que, si es posible, me libres de las grandes tribulaciones y cruces como la
      que te hizo caer tres veces; mas si tu voluntad me las da y mis pecados las exigen, auxíliame con tu gracia y levántame en mis desmayes con tu amor.

      \vspace{2mm}

      Padre nuestro, Ave María y Gloria.

      {!`}Señor, pequé! Tened misericordia de mí.

      Bendita y alabada sea la Pasión y muerte de nuestro Señor Jesucristo y los Dolores de us Madre María Santísima.

      \vspace{2mm}

      \textbf{Décima Estación:} \textit{Jesucristo es desnudado de sus vestidos}.--- {!`}Oh Señor mío Jesucristo!, suplícote me concedas gran conformidad con la pobreza y profundo desprecio de los
      bienes de esta vida, de modo que, así como dejaste tus vestidos por mí, así yo me despoje, al menos, de los supérfluo y lujoso por Ti y por los pobres.

      \vspace{2mm}

      Padre nuestro, Ave María y Gloria.

      {!`}Señor, pequé! Tened misericordia de mí.

      Bendita y alabada sea la Pasión y muerte de nuestro Señor Jesucristo y los Dolores de us Madre María Santísima.

      \vspace{2mm}

      \textbf{Undécima Estación:} \textit{Jesucristo es crucificado}.--- {!`}Oh Señor mío Jesucristo, aunque estás en la Cruz humillado, ajusticiado, deshecho, eres mi Dios, mi Rey y mi Redentor!
      Como a mi Dios te adoro con viva fe, como a mi Rey te saludo y ofrezco cuanto tengo y poseo, como mi Redentor te amo con toda mi alma y te consagro todo mi corazón. 

      \vspace{2mm}

      Padre nuestro, Ave María y Gloria.

      {!`}Señor, pequé! Tened misericordia de mí.

      Bendita y alabada sea la Pasión y muerte de nuestro Señor Jesucristo y los Dolores de us Madre María Santísima.

      \vspace{2mm}

      \textbf{Duodécima Estación:} \textit{Jesús muere en la Cruz}.--- {!`}Oh Señor mío Jesucristo, que en la Cruz mueres por mi!Más me amaste a mi que a Ti, pues quisiste morir por mí.
      Concédeme vivir y morir por Ti, como Tú viviste y moriste por mí. {!`}Dame una buena muerte! Morir en tu gracia, morir en tu amor, morir en tu voluntad, morir en tu cruz.

      \vspace{2mm}

      Padre nuestro, Ave María y Gloria.

      {!`}Señor, pequé! Tened misericordia de mí.

      Bendita y alabada sea la Pasión y muerte de nuestro Señor Jesucristo y los Dolores de us Madre María Santísima.

      \vspace{2mm}

      \textbf{Decimotercera Estación:} \textit{Jesús es bajado de la Cruz}.--- {!`}Oh Señor mío Jesucristo, muerto y deshecho por mí! Yo venero tu santísimo y divinísimo cuerpo reclinado en
      los brazos de tu piadosísima Madre, y te suplico me concedas un vivo dolor de tanto como a Tí y a tu Madre os hice padecer con mis pecados y gracia para enmendarme de todos ellos.

      \vspace{2mm}

      Padre nuestro, Ave María y Gloria.

      {!`}Señor, pequé! Tened misericordia de mí.

      Bendita y alabada sea la Pasión y muerte de nuestro Señor Jesucristo y los Dolores de us Madre María Santísima.

      \vspace{2mm}

      \textbf{Decimocuarta Estación:} \textit{Jesús es sepultado}.--- {!`}Oh Señor mío Jesucristo!, te suplico me concedas la gracia de morir de tal manera que, por haber participado de tu pasión,
      pueda, al expirar, participar e tu gloria y, en el día del juicio, de tu resurrección. Que tu cruz gobierne mi vida y que tu cruz cobije mi muerte en el sepulcro.

      \vspace{2mm}

      Padre nuestro, Ave María y Gloria.

      {!`}Señor, pequé! Tened misericordia de mí.

      Bendita y alabada sea la Pasión y muerte de nuestro Señor Jesucristo y los Dolores de us Madre María Santísima.

      \vspace{2mm}

      Alma de Cristo, santifícame.

      Cuerpo de Cristo, sálvame.

      Sangre de Cristo, lávame.

      Agua del costado de Cristo, purifícame.

      Pasión de Cristo, confórtame.

      {!`}Oh buen Jesús!, óyeme.

      Dentro de tus llagas escóndeme.
      
      No permitas que me aparte de Ti.

      Del maligno enemigo defiéndeme.

      En la hora de mi muerte, llámame.

      Y mándame ir a Ti.

      Para que con tus santos te alabe.

      Por los siglos de los siglos. Amén.

      %%%%%%%%%%
      % SABADO %
      %%%%%%%%%%

      \begin{center}
            \noindent\textsc{\textcolor{red}{Sabbato (Sábado)}\\ {\large A la Santísima Virgen María}}
      \end{center}

      \hfill\textcolor{red}{Eccli 24, 11-13.15-20}

      \primeraletragranderoja{E}{n}todas las cosas busqué el descanso y en la heredad del Señor fijé mi morada. Entonces el Creador de todas las cosas dió sus órdenes, y me habló; y el que a mi me dió el ser
      descansó en mi tabernáculo, y me dijo: Habita en Jacob, y sea Israel tu herencia, y arráigate en medio de mis escogidos. Y así fijé mi estancia en Sión, y fué el lugar de mi reposo
      la ciudad santa, y en Jerusalén está el trono mío. Y me arraigué en un pueblo glorioso, y en la porción de mi Dios, la cual es su herencia: y mi habitación fué en la plena reunión
      de los santos. Elevada estoy cual cedro sobre el Líbano y cual ciprés sobre el monte Sión. Extendí mis ramas como palma de Cades y como rosal plantado en Jericó; me alcé como hermoso
      olivo en los campos, y como plátano en las plazas junto al agua. Como cinamomo y bálsamo aromático desprendí fragancia. Como mirra exhalé suave olor.

      \vspace{2mm}

      \hfill\textcolor{red}{Lc 1, 26-28.42}

      \primeraletragranderoja{E}{n}aquel tiempo: envió Dios al ángel Gabriel a Nazaret, a una virgen desposada con un varón de nombre José, de la casa de David; el nombre de la virgen era María. 
      Entrando a ella, le dijo: Dios te salve, llena de gracia, el Señor es contigo. {!`}Bendita tú entre las mujeres y bendito el fruto de tu vientre!.

      \vspace{2mm}

      \begin{otherlanguage}{latin}
            \textcolor{red}{P}ater noster, qui es in c{\ae}lis, sanctificétur nomen tuum. Advéniat regnum tuum.
Fiat volúntas tua, sicut in c{\oe}lo et in terra. Panem nóstrum quotidiánum da nobis hódie.
Et dimitte nobis debita nostra, sicut et nos dimittimus debitóribus nostris.
Et ne nos indúcas in tentatiónem: sed libera nos a malo. Amen.

            \vspace{1mm}

            \textcolor{red}{A}ve María, grátia plena, Dóminus tecum; benedicta tu in muliéribus, et benedíctus fructus ventris tui,
Iesus. Sancta Maria, Mater Dei, ora pro nobis peccatóribus, nunc et in hora mortis nostr{\ae}. Amen.

            \vspace{1mm}

            \textcolor{red}{M}emorare, O piissima Virgo Maria, non esse auditum a s{\ae}culo, quemquam ad tua currentem pr{\ae}sidia, tua implorantem auxilia, 
tua petentem suffragia, esse derelictum. Ego tali animatus confidentia, ad te, Virgo Virginum, Mater, curro, ad te venio, coram te gemens 
peccator assisto. Noli, Mater Verbi, verba mea despicere; sed audi propitia et exaudi. Amen.

            \vspace{1mm}

            \textcolor{red}{S}ub tuum pr{\ae}sídium confúgimus, sancta Dei Génetrix; nostras deprecatiónes ne despícias
in necessitátibus; sed a perículis cunctis libera nos semper, Virgo gloriósa et benedícta.

            \vspace{1mm}

            \versiculorespuestaseguido{Ora pro nobis, Sancta Dei Génetrix}{Ut digni efficiámur promissiónibus Christi. Amen}

            \vspace{1mm}

            \versiculorespuestaseguido{Glória Patri, et Filio, et Spirítui Sancto}{Sicut erat in princípio et nunc, et semper et in s{\ae}cula s{\ae}culórum. Amen}

            \vspace{1mm}

            \versiculorespuestaseguido{Ave María puríssima}{Sine labe originali concépta}
      \end{otherlanguage}

      %%%%%%%%%%%
      % DOMINGO %
      %%%%%%%%%%%

      \begin{center}
            \noindent\textsc{\textcolor{red}{Dominica (Domingo)}\\ {\large A la Santísima Trinidad}}
      \end{center}

      Santo, Santo, Santo, Señor Dios de los ejércitos. Llenos están los cielos y la tierra de vuestra gloria. Gloria al Padre, gloria al Hijo, gloria al Espíritu Santo. Amén.

      \vspace{2mm}

      \hfill\textcolor{red}{2 Cor 13, 11.13}

      \primeraletragranderoja{H}{ermanos:}Alegraos, sed perfectos, exhortaos, tened un mismo sentir, vivid en paz; y el Dios de la paz y de la caridad será con vosotros. La gracia de nuestro Señor Jesucristo,
      y la caridad de Dios Padre, y la participación del Espíritu Santo sea con todos vosotros. Amén.

      \vspace{2mm}

      \hfill\textcolor{red}{Rom 11, 33-36}

      \primeraletragranderoja{O}{h}profundidad de las riquezas de la sabiduría y ciencia de Dios! {!`}Cuán inescrutables son sus juicios e incomprensibles sus caminos! Porque {?`}quién conoció los designios del Señor?,
      o {?`}quién primero fué su consejero?, o {?`}quién primero le dió a Él, para que le sea recompensado? Porque de Él y por Él y en Él son todas las cosas; a Él sea gloria por todos los siglos.
      Amén.

      \vspace{2mm}

      \hfill\textcolor{red}{Dan 3, 55-56.52}

      \primeraletragranderoja{B}{endito}eres, Señor, que penetras los abismos y estás sentado sobre Querubines. Bendito eres, Señor, en el firmamento del cielo, y digno de alabanza por todos los siglos
      Aleluya, Aleluya. Bendito eres, Señor Dios de nuestros padres, y digno de alabanza por todos los siglos. Aleluia.

      \vspace{2mm}

      \hfill\textcolor{red}{Mt 28, 18-20}

      \primeraletragranderoja{D}{ijo}Jesús a sus discípulos: Se me ha dado poder en el cielo y en la tierra. Id, pues, y enseñad a todas las gentes, bautizándolas en el nombre del Padre y del Hijo y del espíritu
      Santo; enseñándoles a observar todo cuanto os he mandado. Y mirad que Yo estoy con vosotros todos los días hasta el fin del mundo.

      \vspace{2mm}

      \begin{otherlanguage}{latin}
            \textcolor{red}{P}ater noster, qui es in c{\ae}lis, sanctificétur nomen tuum. Advéniat regnum tuum.
Fiat volúntas tua, sicut in c{\oe}lo et in terra. Panem nóstrum quotidiánum da nobis hódie.
Et dimitte nobis debita nostra, sicut et nos dimittimus debitóribus nostris.
Et ne nos indúcas in tentatiónem: sed libera nos a malo. Amen.

            \vspace{1mm}

            \textcolor{red}{A}ve María, grátia plena, Dóminus tecum; benedicta tu in muliéribus, et benedíctus fructus ventris tui,
Iesus. Sancta Maria, Mater Dei, ora pro nobis peccatóribus, nunc et in hora mortis nostr{\ae}. Amen.

            \vspace{1mm}

            \versiculorespuestaseguido{Glória Patri, et Filio, et Spirítui Sancto}{Sicut erat in princípio et nunc, et semper et in s{\ae}cula s{\ae}culórum. Amen}
      \end{otherlanguage}

      \vspace{2mm}

      A vos, Dios Padre ingénito; a vos Hijo unigénito; a vos, Espíritu Santo Paráclito, santa e individua Trinidad, de todo corazón os confesamos, alabamos y bendecimos. A vos se dé
      la gloria por los siglos de los siglos.

      \vspace{2mm}

      \versiculorespuestaseguido{Bendigamos al Padre y al Hijo y al Espíritu Santo}{Alabémosle y ensalcémosle en todos los siglos}

      \vspace{2mm}

      \primeraletragrande{O}{mnipotente}y sempiterno Dios, que nos has concedido a tus siervos el don de conocer la gloria de la eterna Trinidad en la confesión de la verdadera fe,
      y la de adorar la Unidad en el poder de tu majestad: te rogamos que, por la firmeza de esta misma fe, nos libremos siempre de todas las adversidades. Por Cristo nuestro Señor.
      Amén.

\end{multicols}
\end{document}


