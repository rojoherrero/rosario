% Lc 23, 33-34; Ioan 19, 19.25-27; Lc 23, 44-46
\noindent\textbf{\textsc{V El Señor muere en la Cruz}}\hfill\textcolor{red}{Lc 23, 33-34.39-46}

\vspace{0.25em}

\lettrine[lines=2]{\textcolor{red}{Y}} cuando hibieron llegado al lugar llamado <<Cráneo>>, allí crucificaron a El y a los malhechores, uno a la derecha y otro a la izquierda.
Y Jesús decía: Padre, perdónales, porque no saben lo que hacen. Uno de los malhechores que estaban colgados le insultaba, diciendo: {?`}No eres tú el Mesías? Sálvate a tui mismo
y a nosotros. Mas el otro, respondiendo, le reconvenía, diciendo: {?`}Ni siquiera temes tú a Dios, estando en el mismo suplicio? Nosotros, a la verdad, lo estamos justamente,
pues recibimos el justo pago de lo que hicimos; mas éste nada inconveniente ha hecho. Y decía a Jesús: Acuérdate de mi cuando vinieres en la gloria de tu realeza. Díjole [Jesús]:
En verdad te digo que hoy estarás en el paraíso. Y era ya como la hora sexta, y se produjeron tinieblas sobre toda la tierra hasta la hora nona, habiendo faltado el sol; y se rasgó
por medio el velo del santuario. Y clamando con voz poderosa, Jesús dijo: Padre, en tus manos encomiendo mis espíritu. Y, dicho esto, expiró