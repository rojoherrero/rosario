\documentclass[./rosary.tex]{subfiles}
\newcounter{lux-counter}

\begin{document}
\section*{Misterios Luminosos.}
\begin{itemize}
      \item Tradicional: -
      \item Nuevo: jueves.
\end{itemize}

\stepcounter{lux-counter}
\subsection*{\Roman{lux-counter} Misterio: El Bautismo del Señor en el Jordán}

Y aconteció por aquellos días que vino Jesús desde Nazaret de Galilea y fué bautizado en el Jordán por Juan el Bautista. 
Y al punto subiendo del agua, vió rasgarse los cielos y venir sobre Él el Espíritu Santo como paloma; 
y una voz vino de los cielos: «Tú eres mi Hijo amado, en Ti me agradé». 
\textbf{\emph{Marcos 1, 9-11}}

\begin{center}
      Paternóster, diez Avemarías, Gloria y María, Madre de gracia{\ldots}
\end{center}

\bigskip

\stepcounter{lux-counter}
\subsection*{\Roman{lux-counter} Misterio: Su Autoreveñación en las Bodas de Caná}

Al tercer día hubo una boda en Caná de Galilea, y estaba allí la madre de Jesús. Fue invitado también Jesús con sus discípulos a la boda. 
No tenían vino, porque el vino de la boda se había acabado. En esto dijo la madre de Jesús a este: No tiene vino. 
Díjole Jesús: Mujer, ¿qué nos va a mi y a ti? No es aún llegada mi hora. Dijo la madre a los servidores: Haced lo que Él os diga. 
\textbf{\emph{Juan 2, 1-5}}

\begin{center}
      Paternóster, diez Avemarías, Gloria y María, Madre de gracia{\ldots}
\end{center}

\bigskip

\stepcounter{lux-counter}
\subsection*{\Roman{lux-counter} Misterio: El Anuncio del Rey de Dios}

Después que Juan fue preso vino Jesús a Galilea predicando el Evangelio de Dios y diciendo: Cumplido es el tiempo, 
y el reino de Dios está cercano; arrepentíos y creed en el Evangelio. Llegaron a Cafarnaúm, y luego, el día sábado, 
entrando en la sinagoga, enseñaba. Se maravillaban de su doctrina, pues la enseñaba como quien tiene autoridad y 
no como los escribas. 
\textbf{\emph{Marcos 1, 14-15.21-22}}

\begin{center}
      Paternóster, diez Avemarías, Gloria y María, Madre de gracia{\ldots}
\end{center}

\bigskip

\stepcounter{lux-counter}
\subsection*{\Roman{lux-counter} Misterio: La Transfiguración}

Seis días después tomó Jesús a Pedro, a Santiago y a Juan, su hermano, y los llevo aparte, a un monte alto. 
Y se transfiguró ante ellos; brilló su rostro como el sol y sus vestidos se volvieron blancos como la luz. Aún estaba el hablando, 
cuando los cubrió una nobe resplandeciente, y salió de la nube una voz que decía: Este es mi Hijo amado, 
en quien tengo mi complacencia; escichadle. \textbf{\emph{Mateo 17, 1-3.5}}

\begin{center}
      Paternóster, diez Avemarías, Gloria y María, Madre de gracia{\ldots}
\end{center}

\bigskip

\stepcounter{lux-counter}
\subsection*{\Roman{lux-counter} Misterio: La Institución de la Eucaristía}

Mientras comían, Jesús tomó pan, lo bendijo, lo partió y, dándoselo a los discípulos, dijo: Tomad y comed, éste es mi cuerpo. 
Y tomando un cáliz y dando gracias, se lo dió, diciendo: Bebed de él todos, que está es mi sangre del Nuevo Testamento, 
que será derramada por muchos para remisión de los pecados. \textbf{\emph{Mateo 26, 26-28}}

\begin{center}
      Paternóster, diez Avemarías, Gloria y María, Madre de gracia{\ldots}
      
      Oraciones finales (\cpageref{sec:final-prayer}).
\end{center}

\end{document}
