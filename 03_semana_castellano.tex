\documentclass[10pt]{article}
\usepackage[utf8]{inputenc}
\usepackage[latin,spanish]{babel}
%\usepackage[spanish]{babel}
\usepackage[T1]{fontenc}
\usepackage[a4paper, margin=10mm]{geometry}
\usepackage{lettrine}
\usepackage{xcolor}
\usepackage{yfonts}
\usepackage{gregoriotex}
\usepackage{multicol}
\usepackage{enumitem}
%\setlength\parindent{0pt}
\pagenumbering{gobble}

\setlength{\columnseprule}{0.5pt}
\def\columnseprulecolor{\color{red}}

\newcommand{\primeraletragranderoja}[2]{
    \lettrine[lines=2]{\textcolor{red}{#1}}{#2}
}

\newcommand{\primeraletragranderojasola}[1]{
    \lettrine[lines=2]{\textcolor{red}{#1}}{}
}

\newcommand{\primeraletragrande}[2]{
    \lettrine[lines=2]{#1}#2
}

\newcommand{\letraroja}[2]{
    \textcolor{red}{#1}#2
}

\newcommand{\versiculo}[1]{
    \textcolor{red}{\Vbar.} #1.
}

\newcommand{\respuesta}[1]{
    \textcolor{red}{\Rbar.} #1.
}

\newcommand{\textopequenorojo}[1]{
    {\textcolor{red}{\small{#1}}}
}

\newcommand{\versiculorespuesta}[2]{
    \versiculo{#1}\\
    \respuesta{#2}
}

\newcommand{\versiculorespuestaseguido}[2]{
    \versiculo{#1}\respuesta{#2}
}

\newcommand{\redcross}{
    \textcolor{red}{\grecross}
}

\newcommand{\lineahorizontal}[2]{
    \begin{center}
        {\rule{#1cm}{#2pt}}
    \end{center}
}

\newcommand{\lineahorizontalroja}[2]{
    \begin{center}
        \textcolor{red}{\rule{#1cm}{#2pt}}
    \end{center}
}

\newcommand{\latinderecha}[1]{
    \begin{otherlanguage}{latin}
        \begin{rightcolumn}
            \input{#1}    
        \end{rightcolumn}
    \end{otherlanguage}
}

\newcommand{\castellanoizquierda}[1]{
    \begin{leftcolumn}
        \input{#1}
    \end{leftcolumn}
}

\newcommand{\castellanoizquierdasincro}[1]{
    \begin{leftcolumn*}
        \input{#1}
    \end{leftcolumn*}
}

\newcommand{\castellanoizquierdasincronota}[2]{
    \begin{leftcolumn*}[#1]
        \input{#2}
    \end{leftcolumn*}
}

\newcommand{\filacastellanolatin}[2]{
    {    
        \begin{leftcolumn}
            \input{#1}
        \end{leftcolumn}
        \begin{otherlanguage}{latin}
            \begin{rightcolumn}
                \input{#2}    
            \end{rightcolumn}
        \end{otherlanguage}
    }
}

\newcommand{\filacastellanolatinsincro}[2]{
    {    
        \begin{leftcolumn*}
            \input{#1}
        \end{leftcolumn*}
        \begin{otherlanguage}{latin}
            \begin{rightcolumn}
                \input{#2}    
            \end{rightcolumn}
        \end{otherlanguage}
    }
}

\newcommand{\filacastellanolatinsincronota}[3]{
    \castellanoizquierdasincronota{#1}{#2}
    \latinderecha{#3}
}

\newcommand{\titulomisterios}[2]{
    {    
        \begin{minipage}[t]{0.595\textwidth}
            \subsection*{#1}
        \end{minipage}\begin{minipage}[t]{0.395\textwidth}
            \begin{flushright}
                \textcolor{red}{#2}
            \end{flushright}
        \end{minipage}
    }
}

\newcommand{\titulomisterio}[2]{
    {
        \begin{minipage}[t]{0.595\textwidth}
            \section*{#1}
        \end{minipage}\begin{minipage}[t]{0.395\textwidth}
            \begin{flushright}
                \textcolor{red}{#2}
            \end{flushright}
        \end{minipage}
    }
}

\newcommand{\iralfinal}{
    \begin{center}
        \textcolor{red}{Una vez terminamos nos vamos a la \cpageref{final-prayer} para las oraciones finales.}
    \end{center}
}




\begin{document}

\begin{center}
      \textsc{\Large La Semana Santificada}
\end{center}

\begin{multicols}{2}

      \begin{center}
            \textsc{\textcolor{red}{Feria II (Lunes)}\\ {\large A las Benditas almas del Purgatorio}}
      \end{center}

      \hfill\textcolor{red}{Apoc 14, 13}

      En aquellos días: Oí la voz del cielo, que me decía: Escribe: Bienaventurados los muertos que mueren en el Señor. Ya desde ahora dice el Espíritu que descansen
      de sus trabajos, puesto que sus obras los van acompañando.\newline
      
      \hfill\textcolor{red}{Ioann 6, 37-40}

      En aquel tiempo: Dijo Jesús a las turbas de los judíos: Todos los que el Padre me da, vendrán a mi; y al que viniere a mi, no le desecharé; porque he descendido
      del cielo, no para hacer mi voluntad, sino la de aquel que me ha enviado. Y la voluntad de mi Padre, que me ha enviado, es que yo no pierda ninguno de los que
      me ha dado, sino que los resucite a todos en el último día. Por tanto, la voluntad de mi Padre, que me ha enviado, es que todo el que ve al Hijo, y cree en él,
      tenga vida eterna, y yo le resucitaré en el último día.\newline
      
      \begin{otherlanguage}{latin}
            \versiculorespuestaseguido{Oremos por los fieles difuntos}
    {Concedeles el descanso eterno y brille para ellos laa luz perpétua}

            \versiculorespuestaseguido{Descansen en paz}{Amén}

            \textcolor{red}{P}adre nuestro, que estás en los cielos, santificado sea tu Nombre. Venga a nosotros tu Reino.
Hágase tu voluntad, así en la tierra como en el cielo. El pan nuestro de cada día dánosle hoy.
Y perdónamos nuestras deuda, así como nosotros perdonamos a nuestros deudores.
Y no nos dejes caer en la tentación: mas líbranos del mal. Amén.

            \textcolor{red}{D}ios te salve, María, llena eres de gracia, el Señor es contigo; bendita eres entre todas las mujeres,
y bendito es el fruto de tu vientre, Jesús. Santa María, Madre de Dios, ruega por nosotros pecadores,
ahora u en el hora de nuestra muerte. Amén.

            \versiculorespuestaseguido{Gloria al Padre, al Hijo, y al Espíritu Santo}{Como era en el principio, ahora, y siempre, y por los siglos de los siglos. Amén}
      \end{otherlanguage}

      \begin{center}
            \textsc{\textcolor{red}{Feria III (Martes)}\\ {\large Al Ángel de la Guarda}}
      \end{center}

      \hfill\textcolor{red}{Ps 90, 10-12}

      No se te acercará mal alguno, y no se allegará a tu morada la desgracia. Pues dio a sus ángeles órdenes acerca de ti, para que te guarden en todos tus pasos.
      Te llevarán en las palmas de sus manos, para que tu pie no tropiece en alguna piedra.\newline
      
      \hfill\textcolor{red}{Gen 23, 20-22}
      
      Yo mandaré a un ángel ante ti, para que te defienda en el camino y te haga llegar al lugar que te he dispuesto. Acátale, y escucha su voz,
      no le resista, porque no perdonará vuestras rebeliones y porque lleva mi nombre. Pero si le escuchas y haces cuanto él te diga, yo seré
      enemigo de tus enemigos y afligiré a los que te aflijan.\newline

      \versiculorespuestaseguido{Dios mío, os cantaré en la presencia de vuestros Ángeles}{Os adoraré en vuestro santo templo y cantaré a vuestro nombre}

      \begin{otherlanguage}{latin}
            \textcolor{red}{P}adre nuestro, que estás en los cielos, santificado sea tu Nombre. Venga a nosotros tu Reino.
Hágase tu voluntad, así en la tierra como en el cielo. El pan nuestro de cada día dánosle hoy.
Y perdónamos nuestras deuda, así como nosotros perdonamos a nuestros deudores.
Y no nos dejes caer en la tentación: mas líbranos del mal. Amén.

            \textcolor{red}{D}ios te salve, María, llena eres de gracia, el Señor es contigo; bendita eres entre todas las mujeres,
y bendito es el fruto de tu vientre, Jesús. Santa María, Madre de Dios, ruega por nosotros pecadores,
ahora u en el hora de nuestra muerte. Amén.

            \versiculorespuestaseguido{Gloria al Padre, al Hijo, y al Espíritu Santo}{Como era en el principio, ahora, y siempre, y por los siglos de los siglos. Amén}

            \textcolor{red}{Á}ngel de Dios, bajo cuya custodia me puso el Señor con amorosa piedad, a mi, que soy vuestro encomendado, alumbradme,
guardadme, regidme y gobernadme. Amén.
      \end{otherlanguage}
      \newline

      \textbf{Oremos:} Oh Dios, que con admirable providencia os habéis dignado enviarnos a vuestro santos Ángeles para que nos guarden; concedenos, humildemente os pedimos,
      que seamos siempre defendidos con su protección y gocemos un día eternamente de su compañía en el Cielo. Por Jesucristo nuestro Señor. Amén.

      \begin{center}
            \textsc{\textcolor{red}{Feria IV (Miércoles)}\\ {\large Al Glorioso Patriarca San José}}
      \end{center}

      {!`}Oh gloriosísimo Padre de Jesús, Esposo de María, Patriarca y Protector de la santa Iglesia, a quien el Padre Eterno confió el cuidado de gobernar, regir y defender en la tierra
      la Sagrada Familia! Protégenos también a nosotros, que pertenecemos. como fieles católicos, a la santa familia de tu Hijo, que es la Iglesia,y alcánzanos los bienes necesarios de esta
      vida, y sobre todo los auxilios espirituales para la vida eterna. Alcánzanos especialmente estas tres gracias: la de no cometer jamás ningún pecado mortal, principalmente contra la castidad;
      la de un sincero amor y devoción a Jesús y María, y la de una buena muerte, recibiendo bien los últimos sacramentos. Amén.\newline

      \hfill\textcolor{red}{Matth 1, 18-21}

      Estando desposada la MAdre de Jesús, María, con José, sin que antes hubiesen estado juntos, se halló que había concebido en su seno por obra del Espíritu Santo. Mas José, su esposo,
      siendo como era justo, y no queriendo infamarla, deliberó dejarla secretamente. Estando él en este pensamiento, he aquí que un ángel del Señor se le apareció en suerños, diciendo:
      José, hijo de David, no tengas recelos en recibir a María tu esposa, porque lo que se ha engendrado en su seno, es obra del Espíritu Santo. ASí que dará a luz un hijo, a quien pondrás
      el nombre de Jesús, pues él es el que ha de salvar a su pueblo de sus pecados.\newline

      \begin{otherlanguage}{latin}
            \textcolor{red}{P}adre nuestro, que estás en los cielos, santificado sea tu Nombre. Venga a nosotros tu Reino.
Hágase tu voluntad, así en la tierra como en el cielo. El pan nuestro de cada día dánosle hoy.
Y perdónamos nuestras deuda, así como nosotros perdonamos a nuestros deudores.
Y no nos dejes caer en la tentación: mas líbranos del mal. Amén.

            \textcolor{red}{D}ios te salve, María, llena eres de gracia, el Señor es contigo; bendita eres entre todas las mujeres,
y bendito es el fruto de tu vientre, Jesús. Santa María, Madre de Dios, ruega por nosotros pecadores,
ahora u en el hora de nuestra muerte. Amén.

            \versiculorespuestaseguido{Gloria al Padre, al Hijo, y al Espíritu Santo}{Como era en el principio, ahora, y siempre, y por los siglos de los siglos. Amén}

            \versiculorespuestaseguido{San José, ruega por nosotros}{Para que seamos dignos de alcanzar las promesas de Cristo. Amén}
      \end{otherlanguage}
      \newline

      Oh benignísimo Jesús: así como tu amado padre te condujo de Belén a Egipto para librarte del tirano Herodes, así te suplicamos humildemente, por intercesión de San José, que nos libres
      de los que quieren dañar nuestras almas o nuestros cuerpos, nos des fortaleza y salvación en nuestras persecuciones, y en medio del desierto de esta vida nos protejas hasta que volvamos
      a la Patria Celestial. Amén.

      \begin{center}
            \textsc{\textcolor{red}{Feria V (Jueves)}\\ {\large Al Santísimo Sacramento del altar}}
      \end{center}

      \hfill\textcolor{red}{1 Corint 11, 23-29}

      Hermanos: Yo recibí del Señor lo que también os tengo enseñado, y es que el Señor Jesús la noche misma en que había de ser entregado, tomó el pan, y dando gracias, lo partió, y dijo:
      Tomad y comed; este es mi cuerpo, que por vosotros será entregado; haced esto en memoria mía. Y de la misma manera tomó también el cáliz después de haber cenado, diciendo: Este es el
      Nuevo Testamento en mi sangre. Haced esto, cuantas veces le bebiereis, en memoria mía. Pues todas las veces que comiereis este pan, y bebiereis este cáliz, anunciaréis la muerte del
      Señor hasta que venga. De manera que cualquiera que comiere este pan, o bebiere el cáliz del Señor indignamente, reo será del Cuerpo y la Sangre del Señor. Por lo tanto, examínese a
      sí mismo el hombre; y de esta suerte, coma del aquel pan y beba de aquel cáliz; porque quien le come y bebe indignamente, se traga y bebe su propia condenación, no haciendo el debido
      discernimiento del cuerpo del Señor.\newline

      \hfill\textcolor{red}{Ioann 6, 56-59}

      En aquel tiempo: Dijo Jesús a las turbas de los judíos: Mi carne verdaderamente es comida, y mi Sangre es verdaderamente bebida. Quien como mi carne y bebe mi Sangre, en mi mora y yo en él.
      Así como el Padre, que me ha enviado, vive, y yo vivo por mi Padre; así quien me come, también él vivirá por mi. Este es el Pan que ha bajado del Cielo. No sucederá como a vuestros padres,
      que comieron el maná y, no obstante esto, murieron. Quien come este pan vivirá eternamente.\newline

      \begin{otherlanguage}{latin}
            \textcolor{red}{P}adre nuestro, que estás en los cielos, santificado sea tu Nombre. Venga a nosotros tu Reino.
Hágase tu voluntad, así en la tierra como en el cielo. El pan nuestro de cada día dánosle hoy.
Y perdónamos nuestras deuda, así como nosotros perdonamos a nuestros deudores.
Y no nos dejes caer en la tentación: mas líbranos del mal. Amén.

            \textcolor{red}{D}ios te salve, María, llena eres de gracia, el Señor es contigo; bendita eres entre todas las mujeres,
y bendito es el fruto de tu vientre, Jesús. Santa María, Madre de Dios, ruega por nosotros pecadores,
ahora u en el hora de nuestra muerte. Amén.

            \versiculorespuestaseguido{Gloria al Padre, al Hijo, y al Espíritu Santo}{Como era en el principio, ahora, y siempre, y por los siglos de los siglos. Amén}
      \end{otherlanguage}
      \newline

      \textbf{Oremos:} Oh Dios, que dejasteis la memoria de vuestra Pasión en ese Sacramento admirable: concedednos que de tal suerte veneremos los sagrados misterios de vuestro Cuerpo 
      y vuestra Sangre, que experimentemos continuamente en nuestras almas el fruto de vuestra redención: Vos que vivís y reináis con Dios Padre, en la unidad del Espíritu Santo Dios, 
      por los siglos de los siglos, Amén.

      \begin{center}
            \textsc{\textcolor{red}{Feria VI (Viernes)}\\ {\large A la Pasión de N.S. Jesucristo}}
      \end{center}

      \hfill\textcolor{red}{Isai 53, 2-6}

      Sube ante El como un retoño, como retoño de raíz en tierra árida. No hay en él parecer, no hay hermosura que atraiga las miradas, no hay en él belleza que agrade. Despreciado, 
      desecho de los hombres, varón de dolores, conocedor de todos los quebrantos, ante quien se vuelve el rostro, menospreciado, estimado en nada. Pero fue él, ciertamente, quien tomó 
      sobre sí nuestras enfermedades y cargó con nuestros dolores, y nosotros le tuvimos por castigo y herido por Dios y humillado. Fue traspasado por nuestras iniquidades y molido por 
      nuestros pecados. El castigo salvador pesó sobre él, y en sus llagas hemos sido curados. Todos nosotros andábamos errantes, como ovejas, y Yavé cargó sobre él la iniquidad de 
      todos nosotros.\newline

      \hfill\textcolor{red}{Marc 14, 32-38}

      Llegaron a un lugar cuyo nombre era Getsemaní, y dijo a sus discípulos: Sentaos aquí mientras voy a orar. Tomando consigo a Pedro, a Santiago y a Juan, comenzó a sentir temor y angustia,
      y les dicia: Triste está mi alma hasta la muerte; permaneced aquí y velad. Adelantándose un poco, cayó en tierra y oraba que, si era posible, pasase de Él aquella hora. Decía: Abba, Padre,
      todo te es posible; aleja de mi este cáliz; mas no se haga lo que yo quiero, sino lo que quieres tú. Vino y los encontró dormidos, y dijo a Pedro: Simón, {?`}Duermes?{?`}No has podido
      velar una hora? Velad y orad para que no entréis en tentación; el espíritu está pronto, mas las carne es flaca.\newline

      \begin{otherlanguage}{latin}
            \textcolor{red}{P}adre nuestro, que estás en los cielos, santificado sea tu Nombre. Venga a nosotros tu Reino.
Hágase tu voluntad, así en la tierra como en el cielo. El pan nuestro de cada día dánosle hoy.
Y perdónamos nuestras deuda, así como nosotros perdonamos a nuestros deudores.
Y no nos dejes caer en la tentación: mas líbranos del mal. Amén.

            \textcolor{red}{D}ios te salve, María, llena eres de gracia, el Señor es contigo; bendita eres entre todas las mujeres,
y bendito es el fruto de tu vientre, Jesús. Santa María, Madre de Dios, ruega por nosotros pecadores,
ahora u en el hora de nuestra muerte. Amén.

            \versiculorespuestaseguido{Gloria al Padre, al Hijo, y al Espíritu Santo}{Como era en el principio, ahora, y siempre, y por los siglos de los siglos. Amén}
      \end{otherlanguage}

      \begin{center}
            \noindent\textsc{\textcolor{red}{Sabbato (Sábado)}\\ {\large A la Santísima Virgen María}}
      \end{center}

      \hfill\textcolor{red}{Eccli 24, 11-13.15-20}

      En todas las cosas busqué el descanso y en la heredad del Señor fijé mi morada. Entonces el Creador de todas las cosas dió sus órdenes, y me habló; y el que a mi me dió el ser
      descansó en mi tabernáculo, y me dijo: Habita en Jacob, y sea Israel tu herencia, y arráigate en medio de mis escogidos. Y así fijé mi estancia en Sión, y fué el lugar de mi reposo
      la ciudad santa, y en Jerusalén está el trono mío. Y me arraigué en un pueblo glorioso, y en la porción de mi Dios, la cual es su herencia: y mi habitación fué en la plena reunión
      de los santos. Elevada estoy cual cedro sobre el Líbano y cual ciprés sobre el monte Sión. Extendí mis ramas como palma de Cades y como rosal plantado en Jericó; me alcé como hermoso
      olivo en los campos, y como plátano en las plazas junto al agua. Como cinamomo y bálsamo aromático desprendí fragancia. Como mirra exhalé suave olor.\newline

      \hfill\textcolor{red}{Luc 1, 26-28.42}

      En aquel tiempo: envió Dios al ángel Gabriel a Nazaret, a una virgen desposada con un varón de nombre José, de la casa de David; el nombre de la virgen era María. 
      Entrando a ella, le dijo: Dios te salve, llena de gracia, el Señor es contigo. {!`}Bendita tú entre las mujeres y bendito el fruto de tu vientre!.\newline

      \begin{otherlanguage}{latin}
            \textcolor{red}{P}adre nuestro, que estás en los cielos, santificado sea tu Nombre. Venga a nosotros tu Reino.
Hágase tu voluntad, así en la tierra como en el cielo. El pan nuestro de cada día dánosle hoy.
Y perdónamos nuestras deuda, así como nosotros perdonamos a nuestros deudores.
Y no nos dejes caer en la tentación: mas líbranos del mal. Amén.

            \textcolor{red}{D}ios te salve, María, llena eres de gracia, el Señor es contigo; bendita eres entre todas las mujeres,
y bendito es el fruto de tu vientre, Jesús. Santa María, Madre de Dios, ruega por nosotros pecadores,
ahora u en el hora de nuestra muerte. Amén.

            \versiculorespuestaseguido{Gloria al Padre, al Hijo, y al Espíritu Santo}{Como era en el principio, ahora, y siempre, y por los siglos de los siglos. Amén}

            \textcolor{red}{A}cordáos, {!`}oh piadosísima Virgen María!, que jamás se ha oído decir que ninguno de los que han acudido a vuestra protección, 
implorando vuestro auxilio, haya sido desamparado. Animado por esta confianza, a Vos acudo, Madre, Virgen de las vírgenes, y gimiendo 
bajo el peso de mis pecados me atrevo a comparecer ante Vos. Madre de Dios, no desechéis mis súplicas, antes bien, escuchadlas y 
acogedlas benignamente.

            \textcolor{red}{B}ajo tu amparo nos acogemos, Santa Madre de Dios; no deseches las súplicas que te dirigimos en nuestras necesidades;
antes bien, líbranos siempre de todo peligro, {!`}Oh Virgen gloriosa y bendita!

            \versiculorespuestaseguido{Ruega por nos, Santa Madre de Dios}{Para que seamos dignos de alcanzar la promesas Cristo. Amén}

            \versiculorespuestaseguido{Ave María purísima}{Sin pecado concebida}
      \end{otherlanguage}

      \begin{center}
            \noindent\textsc{\textcolor{red}{Dominica (Domingo)}\\ {\large A la Santísima Trinidad}}
      \end{center}

      Santo, Santo, Santo, Señor Dios de los ejércitos. Llenos están los cielos y la tierra de vuestra gloria. Gloria al Padre, gloria al Hijo, gloria al Espíritu Santo. Amén.\newline

      \hfill\textcolor{red}{2 Corint 13, 11.13}

      Hermanos: Alegraos, sed perfectos, exhortaos, tened un mismo sentir, vivid en paz; y el Dios de la paz y de la caridad será con vosotros. La gracia de nuestro Señor Jesucristo,
      y la caridad de Dios Padre, y la participación del Espíritu Santo sea con todos vosotros. Amén.\newline

      \hfill\textcolor{red}{Rom 11, 33-36}

      {!`}Oh profundidad de las riquezas de la sabiduría y ciencia de Dios! {!`}Cuán inescrutables son sus juicios e incomprensibles sus caminos! Porque {?`}quién conoció los designios del Señor?,
      o {?`}quién primero fué su consejero?, o {?`}quién primero le dió a Él, para que le sea recompensado? Porque de Él y por Él y en Él son todas las cosas; a Él sea gloria por todos los siglos.
      Amén.\newline

      \hfill\textcolor{red}{Dan 3, 55-56.52}

      Bendito eres, Señor, que penetras los abismos y estás sentado sobre Querubines.\versiculo{Bendito eres, Señor, en el firmamento del cielo, y digno de alabanza por todos los siglos}
      Aleluia, Aleluia.\versiculo{Bendito eres, Señor Dios de nuestros padres, y digno de alabanza por todos los siglos. Aleluia.}\newline

      \hfill\textcolor{red}{Matth 28, 18-20}

      Dijo Jesús a sus discípulos: Se me ha dado poder en el cielo y en la tierra. Id, pues, y enseñad a todas las gentes, bautizándolas en el nombre del Padre y del Hijo y del espíritu
      Santo; enseñándoles a observar todo cuanto os he mandado. Y mirad que Yo estoy con vosotros todos los días hasta el fin del mundo.\newline

      \begin{otherlanguage}{latin}
            \textcolor{red}{P}adre nuestro, que estás en los cielos, santificado sea tu Nombre. Venga a nosotros tu Reino.
Hágase tu voluntad, así en la tierra como en el cielo. El pan nuestro de cada día dánosle hoy.
Y perdónamos nuestras deuda, así como nosotros perdonamos a nuestros deudores.
Y no nos dejes caer en la tentación: mas líbranos del mal. Amén.

            \textcolor{red}{D}ios te salve, María, llena eres de gracia, el Señor es contigo; bendita eres entre todas las mujeres,
y bendito es el fruto de tu vientre, Jesús. Santa María, Madre de Dios, ruega por nosotros pecadores,
ahora u en el hora de nuestra muerte. Amén.

            \versiculorespuestaseguido{Gloria al Padre, al Hijo, y al Espíritu Santo}{Como era en el principio, ahora, y siempre, y por los siglos de los siglos. Amén}
      \end{otherlanguage}
      \newline

      A vos, Dios Padre ingénito; a vos Hijo unigénito; a vos, Espíritu Santo Paráclito, sata e individua Trinidad, de todo corazón os confesamos, alabamos y bendecimos. A vos se dé
      la gloria por los siglos de los siglos.

      \versiculorespuestaseguido{Bendigamos al Padre y al Hijo y al Espíritu Santo}{Alabémosle y ensalcémosle en todos los siglos}\newline

      \textbf{Oremos:} Omnipotente y sempiterno Dios, que nos has concedio a tus siervos el don de conocer la gloria de la eterna Trinidad en la confesión de la verdadera fe,
      y la de adorar la Unidad en el poder de tu majestad: te rogamos que, por la firmeza de esta misma fe, nos libremos siempre de todas las adversidades. Por Cristo nuesrto Señor.
      Amén.

\end{multicols}
\end{document}


