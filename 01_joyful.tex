\documentclass[./rosary.tex]{subfiles}
\newcounter{joyful-counter}

\begin{document}
\section*{Misterios Gozosos}

\begin{itemize}
      \item Tradicional: lunes, jueves y domingos.
      \item Nuevo: lunes y sábados
\end{itemize}

\stepcounter{joyful-counter}
\subsection*{\Roman{joyful-counter} Misterio: La Anunciación de la Santísima Virgen María}
Y habiendo entrado a ella, dijo: «Dios te salve, llena de gracia, el Señor es contigo, bendita tu entre las mujeres».
Ella, al oír estas palabras, se turbó, y discurría que podría ser esta salucatión. Y le dijo el ángel: «No temas, María,
pues hallaste gracia a los ojos de Dios. He aquí que concebirás en tu seno y darás a luz un Hijo, a quien daras por
nombre Jesús. Este será grande, y será llamado Hijo del Altísimo, y le dará el Señor Dios el trono de DAvid su padre,
y reinará sobre la casa de Jacob etérnamete, y su reinado no tendrá fin». \textbf{\emph{Lucas 1, 27-33}}

\begin{center}
      Paternóster, diez Avemarías, Gloria y María, Madre de gracia{\ldots}
\end{center}
\bigskip

\stepcounter{joyful-counter}
\subsection*{\Roman{joyful-counter} Misterio: La Visitación de Nuestra Señora}

Por aquellos días, levantándose María, se dirigió presurosa a la montaña, a un ciudad de Judá, y entró en la casa de Zacarías y saludó a Isabel.
Y aconteció que, al oir Isabel la salutación de María, dió saltos de gozo el niño en su seno, y fue llena Isabel del Espíritu Santo,
y levantó la voz con gran clamor y dijo: «Bendita tu entre las mujeres y bendito el fruto de tu vientre. ¿Y de dónde a mí esto que venga la madre de mi Señor a mí?
Porque he aquí que, como sonó la voz de tu salutación en mi oídos, dió saltos de alborozo el niño en mi seno. 
\textbf{\emph{Lucas 1, 39-45}}

\begin{center}
      Paternóster, diez Avemarías, Gloria y María, Madre de gracia{\ldots}
\end{center}

\bigskip

\stepcounter{joyful-counter}
\subsection*{\Roman{joyful-counter} Misterio: La Natividad del Nuestro Señor Jesucristo}

Y dió a luz su hijo primogénito, y le envolvió en pañales y le recostó en un pesebre, pues no había para ellos lugar en el mesón.
Y había unos pastores en aquella misma comarca, que pernoctaban al raso y velaban por turno para guardar su ganado,
y un ángel del  Señor se presentó ante ellos. Y les dijo el Ángel: «No tenáis, pues he aquí que os traigo una buena nueva,
que será de grande alegría para todo el pueblo: que os ha nacido hoy en la ciudad de David un Salvador, que es el Mesías, el Señor». 
\textbf{\emph{Lucas 2, 7-8.10-11}}

\begin{center}
      Paternóster, diez Avemarías, Gloria y María, Madre de gracia{\ldots}
\end{center}

\bigskip

\stepcounter{joyful-counter}
\subsection*{\Roman{joyful-counter} Misterio: La Purificación de nuestra Serñora y La Presentación del Niño Jesús en el Templo}

Y cuando se les cumplieron los días de la purificación según la ley de Moisés (Levítico 12, 6),
le subieron a Jerusalen para presentarle al Señor, según está escrito en la Ley del Señor que «todo primogénito
del sexo masculino será consagrado al Señor\footnote{Éxodo 13, 2; 12, 15\label{primogenito}}», y para ofrecer como sacrificio,
según lo que se ordena en la Ley del Señor, «un par de tórtolas o dos palominos\footnote{Levítico 12, 8; 5, 11\label{sacrificio}}». 
\textbf{\emph{Lucas 2, 22-24}}

\begin{center}
      Paternóster, diez Avemarías, Gloria y María, Madre de gracia{\ldots}
\end{center}

\bigskip

\stepcounter{joyful-counter}
\subsection*{\Roman{joyful-counter} Misterio: La pérdida y hallazgo del Niño Jesús en el Templo}

Y no hallándole, se tornaron a Jerusalén para buscarle. Y sucedió que después de tres días le hallaron en el templo,
sentado en medio de los maestros, escuchándolos y haciéndoles preguntas; y se pasmaban todos los que le oían de su
inteligencia y de sus respuestas. Y sus padres, al verle, quedaron sorprendidos; y le dijo su madre:
«Hijo, ¿por qué lo niciste así con nosotros? Mira que tu padre y yo, llenos de aflicción, te andábamos buscando». 
\textbf{\emph{Lucas 2, 46-48}}

\begin{center}
      Paternóster, diez Avemarías, Gloria y María, Madre de gracia{\ldots}
      
      Oraciones finales (\cpageref{sec:final-prayer}).
\end{center}

\end{document}