\documentclass[./main.tex]{subfiles}
\newcounter{joyful-counter}
\newcounter{sorrowful-counter}
\newcounter{glorious-counter}
\newcounter{lux-counter}

\begin{document}

\chapter*{Santo Rosario Meditado}

\begin{longtable} { p{0.475\textwidth} p{0.475\textwidth} }
      \label{crossSignal}
      \destacado{Por la señal de la santa cruz}, de nuestros enemigos libranos Señor, Dios nuestro. En el Nombre del Padre,
      y del Hijo, y del Espíritu Santo. Amén.
          &
      \destacado{Per signum crucis} de inimícis nostris libera nos, Deus noster. In Nómine Pátris, et Filii, et Spíritus Sancti. Amen.\\\\
\end{longtable}

\label{contrition}
\destacado{Señor mío Jesucristo}, Dios y Hombre verdadero, Creador y Redentor mío: por ser vos quién sois, y porque os amo sobre todas las cosas,
me pesa de todo corazón de haberos ofendido, propongo firmemente nunca más pecar, y apartarme de todas las ocasiones de ofenderos,
confesarme, y cumplir la penitencia que me fuere impuesta; ofrézcoos mi vida, obras y trabajos en satisfacción de todos mis pecados;
y confío en vuestra bondad y misericordia infinita me los perdonaréis por los merecimientos de vuestra preciosísima sangre, pasión y muerte,
y me daréis gracia para enmendarme y para perseverar en vuestro santo servicio hasta el fin de mi vida. Amén.

\begin{longtable} { p{0.475\textwidth} p{0.475\textwidth} }
      Abre, Señor, mis labios. -- Y mi boca cantará tus alabanzas                                                           
            & 
      Dómine, lábia mea apéries. -- Et os meum annuntiábit laudem tuam\\
      Apresútare, Señor, a socorrerme. -- Ven, oh Dios, en mi ayuda                                                    
            & 
      Deum, in adjutórium meum inténde. -- Dómine, ad adjuvándum me festina\\\\

      \label{glory}
      \destacado{Gloria al Padre}, al Hijo, y al Espíritu Santo. -- Como era en el principio, ahora, y siempre, y por los siglos de los siglos. Amén.                   
            & 
      \destacado{Glória Patri}, et Filio, et Spíritui Sancto. -- Sicut erat in princípio et nunc, et semper et in sæcula sæculórum, Amen.\\\\
      
      \destacado{María}, Madre de gracia, Madre de Misericordia. Defendednos del enemigo y amparadnos ahora y en la hora de nuestra muerte. Amén.                               
            & 
      \destacado{Maria} Mater gratiæ, Mater Misericordiæ. Tu nos ab hoste protege. Et mortis hora suspice.

\end{longtable}

%%%%%%%%%%%%%%%%%%%%%%%%%%%%
% INICIO MISTERIOS GOZOSOS %
%%%%%%%%%%%%%%%%%%%%%%%%%%%%
\section*{Misterios Gozosos}

\begin{itemize}
      \item Tradicional: lunes y jueves.
      \item Nuevo: lunes y sábados
\end{itemize}

\stepcounter{joyful-counter}
\subsection*{\Roman{joyful-counter} Misterio: La Anunciación de la Santísima Virgen María}
Y habiendo entrado a ella, dijo: «Dios te salve, llena de gracia, el Señor es contigo, bendita tu entre las mujeres».
Ella, al oír estas palabras, se turbó, y discurría que podría ser esta salutación. Y le dijo el ángel: «No temas, María,
pues hallaste gracia a los ojos de Dios. He aquí que concebirás en tu seno y darás a luz un Hijo, a quien darás por
nombre Jesús. Este será grande, y será llamado Hijo del Altísimo, y le dará el Señor Dios el trono de David su padre,
y reinará sobre la casa de Jacob etérnamente, y su reinado no tendrá fin» (\textbf{\emph{Lucas 1, 27-33}}).

\begin{center}
      Paternóster, diez Avemarías, Gloria y María, Madre de gracia.
\end{center}

\stepcounter{joyful-counter}
\subsection*{\Roman{joyful-counter} Misterio: La Visitación de Nuestra Señora}
Por aquellos días, levantándose María, se dirigió presurosa a la montaña, a un ciudad de Judá, y entró en la casa de Zacarías y saludó a Isabel.
Y aconteció que, al oir Isabel la salutación de María, dió saltos de gozo el niño en su seno, y fue llena Isabel del Espíritu Santo,
y levantó la voz con gran clamor y dijo: «Bendita tu entre las mujeres y bendito el fruto de tu vientre. ¿Y de dónde a mí esto que venga la madre de mi Señor a mí?
Porque he aquí que, como sonó la voz de tu salutación en mi oídos, dió saltos de alborozo el niño en mi seno (\textbf{\emph{Lucas 1, 39-45}}).

\begin{center}
      Paternóster, diez Avemarías, Gloria y María, Madre de gracia.
\end{center}

\stepcounter{joyful-counter}
\subsection*{\Roman{joyful-counter} Misterio: La Natividad del Nuestro Señor Jesucristo}
Y dió a luz su hijo primogénito, y le envolvió en pañales y le recostó en un pesebre, pues no había para ellos lugar en el mesón.
Y había unos pastores en aquella misma comarca, que pernoctaban al raso y velaban por turno para guardar su ganado,
y un ángel del Señor se presentó ante ellos. Y les dijo el Ángel: «No temáis, pues he aquí que os traigo una buena nueva,
que será de grande alegría para todo el pueblo: que os ha nacido hoy en la ciudad de David un Salvador, que es el Mesías, el Señor» (\textbf{\emph{Lucas 2, 7-8.10-11}}).

\begin{center}
      Paternóster, diez Avemarías, Gloria y María, Madre de gracia.
\end{center}

\stepcounter{joyful-counter}
\subsection*{\Roman{joyful-counter} Misterio: La Presentación del Niño Jesús en el Templo}
Y cuando se les cumplieron los días de la purificación según la ley de Moisés (Levítico 12, 6),
le subieron a Jerusalén para presentarle al Señor, según está escrito en la Ley del Señor que «todo primogénito
del sexo masculino será consagrado al Señor\footnote{Éxodo 13, 2; 12, 15}», y para ofrecer como sacrificio,
según lo que se ordena en la Ley del Señor, «un par de tórtolas o dos palominos\footnote{Levítico 12, 8; 5, 11}» (\textbf{\emph{Lucas 2, 22-24}}).

\begin{center}
      Paternóster, diez Avemarías, Gloria y María, Madre de gracia.
\end{center}

\stepcounter{joyful-counter}
\subsection*{\Roman{joyful-counter} Misterio: La pérdida y hallazgo del Niño Jesús en el Templo}
Y no hallándole, se tornaron a Jerusalén para buscarle. Y sucedió que después de tres días le hallaron en el templo,
sentado en medio de los maestros, escuchándolos y haciéndoles preguntas; y se pasmaban todos los que le oían de su
inteligencia y de sus respuestas. Y sus padres, al verle, quedaron sorprendidos; y le dijo su madre:
«Hijo, ¿por qué lo hiciste así con nosotros? Mira que tu padre y yo, llenos de aflicción, te andábamos buscando» (\textbf{\emph{Lucas 2, 46-48}}).

\begin{center}
      Paternóster, diez Avemarías, Gloria y María, Madre de gracia.\\
      Oraciones finales (\cpageref{final-prayer}).
\end{center}
%%%%%%%%%%%%%%%%%%%%%%%%%%%
% FINAL MISTERIOS GOZOSOS %
%%%%%%%%%%%%%%%%%%%%%%%%%%%

%%%%%%%%%%%%%%%%%%%%%%%%%%%%%%
% INICIO MISTERIOS LUMINOSOS %
%%%%%%%%%%%%%%%%%%%%%%%%%%%%%%
\section*{Misterios Luminosos.}
\begin{itemize}
      \item Tradicional: -
      \item Nuevo: jueves.
\end{itemize}

\stepcounter{lux-counter}
\subsection*{\Roman{lux-counter} Misterio: El Bautismo del Señor en el Jordán}
Y aconteció por aquellos días que vino Jesús desde Nazaret de Galilea y fué bautizado en el Jordán por Juan el Bautista. 
Y al punto subiendo del agua, vió rasgarse los cielos y venir sobre Él el Espíritu Santo como paloma; 
y una voz vino de los cielos: «Tú eres mi Hijo amado, en Ti me agradé» (\textbf{\emph{Marcos 1, 9-11}}).

\begin{center}
      Paternóster, diez Avemarías, Gloria y María, Madre de gracia.
\end{center}

\stepcounter{lux-counter}
\subsection*{\Roman{lux-counter} Misterio: La auto revelación de Jesús en las Bodas de Caná}
Al tercer día hubo una boda en Caná de Galilea, y estaba allí la madre de Jesús. Fue invitado también Jesús con sus discípulos a la boda. 
No tenían vino, porque el vino de la boda se había acabado. En esto dijo la madre de Jesús a este: No tiene vino. 
Díjole Jesús: Mujer, ¿qué nos va a mi y a ti? No es aún llegada mi hora. Dijo la madre a los servidores: Haced lo que Él os diga (\textbf{\emph{Juan 2, 1-5}}).

\begin{center}
      Paternóster, diez Avemarías, Gloria y María, Madre de gracia.
\end{center}

\stepcounter{lux-counter}
\subsection*{\Roman{lux-counter} Misterio: El Anuncio del Rey de Dios}
Después que Juan fue preso vino Jesús a Galilea predicando el Evangelio de Dios y diciendo: Cumplido es el tiempo, 
y el reino de Dios está cercano; arrepentíos y creed en el Evangelio. Llegaron a Cafarnaúm, y luego, el día sábado, 
entrando en la sinagoga, enseñaba. Se maravillaban de su doctrina, pues la enseñaba como quien tiene autoridad y 
no como los escribas (\textbf{\emph{Marcos 1, 14-15.21-22}}).

\begin{center}
      Paternóster, diez Avemarías, Gloria y María, Madre de gracia.
\end{center}

\stepcounter{lux-counter}
\subsection*{\Roman{lux-counter} Misterio: La Transfiguración}
Seis días después tomó Jesús a Pedro, a Santiago y a Juan, su hermano, y los llevo aparte, a un monte alto. 
Y se transfiguró ante ellos; brilló su rostro como el sol y sus vestidos se volvieron blancos como la luz. Aún estaba el hablando, 
cuando los cubrió una nube resplandeciente, y salió de la nube una voz que decía: Este es mi Hijo amado, 
en quien tengo mi complacencia; escuchadle (\textbf{\emph{Mateo 17, 1-3.5}}).

\begin{center}
      Paternóster, diez Avemarías, Gloria y María, Madre de gracia.
\end{center}

\stepcounter{lux-counter}
\subsection*{\Roman{lux-counter} Misterio: La Institución de la Eucaristía}
Mientras comían, Jesús tomó pan, lo bendijo, lo partió y, dándoselo a los discípulos, dijo: Tomad y comed, éste es mi cuerpo. 
Y tomando un cáliz y dando gracias, se lo dió, diciendo: Bebed de él todos, que está es mi sangre del Nuevo Testamento, 
que será derramada por muchos para remisión de los pecados (\textbf{\emph{Mateo 26, 26-28}}).

\begin{center}
      Paternóster, diez Avemarías, Gloria y María, Madre de gracia.\\
      Oraciones finales (\cpageref{final-prayer}).
\end{center}
%%%%%%%%%%%%%%%%%%%%%%%%%%%%%
% FINAÑ MISTERIOS LUMINOSOS %
%%%%%%%%%%%%%%%%%%%%%%%%%%%%%

%%%%%%%%%%%%%%%%%%%%%%%%%%%%%%
% INICIO MISTERIOS DOLOROSOS %
%%%%%%%%%%%%%%%%%%%%%%%%%%%%%%
\section*{Misterios Dolorosos}
\begin{itemize}
      \item Tradicional: martes y viernes.
      \item Nuevo: martes y viernes.
\end{itemize}

\stepcounter{sorrowful-counter}
\subsection*{\Roman{sorrowful-counter} Misterio: La Agonía de Nuestro Señor en el Huerto de los Olivos}
Y lleva consigo a Pedro y a Santiago y a Juan, y comenzó a sentir espanto y abatimiento; y le dice:
«triste en gran manera está mi corazón hasta la muerte; quedad aquí y velad». Y apartándose un poco,
caía sobre tierra, y rogaba que, a ser posible, pasase el Él aquella hora, y decía: «Abba, Padre, todas las cosas te son posibles:
traspasa de mi este cáliz; más no se haga lo que yo quiero, sino lo que tú quieres» (\textbf{\emph{Marcos 14, 33-36}}).

\begin{center}
      Paternóster, diez Avemarías, Gloria y María, Madre de gracia.
\end{center}

\stepcounter{sorrowful-counter}
\subsection*{\Roman{sorrowful-counter} Misterio: La Flagelación de Nuestro Señor Jesucristo}
«Yo no hallo en Él delito alguno. Es costumbre vuestra que yo os suelte un preso por la Pascua:
¿queréis, pues, que os suelte al rey de los Judíos?». Gritaron, pues, de nuevo, diciendo: «No, a ése, sino a Barrabás».
Era este Barrabás un salteador. Entonces, pues, tomó Pilato a Jesús y le azotó (\textbf{\emph{Juan 18,38-40;19,1}}).

\begin{center}
      Paternóster, diez Avemarías, Gloria y María, Madre de gracia.
\end{center}

\stepcounter{sorrowful-counter}
\subsection*{\Roman{sorrowful-counter} Misterio: La Coronación de espinas de Nuestro Señor Jesucristo}
Entonces los soldados del gobernador, tomando a Jesús y conduciéndole al pretorio, reunieron en torno a Él toda la cohorte. 
Y habiéndole quitado sus vestidos, le envolvieron en una clámide de grana, y trenzando una corona de espinas, 
la pusieron sobre su cabeza, y una caña en su mano derecha; y doblando la rodilla delante de Él, 
le mofaban, diciendo: «Salud, Rey de los judíos». Y escupiendo en Él, tomaron la caña y le daban golpes en la cabeza (\textbf{\emph{Mateo 27, 27-30}}).

\begin{center}
      Paternóster, diez Avemarías, Gloria y María, Madre de gracia.
\end{center}

\stepcounter{sorrowful-counter}
\subsection*{\Roman{sorrowful-counter} Misterio: El Señor con la cruz a cuestas}
Entonces, pues, se le entregó para que fuera crucificando. Se apoderaron, pues, de Jesús, y llevando a cuestas su cruz, salió hacia el lugar llamado el Cráneo, 
que en hebreo se dice Gólgota (\textbf{\emph{Juan 19, 16-17}}). Y como le hubieron sacado, echaron mano de un tal Simón de Cirene que venía del campo, 
le pusieron en hombros la cruz para que la llevase detrás de Jesús (\textbf{\emph{Lc. 23, 26}}).

\begin{center}
      Paternóster, diez Avemarías, Gloria y María, Madre de gracia.
\end{center}

\stepcounter{sorrowful-counter}
\subsection*{\Roman{sorrowful-counter} Misterio: Crucifixión y Muerte del Redentor}
Cuando llegaron al lugar llamado Calvario, le crucificaron allí, y a los dos malhechores, uno a la derecha y otro a la izquierda. Jesús decía: Padre, perdónalos, porque no saben los que hacen. 
Dividiendo sus vestidos, echaron suertes sobre ellos (\textbf{\emph{Lucas 23, 33-34}}). Escribió Pilato un título y lo puso sobre la cruz; estaba escrito: \emph{Jesús Nazareno, Rey de los judíos}. 
Estaba junto a la cruz de Jesús su Madre y la hermana de su Madre, María la debajo Cleofás y María Magdalena. Jesús, viendo a su Madre y al discípulo a quien amaba, que estaba allí, 
dijo a la Madre: Mujer, he ahí a tu hijo. Luego dijo al discípulo: He ahí a tu Madre. Y desde aquella hora el discípulo la recibió en su casa (\textbf{\emph{Juan 19, 19.25-27}}). Era ya como
la hora de sexta, y las tinieblas cubrieron toda la tierra hasta la hora de nona, obscurecióse el sol y el velo del templo se rasgó por medio. Jesús, dando una gran voz, dijo: Padre, en tus 
manos entrego mi espíritu; y diciendo esto expiró (\textbf{\emph{Lucas 23, 44-46}}).

\begin{center}
      Paternóster, diez Avemarías, Gloria y María, Madre de gracia.\\
      Oraciones finales (\cpageref{final-prayer}).
\end{center}
%%%%%%%%%%%%%%%%%%%%%%%%%%%%%
% FINAL MISTERIOS DOLOROSOS %
%%%%%%%%%%%%%%%%%%%%%%%%%%%%%

%%%%%%%%%%%%%%%%%%%%%%%%%%%%%%
% INICIO MISTERIOS GLORIOSOS %
%%%%%%%%%%%%%%%%%%%%%%%%%%%%%%
\section*{Misterios Gloriosos.}
\begin{itemize}
      \item Tradicional: miércoles, sábados y domingos.
      \item Nuevo: miércoles y domingos.
\end{itemize}

\stepcounter{glorious-counter}
\subsection*{\Roman{glorious-counter} Misterio: La Resurrección del Señor}
Pasado el sábado, ya para amanecer el día primero de la semana, vino María Magdalena con la otra María al sepulcro. Y sobrevino un gran terremoto, 
pues un ángel del Señor bajó del cielo y acercándose removió la piedra del sepulcro y se sentó sobre ella. Era su aspecto como el relámpago, 
y su vestidura blanca como la nueve. El ángel, dirigiéndose a las mujeres, dijo: No temáis vosotras, pues sé que buscáis a Jesús el crucificado. 
No está aquí; ha resucitado, según lo había dicho. Venid y ved el sitio donde fue puesto. Id luego y decid a sus discípulos que ha resucitado de entre los muertos 
y que os precede a Galilea; allí le veréis. Es lo que tenía que deciros (\textbf{\emph{Mateo 28, 1-3.5-7}}).

\begin{center}
      Paternóster, diez Avemarías, Gloria y María, Madre de gracia.
\end{center}

\stepcounter{glorious-counter}
\subsection*{\Roman{glorious-counter} Misterio: Las Ascensión Jesucristo a los cielos}
Los llevó hasta cerca de Betania, y levantando sus manos les bendijo (\textbf{\emph{Lucas 24, 50}}). Y les dijo: «Id al mundo entero y predicad el Evangelio a toda la creación. 
El que creyere y fuere bautizado, se salvará, mas el que no creyere, será condenado. Con esto el Señor Jesús, después de hablarles, fue elevado al cielo y se sentó a la diestra de Dios. 
Y ellos, partiéndose de allí, predicaron por todas partes, cooperando el Señor y confirmando la palabra con las señales que la acompañaban (\textbf{\emph{Marcos 16, 15-16.19-20}}).

\begin{center}
      Paternóster, diez Avemarías, Gloria y María, Madre de gracia.
\end{center}

\stepcounter{glorious-counter}
\subsection*{\Roman{glorious-counter} Misterio: La Venida del Espíritu Santo sobre los Apóstoles}
Mas el Paráclito, el Espíritu Santo, que enviará el Padre en mi nombre, Él os enseñará todas las cosas y os recordará todas las cosas que os dije yo (\textbf{\emph{Jn. 14, 26}}). 
Y al cumplirse el día de Pentecostés, estaban todos juntos en el mismo lugar. Y se produjo de súbito desde el cielo un estruendo como de viento que soplaba vehementemente, 
y llenó toda la casa donde se hallaban sentados. Y vieron aparecer lenguas como de fuego, que, repartiéndose, se posaban sobre cada uno de ellos (\textbf{\emph{Hechos 2, 1-4}}).

\begin{center}
      Paternóster, diez Avemarías, Gloria y María, Madre de gracia.
\end{center}

\stepcounter{glorious-counter}
\subsection*{\Roman{glorious-counter} Misterio: La Asunción de María Santísima a los cielos}
Mi alma magnifica al Señor y exulta de júbilo mi espíritu al Señor, mi Salvador, porque ha mirado la humildad de su sierva; por eso todas las generaciones
me llamarán bienaventurada, porque ha hecho en mi maravillas el Poderoso, cuyo nombre es santo (\textbf{\emph{Lucas 1, 46-49}}). Y se abrió el templo de Dios, que está en el cielo, y 
fué vista el arca de la alianza en el templo, y se produjeron relámpagos, y voces, y truenos, y temblor de tierra, y fuerte granizada (\textbf{\emph{Apocalipsis 11, 19}}).

\begin{center}
      Paternóster, diez Avemarías, Gloria y María, Madre de gracia.
\end{center}

\stepcounter{glorious-counter}
\subsection*{\Roman{glorious-counter} Misterio: La Coronación de la Santísima Virgen María}
¿Quién es esa que aparece resplandeciente como la aurora, hermosa cual luna, deslumbradora como el sol, imponente como batallones? \textbf{\emph{Cant 6,10}}. Y una gran señal 
fué vista en el cielo: una Mujer vestida del sol, y la luna debajo de sus  pies, y sobre su cabeza una corona de doce estrellas (\textbf{\emph{Apocalipsis 12, 1}}). 
Tiene Él escrito en su vestido y en su manto Rey de reyes y Señor de los que dominan (\textbf{\emph{Ap. 18, 16}}). 
Está la Reina a su derecha, adornada con oro finísimo (\textbf{\emph{Sal. 44, 10}}).

\begin{center}
      Paternóster, diez Avemarías, Gloria y María, Madre de gracia.
\end{center}
%%%%%%%%%%%%%%%%%%%%%%%%%%%%%
% FINAL MISTERIOS GLORIOSOS %
%%%%%%%%%%%%%%%%%%%%%%%%%%%%%

\end{document}